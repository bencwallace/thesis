\chapter{An implicit function theorem}
\label{sec:IFT}

In this appendix, we prove Proposition~\ref{prop:IFT}.

\section{Implicit function theorem with a parameter}

We make use of \cite[Chapter 4, Theorem~9.3]{LS14},
which is a version of the implicit function theorem that allows
for a continuous, rather than differentiable, parameter.
While the precise statement of \cite[Chapter 4, Theorem~9.3]{LS14}
takes this parameter from an open subset of a Banach space,
by \cite[Chapter 4, Theorem~9.2]{LS14}, the
parameter can in fact be taken from an arbitrary metric space.
With this minor change, we restate \cite[Chapter 4, Theorem~9.3]{LS14}
as the following proposition.

\begin{prop}
\label{prop:LS-IFT}
Let $A$ be a metric space, let $W,X$ be Banach spaces,
and let $B \subset W$ be an open subset.
Let $F : A \times B \to X$ be continuous,
and suppose that $F$ is $C^1$ in its second argument.
Let $(\alpha, \beta) \in A \times B$ be a point such that
$F(\alpha, \beta) = 0$ and $D_2 F(\alpha, \beta)^{-1}$ exists.
Then there are open balls $M \ni \alpha$ and $N \ni \beta$
and a unique continuous mapping $f : M \to N$ such that $F(\xi, f(\xi)) = 0$
for all $\xi \in M$.
\end{prop}

We also use the following lemma, which is a small modification of
\cite[Chapter 3, Theorem~11.1]{LS14}. In particular, it considers
functions that may only be left- or right-differentiable.

\begin{lemma}
\label{lem:IFT-C1}
Let $F$ be a mapping as in the previous proposition
with $A \subset \R^{m_1} \times \R^{m_2}$.
In addition, suppose that $F$ is left-differentiable (respectively, right-differentiable)
in $\alpha_2$ at $(\alpha, \beta)$, with $\alpha = (\alpha_1, \alpha_2)$.
If $f$ is a continuous mapping defined in a neighbourhood of
$\alpha$, such that $F(\xi, f(\xi)) = 0$,
then $f$ is left-differentiable (respectively, right-differentiable) in $\alpha_2$ at $\alpha$.
\end{lemma}

\section{Main result}

The above results lead to the following proposition, which we apply
in the proofs of Propositions~\ref{prop:nuzhat} and \ref{prop:changevariables1}.
Recall that $D(\delta, r)$ is defined in \refeq{Ddef}.

\begin{prop}
\label{prop:IFT-re}
Let $\delta > 0$, and let $r_1, r_2$ be continuous positive-definite functions on $[0, \delta]$.
Set
\begin{equation}
    D(\delta, r_1, r_2)
    =
    \{ (w, x, y, z) \in D(\delta, r_1) \times \R^n : |z| \leq r_2(x) \},
\end{equation}
and let $F$ be a continuous function on $D(\delta, r_1, r_2)$ that is $C^1$ in $(x, z)$.
Suppose that for all $(\bar w, \bar x) \in [0, \delta]^2$ there exists $\bar z$
such that both $F(\bar w, \bar x, 0, \bar z) = 0$
and $D_Y F(\bar w, \bar x, 0, \bar z)$ is invertible.
Then there is a continuous positive-definite function $r$ on $[0, \delta]$ and
a continuous map $f : D(\delta, r) \to \R^n$
that is $C^1$ in $x$
and such that $F(w, x, y, f(w, x, y)) = 0$
for all $(w, x, y) \in D(\delta, r)$.
Moreover, if $F$ is left-differentiable
(respectively, right-differentiable) in $y$ at some point $(w, x, y, z)$,
then $f$ is left-differentiable (respectively, right-differentiable) at $(w, x, y)$.
\end{prop}

\begin{proof}
Take any $(\bar w, \bar x) \in [0, \delta] \times (0, \delta]$
and let $R(\bar w, \bar x)$ be the maximal radius $s$ such that
for all $(w, x, y) \in B(\bar w, \bar x, 0; s)$ there exists $z$
such that both $F(w, x, y, z) = 0$ and $D_Z F(w, x, y, z)$ is
invertible. By continuity of $(D_Z F(w, x, y, z))^{-1}$ near
$(\bar w, \bar x, 0, \bar z)$, and by Proposition~\ref{prop:LS-IFT}
(applied to the restriction of $F$ to $A \times B$, for some
$A \ni (\bar w, \bar x, 0)$ and an open set $B \ni \bar z$),
we have $R(\bar w, \bar x) > 0$ and there is a continuous function
\begin{equation}
f_{\bar w,\bar x} : B(\bar w, \bar x, 0; R(\bar w, \bar x)) \to \R^n
\end{equation}
such that $F(w, x, y, f_{\bar w,\bar x}(w, x, y)) = 0$
for all $(w, x, y) \in B(\bar w, \bar x, 0; R(\bar w, \bar x))$.
Moreover, the unique solution to $F(w, x, y, z) = 0$
is given by $z = f_{\bar w,\bar x}(w, x, y)$ for all such $(w, x, y)$.
By an application of Lemma~\ref{lem:IFT-C1}
(with $\alpha_1 = (w, x), \alpha_2 = y$),
we see that $f_{\bar w, \bar x}$ is
left- or right-differentiable in $y$ wherever $F$ is.
By another application of Lemma~\ref{lem:IFT-C1} (with $\alpha_1 = (w, y), \alpha_2 = x$),
we see that $f_{\bar w, \bar x}$ is $C^1$ in $x$.

Set $R(\bar w, 0) = 0$ for all $\bar w \in [0, \delta]$, and
let
\begin{equation}
D_f = \bigcup_{(\bar w,\bar x)\in [0, \delta]^2} B(\bar w, \bar x, 0; R(\bar w, \bar x)).
\end{equation}
We define $f(w, 0, 0) = 0$ and, for $x > 0$,
\begin{equation}
f(w, x, y) = f_{\bar w,\bar x}(w, x, y)
  \quad\text{for}\quad
(w, x, y) \in B(\bar w, \bar x, 0; R(\bar w, \bar x)).
\end{equation}
By uniqueness, this function is well-defined.
Continuity of $f$ at $(w, 0, 0)$
follows from the fact that $|f(w, x, y)| \le r_2(x)$.
The remaining desired regularity properties of $f$
follow from those of the $f_{\bar w,\bar x}$.
It remains to show that $D(\delta,r) \subset D_f$
for some continuous positive-definite function $r$ on $[0, \delta]$.

First, let us show that $R$ is continuous on $[0, \delta]^2$.
Let $\bar x > 0$ and fix $0 < \epsilon < R(\bar w, \bar x)$.
Then for any $(\bar w', \bar x') \in [0,\delta] \times (0, \delta]$ such that
$|(\bar w, \bar x) - (\bar w', \bar x')| < \epsilon$,
we have $B(\bar w', \bar x', 0; R(\bar w, \bar x) - \epsilon) \subset B(\bar w, \bar x, 0; R(\bar w, \bar x))$
by maximality of $R$.
It follows that $R(\bar w', \bar x') \geq R(\bar w, \bar x) - \epsilon$.
By a similar argument, $R(\bar w', \bar x') \leq R(\bar w, \bar x) + \epsilon$,
so $|R(\bar w, \bar x) - R(\bar w', \bar x')| \leq \epsilon$.
Thus, $R$ is continuous on $[0, \delta] \times (0, \delta]$.
Continuity at $\bar x = 0$ follows from the fact that $R(\bar w, \bar x) \le r_1(\bar x)$
uniformly in $\bar w$.

For $\bar x \in [0,\delta]$, let
\begin{equation}
r(\bar x) = \inf (R(\bar w, \bar x) : \bar w \in [0, \delta]).
\end{equation}
Since $R(\cdot, \bar x)$ is continuous, $r(\bar x) > 0$
for $\bar x > 0$. Moreover, $0 \le r(0) \le r_1(0) = 0$, so $r$ is positive-definite.
Continuity of $r$ follows from joint continuity of $R$.
For any $(w, x, y) \in D(\delta, r)$ (with this choice of $r$),
\begin{equation}
|(w, x, y) - (w, x, 0)| = |y| < r(x) \leq R(w, x),
\end{equation}
so $(w, x, y) \in B(w, x, 0; R(w, x))$.
We conclude that $D(\delta, r) \subset D_f$.
\end{proof}