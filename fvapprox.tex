%% Use this to reset the appendix counter.  Note that the FoGS
%% requires that the word ``Appendices'' appear in the table of
%% contents either before each appendix lable or as a division
%% denoting the start of the appendices.  We take the latter option
%% here.  This is ensured by making the \texttt{appendicestoc} option
%% a default option to the UBC thesis class.

%%% If you only have one appendix, please uncomment the following line.
\renewcommand{\appendicesname}{Appendix}

% Start appendices
\appendix

% First appendix (from saw-sa)
\chapter{Proof of finite-volume approximation}
\label{sec:finvol}

In this appendix, we prove Proposition~\ref{prop:finvol}.

\todo{Other things that could be included are the proof of the integral
representation and bounds on the critical point. This would also be a
good place to briefly explain how $\gamma < 0$ is handled.}

%%%%%%%%%%%%%%%%%%%%%%%%%%%%%%%%%%%%%%%%%%%%%%%%%%%%%%%%%%%%%%%%%%%%%%%%%%%%%%%
%%%%%%%%%%%%%%%%%%%%%%%%%%%%%%%%%%%%%%%%%%%%%%%%%%%%%%%%%%%%%%%%%%%%%%%%%%%%%%%

\section{Proof}

Before proving the proposition, we require some preliminaries.
Let $P^n$ be the projection
of $\Zd$ onto the discrete torus of side $n$,
which we denote $\Z_n^d$.
Then $P^n$ has a natural action
on the path space $(\Zd)^{[0,\infty)}$. We let
$X^n = P^n(X)$ be the projection of $X$
and note that $X^n$ is a simple random walk on $\Z^d_n$.

We call $h = (h_x)_{x\in\Zd}$ a \emph{field of path functionals} if
$h_x : (\Zd)^{[0,\infty)} \to \R$ is a function on continuous-time paths
for each $x \in \Zd$;
a simple example is given by the local time functional.
We assume that the \emph{random} field $h(X) = (h_x(X))_{x\in\Zd}$
has finite support almost surely, i.e.,
with probability $1$, $h_x(X) = 0$ for all but finitely many $x$.
Denote by $h(X^n)$ the corresponding random field for $X^n$, i.e., for $x \in \Z_n^d$,
\begin{equation}
h_x(X^n) = \sum_{y\in\Zd} h_{x+ny}(X).
\end{equation}

Given a positive integer $k$, we define
$Q_k \subset \Z^d$ by $Q_k = \{y \in \Z^d : 0 \le y_i < k, \;   i=1,\ldots,d\}$.
Then, for integers $n,k \ge 1$,
\begin{equation}
\label{e:ffold1}
    \sum_{y \in Q_k} h_{x+ny}(X^{kn})
  = \sum_{y \in Q_k} \sum_{z\in\Zd} h_{x+ny+knz}(X)
  = \sum_{y\in\Zd} h_{x+ny}(X)
  = h_x(X^n),
\end{equation}
and it follows by summation over $x \in \Z^d_n$ that
\begin{equation}
\label{e:ffold2}
\sum_{x\in\Z^d_{kn}} h_x(X^{kn})
  =
\sum_{x\in\Z^d_n} h_x(X^n).
\end{equation}

\begin{lemma}
\label{lem:mono}
Let $n,k \ge 1$ and let $f$ and $g$ be nonnegative fields of path functionals
with finite support almost surely.
Then
\begin{equation}
\sum_{x\in\Z^d_{kn}} f_x(X^{kn}) g_x(X^{kn})
  \leq
\sum_{x\in\Z^d_n} f_x(X^n) g_x(X^n).
\end{equation}
\end{lemma}

\begin{proof}
By \eqref{e:ffold2} and \eqref{e:ffold1},
\begin{equation}
\sum_{x\in\Z_{kn}^d} f_x(X^{kn}) g_x(X^{kn})
  =
\sum_{x\in\Z_n^d}
\sum_{y \in Q_k}
  f_{x+ny}(X^{kn}) g_{x+ny}(X^{kn}).
\lbeq{mono}
\end{equation}
By nonnegativity and two more applications of \eqref{e:ffold1},
\begin{align}
\sum_{x\in\Z_n^d}
\sum_{y \in Q_k}
f_{x+ny}(X^{kn}) g_{x+ny}(X^{kn})
  &\le \sum_{x\in\Z_n^d}
      \left(\sum_{y \in Q_k} f_{x+ny}(X^{kn})\right)
      \left(\sum_{y \in Q_k} g_{x+ny}(X^{kn})\right) \nonumber \\
  &= \sum_{x\in\Z_n^d} f_x(X^n) g_x(X^n).
\end{align}
This completes the proof.
\end{proof}

For $L \geq 2$ and $N \geq 1$
note that $\Lambda_N$ is the torus $\Z^d_n$ with $n=L^N$.
Thus, $X^{L^N}$ is the simple random walk on $\Lambda_N$.
For $F_T = F_T(X)$ any one of the functions $L_T^x,I_T,C_T$
of $X$ defined in \eqref{e:LTx-def}--\eqref{e:CTdef},
we write $F_{N,T} = F_T(X^{L^N})$. For instance, with $n=L^N$,
\begin{equation}
    L^x_{N,T} = \int_0^T \1_{X^{n}_t=\;x} \; dt,
    \quad I_{N,T} = \sum_{x \in \Lambda_N}(L_{N,T}^x)^2 .
\end{equation}
We apply Lemma~\ref{lem:mono} with $k = L$ and $n = L^N$ for three
choices of $f$, $g$:
\begin{alignat}{2}
\label{e:ILT-mon}
I_{N+1,T} &\leq I_{N,T}
	\quad
&&(f_x=g_x=L_T^x),
	\\
\label{e:CSA-mon}
C_{N+1,T} &\leq C_{N,T}
	\quad
&&(f_x=\textstyle{\sum_{e\in \Ucal}L_T^{x+e}},\; g_x=L_T^x),
	\\
\lbeq{nabL}
\sum_{x\in\Lambda_{N+1}} |\nabla^e L^x_{N+1,T}|^2
	&\leq
\sum_{x\in\Lambda_N} |\nabla^e L^x_{N,T}|^2
	\quad
&& (f_x = g_x = \left|\nabla^e L_T^x\right|).
\end{alignat}
Summation of \refeq{nabL} over unit vectors $e\in\Zd$ also gives
\begin{align}
\label{e:gradLT-mon}
\sum_{x\in\Lambda_{N+1}} |\nabla L^x_{N+1,T}|^2
  \leq
\sum_{x\in\Lambda_N} |\nabla L^x_{N,T}|^2.
\end{align}

Recall that we are identifying the vertices of $\Lambda_N$ with nested subsets of $\Zd$.
% centred at the origin (approximately if $L$ is even),
% with $\Lambda_{N+1}$ paved by $L^d$ translates of $\Lambda_N$.
We can thus define $\partial \Lambda_N$ to be the inner vertex boundary of $\Lambda_N$.
We denote the expectation of $X^{L^N}$ started from $0$ by $E^{\Lambda_N}_0$
and define
\begin{align}
\label{e:cN}
c_{N,T}(x)
    &=
E^{\Lambda_N}_0 \left( e^{-U_{\gcc,\gamma,T}} \1_{X(T)=b} \right),
	\quad
x \in \Lambda_N \\
c_{N,T}
    &=
E^{\Lambda_N}_0 \left( e^{-U_{\gcc,\gamma,T}} \right).
\end{align}
The finite-volume two-point function and susceptibility
are defined by
\begin{align}
G_x(\gcc,\gamma,\nu)
    &= \int_0^\infty c_{N,T}(x) e^{-\nu T} \; dT, \\
\chi_N(\gcc, \gamma, \nu)
    &= \int_0^\infty c_{N,T} e^{-\nu T} \; dT
    .
    \label{e:chiNdef}
\end{align}
Here and throughout this section we have dropped the parameter $n$
from the notation since $n = 0$.

\begin{prop}
\label{prop:finvol-re}
Let $d >0$, $\gcc >0$ and $\gamma < \gcc$. For all $\nu \in \R$,
\begin{equation}
\label{e:Givlc}
\lim_{N \to \infty}
G_x(\gcc,\gamma,\nu)
=
G_x(\gcc,\gamma,\nu)
\end{equation}
and
\begin{equation}
\label{e:chilim}
\lim_{N\to\infty}\chi_N(\gcc,\gamma,\nu)=   \chi(\gcc,\gamma,\nu).
\end{equation}
\end{prop}

\begin{proof}
We only prove the case $\gamma \ge 0$. The proof for $\gamma < 0$ can be found in
\cite{BSW-saw-sa}.

Fix $x \in \Zd$, and consider $N$ sufficiently large that $x$ can be identified
with points in $\Lambda_N$.
By \eqref{e:V2}, \eqref{e:ILT-mon} and \eqref{e:gradLT-mon}
% (if $0 \le \gamma < \gcc$),
% or by \eqref{e:V}, \eqref{e:ILT-mon} and \eqref{e:CSA-mon} (if $\gamma < 0$),
\begin{equation}
\label{e:ctmon}
c_{N,T}(x) \leq c_{N+1,T}(x).
\end{equation}
Thus, \eqref{e:Givlc} follows by monotone convergence, once we show that
\begin{equation}
\lim_{N\to\infty} c_{N,T}(x) = c_T(x).
\end{equation}

% This follows as in \cite[(2.8)]{BBS-saw4}.
To show this, we define
\begin{align}
c_{N,T}^*(x)
  &=
E^{\Lambda_N}_0
\left(
  e^{-U_{\gcc,\gamma,T}} \1_{X(T)=b} \1_{\{X([0, T]) \cap \partial \Lambda_N \neq \varnothing\}}
\right) \\
c_T^*(x)
  &=
E_0
\left(
  e^{-U_{\gcc,\gamma,T}} \1_{X(T)=b} \1_{\{X([0, T]) \cap \partial \Lambda_N \neq \varnothing\}}
\right).
\end{align}
Since walks which do not reach $\partial \Lambda_N$ make equal contributions to both
$c_T(x)$ and $c_{N,T}(x)$,
we have
\begin{equation}
c_T(x) - c_T^*(x) = c_{N,T}(x) - c_{N,T}^*(x).
\end{equation}
Thus,
\begin{align}
|c_T(x) - c_{N,T}(x)|
= |c_T^*(x) - c_{N,T}^*(x)|
\leq c_T^*(x) + c_{N,T}^*(x).
\end{align}
Let $P^{\Lambda_N}_0$ and $P_0$ be the measures
associated with $E^{\Lambda_N}_0$ and $E_0$, respectively.
With $Y_t$ a rate-$2d$ Poisson process with measure ${\sf P}$,
\begin{align}
 c_T^*(x) + c_{N,T}^*(x)
  &\leq P_0 (X([0, T]) \cap \partial\Lambda_N \neq \varnothing)
    + P^{\Lambda_N}_0 (X([0, T]) \cap \partial\Lambda_N \neq \varnothing) \nonumber \\
  &\leq 2 {\sf P} (Y_T \geq \diam{\Lambda_N}) \to 0
\end{align}
as $N\to\infty$.
This completes the proof of \eqref{e:Givlc}.

Finally, by monotone convergence of $G_N$ to $G$,
for $\nu \in \R$,
\begin{equation}
\lim_{N\to\infty} \chi_N(g, \gamma, \nu)
    = \sum_{x\in\Zd} \lim_{N\to\infty} G_{x,N}(\gcc,\gamma,\nu) \1_{x\in\Lambda_N}
    = \chi(g, \gamma, \nu),
\end{equation}
which proves \eqref{e:chilim}.
\end{proof}