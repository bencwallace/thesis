\chapter{Analysis of critical behaviour}
\label{sec:chi-G-xi}

\setcounter{footnote}{0}

% \commentbw{Generally I am having a hard time to learn the structure of the thesis and
% of the proof.  Somewhere as early as possible I would like to see a description
% of the structure of the thesis:
% - the main ingredients of the proof of Thm 1.7.1
% - where these ingredients are proved
% - what are you including about the proof and what results are you citing
%   from the literature.
% This relates to one of my comments about Chapter 2 concerning the roles
% of Thms 2.7.1 and 2.8.1.}

In this chapter, we prove Theorem~\ref{thm:mr} using Theorem~\ref{thm:rhatflow}.
For simplicity, we drop the parameter $n$ from the notation. In order to employ
Theorem~\ref{thm:rhatflow}, we fix
\begin{equation}
\nu_0 = \hat\nu_0^c(m^2, g_0, \gamma_0),
	\quad
z_0 = \hat z_0^c(m^2, g_0, \gamma_0).
\end{equation}
Then Theorem~\ref{thm:rhatflow} defines a sequence
\begin{equation}
(U_j, K_j) \in \domRG_j,
	\quad
0 \le j \le N
\end{equation}
for any $N$. Moreover, $U_j$ is independent of the volume, so we actually have
an \emph{infinite} sequence
\begin{equation}
U_j \in \DV_j
	\quad
j \ge 0.
\end{equation}
Throughout this section we write $U_j$ as
\begin{equation}
U_{j;x}
	=
g_j \tau_y^2 + \nu_j \tau_y + z_j \tau_{\Delta,y} + u_j
- \lambda_{0,j} f_0 \1_{y=0}
- \lambda_{x,j} f_x \1_{y=x}
- \tfrac12 (\1_{y=0} q_{0,j} + \1_{y=x} q_{x,j}) \sigma_0 \sigma_x.
\end{equation}


%%%%%%%%%%%%%%%%%%%%%%%%%%%%%%%%%%%%%%%%%%%%%%%%%%%%%%%%%%%%%%%%%%%%%%%%%%%%%%%
%%%%%%%%%%%%%%%%%%%%%%%%%%%%%%%%%%%%%%%%%%%%%%%%%%%%%%%%%%%%%%%%%%%%%%%%%%%%%%%

\section{Susceptibility}
\label{sec:suscept}

% By Theorem~\ref{thm:rhatflow}, for any $N$ there exist
% $(V_N, K_N) \in \domRG_N$ such that \eqref{e:IcircKnew} becomes
The proof of Theorem~\ref{thm:mr}(ii) involves some small changes to the proof
of the $\gamma_0 = 0$ case in \cite{BBS-saw4-log}. Rather than specifying the
individual changes that need to be made, here we sketch the complete argument.

Since the only scale-$N$ blocks are the empty set and $\Lambda$, at scale $j = N$
the representation \eqref{e:IcircKnew} becomes
\begin{equation}
\label{e:ZNINKN}
Z_N = e^{\zeta_N} (I_N(\Lambda) + K_N(\Lambda)).
\end{equation}
In particular, \eqref{e:DVdef}--\eqref{e:domRG},
\eqref{e:Vnormdef}, \eqref{e:Wbilinbd}, and \eqref{e:T0dom} imply that
\begin{align}
\label{e:unu-bd}
u_N |\Lambda_N| = O(1),
	&\qquad
\nu_N = O(L^{-2 N} g_N),
	\\
\label{e:WKbd}
\|W_N\|_{T_0(\ell_N)} \le O(\chicCov_N g_N^2),
	&\qquad
\|K_N\|_{T_0(\ell_N)} \le O(\chicCov_N g_N^3).
\end{align}
Now by \eqref{e:ZNINKN} and \eqref{e:chibarm} together with the definitions of
$I_N$ and $V_N$,
\begin{equation}
\label{e:chiNhat-approx}
\hat\chi_N(m^2, \gcc_0, \gamma_0, \nu_0^c, z_0^c)
	=
\frac{1}{m^2}
	+
\frac{1}{m^4 |\Lambda|}
\new{
\frac
{-\nu_N |\Lambda| + D^2 W^0_N(0; \1, \1) + D^2 K^0_N(0; \1, \1)}
{(1 + W^0_N(0) + K^0_N(0))}}.
\end{equation}
% \commentbw{(3.1.4):  dropped label $\bulk$ for bulk on RHS.  Also we now know
% that $D^2 W = 0$.  There should be no $e^{u_N |\Lambda|}$ in denominator.}

\new{
\begin{rk}
In fact, with a bit more work it can be shown that $D^2 W^0_N(0; \1, \1) = 0$.
However, we will not need this here.
\end{rk}}

Using Lemma~\ref{lem:deriv-norm-bds} with $s = 0$ (recall \eqref{e:elldef-zz})
\new{together with \eqref{e:WKbd}},
we see that the last term vanishes as $N\to\infty$ leaving
% By \eqref{e:Wbilinbd}, \eqref{e:VKbds}, and \eqref{e:VjKjDj-hat} with $s = 0$
% in the choice of $\ell_j$, the second term vanishes as $N\to\infty$ leaving
\begin{equation}
\label{e:chi-m-hat}
\hat\chi(m^2, \gcc_0, \gamma_0, \nu_0, z_0)
	=
\lim_{N\to\infty} \hat\chi_N(m^2, \gcc_0, \gamma_0, \nu_0, z_0)
	=
\frac{1}{m^2}.
\end{equation}
In order to identify the asymptotics of $m^2$ as $\nu$ approaches the
critical point, we will need information about the derivative of $\hat\chi$ with
respect to $\nu_0$. Let us denote by $F'$ the derivative of a function $F$ with
respect to $\nu_0$. By \eqref{e:chiNhat-approx}, the derivative $\hat\chi_N'$
will contain a term $-\nu_N'/m^4$. An argument using Lemma~\ref{lem:deriv-norm-bds}
shows that the remaining terms are of strictly higher order. Together with a careful
analysis of the derivatives of the renormalisation group flow with respect to the initial
condition $\nu_0$ (as in \cite[Section~\ref{log-sec:pfmr}]{BBS-saw4-log} for $\gamma = 0$),
we get
\begin{equation}
\label{e:chiprime-m-hat}
\hat\chi'(m^2, g_0, \gamma_0, \nu_0^c, z_0^c)
	\sim
-\frac{1}{m^4} \frac{c(\hat\gcc_0, \gamma_0)}{(\hat\gcc_0\bubble_{m^2})^{(n+2)/(n+8)}}
	\quad
\text{as $(m^2,\gcc_0,\gamma_0) \to (0,\hat\gcc_0,\hat\gamma_0)$},
\end{equation}
where $c$ is a continuous function. The bubble diagram $\bubble_{m^2}$ was defined
in \eqref{e:bubble} and its logarithmic divergence as $m^2\downarrow0$ is ultimately
the source of the logarithmic corrections in Theorem~\ref{thm:mr}.
% The proof for $\gamma_0 \ne 0$ is almost entirely a direct adaptation of the proof
% found there using the critical parameters $\hat\nu_0^c, \hat z_0^c$ constructed
% in Theorem~\ref{thm:rhatflow} rather than their $\gamma_0 = 0$ analogues.

\begin{rk}
There is one aspect of the proof of \eqref{e:chiprime-m-hat} that must be modified when
$\gamma_0 = 0$: This is the verification of the third bound
in the base case ($j = 0$) of the inductive hypothesis \cite[\eqref{log-e:induct1}]{BBS-saw4-log}.
This will be done in Section~\ref{sec:Ksmooth} (see Remark~\ref{rk:DK-base-case}).
\end{rk}

%%%%%%%%%%%%%%%%%%%%%%%%%%%%%%%%%%%%%%%%%%%%%%%%%%%%%%%%%%%%%%%%%%%%%%%%%%%%%%%

\subsection{Change of parameters}
\label{sec:nuztilde}

We wish to recover the asymptotics of $\chi$ from \eqref{e:chi-m-hat} and
\eqref{e:chiprime-m-hat}. By \eqref{e:chichibar},
\begin{equation}
\label{e:chichihat}
\chi_N(\gcc, \gamma, \nu)
	=
(1 + z_0) \hat\chi_N(m^2, \gcc_0, \gamma_0, \nu_0, z_0),
\end{equation}
whenever the variables on the left- and right-hand sides satisfy
\begin{equation}
\label{e:gg0-re}
\gcc_0 = (\gcc_0 - \gamma) (1 + z_0)^2,
\quad
\nu_0 = \nu (1 + z_0) - m^2,
\quad
\gamma_0 = \frac{1}{4d} \gamma (1 + z_0)^2.
\end{equation}
On the other hand, \eqref{e:chi-m-hat} is contingent on the initialization of
the renormalisation group with the critical parameters
\begin{equation}
\label{e:crit-constraint}
\nu_0 = \hat \nu_0^c(m^2, \gcc_0, \gamma_0),
	\quad
z_0   = \hat z_0^c(m^2, \gcc_0, \gamma_0).
\end{equation}

Given $\gcc,\gamma,\nu$,
the relations \eqref{e:gg0-re} leave free two of the variables
$(m^2, \gcc_0, \gamma_0, \nu_0, z_0)$.
More generally, if any three of the variables
$(\gcc, \gamma, \nu, m^2, \gcc_0, \gamma_0, \nu_0, z_0)$
are fixed, then two of the remaining variables are free.
In the following two propositions, which together form an extension of
\cite[Proposition~\ref{log-prop:changevariables}]{BBS-saw4-log},
we fix three variables and show that the addition of the constraints
\eqref{e:crit-constraint}
% \begin{equation}
% \label{e:crit-constraint}
% \nu_0 = \hat \nu_0^c(m^2, \gcc_0, \gamma_0),
% 	\quad
% z_0   = \hat z_0^c(m^2, \gcc_0, \gamma_0)
% \end{equation}
allows us to uniquely specify the two remaining variables.
For this, we make use of the following version of the
implicit function theorem, which we prove in Appendix~\ref{sec:IFT}.

\begin{prop}
\label{prop:IFT}
Let $\delta > 0$, and let $r_1, r_2$ be continuous positive-definite functions on $[0, \delta]$.
Recalling \eqref{e:Ddef}, set
\begin{equation}
D(\delta, r_1, r_2)
	=
\{ (w, x, y, z) \in D(\delta, r_1) \times \R^n : |z| \leq r_2(x) \},
\end{equation}
and let $F$ be a continuous function on $D(\delta, r_1, r_2)$ that is $C^1$ in $(x, z)$.
Suppose that for all $(\bar w, \bar x) \in [0, \delta]^2$ there exists $\bar z$
such that both $F(\bar w, \bar x, 0, \bar z) = 0$
and $D_Y F(\bar w, \bar x, 0, \bar z)$ is invertible.
Then there is a continuous positive-definite function $r$ on $[0, \delta]$ and
a continuous map $f : D(\delta, r) \to \R^n$
that is $C^1$ in $x$
and such that $F(w, x, y, f(w, x, y)) = 0$
for all $(w, x, y) \in D(\delta, r)$.
Moreover, if $F$ is left-differentiable
(respectively, right-differentiable) in $y$ at some point $(w, x, y, z)$,
then $f$ is left-differentiable (respectively, right-differentiable) at $(w, x, y)$.
\end{prop}
Our first application of this result is Proposition~\ref{prop:changevariables1},
in which the three fixed variables are $(m^2, \gcc_0, \gamma)$.

\begin{prop}
\label{prop:changevariables1}
There exist $\delta_* > 0$,
a continuous positive-definite function $r_* : [0, \delta_*] \to [0, \infty)$,
and continuous functions $(\nu^*, \gcc_0^*, \gamma_0^*, \nu_0^*, z_0^*)$
defined for $(m^2, \gcc, \gamma) \in D(\delta_*, r_*)$, such that
\eqref{e:gg0-re} and \eqref{e:crit-constraint} hold with $\nu = \nu^*$ and
$(\gcc_0, \gamma_0, \nu_0, z_0) = (\gcc^*_0, \gamma^*_0, \nu^*_0, z^*_0)$.
Moreover,
\begin{gather}
\label{e:gznustarbd}
\gcc_0^* = \gcc_0 + O(\gcc_0^2),
\quad
\nu_0^* = O(\gcc_0),
\quad
z_0^* = O(\gcc_0).
\end{gather}
\end{prop}

\begin{proof}
Suppose we have found the desired continuous functions $(\gcc_0^*, \gamma_0^*)$
and that $\gcc_0^*$ satisfies the first bound in \eqref{e:gznustarbd}.
Then the functions defined by
\begin{equation}
\nu_0^* = \hat\nu_0^c(m^2, \gcc_0^*, \gamma_0^*),
	\quad
z_0^* = \hat z_0^c(m^2, \gcc_0^*, \gamma_0^*),
	\quad
\nu^* = \frac{\nu_0^* + m^2}{1 + z_0^*}
\end{equation}
are continuous, satisfy \eqref{e:gg0-re}, and satisfy the remaining bounds in
\eqref{e:gznustarbd} by \eqref{e:hat-est}.

In order to construct $(g_0^*, \gamma_0^*)$, we first solve the third equation
of \eqref{e:gg0-re}, and then solve the first equation of \eqref{e:gg0-re}.
To this end, we begin by defining
\begin{equation}
f_1(m^2, \gcc_0, \gamma, \gamma_0)
	=
\gamma_0 - (4d)^{-1} \gamma (1 + \hat z_0^c(m^2, \gcc_0, \gamma_0))^2
\end{equation}
for $(m^2, \gcc_0, \gamma_0) \in D(\delta, \hat r)$
and $|\gamma| \le \hat r(\gcc_0)$.
% (recall that $\hat r$ is defined in Proposition~\ref{prop:nuzhat});
Although $f_1$ is well-defined
for any $\gamma \in \R$, we restrict the domain in preparation
for our application of Proposition~\ref{prop:IFT}.
Note that $f_1$ is $C^1$ in $\gamma$ and
$f_1(\cdot, \cdot, \gamma, \cdot) \in C^{0,1,\pm}(D(\delta, \hat r))$ for any $\gamma$.
The equation $f_1(m^2, \gcc_0, \gamma, \gamma_0) = 0$
has the solution $\gamma_0 = 0$ when $\gamma = 0$
and, for any $\gamma_0 \neq 0$,
\begin{equation}
\ddp{f_1}{\gamma_0}
	=
1 - (2d)^{-1} \gamma (1 + \hat z_0^c(m^2, \gcc_0, \gamma_0)) \ddp{\hat z_0^c}{\gamma_0}.
\end{equation}
By Theorem~\ref{thm:rhatflow}, the one-sided $\gamma_0$ derivatives of $\hat z_0^c$ exist
at $\gamma_0 = 0$. Thus, the $\gamma_0$ derivative of $f_1$ is well-defined
and equal to $1$ when $\gamma = 0$ for any small $\gamma_0$ (including $\gamma_0 = 0$).
It follows by Proposition~\ref{prop:IFT}
(with $w = m^2$, $x = \gcc_0$, $y = \gamma$, $z = \gamma_0$ and $r_1 = r_2 = \hat r$)
that there exists a continuous function $\gamma^{(1)}_0(m^2, \gcc_0, \gamma)$
on $D(\delta, r^{(1)})$ (for some continuous positive-definite function $r^{(1)}$ on $[0, \delta]$)
such that $f_1(m^2, \gcc_0, \gamma, \gamma^{(1)}_0) = 0$.
Moreover, $\gamma^{(1)}_0$ is $C^1$ in $(\gcc_0, \gamma)$.

Next, we define
\begin{equation}
f_2(m^2, \gcc, \gamma, \gcc_0)
	=
\gcc_0 - (\gcc - \gamma) (1 + \hat z_0^c(m^2, \gcc_0, \gamma^{(1)}_0(m^2, \gcc, \gamma)))^2
\end{equation}
for $(m^2, \gcc_0, \gamma) \in D(\delta, r^{(1)})$ and $\gcc \in [0, \delta_*]$,
where $\delta_* > 0$ will be made sufficiently small below.
Then $f_2(m^2, \gcc, \gamma, \gcc_0) = 0$ is solved by
$(\gamma, \gcc_0) = (0, \gcc_0^*(m^2, \gcc, 0))$,
where $\gcc_0^*(m^2, \gcc, 0)$ was constructed in \cite[\eqref{log-e:ccstar2}]{BBS-saw4-log}.
By \cite[\eqref{log-e:gznustarbd}]{BBS-saw4-log}, $\gcc_0^* = \gcc + O(\gcc^2)$,
so we may restrict the domain of $f_2$ so that $|\gcc_0| \le 2 \gcc$.
Moreover,
\begin{equation}
\ddp{f_2}{\gcc_0}
	=
1 - 2 (\gcc - \gamma) (1 + \hat z_0^c(m^2, \gcc_0, \gamma^{(1)}_0))
\left( \ddp{\hat z_0^c}{\gcc_0} + \ddp{\hat z_0^c}{\gamma_0} \ddp{\gamma^{(1)}_0}{\gcc_0} \right).
\end{equation}
Differentiating both sides of
\begin{equation}
\gamma^{(1)}_0
	=
\frac{1}{4d} \gamma (1 + \hat z_0^c(m^2, \gcc_0, \gamma^{(1)}_0))^2,
\end{equation}
and solving for $\ddp{\gamma^{(1)}_0}{\gcc_0}$, gives
\begin{equation}
\ddp{\gamma^{(1)}_0}{\gcc_0}
	=
\frac{\gamma (1 + \hat z_0^c)
	\ddp{\hat z_0^c}{\gcc_0}}{2 d - \gamma (1 + \hat z_0^c) \ddp{\hat z_0^c}{\gamma_0}},
\end{equation}
where $\hat z_0^c$ and its derivatives are evaluated at $(m^2, \gcc_0, \gamma^{(1)}_0)$.
Thus, $\ddp{\gamma^{(1)}_0}{\gcc_0} = 0$ when $\gamma = 0$.
It follows that $\partial f_2/\partial \gcc_0$
is well-defined when $(\gamma, \gcc_0) = (0, \gcc_0^*(m^2, \gcc, 0))$ and equals
\begin{equation}
1 - 2 \gcc (1 + \hat z_0^c(m^2, \gcc_0^*, 0)) \ddp{\hat z_0^c}{\gcc_0}(m^2, \gcc_0^*, 0),
\end{equation}
% THIS WAS A TYPO IN SAW-SA
which is positive when $\delta_*$ is small, by \eqref{e:hat-est}.
Thus, by Proposition~\ref{prop:IFT}
(with $w = m^2$, $x = \gcc$, $y = \gamma$, $z = \gcc_0$ and $r_1 = r^{(1)}$, $r_2(\gcc) = 2\gcc$),
there exists a function $\gcc_0^*(m^2, \gcc, \gamma) \in C^{0,1,\pm}(D(\delta_*, r^{(2)}))$
(for some continuous positive-definite function $r^{(2)}$ on $[0, \delta_*]$)
such that $f_2(m^2, \gcc, \gamma, \gcc_0^*) = 0$.

By the fact that $\gcc_0^*$ solves $f_2 = 0$,
\begin{equation}
\gcc_0^* = (\gcc - \gamma) + O((\gcc - \gamma)^2).
\end{equation}
Since $|\gamma| \le r^{(2)}(\gcc_0)$ and $r^{(2)}(\gcc_0)$ can be taken
as small as desired, this implies the first estimate in \eqref{e:gznustarbd}.
Thus, by taking $r_*$ sufficiently small, if $|\gamma| \le r_*(\gcc_0)$, then
$|\gamma| \le r^{(2)}(\gcc_0^*(m^2, \gcc, \gamma))$.
Thus, for $\gcc < \delta_*$ and $|\gamma| \leq r_*(\gcc)$,
we can define
\begin{equation}
\gamma_0^*(m^2, \gcc, \gamma)
	=
\gamma^{(1)}_0(m^2, \gcc_0^*(m^2, \gcc, \gamma), \gamma),
\end{equation}
which completes the proof.
\end{proof}

Using Proposition~\ref{prop:changevariables1}, it is possible to
identify the critical point $\nu_c$, as follows.
By \eqref{e:chi-m-hat}, \eqref{e:chichihat}, Proposition~\ref{prop:finvol}, and Proposition~\ref{prop:changevariables1},
\begin{equation}
\label{e:chistar}
\chi(\gcc, \gamma, \nu^*) = \frac{1 + z_0^*}{m^2} = \frac{1 + O(\gcc)}{m^2}.
\end{equation}
Thus, with $\nu = \nu^*$, we see that $\chi < \infty$ when $m^2 > 0$, and
$\chi = \infty$ when $m^2 = 0$.
By \eqref{e:nuc-def}, this implies that
\begin{equation}
\label{e:nustarbd}
\nu_c(\gcc, \gamma) = \nu^*(0, \gcc, \gamma) = O(\gcc),
	\quad
\nu_c(\gcc, \gamma) < \nu^*(m^2, \gcc, \gamma)
	\quad
(m^2 > 0).
\end{equation}
It follows that
\begin{equation}
\chi(\gcc, \gamma, \nu_c) = \infty,
\end{equation}
which is a fact that cannot be concluded immediately from the definition \refeq{nuc-def}.

In \eqref{e:chistar}, $\chi$ is evaluated at $\nu^* = \nu^*(m^2, \gcc, \gamma)$.
However, in the setting of Theorem~\ref{thm:mr},
we need to evaluate $\chi$ at a \emph{given} value of $\nu$
and then take $\nu \downarrow \nu_c$.
To do so, we must determine a choice of $m^2$ in terms of $\nu$
such that \eqref{e:gg0-re} is satisfied and this choice
must approach $0$ (as it should by \eqref{e:nustarbd})
right-continuously as $\nu\downarrow\nu_c$.
The following proposition carries out this construction.
In the following, the functions $\tilde m^2, \tilde \gcc_0$ should not be
confused with the parameter $\mgen^2, \ggen_0$ that appeared previously
in the $\Wcal_j$ norms (these norms are not used in this chapter).

\begin{prop}
\label{prop:changevariables2}
Write $\nu = \nu_c + \varepsilon$.
There exist functions $\tilde m^2, \tilde \gcc_0, \tilde\gamma_0, \tilde\nu_0, \tilde z_0$
of $(\varepsilon, \gcc, \gamma) \in D(\delta_*, r_*)$
(all right-continuous as $\varepsilon\downarrow 0$)
such that \eqref{e:gg0-re} and \eqref{e:crit-constraint} hold with
\begin{equation}
(m^2, \gcc_0, \gamma_0, \nu_0, z_0) = (\tilde m^2, \tilde \gcc_0, \tilde\gamma_0, \tilde\nu_0, \tilde z_0).
\end{equation}
Moreover,
\begin{gather}
\label{e:mtildebd}
\tilde m^2(0, \gcc, \gamma) = 0,
		\qquad
\tilde m^2(\varepsilon, \gcc, \gamma) > 0
		\quad
(\varepsilon > 0) \\
\label{e:gznutildebd}
\tilde \gcc_0 = \gcc + O(\gcc^2),
		\quad
\tilde \nu_0 = O(\gcc),
		\quad
\tilde z_0 = O(\gcc).
\end{gather}
\end{prop}

\begin{proof}
The proof is a minor modification of the proof in \cite{BBS-saw4-log},
using Proposition~\ref{prop:changevariables1}.
Define
\begin{equation}
\label{e:mtildef}
\tilde m^2
		=
\tilde m^2 (\varepsilon,\gcc,\gamma)
		=
\inf \{m^2 > 0 : \nu^*(m^2, \gcc, \gamma) = \nu_c(\gcc, \gamma) + \varepsilon \},
\end{equation}
on $D(\delta_*, r_*)$. By continuity of $\nu^*$, the infimum is attained and
\begin{equation}
\nu_c(\gcc, \gamma) + \varepsilon
	=
\nu^*(\tilde m^2(\varepsilon, \gcc, \gamma), \gcc, \gamma).
\end{equation}
From the above expression, continuity of $\nu^*$, and \eqref{e:nustarbd},
it follows that $\tilde m^2$ is right-continuous as $\varepsilon\downarrow 0$.
It is immediate that \eqref{e:mtildebd} holds.
Also, the functions of $(\varepsilon,\gcc,\gamma)$ defined by
\begin{align}
&\tilde\nu_0 = \nu_0^*(\tilde m^2, \gcc, \gamma),
	\quad
\tilde z_0 = z_0^*(\tilde m^2, \gcc, \gamma),
	\\
&\tilde \gcc_0 = (\gcc - \gamma) (1 + \tilde z_0)^2,
	\quad
\tilde\gamma_0 = \frac{1}{4d} \gamma (1 + \tilde z_0)^2
\end{align}
are right-continuous as $\varepsilon \downarrow 0$ and satisfy \eqref{e:gg0-re}.
The bounds \eqref{e:gznutildebd} follow from the definitions
and \eqref{e:gznustarbd}, and the proof is complete.
\end{proof}

%%%%%%%%%%%%%%%%%%%%%%%%%%%%%%%%%%%%%%%%%%%%%%%%%%%%%%%%%%%%%%%%%%%%%%%%%%%%%%%

\subsection{Conclusion of the argument}
\label{sec:suscept-conc}

We sketch the remainder of the argument, which follows as in
\cite[Section~\ref{log-sec:chvar}]{BBS-saw4-log}.
By Proposition~\ref{prop:finvol}, \eqref{e:chichihat}, \eqref{e:chi-m-hat},
and Proposition~\ref{prop:changevariables2},
\begin{equation}
\label{e:chi-renorm}
\chi(\gcc, \gamma, \nu)
	=
\frac{1 + \tilde z_0}{\tilde m^2}.
\end{equation}
Similarly, from \eqref{e:chiprime-m-hat} (using \eqref{e:chi-renorm}), we get
\begin{equation}
\label{e:diff-reln}
\new{\chi'(g, \gamma, \nu)}
	\sim
-\chi^2(g, \gamma, \nu)
\frac{c_0(g, \gamma)}{(\tilde g_0 {\sf B}_{\tilde m^2})^{\frac{n+2}{n+8}}}
\end{equation}
% \commentbw{(3.1.33): LHS should be derivative}
with $c_0(g, \gamma) = \lim_{\varepsilon\downarrow0} c(\tilde g_0, \tilde\gamma_0)$.
% By a careful analysis of the derivatives of the renormalisation group flow with
% respect to the initial condition $\nu_0$, it can be shown that
% \begin{equation}
% \label{e:chiprime-m-hat}
% \ddp{\hat\chi}{\nu_0} \left(m^2,\gcc_0, \gamma_0,\hat\nu_0^c, \hat z_0^c \right)
% 	\sim
% -\frac{1}{m^4} \frac{c(\gcc_0^*, \gamma_0)}{(\gcc_0^*\bubble_{m^2})^{1/4}}
% 	\quad
% \text{as $(m^2,\gcc_0,\gamma_0) \to (0,\gcc_0^*,\gamma_0^*)$},
% \end{equation}
% where $c$ is a continuous function and the \emph{bubble diagram} $\bubble_{m^2}$ is
% is asymptotic to $(2\pi^2)^{-1} \log m^{-2}$, as $m^2 \downarrow 0$, when $d = 4$.
% This was shown for $\gamma_0 = 0$ in \cite[Section~\ref{log-sec:pfmr}]{BBS-saw4-log}.
% The proof for $\gamma_0 \ne 0$ is almost entirely a direct adaptation of the proof
% found there using the critical parameters $\hat\nu_0^c, \hat z_0^c$ constructed
% in Theorem~\ref{thm:rhatflow} rather than their $\gamma_0 = 0$ analogues.
By exactly the same argument as in \cite[Section~\ref{log-sec:pfsuscept}]{BBS-saw4-log},
the differential relation \eqref{e:diff-reln}
% \eqref{e:chi-renorm} and \eqref{e:chiprime-m-hat} allow us to obtain
% a differential relation between $\ddp{\chi}{\nu}$ and $\chi$,
can be solved, which gives the result of Theorem~\ref{thm:mr}(ii).

\begin{rk}
It is a consequence of \eqref{e:chieps-asympt} and \eqref{e:chi-renorm} that
\begin{equation}
\label{e:mass-epsilon-asympt}
\tilde m^2
	\sim
\tilde A_{g,n}^{-1} \varepsilon (\log \varepsilon^{-1})^{-\frac{n + 2}{n + 8}}
	\quad
\text{as $\varepsilon \downarrow 0$}.
\end{equation}
\end{rk}

%%%%%%%%%%%%%%%%%%%%%%%%%%%%%%%%%%%%%%%%%%%%%%%%%%%%%%%%%%%%%%%%%%%%%%%%%%%%%%%
%%%%%%%%%%%%%%%%%%%%%%%%%%%%%%%%%%%%%%%%%%%%%%%%%%%%%%%%%%%%%%%%%%%%%%%%%%%%%%%

\section{Two-point function}

Our analysis of the two-point function and finite-order correlation length is
based on the following proposition.

\begin{prop}
\label{prop:R}
Let $d=4$, $n \ge 0$, $\varepsilon \in (0,\delta)$ with $\delta$ sufficiently small,
and $\nu = \nu_c + \varepsilon$.
Let $x \in \Z^4$ with $x \neq 0$.
Fix $s = 0$ or $s > 1$.
For $L$ sufficiently large and for $g > 0$ sufficiently small (depending on $s$),
\begin{equation}
\label{e:Gab-to-sum-Rqj}
\frac{1}{1+\tilde z_0} G_x(g, \gamma, \nu)
	=
(1 + O(\gbar_{j_x})) G_x(0, 0, \tilde m^2) + R_x(\tilde m^2)
\end{equation}
and the remainder $R_x$ satisfies the bound
\begin{equation}
\label{e:Rab-bound}
|R_{x}(m^2)|
	\le
\frac{O(\gbar_{j_{x}}) }{|x|^2}
	\times
\begin{cases}
1,				& (m|x|\le 1) \\
(m|x|)^{-2s},	& (m|x|\ge 1)
\end{cases}
\end{equation}
with the  constant depending on $L$ and $s$.
\end{prop}

\begin{proof}
% [Proof of Proposition~\ref{prop:R} (assuming Theorem~\ref{thm:step-mr-fv})]
Let $D_{\sigma_0}$ and $D_{\sigma_x}$ denote differentiation with respect to
$\sigma_0$ and $\sigma_x$, respectively, evaluated with all fields set to $0$.
By \eqref{e:generating-fn}, \eqref{e:ZNINKN}, and \eqref{e:IcircKnew},
\begin{equation}
\lbeq{GK}
\frac{1}{1+\tilde z_0} G_{x,N}(g,\gamma,\nu)
	=
\frac{1}{2} (q_{0,N} + q_{x,N})
	+
\new{
\frac{D^2_{\sigma_0\sigma_x}K^0_{N}}{1 + K^0_{N}}
	-
\frac{\left(D_{\sigma_0}K^0_{N}\right) \left(D_{\sigma_x}K^0_{N}\right)}{(1 + K^0_{N})^2}},
\end{equation}
% \commentbw{(3.2.3): missing explanation of why no $W$ on RHS}
where the quantities on the right-hand side are evaluated at
$(\tilde m^2, \tilde g_0, \tilde\gamma_0, \tilde\nu_0, \tilde z_0)$.
\new{No $W_N$ term appears on the right-hand side since $W_N$ is quadratic in $\Vp$
and $\Vp$ has no constant part.}
By \eqref{e:WKbd} and Lemma~\ref{lem:deriv-norm-bds}, the last two terms
vanish as $N \to \infty$ leaving
\begin{equation}
\frac{1}{1+\tilde z_0} G_x(g, \gamma, \nu) = \frac{1}{2} (q_{0,\infty} + q_{x,\infty}).
\end{equation}

Now it is a straightforward computation using \eqref{e:lampt}--\eqref{e:deltanuw1}
and \eqref{e:Rplusdef} to show that
\begin{equation}
\label{e:q}
q_{u,\infty}
	=
\lambda_{\pp, j_\qq} \lambda_{\qq, j_\qq}  G_x(0, 0, \tilde m^2)
	+
\sum_{i = j_\qq}^\infty R^{q_u}_i,
\quad u = 0, x
\end{equation}
where $R^{q_u}_i$ is the coefficient of $\1_{y=u}\sigma_0\sigma_x$
(recall \refeq{Vy}) in $R_{+,i}$.
Moreover, as in \cite[\eqref{phi4-e:lam-star}]{ST-phi4} and \cite[Corollary~\ref{phi4-cor:vx}]{ST-phi4},
\begin{equation}
\lambda_{u,j_\qq} = 1 + O(\chicCov_{j_\qq} \gbar_{j_\qq}).
\end{equation}
It follows that
\begin{equation}
\frac{1}{1+\tilde z_0} G_x(g, \gamma, \nu)
	=
(1 + O(\gbar_{j_x})) G_x(0, \tilde m^2) + R_x
\end{equation}
with
\begin{equation}
\label{e:Rabdef}
R_x = \frac{1}{2} \sum_{i=j_\qq}^\infty (R^{q_0}_i + R^{q_x}_i).
\end{equation}

By the first bound of \eqref{e:RKplus} and the definition \eqref{e:Vnormdef}
of the $\Vcal$ norm,
\begin{align}
\label{e:vq-new}
    |R^{q_u}_{+,i}|
&
\le O(\ell_{\sigma,i}^{-2}\chicCov_i\gbar_i^{3}).
\end{align}
We insert the definition of $\ell_{\sigma,j}$ from \refeq{elldef-zz} into \refeq{vq-new}.
We also use $\ggen_j^{-2} = O(\gbar_j^{-2})$, $\chicCov_i \le 1$, $\ell_0^2 \le O(1)$,
as well as $\gbar_{j} \leq O(\gbar_{j_{x}})$ for $j \geq j_x$.
The definitions of
the coalescence scale $j_x$ and the mass scale $j_m$ imply that  $L^{-2j_x} \le O(|x|^{-2})$
and $L^{ -  (j_{x} - j_m)_+} \le O((m|x|)^{-1})$.
All this leads to
\begin{align}
\sum_{j = j_{x}}^\infty |R^{q_u}_j|
	&\leq
L^{-2j_{x} - 2s (j_{x} - j_m)_+}
\sum_{j = j_{x}}^\infty O(\gbar_{j}) 4^{-(j - j_{x})}
%\nnb
%&\leq
%L^{-2j_{x} - 2s (j_{x} - j_m)_+} O(\gbar_{j_{x}})
	\nnb
	&\leq
|x|^{-2} (m|x|)^{-2s} O(\gbar_{j_{x}})
.
\label{e:vq-sum}
\end{align}
This gives the desired estimate \eqref{e:Rab-bound}.
\end{proof}

A version of this result with $s = 0$ and $\gamma = 0$ was obtained in \cite{BBS-saw4,ST-phi4}.
This version is sufficient for studying the \emph{critical} two-point function with $\gamma = 0$.
With the extension to $\gamma \ne 0$, we can complete the proof of the first part of
Theorem~\ref{thm:mr}.

\begin{proof}[Proof of Theorem~\ref{thm:mr}(i)]
% With $\nu_c = 0$ (hence $m^2 = 0$),  			% ???
% With $s = 0$, we have $R_x = O(\gbar_{j_x}) G_x(0, 0, 0)$, so
We apply Proposition~\ref{prop:R} with $s = 0$ to get
\begin{equation}
\frac{1}{1 + \tilde z_0} G_x(g, \gamma, \nu)
	=
(1 + O(\gbar_{j_x})) G_x(0, 0, \tilde m^2) + R_x(\tilde m^2).
\end{equation}
By Proposition~\ref{prop:changevariables2}, $\tilde m^2 = 0$
when $\nu = \nu_c$. Since $R_x(0) = O(\gbar_{j_x}) G_x(0, 0, 0)$,
\begin{equation}
\frac{1}{1 + \tilde z_0} G_x(g, \gamma, \nu_c)
	=
(1 + O(\gbar_{j_x})) G_x(0, 0, 0)
\end{equation}
and the result follows from \eqref{e:gjxgjmbd}.
\end{proof}

%%%%%%%%%%%%%%%%%%%%%%%%%%%%%%%%%%%%%%%%%%%%%%%%%%%%%%%%%%%%%%%%%%%%%%%%%%%%%%%
%%%%%%%%%%%%%%%%%%%%%%%%%%%%%%%%%%%%%%%%%%%%%%%%%%%%%%%%%%%%%%%%%%%%%%%%%%%%%%%

\section{Finite-order correlation length}

An elementary ingredient in the proof of Theorem~\ref{thm:mr}(iii) is the following result
for the $g=0$ case, which is independent of $n\ge 0$.
For simplicity, we restrict attention to dimensions $d>2$, as only $d=4$ is used here.
A proof is provided in Appendix~\ref{app:free-moments}.

\begin{prop}\label{prop:Gab-free-moment-estimate}
Let ${\sf c}_p$ be the constant defined by \eqref{e:cpdef}.
For all dimensions $d>2$ and all $p>0$,
as $m^2 \downarrow 0$,
\begin{equation}
\label{e:Gab-free-moment-estimate}
\sum_{x\in\Zd} |x|^p G_{x}(0, 0, m^2)
=
{\sf c}_p^p m^{-(p + 2)} (1 + O(m)).
\end{equation}
In particular, $\xi_p(0,0,\varepsilon) = {\sf c}_p \varepsilon^{-1/2}
(1+O(\varepsilon^{1/2}))$ as $\varepsilon \downarrow 0$.
\end{prop}



\begin{proof}[Proof of Theorem~\ref{thm:mr}]
We multiply \eqref{e:Gab-to-sum-Rqj} by $|x|^p$, sum over $x \in \Z^4$,
and use \eqref{e:chistar}
% \eqref{e:susceptibility-mass-identity},
to obtain
\begin{equation}
\xi_p^p(g,\gamma,\nu)
	=
\sum_{x \in \Z^4} |x|^p \frac{G_{x}(g, \gamma, \nu)}{\chi(g, \gamma, \nu)}
	=
\tilde m^2 \sum_{x \in \Z^4} |x|^p \Big(G_{x}(0, 0, m^2) + r_{x}(\tilde m^2) \Big),
\end{equation}
with
\begin{equation}
\lbeq{rx}
    r_x(m^2) = O(\gbar_{j_x})  G_x(0, 0, m^2) + R_{x}(m^2).
\end{equation}
By
Proposition~\ref{prop:Gab-free-moment-estimate},
this gives (as $\tilde m^2 \downarrow 0$)
\begin{equation}
\lbeq{ximasy}
\xi_p^p(g,\gamma,\nu)
	\sim
{\sf c}_p^p \tilde m^{-p} +
\tilde m^2 \sum_{x \in \Z^4} |x|^p r_{x}(\tilde m^2).
\end{equation}
By \eqref{e:mass-epsilon-asympt}, it suffices to prove that
the first term on the right-hand side of \refeq{ximasy} is dominant.

For the term $O(\gbar_{j_x}) G_x(0, 0, m^2)$ in \refeq{rx},
we apply \eqref{e:gjxgjmbd} to obtain
\begin{align}
\lbeq{easyerror}
&\sum_{x \in \Z^4} \gbar_{j_x} |x|^p G_x(0, 0, \tilde m^2)
	\nnb & \quad \le
\sum_{x : 0 < j_x \le j_{\tilde m}} \frac{c |x|^p}{\log |x|} G_x(0, 0, \tilde m^2)
	+
\frac{c}{\log \tilde m^{-1}} \sum_{x : j_x >j_{\tilde m}}  |x|^p G_x(0, 0, \tilde m^2).
\end{align}
In the first term,
we use $G_x(0, 0, m^2) \le G_x(0, 0, 0) \le O(|x|^{-2})$.
The restriction $j_x \le j_{\tilde m}$ ensures that $|x| \le O(\tilde m^{-1})$.
Therefore the first term is bounded above by a multiple of
$(\tilde m^{-1})^{d+p-2}(\log \tilde m^{-1})^{-1}$, which suffices.
For the term with $j_x > j_{\tilde m}$, we extend the sum to $x \in \Z^4$
and apply Proposition~\ref{prop:Gab-free-moment-estimate}
to obtain a bound of the same form as for the first term.

For the term $R_x$ of \eqref{e:rx}, we use Proposition~\ref{prop:R}
to see that
\begin{equation}\label{e:rab-bound-scales-bis}
|R_x(\tilde m^2)|
	=
O(\gbar_{j_x})
L^{-2j_x - 2s (j_x - j_{\tilde m})_+}.
\end{equation}
We divide $\Z^4$ into shells $S_1 = \{x : |x| < \frac 12 L\}$ and, for $j \ge 2$,
$S_j = \{x : \frac 12 L^{j-1} \le |x| < \frac 12 L^{j}\}$.
The number of points in $S_j$ is bounded by $O(L^{4j})$.
Note that $j_x$ is the unique scale so that
\begin{equation}
   \label{e:Phi-def-jc}
    x \in S_{j_x +1}
   .
\end{equation}
By \eqref{e:rab-bound-scales-bis} with $s>\frac 12 (p+2)$ and \refeq{Phi-def-jc},
\begin{equation}
\label{e:xpr-shells}
\sum_{x\in\Z^4} |x|^p |R_{x}(\tilde m^2)|
	=
\sum_{j = 1}^\infty\sum_{x\in S_j}   |x|^p |R_{x}(\tilde m^2)| \\
    =
\sum_{j = 1}^\infty L^{4j + pj - 2j - 2s (j - j_{\tilde m})_+} O(\gbar_{j}),
\end{equation}
with an $L$-dependent constant.
By Lemma~\ref{lem:mass-scale-sum} below (with $a=p+2$ and $b=1$),
we obtain
\begin{equation}\label{e:xpr-mbound}
\tilde m^2 \sum_{x\in\Z^4} |x|^p |R_{x}(\tilde m^2)|
	=
O\big(\tilde m^{-p}(\log \tilde m^{-1})^{-1}\big).
\end{equation}
The first term on the right-hand side of \refeq{ximasy} therefore dominates,
and the proof is complete.
\end{proof}

The estimate used to obtain \eqref{e:xpr-mbound} is given by the following lemma,
which is stated more generally for use in the proof of Proposition~\ref{prop:Gab-free-moment-estimate}.

\begin{lemma} \label{lem:mass-scale-sum}
Let $L>1$, $2s> a > 0$, $b \geq 0$, and let $\gbar_0>0$ be sufficiently small.
Then
\begin{equation} \label{e:gmsumbd}
\sum_{j=1}^\infty L^{aj - 2s (j - j_m)_+}
\gbar_j^b = O(m^{-a} \gbar_{j_m}^b) = O(m^{-a} (\log m^{-1})^{-b}).
\end{equation}
\end{lemma}

\begin{proof}
We divide the sum at the mass scale as
\begin{equation} \label{e:xpr-2sums}
\sum_{j=1}^\infty L^{aj - 2s (j - j_m)_+} \gbar_j^{b}
= \sum_{j=1}^{j_m} L^{aj} \gbar_j^{b} +  \sum_{j=j_m+1}^\infty L^{aj - 2s (j - j_m)} \gbar_{j}^{b}.
\end{equation}
For the second sum on the right-hand side, we use $\gbar_j = O(\gbar_{j_m})$ for $j > j_m$
by \eqref{e:gjxgjmbd},
and obtain
a bound consistent with the first equality of \refeq{gmsumbd}.
For the first term, we use the crude bound
$\gbar_i/\gbar_{i+1} = 1+O(g_0)$ (by
\cite[Lemma~\ref{flow-lem:elementary-recursion}]{BBS-rg-flow}), and find
\begin{equation}
  \sum_{j=1}^{j_m} L^{aj} \gbar_j^b
  \leq
  L^{aj_m} \gbar_{j_m}^b
  \sum_{j=1}^{j_m} ((1+O(\gbar_0))L^{-a})^{j_m-j}
  =
  O(L^{aj_m} \gbar_{j_m}^b),
\end{equation}
for sufficiently small $\gbar_0>0$.
This proves the first equality in \eqref{e:gmsumbd}.
The second equality then follows since
$\gbar_{j_m} = O(\log m^{-1})$ by \eqref{e:gjxgjmbd}.
\end{proof}