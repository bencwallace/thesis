\section{Main results}

The main result is stated below.
\todo{Discuss existence of $\nu_c$}

\begin{theorem}
\label{thm:mr}
Let $d = 4$ and $n \ge 0$. For $L$ sufficiently large (depending on $n$),
there exist $\gcc_* > 0$
and a positive function $\gamma_* : (0, \gcc_*) \to \R$
such that whenever $0 < \gcc < \gcc_*$ and $|\gamma| < \gamma_*(\gcc)$,
there are constants $A_{\gcc,\gamma,n}$ and $B_{\gcc,\gamma,n}$ such that the following hold:

\smallskip\noindent
(i)
The critical two-point function decays as
\begin{equation}
G_x(\gcc,\gamma,\nu_c; n)
    =
A_{\gcc,\gamma} |x|^{-2} (1 + O((\log |x|)^{-1}))
    \quad
\text{as $|x|\to\infty$},
\end{equation}
with $A_{\gcc,\gamma} = (4 \pi)^{-2} (1 + O(\gcc))$ as $\gcc \downarrow 0$.

\smallskip\noindent
(ii)
The susceptibility diverges as
\begin{equation} \label{e:chieps-asympt}
\chi(\gcc, \gamma, \nu_c + \varepsilon; n)
	\sim
B_{\gcc,\gamma,n} \varepsilon^{-1} (\log \varepsilon^{-1})^{(n+2)/(n+8)},
	\quad
\varepsilon\downarrow 0
\end{equation}
with $B_{\gcc,\gamma,n} = ((n + 8) \gcc / 16\pi^2)^{(n+2)/(n+8)} (1 + O(\gcc))$
as $\gcc \downarrow 0$.

\smallskip\noindent
(iii)
For any $p >0$, if $L$ is chosen large and $\gcc_*$ small both depending on $p$,
then the correlation length of order $p$ diverges as
\begin{equation} \label{e:xieps-asympt}
\xi_p(\gcc, \gamma, \nu_c + \varepsilon; n)
	\sim
B_{\gcc,\gamma,n}^{1/2} {\sf c}_p \varepsilon^{-1/2} (\log \varepsilon^{-1})^{(n+2)/2(n+8)},
	\quad
\varepsilon\downarrow 0
\end{equation}
with
\begin{equation}
\label{e:cpdef}
{\sf c}_p^p
	=
\int_{\R^4} |x|^p (-\Delta_{\R^4} + 1)^{-1}_{0x} \; dx.
\end{equation}
\end{theorem}

The $\gamma = 0$ cases of (i) and (ii) were proved by Bauerschmidt, Brydges, and
Slade. The $n > 0$ case with $\gamma \ne 0$ is a new result in this thesis. We
will only discuss the proof of the $\gamma \ge 0$ case, which is of primary
interest. The proof of the $\gamma < 0$ case with $n = 0$ can be found in
\cite{BSW-saw-sa} and the extension to $n > 0$ is straightforward.