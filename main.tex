\section{Main result}
\label{sec:mr}

For any integer $n \ge 1$, let $G_x(g, \gamma, \nu; n)$
denote the two-point point function for the version of the $|\varphi|^4$ model
defined by \eqref{e:two-point-function-phi4}.
We let $G_x(g, \gamma, \nu; 0)$ denote the two-point function of the WSAW-SA,
defined in \eqref{e:Gsa}; this notation will be explained in Section~\ref{sec:spin-walk}.
We employ similar conventions for the susceptibility, correlation length of order $p$,
and critical point, which we denote by
$\chi(g, \gamma, \nu; n)$, $\xi_p(g, \gamma, \nu; n)$, $\nu_c(g, \gamma; n)$, respectively,
with $n \ge 0$ an integer.
When $n = 0$, these correspond to the WSAW-SA, whereas for $n \ge 1$ they correspond to the
$|\varphi|^4$ model. The following theorem is the main result of this thesis.

\begin{theorem}
\label{thm:mr}
Let $d = 4$ and $n \ge 0$. For $L$ sufficiently large (depending on $n$),
there exists $\gcc_* > 0$ and a positive function $\gamma_* : (0, \gcc_*) \to \R$
such that whenever $0 < \gcc < \gcc_*$ and $|\gamma| < \gamma_*(\gcc)$,
there are constants $A_{\gcc,\gamma,n}$ and $B_{\gcc,\gamma,n}$ such that the following hold:

\smallskip\noindent
(i)
The critical two-point function decays as
\begin{equation}
G_x(\gcc,\gamma,\nu_c; n)
    =
A_{\gcc,\gamma,n} |x|^{-2} \big(1 + O((\log |x|)^{-1})\big)
    \quad
\text{as $|x|\to\infty$},
\end{equation}
with $A_{\gcc,\gamma,n} = (4 \pi)^{-2} (1 + O(\gcc))$ as $\gcc \downarrow 0$.

\smallskip\noindent
(ii)
The susceptibility diverges as
\begin{equation} \label{e:chieps-asympt}
\chi(\gcc, \gamma, \nu_c + \varepsilon; n)
	\sim
B_{\gcc,\gamma,n} \varepsilon^{-1} (\log \varepsilon^{-1})^{(n+2)/(n+8)},
	\quad
\varepsilon\downarrow 0
\end{equation}
with $B_{\gcc,\gamma,n} = ((n + 8) \gcc / 16\pi^2)^{(n+2)/(n+8)} (1 + O(\gcc))$
as $\gcc \downarrow 0$.

\smallskip\noindent
(iii)
For any $p >0$, if $L$ is chosen large and $\gcc_*$ small (both depending on $p$),
then the correlation length of order $p$ diverges as
\begin{equation} \label{e:xieps-asympt}
\xi_p(\gcc, \gamma, \nu_c + \varepsilon; n)
	\sim
B_{\gcc,\gamma,n}^{1/2} {\sf c}_p \varepsilon^{-1/2} (\log \varepsilon^{-1})^{(n+2)/2(n+8)},
	\quad
\varepsilon\downarrow 0
\end{equation}
with
\begin{equation}
\label{e:cpdef}
{\sf c}_p^p
	=
\int_{\R^4} |x|^p (-\Delta_{\R^4} + 1)^{-1}_{0x} \; dx.
\end{equation}
\end{theorem}

The $\gamma = 0$ cases of (i) and (ii) were proved by Bauerschmidt, Brydges,
and Slade in \cite{BBS-saw4,BBS-saw4-log};
in fact, the $n = 1$ case of their results was first obtained in
\cite{Hara87,HT87,GK85,FMRS87}.
The $n > 0$ case with $\gamma \ne 0$ is a new result in this thesis. We
will only discuss the proof of the $\gamma \ge 0$ case, which is of primary
interest. The proof of the $\gamma < 0$ case with $n = 0$ can be found in
\cite{BSW-saw-sa} and the extension to $n \ge 1$ is straightforward.

\newer{\begin{rk}\mbox{}
\begin{enumerate}
\item
The behaviour \eqref{e:xieps-asympt} is consistent with predictions for
the correlation length $\xi$ and should be understood as a step towards a
rigorous understanding of $\xi$ in four dimensions (even in the case $\gamma = 0$).

\item
The statement of Theorem~\ref{thm:mr} does not provide a quantitative upper bound
on $\gamma$. However, it should be possible to
extend this result to all $|\gamma| \le C g^3$ for some constant
$C$. We did not pursue this extension here as we expect that the results of
Theorem~\ref{thm:mr} may well hold for $\gamma$ larger than $O(g^3)$. Indeed, these results
should hold for all $\gamma$ below the $\theta$-curve and we know of no particular reason
to expect the $\theta$-curve to scale like $g^3$.

\item
We expect that the results of \cite{ST-phi4} on higher-order correlation functions can
be extended to the $\gamma$-dependent case considered here by the methods used to prove
Theorem~\ref{thm:mr}. For instance, it should be possible to show that
\begin{equation}
\frac1n \langle |\varphi_0|^2 ; |\varphi_x|^2 \rangle_{g,\gamma,\nu_c}
	\sim
C_{g,\gamma,n} |x|^{-4} (\log |x|)^{-2\left(\frac{n+2}{n+8}\right)}
% \frac{C_{g,\gamma,n}}{(\log |x|)^{2\left(\frac{n+2}{n+8}\right)}}
% \frac{1}{|x|^4}.
\end{equation}
\end{enumerate}
\end{rk}}

% \commentbw{OPE?}
