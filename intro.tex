% Parts are the largest structural units, but are optional.
%\part{Thesis}

% Chapters are the next main unit.
\chapter{Introduction}

%%%%%%%%%%%%%%%%%%%%%%%%%%%%%%%%%%%%%%%%%%%%%%%%%%%%%%%%%%%%%%%%%%%%%%%%%%%%%%%%%%%%%%%%%%
%%%%%%%%%%%%%%%%%%%%%%%%%%%%%%%%%%%%%%%%%%%%%%%%%%%%%%%%%%%%%%%%%%%%%%%%%%%%%%%%%%%%%%%%%%

\section{Statistical mechanics}

The state of knowledge of a system with state space $(\Omega, d\lambda)$ can be
expressed by a probability measure $\mu$ on $\Omega$. Let $\Mcal_\lambda(\Omega)$ denote
the set of probability measures on $\Omega$ absolutely continuous with respect to $\lambda$.
For $\mu \in \Mcal_\lambda(\Omega)$, we denote the Radon-Nikodym derivative of $\mu$ with
respect to $\lambda$ by $d\mu/d\lambda$ and define the \emph{entropy} of $\mu$ with respect
to $\lambda$ by
\begin{equation}
h(\mu) = h_\lambda(\mu) = -\int_\Omega \log\frac{d\mu}{d\lambda} \; d\mu.
\end{equation}
In many cases, specific information about
the state of knowledge is available, for example in the form of statements such as
\begin{equation}
\int f \; d\mu \in S_f
\end{equation}
with $f$ running over some collection of $\mu$-integrable functions
that represent \emph{observable} quantities of the system, and $S_f$ a Borel
subsets of $\R$ for each $f$. The \emph{principle of maximum entropy} asserts that, in this
case, the measure best expressing the state of knowledge of the system is given by
\begin{equation}
\hat\mu = \argmax(h_\lambda(\mu) : \mu \in \Mcal_\lambda(\Omega)),
\end{equation}
assuming such a measure exists and is unique.

%%%%%%%%%%%%%%%%%%%%%%%%%%%%%%%%%%%%%%%%%%%%%%%%%%%%%%%%%%%%%%%%%%%%%%%%%%%%%%%%%%%%%%%%%%

\subsection{Microcanonical ensemble}

Consider an \emph{isolated} physical system on $\Omega$, that is, one that cannot exchange
energy with its surroundings. Such a system can be determined by a choice of function
$H : \Omega \to \R$, called the \emph{Hamiltonian}. The value $H(\omega)$ represents the
total energy of the system in state $\omega\in\Omega$.

\begin{example}
Let $\Omega = U^n \times \R^{3n}$, where $U \subset \R^3$, and denote a generic element of
$\Omega$ by $(q, p)$, where $q \in U^n$ and $p \in \R^{3n}$. Then $\Omega$ is the state
space of a system of $n$ point particles $i = 1, \ldots, n$ with positions $q_i \in U$
and momenta $p_i \in \R^3$. Given a $C^1$ Hamiltonian $H : \Omega \to \R$,
the dynamics of such a system is determined by \emph{Hamilton's equations}
\begin{align}
\dd{q}{t}   &= \nabla_q H(q(t), p(t)) \\
-\dd{p}{t}  &= \nabla_p H(q(t), p(t)).
\end{align}
An immediate consequence of these equations and the chain rule is the \emph{principle
of conservation of energy}:
\begin{equation}
\dd{H}{t}(q(t), p(t)) = 0.
\end{equation}
Thus, a Hamiltonian system with initial configuration $(q(0), p(0))$ of energy
$E = H(q(0), p(0))$ will evolve on the constant energy shell $S_E = H^{-1}(E)$.

% Let us assume that $H$ has compact level sets so that the solutions to Hamilton's
% equations can be extended globally in time. In general, especially when $n$ is large,
% a Hamiltonian system is likely to be too complicated to understand in a precise way.
% The basic idea of statistical mechanics is to leverage this complexity by assuming
% (or in some cases proving) that, as $t\to\infty$, such a system will settle into a
% state of equilibrium in which all possible states are ``equiprobable''.
\end{example}

% Although one can make sense of the notion of an equiprobable probability distribution
% on $S_E$ in the above example, it is simpler (and more relevant to this thesis) to
% consider the case when $S_E$ is finite.
If $S_E = H^{-1}(E)$ is finite\footnote{We can view such a system as an approximation
to a traditional continuous system with $S_E$ uncountable.}, then one can easily determine
the maximum entropy measure on $S_E$. Indeed, the maximum entropy measure on a finite
space is simply the uniform measure. We define the
\emph{microcanonical distribution} on a finite set $F$ to be the uniform
measure on $F$.

%%%%%%%%%%%%%%%%%%%%%%%%%%%%%%%%%%%%%%%%%%%%%%%%%%%%%%%%%%%%%%%%%%%%%%%%%%%%%%%%%%%%%%%%%%

\subsection{Canonical ensemble}

In practice, most systems of interest are not truly isolated: they may exchange
energy with their environments. Everyday experience, however, suggests that
a physical system that is left undisturbed for a sufficiently long time will achieve
\emph{thermal equilibrium}, in which the system's temperature is constant and equal to
that of its surroundings. We define the \emph{canonical ensemble} for a system with
state space $\Omega$ and Hamiltonian $H$ on $\Omega$
to be the maximum entropy distribution subject to the fixed average energy constraint
\begin{equation}
\int H \; d\mu = E.
\end{equation}
It can be shown by the method of Lagrange multipliers that the canonical ensemble is
given by the \emph{Gibbs measure}
\begin{equation}
d\mu_\beta = \frac{1}{Z_\beta} e^{-\beta H} d\lambda,
\end{equation}
where
\begin{equation}
Z_\beta = \int e^{-\beta H} \; d\lambda
\end{equation}
is the normalizing constant, known as the \emph{canonical partition function}.
The quantity $\beta = \beta(E)$, which arises as a Lagrange multiplier, is known as
the \emph{inverse temperature}.

The \emph{free energy} of this system is defined by
\begin{equation}
F_\beta = -\frac{1}{\beta} \log Z_\beta.
\end{equation}
This definition may seem obscure at first, but is elucidated by
a computation of the entropy $h_\lambda(\mu_\beta)$, which implies that
\begin{equation}
F_\beta = E - \frac{1}{\beta} h_\lambda(\mu_\beta).
\end{equation}
This is the famous thermodynamic relation between free energy, internal energy $E$,
temperature $1/\beta$, and entropy.


In the context of spin systems, there is a natural mathematical reason for studying measures
of the above form. This is the Hammersley-Clifford theorem, which states that any Markov
random field (a spatial generalization of a Markov chain) on a graph has a representation
as a Gibbs measure whose Hamiltonian is a sum of ``local'' interactions.

% This situation can be modeled as follows: call the system of interest system $A$ and
% its surroundings system $B$. We assume that $B$ behaves as a \emph{thermal reservoir},
% meaning that it is so large that its temperature may be assumed to be effectively
% constant, even if system $A$ is at a different temperature.
% It is then reasonable to assume that the total system $A \times B$ is isolated.

% \begin{example}
% Let $\Omega_1$ and $\Omega_2$ be the state spaces of systems $A$ and
% $B$ and let $H_i : \Omega_i \to \R$ be their respective Hamiltonians.
% The system $A \times B$ then has state space $\Omega_1 \times \Omega_2$.
% If we assume that the contribution to the energy due to interactions between systems $A$
% and $B$ is negligible, then the Hamiltonian of $A \times B$ is the function $H : \Omega \to \R$
% defined by $H(\omega_1, \omega_2) = H_1(\omega_1) + H_2(\omega_2)$.

% Let us assume that the 
% Let $S^i_{E_i} = H_i^{-1}(E_i)$ and $S_E = H^{-1}(S_E)$ and assume that $|S^i_{E_i}| < \infty$
% for any $E_i$. Then
% \begin{equation}
% S_E = \bigcup_{E_1=0}^E S^1_{E_1} \times S^2_{E-E_2}
% \end{equation}
% is finite for all $E$.
% % Suppose that $S_E$ is compact so that the microcanonical ensemble $\mu_E$ on $S_E$ is well-defined.

% For any $E_2 \in \R$, let $S^2_{E_2} = H_2^{-1}(E_2)$. Then the marginal distribution $\mu^1_E$
% of $\mu$ on $\Omega_1$ is given by
% \begin{equation}
% \mu^1_E(\omega_1)
%   =
% \sum_{\omega_2:\omega\in S_E} \mu_E(\omega)
%   =
% \frac{\# S^2_{E-H_1(\omega_1)}}{\# S_E}
%   \propto
% e^{h^2_{E-H_1(\omega_1)}},
% \end{equation}
% where
% \begin{equation}
% h^2_{E_2} = \log |S^2_{E_2}|
% \end{equation}
% is the \emph{entropy} of the microcanonical ensemble on $S^2_{E_2}$.

% If $E_2 \mapsto h^2_{E_2}$ is $C^1$, then we have the first-order expansion
% \begin{equation}
% h^2_{E - H_1(\omega_1)}
%   =
% h^2_E - \dd{h^2_E}{E} H_1(\omega_1) + O(H_1(\omega_1)^2).
% \end{equation}
% Thus,
% \begin{equation}
% e^{-h^2_{E-H_1(\omega_1)}}
%   \approx
% C e^{-\beta H_1(\omega_1)},
% \end{equation}
% where $\beta = \dd{h^2_E}{E}$ and $C$ is a constant independent of $\omega_1$.
% The assumption that system $B$ is a ``heat bath'' then amounts to saying that
% $E - H_1(\omega_1)$ is such a negligible perturbation of $E$ that the first-order
% approximation above accurately represents the marginal density of $\mu^1_E$.
% \end{example}

%%%%%%%%%%%%%%%%%%%%%%%%%%%%%%%%%%%%%%%%%%%%%%%%%%%%%%%%%%%%%%%%%%%%%%%%%%%%%%%%%%%%%%%%%%

\subsection{Grand canonical ensemble}

Consider a physical system that is free to exchange particles with its environment. We suppose
the ``number of particles'' of this system take values in a measurable set $\interval \subset \R$.
For $T \in \interval$, let $\Omega_T$ be the space of configurations with $T$
particles and let $H_T : \Omega_T \to \R$ be the Hamiltonian on $\Omega_T$.
The state space for the full system is then given by $\Omega = \bigsqcup_{T \in \interval} \Omega_T$,
where $\sqcup$ denotes the disjoint union.
For $\omega\in\Omega$, let $|\omega|$ be the unique value of $T$
such that $\omega \in \Omega_T$.
We define $H : \Omega \to \R$ by $H(\omega) = H_{|\omega|}(\omega)$.

The \emph{grand canonical ensemble} for this system is defined to be the maximum entropy measure
subject to the constraints of fixed average energy and fixed average number of particles. It can
be shown that the grand canonical ensemble is given by a measure
\begin{equation}
d\mu_{\beta,\nu} = \frac{1}{Z_{\beta,\nu}} e^{-\beta (H(\omega) - \nu |\omega|)},
\end{equation}
where $Z_{\beta,\nu}$ is the normalizing constant, known as the \emph{grand canonical partition
function}. The parameters $\beta$ and $\nu$ are both Lagrange multipliers; we call $\nu$ the
\emph{fugacity}. Since the $\nu |\omega|$ term can be absorbed into the Hamiltonian, the measure
$\mu_{\beta,\nu}$ is really just an ordinary Gibbs measure as defined in the previous section.
For this reason, the distinction between canonical and grand canonical ensembles is informal.

\begin{example}
Note that the grand canonical partition function is given by
\begin{equation}
Z_{\beta,\nu}
  =
\int_\interval e^{\nu T} Z_\beta^{(T)} \; dT,
\end{equation}
where $Z^{(T)}_\beta$ is the canonical partition function of $H_T$.
Thus, $Z_{\beta,\nu}$ is the Laplace transform of the canonical partition function
viewed as a function of $T$.
\end{example}

%%%%%%%%%%%%%%%%%%%%%%%%%%%%%%%%%%%%%%%%%%%%%%%%%%%%%%%%%%%%%%%%%%%%%%%%%%%%%%%%%%%%%%%%%%
%%%%%%%%%%%%%%%%%%%%%%%%%%%%%%%%%%%%%%%%%%%%%%%%%%%%%%%%%%%%%%%%%%%%%%%%%%%%%%%%%%%%%%%%%%

\section{Random walks and Gaussian measures}

Most systems of interest do not have a finite (or even countable) state space. Nevertheless,
finite systems serve as natural approximations of real systems. For instance, spatially-extended
systems may be approximated by models on graphs. In this section, we discuss two of the
simplest such models. We begin by introducing some notation and graph-theoretic background.

%%%%%%%%%%%%%%%%%%%%%%%%%%%%%%%%%%%%%%%%%%%%%%%%%%%%%%%%%%%%%%%%%%%%%%%%%%%%%%%%%%%%%%%%%%

\subsection{Setting and notation}

Let $\vertices$ be a countable set. An $n$-component \emph{field} on $\vertices$
is an element $\varphi$ of $(\R^n)^\vertices$.
We denote the components of a field $\varphi$
by $\varphi^i_x \in \R$ for $x \in \vertices$ and $i = 1, \ldots, n$.
The Euclidean inner product on fields is defined by
\begin{equation}
\varphi\cdot\tilde\varphi
  =
\sum_{x\in\vertices} \varphi_x \cdot \tilde\varphi_y
  =
\sum_{i=1}^n \sum_{x\in\vertices} \varphi^i_x \tilde\varphi^i_x
\end{equation}
and the Euclidean norm is
\begin{equation}
|\varphi|^2 = \varphi \cdot \varphi.
\end{equation}
Given any $\vertices\times\vertices$ matrix $M$, we define the field $M \varphi$ by
\begin{equation}
(M \varphi)_x = \sum_{y\in\vertices} M_{xy} \varphi_y
\end{equation}
when the sum is well-defined.

\subsubsection{Graphs and the Laplacian}

Let $\jay$ be a symmetric $\vertices\times\vertices$ matrix satisfying
\begin{equation}
0 < d_x = \sum_{y\in\vertices} J_{xy} < \infty,
  \quad
J_{xy} \ge 0,
  \quad
J_{xx} = 0
\end{equation}
for all $x, y$. Then we can define the diagonal matrix $\diag$ with entries
\begin{equation}
\diag_{xx} = d_x.
\end{equation}
Moreover, if $\edges = \{ \{ x, y \} : J_{xy} \ne 0 \}$, then
$\graph = (\vertices, \edges, \jay)$ is a weighted connected undirected graph.
We write $x \sim y$ if $\{ x, y \} \in \edges$.

The \emph{graph Laplacian} on $\graph$ is defined by
\begin{equation}
\lap = \diag - \jay.
\end{equation}
We also define the \emph{massive Laplacian} with squared \emph{mass} $m^2 > 0$
by
\begin{equation}
m^2 + \lap.
\end{equation}
Note that
\begin{equation}
\varphi \cdot \lap \varphi
  =
\frac{1}{2} \sum_{x,y\in\vertices} J_{xy} |\varphi_x - \varphi_y|^2
  \ge
0,
\end{equation}
so $\lap$ is positive-semidefinite.

\subsubsection{The Green function}

If $m^2 > 0$, then $m^2 + \lap$ is positive-definite, hence invertible with inverse
\begin{equation}
(m^2 + \lap)^{-1} = (m^2 + D)^{-1} \sum_{n=0}^\infty Z^n P^n,
\end{equation}
where
\begin{align}
Z = (m^2 + D)^{-1} D,
  \quad
P = D^{-1} J.
\end{align}
Let $z_x$ denote the diagonal elements of $Z$.
The \emph{Green function} for $m^2 + \lap$ is the kernel of $(m^2 + \lap)^{-1}$, which we define by
\begin{equation}
C(x, y)
  =
(m^2 + d_x)^{-1} \sum_{n=0}^\infty z_x^n P^n_{xy}
\end{equation}
whenever this series converges.

\begin{example}
An important case is when $\jay$ has $\{0, 1 \}$-valued entries. In this case, $d_x$ is the
\emph{degree} of $x$ in $\graph$ and we denote the matrix $\lap$ by $-\Delta$, which has
entries
\begin{equation}
-\Delta_{xy} = d_x \1_{x=y} - \1_{x \sim y}.
\end{equation}
\end{example}

% Let $\jay$ be a $\vertices \times \vertices$
% matrix with $J_{xx} = 0$ and $J_{xy} \ge 0$ for all $x, y \in \vertices$ and suppose
% that $\jay$ has summable rows: $\sum_{y\in\vertices} J_{xy} < \infty$. Let $\diag$ be
% the diagonal matrix with entries $D_{xx} = \sum_{y\in\vertices} J_{xy}$, and let $A = \diag - \jay$.

% \todo{The $Q$ matrix will be $Q = -A$, the matrix $\jay$ will be the (ferromagnetic) Ising interaction,
% and $A$ will be the interaction for the $|\varphi|^4$ model.
% E.g.\ $A = -\Delta + m^2$ has positive diagonal entries. Recall that
% $\Delta_{xy} = \1_{x \sim y} - 2 d \1_{x=y}$.}

%%%%%%%%%%%%%%%%%%%%%%%%%%%%%%%%%%%%%%%%%%%%%%%%%%%%%%%%%%%%%%%%%%%%%%%%%%%%%%%%%%%%%%%%%%

\subsection{Gaussian measures and the free field}

Let $C$ be a positive-definite symmetric $\vertices\times\vertices$ matrix.
The (centered) $n$-component \emph{Gaussian field} on $\vertices$ with
\emph{covariance} $C$ is the unique measure $\mu$ on $(\R^n)^\vertices$ with
characteristic function
\begin{equation}
\hat\mu(\xi)
  \coloneqq
\int_{(\R^n)^\vertices} \mu(d\varphi) \; e^{i \varphi \cdot \xi}
  =
e^{-\frac{1}{2} \xi \cdot C \xi}.
\end{equation}
If $\vertices$ is finite, then the Lebesgue measure $d\varphi$ on $(\R^n)^\vertices$
is well-defined and
\begin{equation}
d\mu(\varphi)
  =
\frac{1}{\sqrt{\det(2\pi C)}} e^{-\frac{1}{2} \varphi\cdot C^{-1}\varphi}
d\varphi.
\end{equation}
This is just a canonical Gibbs measure with a quadratic Hamiltonian, which is essentially
the simplest Gibbs measure besides the independent field with constant
Hamiltonian\footnote{Note that $e^{-\beta H}$ is not integrable if $H$ is linear.}.
By Wick's theorem,
\begin{equation}
\int \mu(d\varphi) \; \varphi_a \cdot \varphi_b = C_{ab}.
\end{equation}
The \emph{Gaussian free field} on $\vertices$ with \emph{mass} $m^2 \ge 0$ is
the Gaussian field on $\vertices$ with covariance equal to the massive Green
function $C = (m^2 + L)^{-1}$.

%%%%%%%%%%%%%%%%%%%%%%%%%%%%%%%%%%%%%%%%%%%%%%%%%%%%%%%%%%%%%%%%%%%%%%%%%%%%%%%%%%%%%%%%%%

\subsection{Random walk}

\todo{For now specialize to unkilled walks. We only really need the transient case, for which
there is no problem with the corresponding Gaussian measure (in what sense? see Friedli-Velenik).
Put some of this in an external storage file.}

% The Laplacian $\lap$ is the generator of a $\vertices$-valued Markov process $X_t$
% that jumps from $x$ at rate $d_x$ and jumps to $y$ (from $x$) with probability
% $P_{xy} = d_x^{-1} J_{xy}$. Let $P_x$ denote the probability of this process conditioned so
% that $X_0 = x$ and define the measure $\mu_T$ on $\Wcal_T$ by
% \begin{equation}
% \mu_T(d\omega) = P_0((X_t = d\omega_t)_{t \le T}).
% \end{equation}
% Then $c_T = 1$ for all $T$ and so $\chi(\nu) = \nu^{-1}$.

% Let $\hat\mu_n$ be the corresponding measure on discrete $n$-step walks.
% Then $c_n(x, y) = P^n_{xy}$, so
% \begin{equation}
% G_{xy}
%   =
% \frac{1 - e^{-\nu}}{\nu} (1 - z P)^{-1}.
% \end{equation}

Let $Q$ be the $\vertices\cup\{\partial\}\times\vertices\cup\{\partial\}$ matrix with
\begin{equation}
Q_{xy}
  =
\begin{cases}
J_{xy},       & x \ne y, \partial, y \in \vertices \\
m^2,          & x \ne \partial, y = \partial \\
-(m^2 + d_x), & x = y \in \vertices \\
0,            & x = \partial
\end{cases}.
\end{equation}
Let $X_t$ be the Markov process with generator $Q$.
That is, $X$ takes jumps from $x$ at rate
\begin{equation}
-Q_{xx}
  =
\begin{cases}
m^2 + d_x,    & x = y \in \vertices \\
0,            & x = y = \partial
\end{cases}
\end{equation}
and jumps from $x$ to $y$ with probability (with $0/0 = 1$)
\begin{equation}
\hat P_{xy}
  =
-Q_{xy} / Q_{xx}
  =
\begin{cases}
z_x P_{xy},             & x \ne \partial, y \in \vertices \\
\frac{m^2}{m^2 + d_x},  & x \ne \partial, y = \partial \\
\1_{y=\partial},        & x = \partial
\end{cases}.
\end{equation}
We call $X_t$ the \emph{continuous-time random walk} on $\graph$.
In continuous time, we let
\begin{equation}
\lambda_T(d\omega) = P_0(d\omega \mid \tau_\partial = T),
\end{equation}
where $\tau_\partial = \inf(t : X_T = \partial)$.

The \emph{discrete-time random walk} on $\graph$ is the Markov chain
$\hat X_n$ defined by
\begin{equation}
\hat X_n = X_{\tau_n},
\end{equation}
where $\tau_n$ is the $n$-th jump time of $X$. The transition matrix of
$\hat X$ is given by $\hat P$.

For $v \in \vertices$, let $P_v$ and $E_v$ denote the probability and expectation
with respect to either the process $X$ or $\hat X$ conditioned so that $X_0 = v$.
We have
\begin{equation}
c_T(x, y) = P_x (X_T = y) = (e^{T Q})_{xy}.
\end{equation}
Thus,
\begin{equation}
\int_0^\infty dT \; e^{-\nu T} c_T(x, y)
  =
\sum_{n=0}^\infty \frac{1}{n!} Q^n_{xy} \int_0^\infty dT \; e^{-\nu T} T^n
  =
\sum_{n=0}^\infty Q^n_{xy} \nu^{-(n+1)}
\end{equation}

\begin{example}
We have
\begin{equation}
\int_0^\infty dT \; \1_{X_T=y}
  =
\sum_{n=0}^\infty \int_{\tau_n}^{\tau_{n+1}} dT \; \1_{\hat X_n=y}
  =
\sum_{n=0}^\infty (\tau_{n+1} - \tau_n) \1_{\hat X_n=y}.
\end{equation}
If $\hat X_n = y$, then $\tau_{n+1} - \tau_n$ is a Poisson random variable with rate $d_y$,
so taking the expectation of the above yields
\begin{equation}
G_{xy} = d_y \sum_{n=0}^\infty c_n(x, y).
\end{equation}
\todo{Show that this is}
\begin{equation}
C(x, y) = (m^2 + d_x)^{-1} E_x \left(\sum_{n=0}^\infty \1_{X_n=y} \right)
\end{equation}
so the sum converges if and only if the expected number of visits to $y$
made a walk with transition $\hat P$ started at $x$ is finite. Thus, $C$ is
well-defined if and only if the walk is transient.
In particular, it is well-defined for $m^2 > 0$. When $\graph = \Zd$, it is
well-defined for $m^2 = 0$ if and only if $d > 2$.
\end{example}

\begin{example}
\todo{On transitive graphs, SRW is a sum of iid random variables. This gives CLT and
invariance principle}
\end{example}

%%%%%%%%%%%%%%%%%%%%%%%%%%%%%%%%%%%%%%%%%%%%%%%%%%%%%%%%%%%%%%%%%%%%%%%%%%%%%%%%%%%%%%%%%%
%%%%%%%%%%%%%%%%%%%%%%%%%%%%%%%%%%%%%%%%%%%%%%%%%%%%%%%%%%%%%%%%%%%%%%%%%%%%%%%%%%%%%%%%%%

\section{Lattice models}

% Let $\graph = (\vertices, \edges)$ be a graph with vertices
% $\vertices$ and (undirected) edges $\edges$.

%%%%%%%%%%%%%%%%%%%%%%%%%%%%%%%%%%%%%%%%%%%%%%%%%%%%%%%%%%%%%%%%%%%%%%%%%%%%%%%%%%%%%%%%%%

\subsection{Models of walks}

In the following, $\interval_T$ will denote one of the following choices (for all $T \ge 0$):
\begin{equation}
\interval_T
  =
\begin{cases}
\{ 0, \ldots, \lfloor T \rfloor \} \\
[0, T]
\end{cases}.
\end{equation}
These two choices will be referred to, respectively, as the \emph{discrete-} and
\emph{continuous-time} cases.
\todo{Unless the graph is transitive, the following depends on $0$.}
Fix a designated vertex $0 \in \vertices$ and
let $\Wcal_T$ denote the set of
right-continuous paths $\omega : \interval_T \to \vertices$ with $\omega(0) = 0$.
We will refer to elements of $\Wcal := \bigcup_{T \geq 0} \Wcal_T$ as \emph{walks}.
A model of walks is determined by a choice of finite measure $d\mu_T$ on
$\Wcal_T$ for each $T$.
In the discrete-time case, we will assume that
$\mu_T = \mu_{\lfloor T \rfloor}$ for all $T$ (both are measures on
$\Wcal_{\lfloor T \rfloor}$ in this case).

Given a model of walks $d\mu_T$, we define the normalizing constant
\begin{equation}
c_T = \int_{\Wcal_T} d\mu_T.
\end{equation}
Given a parameter $\nu \in \R$ (called the \emph{killing rate}),
there is a natural measure $\mu$ on $\Wcal$ defined by
\begin{equation}
\int f \; d\mu
  =
\int_0^\infty dT \; e^{-\nu T} \int_{\Wcal_T} d\mu_T f
\end{equation}
for any continuous function $f$ on $\Wcal$ with compact support.
\todo{When is $\mu$ well-defined?}
The corresponding normalizing constant is denoted
\begin{equation}
\chi(\nu) = \int_0^\infty dT \; e^{-\nu T} c_T
\end{equation}
and is known as the \emph{susceptibility}. In the discrete-time case,
we let $z = e^{-\nu}$ and write
\begin{equation}
\mu(f)
  =
\frac{1 - e^{-\nu}}{\nu} \sum_{n=0}^\infty z^n \int_{\Wcal_n} d\mu_n(\omega) \; f(\omega).
\end{equation}
with
\begin{equation}
\chi(\nu) = \frac{1 - e^{-\nu}}{\nu} \sum_{n=0}^\infty c_n z^n.
\end{equation}

For $a, b \in \vertices$, let $\Wcal_T(a, b) \subset \Wcal_T$ denote the collection of walks
$\omega \in \Wcal_T$ with $\omega_0 = a$ and $\omega_T = x$.
The conditional measure $\mu^{(ab)}_T = \mu_T(\cdot \mid \Wcal_T(a, b))$
is given by
\begin{equation}
\mu^{(ab)}_T(d\omega) = \frac{\mu_T(d\omega) \1_{\Wcal_T(a, b)}(\omega)}{c_T(a, b)},
\end{equation}
where
\begin{equation}
c_T(a, b) = \mu_T(\Wcal_T(a, b)).
\end{equation}
Let $\Wcal(a, b) = \bigcup_{T \ge 0} \Wcal_T(a, b)$. Then
\begin{equation}
\mu^{(ab)}(d\omega) = \frac{\mu(d\omega) \1_{\Wcal(a, b)}(\omega)}{G_x},
\end{equation}
where the \emph{two-point function}
\begin{equation}
G_{ab} = \mu(\Wcal(a, b)) = \int_0^\infty dT \; e^{-\nu T} c_T(a, b).
\end{equation}
In discrete time, we have
\begin{align}
G_{ab} = \frac{1 - e^{-\nu}}{\nu} \sum_{n=0}^\infty c_n(a, b) z^n.
\end{align}

Many models of walks are determined by letting $\mu_T$ be the canonical Gibbs measure
associated to a Hamiltonian $H_T : \Wcal_T \to \R$ and a base measure $d\lambda_T$ on $\Wcal_T$.
That is,
\begin{equation}
d\mu_T(\omega) = \frac{1}{Z_T} e^{-H_T(\omega)} d\lambda_T(\omega).
\end{equation}
In this caes, $\mu$ is the corresponding grand canonical ensemble (with killing rate playing
the role of fugacity) and $c_T$ and $\chi$
are the canonical and grand canonical partition functions, respectively.

% In what follows, we let $Q$ be a $\vertices \times \vertices$ matrix with $Q_{xy} = 0$
% whenever $\{ x, y \} \notin \edges$.

%%%%%%%%%%%%%%%%%%%%%%%%%%%%%%%%%%%%%%%%%%%%%%%%%%%%%%%%%%%%%%%%%%%%%%%%%%%%%%%%%%%%%%%%%%

\subsubsection{Self-avoiding walk}

A \emph{self-avoiding walk} of length $n$ on $\graph$ is a discrete walk $\omega\in\Wcal_n$
that has no self intersections, i.e.\ $\omega_x = \omega_y$ if and only if $x = y$.
We equip the collection of all self-avoiding walks of length $n$ with the uniform measure $\mu_n$.

These measures do not form a consistent family due to the possibility of ``traps''.
That is, the equality
\begin{equation}
\mu_{|\omega|}(\omega) = \sum_{\tilde\omega \supset \omega} \mu_{|\tilde\omega|}(\tilde\omega)
\end{equation}
does not hold for all $\omega\in\Wcal$ (the sum here is over all self-avoiding walks extending
$\omega$).
% \begin{wrapfigure}{R}{0.4\textwidth}
% \vspace{-0.5cm}
% \begin{center}
%   \includegraphics[width=0.3\textwidth]{figures/trapped}
%   \caption{A trapped self-avoiding walk}
%   \label{fig:trap}
% \end{center}
% \vspace{-0.5cm}
% \end{wrapfigure}
For instance, consider the self-avoiding walk $\omega\in\Wcal_7$ on $\Zd$ in
Figure~\ref{fig:trap}.
This walk has positive probability under $\mu_7$ but,
since there are no self-avoiding walks extending $\omega$, the sum on the right-hand side
above is $0$.

As a result, the methods of stochastic processes cannot be directly used to study the self-avoiding
walk. The existence of traps also contributes to the combinatorial difficulty of studying
self-avoiding walk; for instance, it is not clear how to express $c_{n+1}$ (the number of
$(n+1)$-step self-avoiding walks) in terms of $c_n$.

% \begin{figure}[!htb]
% \centering
% \caption{A trapped self-avoiding walk}
% \includegraphics{figures/trapped}
% \end{figure}

%%%%%%%%%%%%%%%%%%%%%%%%%%%%%%%%%%%%%%%%%%%%%%%%%%%%%%%%%%%%%%%%%%%%%%%%%%%%%%%%%%%%%%%%%%

\subsubsection{Weakly self-avoiding walk with self-attraction}

Define the \emph{local time} up to time $T$ of $\omega \in \Wcal$ at $x \in \Zd$ by
\begin{equation}
\lt^x_T(\omega) = \int_0^T \1_{\omega(S)=x} \; dS.
\end{equation}
In the discrete-time case, $\lt^x_n$ is the number of times $\omega$ visits $x$
and is bounded by $n$. In the continuous-time case, $\lt^x_T$ is almost surely
finite for the continuous-time simple random walk.

We define the \emph{intersection local time}
\begin{equation}
\label{e:ITdef}
I_T(\omega) = \sum_{x\in\vertices} (\lt^x_T)^2
  =
\int_0^T \!\! \int_0^T \1_{\omega(S_1)=\omega(S_2)} \; dS_1 dS_2
\end{equation}
and the \emph{contact self-attraction}
\begin{equation}
\label{e:CTdef}
C_T(\omega) =
  \sum_{x \in \vertices} \sum_{y \sim x} \lt_T^x(\omega) \lt_T^y(\omega)
  = \int_0^T ds \int_0^T dt \; \1_{\omega_{s} \sim \omega_{t}}
\end{equation}
up to time $T$.
% Recall that we have set the inverse temperature equal to $1$.
Given a parameter $\gcc > 0$,
and $\gamma \in \R$, we define
\begin{equation}
\label{e:Udef-neg}
U_{\gcc,\gamma}(f)
=
\gcc \sum_{x\in\vertices} f_x^2
- \frac{\gamma}{2d}
\sum_{x\in\vertices} \sum_{y \sim x} f_x f_y
\end{equation}
for $f : \vertices \to \R$.
The \emph{weakly self-avoiding walk with self-attraction} (WSAW-SA) is defined via
the Hamiltonian
\begin{equation}
H_T(\omega) = U_{\gcc,\gamma}(L_T(\omega)).
\end{equation}
We denote the canonical partition function by
\begin{equation}
c_T = c_T(\gcc, \gamma) = E_0 \left( e^{-\gcc I(T) + \gamma C(T)} \right),
\end{equation}
where $0 \in \vertices$ is fixed, and the susceptibility by
\begin{equation}
\chi(\gcc, \gamma, \nu) = \int_0^\infty c_T e^{-\nu T} \; dT.
\end{equation}

In the case $\gamma = 0$, the discrete-time version of this model is known as
the \emph{Domb-Joyce model}. In continuous-time, it is the \emph{continuous-time
weakly self-avoiding walk} (WSAW).

%%%%%%%%%%%%%%%%%%%%%%%%%%%%%%%%%%%%%%%%%%%%%%%%%%%%%%%%%%%%%%%%%%%%%%%%%%%%%%%%%%%%%%%%%%

\subsection{Spin systems}

Assume for now that $\graph$ is a finite graph.

Let $S \subset \R^n$. A \emph{spin system} on $\graph$ with spins in $S$ is given
by a probability measure $\mu$ on $\Omega = S^V$. Such a measure is often given by a
Gibbs measure
\begin{equation}
\frac{1}{Z} e^{-H(\varphi)} d\varphi,
\end{equation}
where $d\varphi = \prod_{x\in\vertices} d\varphi_x$ is the Lebesgue measure on
$\Omega$. A spin system is said to be \emph{ferromagnetic} when the measure can
be written as
\begin{equation}
\frac{1}{Z} e^{-\tilde H(\varphi)} \prod_{x\in\vertices} \mu(d\varphi_x)
\end{equation}
for some measure $\mu$ on $S$, with
\begin{equation}
\tilde H(\varphi) = -\sum_{x, y \in \vertices} J_{xy} \varphi_x \cdot \varphi_y
\end{equation}
and $J_{xy} \ge 0$ for all $x, y$.

%%%%% MOVE EARLIER %%%%%%%%%%%%%%%%
% Given a configuration $\varphi \in \Omega$
% of spins, we will denote the $i$-th component of $\varphi_x \in S$ by $\varphi^i_x$.
% We define the inner product
% \begin{equation}
% \varphi \cdot \tilde\varphi = \sum_{i=1}^n \sum_{x\in\vertices} \varphi^i_x \tilde\varphi^i_x
% \end{equation}
% and we let a $V \times V$ matrix $M$ act on $\varphi$ via
% \begin{equation}
% (M\varphi)_x = \sum_{y\in\vertices} M_{xy} \varphi_x.
% \end{equation}

We define the \emph{two-point function} (when it exists)
\begin{equation}
G_x(\mu) = \frac{1}{n} \int d\mu(\varphi) \; \varphi_0 \cdot \varphi_x
\end{equation}
and the \emph{susceptibility}
\begin{equation}
\chi(\mu) = \sum_{x\in\vertices} G_x(\mu).
\end{equation}
Note that this definition of the susceptibility is consistent with the corresponding
definition for walks.

%%%%%%%%%%%%%%%%%%%%%%%%%%%%%%%%%%%%%%%%%%%%%%%%%%%%%%%%%%%%%%%%%%%%%%%%%%%%%%%%%%%%%%%%%%

\subsection{Gaussian fields}

The Gaussian field with covariance $C$ is a spin system with Hamiltonian (for finite
$\vertices$)
\begin{equation}
H(\varphi) = \frac{1}{2} \varphi \cdot C^{-1} \varphi.
\end{equation}
By Wick's theorem, two-point function is just the covariance $C$.
Since $J$ has nonnegative entries, the Gaussian free field (which must be massive when
$\vertices$ is finite) is ferromagnetic.

%%%%%%%%%%%%%%%%%%%%%%%%%%%%%%%%%%%%%%%%%%%%%%%%%%%%%%%%%%%%%%%%%%%%%%%%%%%%%%%%%%%%%%%%%%

\subsubsection{The $|\varphi|^4$ spin model}

The $|\varphi|^4$ spin model is given by a quartic perturbation to the Gaussian free field.
Precisely, the Hamiltonian has the form
\begin{equation}
H(\varphi)
  =
\sum_{x\in\vertices}
\left(
  \frac{1}{4} g |\varphi_x|^4
    +
  \frac{1}{2} \nu |\varphi_x|^2
    +
  \frac{1}{2} \varphi_x \cdot (A \varphi)_x
\right)
\end{equation}
for $g > 0$, $\nu\in\R$. This is a ferromagnetic spin system.

Here, $m^2$ need not be strictly positive (this was required for the DGFF for reasons of
integrability). In fact, an important special case for us will be $A = -\Delta$.

%%%%%%%%%%%%%%%%%%%%%%%%%%%%%%%%%%%%%%%%%%%%%%%%%%%%%%%%%%%%%%%%%%%%%%%%%%%%%%%%%%%%%%%%%%

\subsubsection{The $O(n)$ spin model}

Consider the $|\varphi|^4$ model as above. Suppose that the $A$ are constant:
there exists $d_0$ such that $d_x = d_0$ for all $x$.
Then the $|\varphi|^4$ Hamiltonian can be written as
\begin{equation}
\sum_{x\in\vertices}
\left(
  \frac{1}{4} g |\varphi_x|^4
    +
  \frac{1}{2} (\nu + d_0) |\varphi_x|^2
\right)
  -
\frac{1}{2} \varphi \cdot \jay \varphi
\end{equation}

If we take the limit $g\to\infty$ with $\nu = -(d_0 + g / 2)$, the
$|\varphi|^4$ spin model converges
weakly to the $O(n)$ spin model, given by the Hamiltonian
\begin{equation}
H(\varphi) = -\frac{1}{2} \varphi \cdot \jay \varphi
\end{equation}
on the state space $S = S^{n-1} \subset \R^n$ the $(n-1)$-sphere.
This again is a ferromagnetic spin system.
The case $n = 1$ is the celebrated \emph{Ising model}. When $n = 2, 3$,
we get the \emph{XY model} and the \emph{classical Heisenberg model}.

%%%%%%%%%%%%%%%%%%%%%%%%%%%%%%%%%%%%%%%%%%%%%%%%%%%%%%%%%%%%%%%%%%%%%%%%%%%%%%%%%%%%%%%%%%

\subsubsection{Infinite-volume Gibbs measures and phase transitions}

\todo{This needs to be reworked. Don't need that much information on Gibbs measure
since we will be dealing with the single-phase regime.}

The presence of a phase transition in a physical system is signalled by an abrupt
(i.e.\ non-analytic) change in an observable quantity as a parameter is varied. In fact,
a $p$-th order phase transition is usually said to occur when the \emph{free energy}
\begin{equation}
F(\beta) = -\frac{1}{\beta} \log Z
\end{equation}
has a discontinuous $p$-th derivative (but continuous derivative of order $p - 1$).
But when the Hamiltonian $H$ is a smooth, then so is the expression defining the free
energy above. The solution to this problem is to study the limit of the free energy
when the underlying graph $\graph$ becomes infinite.

In other words, one takes a sequence $\Lambda_N \subset \graph$ of subgraphs such that
$\Lambda_N \uparrow \graph$, defines a Hamiltonian $H_N$ on $S^{\Lambda_N}$ for each $N$,
and attempts to take the limit
\begin{equation}
\lim_{N\to\infty} -\frac{1}{\beta} \log Z_N,
\end{equation}
where $Z_N$ is the partition function for the finite-volume Gibbs measure with Hamiltonian
$H_N$. An immediate issue one faces is how to define the $H_N$ in an appropriate, consistent
manner.

Besides the case of Gaussian fields, the examples above do not generalize directly to an infinite
graph as the Hamiltonian is no longer well-defined in that case. This obstacle was overcome in the
work of Dobrushin, Lanford, and Ruelle. We briefly sketch their ideas in this section.

For every finite subset $A \subset \vertices$, let $\Phi_A : \Omega \to \R$ be a function that
depends only on the values of the spins in $A$. The collection $\Phi = (\Phi_A)_{A\subset\vertices}$
is known as a \emph{potential}. The Hamiltonian induced by $\Phi$ on a finite
subset $\Lambda \subset \vertices$ is defined by
\begin{equation}
H^\Phi_\Lambda(\varphi) = \sum \Phi_A(\varphi),
\end{equation}
where the sum is over all finite subset $A \subset \vertices$ that intersect $\Lambda$.
For simplicity, suppose that the space of spins $S$ is either a discrete set equipped with the
counting measure or is equipped with the Lebesgue measure. Let us also assume that the Hamiltonians
$H^\Phi_\Lambda$ are bounded below. Thus, the  Gibbs measures corresponding to $H^\Phi_\Lambda$ are
all well-defined.

For $\Lambda \subset \vertices$ and $\varphi \in \Omega$, let $\varphi_\Lambda \in S^\Lambda$
denote the restriction of $\varphi$ to $\Lambda$.
A measure $\mu$ on $\Omega$ is said to be a \emph{Gibbs measure} for the potential $\Phi$
if for $\mu$-almost
every choice of boundary condition $\tilde\varphi$, the conditional distribution
$\mu(d\varphi \mid \varphi_{\Lambda^c} = \tilde\varphi_{\Lambda^c})$ is given by the finite-volume Gibbs
measure
\begin{equation}
\mu^\Phi_\Lambda(\varphi \mid \tilde\varphi)
  :=
\frac{1}{Z^\Phi_\Lambda(\tilde\varphi)}
e^{-H^\Phi_\Lambda(\varphi_\Lambda \tilde\varphi_{\Lambda^c})}
d\varphi_\Lambda,
\end{equation}
where $\varphi_\Lambda\tilde\varphi_{\Lambda^c} \in \Omega$ is the concatenation of
$\varphi_\Lambda$ and $\tilde\varphi_{\Lambda^c}$
and where $d\varphi_\Lambda$ is the Lebesgue measure on $S^\Lambda$.
We denote the collection of all such Gibbs measures by $\gibbs(\Phi)$.

It is straightforward to verify that any finite-volume Gibbs measure is a Gibbs measure
in the above sense. A common procedure for constructing other elements of $\gibbs(\Phi)$
is to take an \emph{infinite-volume limit}.

To do so, let $\Lambda_N$ be a sequence of finite subsets of $\vertices$ such that
$\Lambda_N \uparrow \vertices$
and let $\varphi^{(N)}$ be a sequence of spin configurations in $\Omega$. It can be verified
that, if the sequence of measures $\mu_N = \mu^\Phi_{\Lambda_N}(\cdot \mid \varphi^{(N)})$ converges weakly
to a measure $\mu$, then $\mu\in\gibbs(\Phi)$. Any measure $\mu$ constructed in this way
can be recovered from the limiting expectations $\lim_{N\to\infty} \mu_N(f)$ of sufficiently
many \emph{observables}, which are bounded continuous functions $f : \Omega \to \R$.
% It suffices to study the limiting properties of $\mu_N(f)$.

\begin{example}[Periodic boundary conditions]
Let $\graph = \Zd$ and suppose we are given a translation-invariant potential $\Phi$.
That is, $\Phi_A = \Phi_{A + i}$ for any $i \in \Zd$.
Suppose, moreover, that $\Phi$ is of finite range, i.e.\ $\Phi_A = 0$ whenever
$|A| > R$ for some $R$. Let $\Lambda_N \subset \Zd$ be a sequence of finite boxes
with $|\Lambda_N| > R$ for each $N$.

Given a configuration $\varphi \in \Omega$, let $\varphi^N \in \Omega$
be the periodic extension of the restriction of $\varphi_{\Lambda_N}$.
We define a potential $\Phi^N$ by
\begin{equation}
\Phi^N_A(\varphi) = \Phi_A(\varphi^N) \1_{A \subset \Lambda_N}.
\end{equation}
The Gibbs measure $\mu_N$ on $\Lambda_N$ with \emph{periodic boundary conditions}
with potential $\Phi$ is defined to be the Gibbs measure on $\Lambda_N$ with
potential $\Phi^N$ (and any choice of boundary conditions).
% \begin{equation}
% \mu_N(d\varphi) = \frac{1}{Z_N} \exp\left(-H^{\Phi^N}(\varphi)\right) d\varphi,
% \end{equation}
% where $H^{\Phi^N}$ is the Hamiltonian on $\Lambda_N$ associated to the potential $\Phi_N$.
It is shown in \cite[Example 4.20]{Georgii11} that, if $\mu_N$ converges weakly to a
measure $\mu$, then $\mu \in \gibbs(\Phi)$.
\end{example}

\begin{example}
Let $\graph = \Zd$ and let $\Lambda_N = \Zd/L^N\Zd$ for $L > 1$.
Let $G_{N,x}$ denote the two-point function of a spin system on $\Lambda_N$.
We define the two-point function on $\Zd$ by the limit
\begin{equation}
G_x = \lim_{N\to\infty} G_{x,N}.
\end{equation}
By the previous example, if the limit $\mu$ of the $\mu_N$ exists, then
\begin{equation}
G_x = \frac{1}{n} \int d\mu(\varphi) \varphi_0 \cdot \varphi_x.
\end{equation}
\end{example}

%%%%%%%%%%%%%%%%%%%%%%%%%%%%%%%%%%%%%%%%%%%%%%%%%%%%%%%%%%%%%%%%%%%%%%%%%%%%%%%%%%%%%%%%%%
%%%%%%%%%%%%%%%%%%%%%%%%%%%%%%%%%%%%%%%%%%%%%%%%%%%%%%%%%%%%%%%%%%%%%%%%%%%%%%%%%%%%%%%%%%

\section{Critical behaviour and universality}

\todo{This description of critical behaviour is too narrow and does not apply to walks}

In this section, we let $\graph = \Zd$.

Let $\gibbs_\beta$ denote the set of all Gibbs measures for some potential at inverse temperature
$\beta$. A \emph{phase transition} is said to occur at inverse temperature $\beta$ if
$|\gibbs_\beta| > 1$. Many systems possess a unique \emph{critical point} $\beta_c$, such that
\begin{equation}
|\gibbs_\beta|
\begin{cases}
=1,  & \beta < \beta_c \\
> 1, & \beta > \beta_c
\end{cases}
\end{equation}
and it is usually expected that $|\gibbs_{\beta_c}| = 1$.
Moreover, such systems tend to exhibit \emph{critical behaviour} when $\beta = \beta_c$:
roughly speaking, this means that a number of observables scale according to a power
law (sometimes with logarithmic corrections) at or near $\beta_c$, whereas these same
observables exhibit exponential decay away from $\beta_c$.

A strong indicator of critical behaviour is the divergence of the
\emph{correlation length}, defined for any model (of walks or spins) with two-point
function $G_x(\mu)$ by
\begin{equation}
\xi(\mu) = \limsup_{k\to\infty} \frac{-k}{\log G_{ke}(\mu)},
\end{equation}
where $e \in \Zd$ is a unit vector.
In other words, the correlation length is the exponential rate of decay of the
two-point function.
A related quantity is the \emph{correlation length of order $p$}, defined by
\begin{equation}
\xi_p(\mu) = \left(\frac{\sum_{x\in\Zd} |x|^p G_x(\mu)}{\chi(\mu)}\right)^{1/p}.
\end{equation}
\todo{Heuristic relation bewteen $\xi$ and $\xi_p$.}

It is expected that
\begin{align}
G_x       &\sim C_1 |x|^{-(d - 2 + \eta)}, \\
\chi(\nu) &\sim C_2 (\nu - \nu_c)^{-\gamma}, \\
\xi       &\sim C_3 (\nu - \nu_c)^{-\nubar}, \\
\xi_p     &\sim C_4 (\nu - \nu_c)^{-\nubar}
\end{align}
possibly with logarithmic corrections.
For walks, it is expected that
\begin{align}
c_T                       &\sim C_5 e^{-\nu_c T} T^{-\gamma}, \\
\langle |X_T|^2 \rangle   &\sim C_6 T^{-\nubar}.
\end{align}

%%%%%%%%%%%%%%%%%%%%%%%%%%%%%%%%%%%%%%%%%%%%%%%%%%%%%%%%%%%%%%%%%%%%%%%%%%%%%%%%%%%%%%%%%%

\subsection{The DGFF}

The two-point function is just the massive Green function $(-\Delta + m^2)^{-1}$
which, on $\Zd$, has the well-known Ornstein-Uhlenbeck decay \todo{(show this;
use random walks?)}.
Moreover,
\begin{equation}
\chi
  =
\sum_{x\in\vertices} (-\Delta + m^2)^{-1}_{0x}
  =
\sum_{x\in\vertices} \sum_{n=0}^\infty z^n P^n_{0x}
  =
\sum_{n=0}^\infty z^n
  =
(1 - z)^{-1}.
\end{equation}
Thus, there is a critical point at $m^2 = 0$ ($z = 1$).

\todo{See candidacy report or preliminary version of it.}

\todo{For the Green function, see Theorem 1.5.4 in Lawler--Intersections of Random Walks}

%%%%%%%%%%%%%%%%%%%%%%%%%%%%%%%%%%%%%%%%%%%%%%%%%%%%%%%%%%%%%%%%%%%%%%%%%%%%%%%%%%%%%%%%%%

\subsection{Universality}

Critical behaviour should, roughly speaking, only depend on ``tail properties''
(global geometry, range of interaction, and symmetries).

%%%%%%%%%%%%%%%%%%%%%%%%%%%%%%%%%%%%%%%%%%%%%%%%%%%%%%%%%%%%%%%%%%%%%%%%%%%%%%%%%%%%%%%%%%

\subsection{Scaling limits}

%%%%%%%%%%%%%%%%%%%%%%%%%%%%%%%%%%%%%%%%%%%%%%%%%%%%%%%%%%%%%%%%%%%%%%%%%%%%%%%%%%%%%%%%%%
%%%%%%%%%%%%%%%%%%%%%%%%%%%%%%%%%%%%%%%%%%%%%%%%%%%%%%%%%%%%%%%%%%%%%%%%%%%%%%%%%%%%%%%%%%

\section{Relations between models}

There are a number of close relationships between models of walks and ferromagnetic spin
systems given by a Gibbs measure.

%%%%%%%%%%%%%%%%%%%%%%%%%%%%%%%%%%%%%%%%%%%%%%%%%%%%%%%%%%%%%%%%%%%%%%%%%%%%%%%%%%%%%%%%%%

\subsection{The SRW and DGFF}

\todo{See candidacy report or preliminary version of it. See also misc notes}

For the discrete-time simple random walk, the canonical partition function $c_n$,
which is the number of such walks, is just $(2 d)^n$.

%%%%%%%%%%%%%%%%%%%%%%%%%%%%%%%%%%%%%%%%%%%%%%%%%%%%%%%%%%%%%%%%%%%%%%%%%%%%%%%%%%%%%%%%%%

\subsection{The Kac-Siegert representation}

\todo{See notes on this}

%%%%%%%%%%%%%%%%%%%%%%%%%%%%%%%%%%%%%%%%%%%%%%%%%%%%%%%%%%%%%%%%%%%%%%%%%%%%%%%%%%%%%%%%%%

\subsection{Self-avoiding walk representations}

\todo{High-temperature expansion of spin system (can be used later to motivate polymer
expansion) to get loop models, De Gennes $n\downarrow0$ limit, McKane/Parisi-Sourlas
and supersymmetry, Grassmann integration and BIS representation (sufficiently general
for WSAW-SA)}