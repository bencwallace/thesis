% Parts are the largest structural units, but are optional.
%\part{Thesis}

% Chapters are the next main unit.
\chapter{Introduction}

%%%%%%%%%%%%%%%%%%%%%%%%%%%%%%%%%%%%%%%%%%%%%%%%%%%%%%%%%%%%%%%%%%%%%%%%%%%%%%%
%%%%%%%%%%%%%%%%%%%%%%%%%%%%%%%%%%%%%%%%%%%%%%%%%%%%%%%%%%%%%%%%%%%%%%%%%%%%%%%

\section{Statistical mechanics}

%%%%%%%%%%%%%%%%%%%%%%%%%%%%%%%%%%%%%%%%%%%%%%%%%%%%%%%%%%%%%%%%%%%%%%%%%%%%%%%

\subsection{The principle of maximum entropy}

Let $(\Omega, \lambda)$ be a measure space. The state of knowledge of a system on $\Omega$ can be expressed by a probability measure $\mu$ on $\Omega$. Let
$\Mcal_\lambda(\Omega)$ denote the set of probability measures on $\Omega$ absolutely continuous with respect to $\lambda$. For
$\mu \in \Mcal_\lambda(\Omega)$, we denote the Radon-Nikodym derivative of
$\mu$ with respect to $\lambda$ by $d\mu/d\lambda$ and define the
\emph{entropy} of $\mu$ with respect to $\lambda$ by
\begin{equation}
h(\mu) = h_\lambda(\mu) = -\int_\Omega \log\frac{d\mu}{d\lambda} \; d\mu.
\end{equation}
In many cases, specific information about the system under consideration is available, so that we may restrict our attention to a subspace $M \subset \Mcal_\lambda(\Omega)$.
% for example in the form of statements such as
% \begin{equation}
% \int f \; d\mu \in S_f
% \end{equation}
% with $f$ running over some collection of $\mu$-integrable functions
% that represent \emph{observable} quantities of the system, and $S_f$ a Borel
% subsets of $\R$ for each $f$.
The \emph{principle of maximum entropy} \cite{Jaynes57} asserts that, in this
case, the measure best expressing the state of knowledge of the system is given by
\begin{equation}
\hat\mu = \argmax(h_\lambda(\mu) : \mu \in M),
\end{equation}
assuming such a measure exists and is unique. One may for instance regard the choice of a measure not attaining this maximum to be tantamount to the imposition of additional constraints on the system.

%%%%%%%%%%%%%%%%%%%%%%%%%%%%%%%%%%%%%%%%%%%%%%%%%%%%%%%%%%%%%%%%%%%%%%%%%%%%%%%

\subsection{Microcanonical ensemble}

Consider an \emph{isolated} physical system on $\Omega$, that is, one that cannot exchange energy with its surroundings. Such a system can be determined by a choice of function $H : \Omega \to \R$, called the \emph{Hamiltonian}. The value $H(\omega)$ represents the total energy of the system in state
$\omega\in\Omega$.

\begin{example}
Let $\Omega = U^n \times \R^{3n}$, where $U \subset \R^3$, and denote a generic element of $\Omega$ by $(q, p)$, where $q \in U^n$ and $p \in \R^{3n}$. Then $\Omega$ is the state space of a system of $n$ point particles
$i = 1, \ldots, n$ with positions $q_i \in U$ and momenta $p_i \in \R^3$. Given a $C^1$ Hamiltonian $H : \Omega \to \R$, the dynamics of such a system is determined by \emph{Hamilton's equations}
\begin{align}
\dd{q}{t}   &= \nabla_q H(q(t), p(t)) \\
-\dd{p}{t}  &= \nabla_p H(q(t), p(t)).
\end{align}
An immediate consequence of these equations and the chain rule is the \emph{principle of conservation of energy}:
\begin{equation}
\dd{H}{t}(q(t), p(t)) = 0.
\end{equation}
Thus, a Hamiltonian system with initial configuration $(q(0), p(0))$ of energy
$E = H(q(0), p(0))$ will evolve on the constant energy shell $S_E = H^{-1}(E)$.

% Let us assume that $H$ has compact level sets so that the solutions to Hamilton's
% equations can be extended globally in time. In general, especially when $n$ is large,
% a Hamiltonian system is likely to be too complicated to understand in a precise way.
% The basic idea of statistical mechanics is to leverage this complexity by assuming
% (or in some cases proving) that, as $t\to\infty$, such a system will settle into a
% state of equilibrium in which all possible states are ``equiprobable''.
\end{example}

% Although one can make sense of the notion of an equiprobable probability distribution
% on $S_E$ in the above example, it is simpler (and more relevant to this thesis) to
% consider the case when $S_E$ is finite.
If $S_E = H^{-1}(E)$ is finite\footnote{We can view such a system as an approximation to a traditional continuous system with $S_E$ uncountable.}, then one can easily determine that the maximum entropy measure on $S_E$ is simply the uniform measure. For a system with finite state space $\Omega$ and Hamiltonian $H$, we define the \emph{microcanonical distribution} with energy $E$ to be the uniform measure on $S_E$.

%%%%%%%%%%%%%%%%%%%%%%%%%%%%%%%%%%%%%%%%%%%%%%%%%%%%%%%%%%%%%%%%%%%%%%%%%%%%%%%

\subsection{Canonical ensemble}

In practice, most systems of interest are not truly isolated: they may exchange
energy with their environments. Everyday experience, however, suggests that
a physical system that is left undisturbed for a sufficiently long time will achieve \emph{thermal equilibrium}, in which the system's temperature is constant and equal to that of its surroundings. We define the \emph{canonical ensemble} for a system with state space $\Omega$ and Hamiltonian $H$ on
$\Omega$ to be the maximum entropy distribution subject to the fixed average energy constraint
\begin{equation}
\int H \; d\mu = E.
\end{equation}
It can be shown by the method of Lagrange multipliers that the canonical ensemble is given by the \emph{Gibbs measure}
\begin{equation}
d\mu_\beta = \frac{1}{Z_\beta} e^{-\beta H} d\lambda,
\end{equation}
where
\begin{equation}
Z_\beta = \int e^{-\beta H} \; d\lambda
\end{equation}
is the normalizing constant, known as the \emph{canonical partition function}.
The quantity $\beta = \beta(E)$, which arises as a Lagrange multiplier, is known as the \emph{inverse temperature}.

The \emph{free energy} of this system is defined by
\begin{equation}
F_\beta = -\frac{1}{\beta} \log Z_\beta.
\end{equation}
This definition may seem obscure at first, but is elucidated by a computation of the entropy $h_\lambda(\mu_\beta)$ with $\beta = \beta(E)$, which implies that
\begin{equation}
F_\beta = E - \frac{1}{\beta} h_\lambda(\mu_\beta).
\end{equation}
This is the famous thermodynamic relation between free energy, internal energy $E$, temperature $1/\beta$, and entropy.


In the context of spin systems, there is a natural mathematical reason for studying measures of the above form. This is the Hammersley-Clifford theorem, which states that any Markov random field (a spatial generalization of a Markov chain) on a graph has a representation as a Gibbs measure whose Hamiltonian is a sum of ``local'' interactions.

%%%%%%%%%%%%%%%%%%%%%%%%%%%%%%%%%%%%%%%%%%%%%%%%%%%%%%%%%%%%%%%%%%
% COMMENTED OUT: Canonical measure as marginal of microcanonical %
%%%%%%%%%%%%%%%%%%%%%%%%%%%%%%%%%%%%%%%%%%%%%%%%%%%%%%%%%%%%%%%%%%
% This situation can be modeled as follows: call the system of interest system $A$ and
% its surroundings system $B$. We assume that $B$ behaves as a \emph{thermal reservoir},
% meaning that it is so large that its temperature may be assumed to be effectively
% constant, even if system $A$ is at a different temperature.
% It is then reasonable to assume that the total system $A \times B$ is isolated.

% \begin{example}
% Let $\Omega_1$ and $\Omega_2$ be the state spaces of systems $A$ and
% $B$ and let $H_i : \Omega_i \to \R$ be their respective Hamiltonians.
% The system $A \times B$ then has state space $\Omega_1 \times \Omega_2$.
% If we assume that the contribution to the energy due to interactions between systems $A$
% and $B$ is negligible, then the Hamiltonian of $A \times B$ is the function $H : \Omega \to \R$
% defined by $H(\omega_1, \omega_2) = H_1(\omega_1) + H_2(\omega_2)$.

% Let us assume that the 
% Let $S^i_{E_i} = H_i^{-1}(E_i)$ and $S_E = H^{-1}(S_E)$ and assume that $|S^i_{E_i}| < \infty$
% for any $E_i$. Then
% \begin{equation}
% S_E = \bigcup_{E_1=0}^E S^1_{E_1} \times S^2_{E-E_2}
% \end{equation}
% is finite for all $E$.
% % Suppose that $S_E$ is compact so that the microcanonical ensemble $\mu_E$ on $S_E$ is well-defined.

% For any $E_2 \in \R$, let $S^2_{E_2} = H_2^{-1}(E_2)$. Then the marginal distribution $\mu^1_E$
% of $\mu$ on $\Omega_1$ is given by
% \begin{equation}
% \mu^1_E(\omega_1)
%   =
% \sum_{\omega_2:\omega\in S_E} \mu_E(\omega)
%   =
% \frac{\# S^2_{E-H_1(\omega_1)}}{\# S_E}
%   \propto
% e^{h^2_{E-H_1(\omega_1)}},
% \end{equation}
% where
% \begin{equation}
% h^2_{E_2} = \log |S^2_{E_2}|
% \end{equation}
% is the \emph{entropy} of the microcanonical ensemble on $S^2_{E_2}$.

% If $E_2 \mapsto h^2_{E_2}$ is $C^1$, then we have the first-order expansion
% \begin{equation}
% h^2_{E - H_1(\omega_1)}
%   =
% h^2_E - \dd{h^2_E}{E} H_1(\omega_1) + O(H_1(\omega_1)^2).
% \end{equation}
% Thus,
% \begin{equation}
% e^{-h^2_{E-H_1(\omega_1)}}
%   \approx
% C e^{-\beta H_1(\omega_1)},
% \end{equation}
% where $\beta = \dd{h^2_E}{E}$ and $C$ is a constant independent of $\omega_1$.
% The assumption that system $B$ is a ``heat bath'' then amounts to saying that
% $E - H_1(\omega_1)$ is such a negligible perturbation of $E$ that the first-order
% approximation above accurately represents the marginal density of $\mu^1_E$.
% \end{example}

%%%%%%%%%%%%%%%%%%%%%%%%%%%%%%%%%%%%%%%%%%%%%%%%%%%%%%%%%%%%%%%%%%%%%%%%%%%%%%%

\subsection{Grand canonical ensemble}

Consider a physical system that is free to exchange both energy and particles with its environment. We suppose the ``number of particles'' of this system takes values in a measurable set $\interval \subset \R$. For
$T \in \interval$, let $\Omega_T$ be a space of $T$-particle configurations and let $H_T : \Omega_T \to \R$ be the Hamiltonian on $\Omega_T$. The state space for the full system is then given by
$\Omega = \bigsqcup_{T \in \interval} \Omega_T$, where $\sqcup$ denotes the disjoint union. For $\omega\in\Omega$, let $|\omega|$ be the unique value of
$T$ such that $\omega \in \Omega_T$. We define $H : \Omega \to \R$ by
$H(\omega) = H_{|\omega|}(\omega)$.

The \emph{grand canonical ensemble} for this system is defined to be the maximum entropy measure subject to the constraints of fixed average energy and fixed average number of particles. It can be shown that the grand canonical ensemble is given by a measure
\begin{equation}
d\mu_{\beta,\nu}
  =
\frac{1}{Z_{\beta,\nu}} e^{-\beta H(\omega) - \nu |\omega|},
\end{equation}
where $Z_{\beta,\nu}$ is the normalizing constant, known as the \emph{grand canonical partition function}. We call $\nu$ the \emph{fugacity}. Since the
$\nu |\omega|$ term can be absorbed into the Hamiltonian, the measure $\mu_{\beta,\nu}$ is really just an ordinary Gibbs measure as defined in the previous section. For this reason, the distinction between canonical and grand canonical ensembles is an informal one.

\begin{rk}
Note that the grand canonical partition function is given by
\begin{equation}
Z_{\beta,\nu}
  =
\int_\interval e^{-\nu T} Z_\beta^{(T)} \; dT,
\end{equation}
where $Z^{(T)}_\beta$ is the canonical partition function of $H_T$. Thus,
$Z_{\beta,\nu}$ is the Laplace transform of the canonical partition function
viewed as a function of $T$.
\end{rk}

%%%%%%%%%%%%%%%%%%%%%%%%%%%%%%%%%%%%%%%%%%%%%%%%%%%%%%%%%%%%%%%%%%%%%%%%%%%%%%%
%%%%%%%%%%%%%%%%%%%%%%%%%%%%%%%%%%%%%%%%%%%%%%%%%%%%%%%%%%%%%%%%%%%%%%%%%%%%%%%

\section{Graphs}

Most systems of interest do not have a finite (or even countable) state space. Nevertheless, finite systems serve as natural approximations of real systems. For instance, a natural way to approximate spatially-extended systems is by defining models on graphs.

%%%%%%%%%%%%%%%%%%%%%%%%%%%%%%%%%%%%%%%%%%%%%%%%%%%%%%%%%%%%%%%%%%%%%%%%%%%%%%%

\subsection{Graphs and the Laplacian}

Let $\vertices$ be a countable set.
Let $\jay$ be a symmetric $\vertices\times\vertices$ matrix satisfying
\begin{equation}
0 < d_x = \sum_{y\in\vertices} J_{xy} < \infty,
  \quad
J_{xy} \ge 0,
  \quad
J_{xx} = 0
\end{equation}
for all $x, y$. Then we can define the diagonal matrix $\diag$ with entries
\begin{equation}
\diag_{xx} = d_x.
\end{equation}
Moreover, if $\edges = \{ \{ x, y \} : J_{xy} \ne 0 \}$, then
$\graph = (\vertices, \edges, \jay)$ is a weighted connected undirected graph.
We write $x \sim y$ if $\{ x, y \} \in \edges$. We will say that $\graph$ is
$d_0$-regular if $d_x = d_0$ for all $x$.

The \emph{graph Laplacian} on $\graph$ is defined by
\begin{equation}
\lap = \diag - \jay.
\end{equation}
We also define the \emph{massive Laplacian} with squared \emph{mass} $m^2 > 0$
by
\begin{equation}
m^2 + \lap.
\end{equation}
Note that
\begin{equation}
\varphi \cdot \lap \varphi
  =
\frac{1}{2} \sum_{x,y\in\vertices} J_{xy} |\varphi_x - \varphi_y|^2
  \ge
0,
\end{equation}
so $\lap$ is positive-semidefinite.

\begin{example}
An important case is when $\jay$ has $\{0, 1 \}$-valued entries. In this case, $d_x$ is the \emph{degree} of $x$ in $\graph$ and we denote the matrix $\lap$ by $-\Delta$, which has entries
\begin{equation}
-\Delta_{xy} = d_x \1_{x=y} - \1_{x \sim y}.
\end{equation}
\end{example}

%%%%%%%%%%%%%%%%%%%%%%%%%%%%%%%%%%%%%%%%%%%%%%%%%%%%%%%%%%%%%%%%%%%%%%%%%%%%%%%

\subsection{The Green function}

If $m^2 > 0$, then $m^2 + \lap$ is positive-definite, hence invertible with inverse
\begin{equation}
(m^2 + \lap)^{-1} = (m^2 + D)^{-1} \sum_{n=0}^\infty Z^n P^n,
\end{equation}
where
\begin{align}
Z = (m^2 + D)^{-1} D,
  \quad
P = D^{-1} J.
\end{align}
Let $z_x$ denote the diagonal elements of $Z$. The \emph{Green function} for
$m^2 + \lap$ is the kernel of $(m^2 + \lap)^{-1}$, which we define by
\begin{equation}
C(x, y)
  =
(m^2 + d_x)^{-1} \sum_{n=0}^\infty z_x^n P^n_{xy}
\end{equation}
whenever this series converges.

% Let $\jay$ be a $\vertices \times \vertices$
% matrix with $J_{xx} = 0$ and $J_{xy} \ge 0$ for all $x, y \in \vertices$ and suppose
% that $\jay$ has summable rows: $\sum_{y\in\vertices} J_{xy} < \infty$. Let $\diag$ be
% the diagonal matrix with entries $D_{xx} = \sum_{y\in\vertices} J_{xy}$, and let $A = \diag - \jay$.

% \todo{The $Q$ matrix will be $Q = -A$, the matrix $\jay$ will be the (ferromagnetic) Ising interaction,
% and $A$ will be the interaction for the $|\varphi|^4$ model.
% E.g.\ $A = -\Delta + m^2$ has positive diagonal entries. Recall that
% $\Delta_{xy} = \1_{x \sim y} - 2 d \1_{x=y}$.}

%%%%%%%%%%%%%%%%%%%%%%%%%%%%%%%%%%%%%%%%%%%%%%%%%%%%%%%%%%%%%%%%%%%%%%%%%%%%%%%
%%%%%%%%%%%%%%%%%%%%%%%%%%%%%%%%%%%%%%%%%%%%%%%%%%%%%%%%%%%%%%%%%%%%%%%%%%%%%%%

\section{Spin systems}

An $n$-component \emph{field} or \emph{spin configuration} on $\vertices$
is an element $\varphi$ of $(\R^n)^\vertices$. We denote the components of a field $\varphi$ by $\varphi^i_x \in \R$ for $x \in \vertices$ and $i = 1, \ldots, n$. The Euclidean inner product on fields is defined by
\begin{equation}
\varphi\cdot\tilde\varphi
  =
\sum_{x\in\vertices} \varphi_x \cdot \tilde\varphi_y
  =
\sum_{i=1}^n \sum_{x\in\vertices} \varphi^i_x \tilde\varphi^i_x
\end{equation}
and the Euclidean norm is
\begin{equation}
|\varphi|^2 = \varphi \cdot \varphi.
\end{equation}
Given any $\vertices\times\vertices$ matrix $M$, we define the field
$M \varphi$ by
\begin{equation}
(M \varphi)_x = \sum_{y\in\vertices} M_{xy} \varphi_y
\end{equation}
when the sum is well-defined.

A \emph{spin system} is just a probability measure $d\mu$ on
$\Omega = S^\vertices$. The elements $\varphi \in \Omega$ are called
\emph{fields} or \emph{spin configurations}.

Suppose that $S \subset \R^n$ is endowed with a measure $d\lambda^0$ and let
$H : \Omega \to \R$ be a measurable function. We wish to study spin systems given by Gibbs measures of the form
\begin{equation}
d\mu_\beta(\varphi) = \frac{1}{Z_\beta} e^{-\beta H(\varphi)} d\lambda(\varphi),
\end{equation}
where
\begin{equation}
d\lambda(\varphi) = \prod_{x\in\vertices} d\lambda^0(\varphi_x).
\end{equation}

However, there are some problems with this definition when $\vertices$ is infinite. For one, $d\lambda$ may be pathological: for example, if $S = \R$ and
$d\lambda^0$ is Lebesgue measure, then it can be shown that $d\lambda$ takes values in $\{0, \infty\}$. Another problem is that it may be difficult to define a reasonable choice of $H$ on the infinite product space $\Omega$.

For this reason, we temporarily restrict our attention to finite graphs, and define spin systems on such graphs as above. In many cases, the Hamiltonian will depend on one or more parameters; in this case, adjusting $\beta$ is equivalent to rescaling these parameters and so we will usually drop $\beta$
(or set $\beta = 1$).

We will mainly be concerned with \emph{ferromagnetic} spin systems, for which the Hamiltonian has the form
\begin{equation}
H(\varphi) = \sum_{x, y \in \vertices} \varphi_x \cdot J_{xy} \varphi_y,
\end{equation}
where (recall) $J_{xy} \ge 0$. Thus, $H(\varphi)$ is lower when the spins align.

Let us fix a designated vertex\footnote{In general, some quantities we define will depend on the choice of vertex. However, we will ultimately be interested in the transitive graph $\graph = \Zd$ for which this choice is irrelevant.} $0 \in \vertices$. We define the \emph{two-point function} for such a spin system $\mu_\beta$ by
\begin{equation}
G_x(\mu_\beta) = \frac{1}{n} \int d\mu_\beta(\varphi) \; \varphi_0 \cdot \varphi_x
\end{equation}
when this is well-defined. \todo{This should be the truncated correlation.}
If it is, we also define the \emph{susceptibility}
\begin{equation}
\chi(\mu_\beta) = \sum_{x\in\vertices} G_x(\mu_\beta).
\end{equation}

%%%%%%%%%%%%%%%%%%%%%%%%%%%%%%%%%%%%%%%%%%%%%%%%%%%%%%%%%%%%%%%%%%%%%%%%%%%%%%%

\subsection{Gaussian measures and the free field}

Let $S = \R^n$ and let $C$ be a positive-definite symmetric $\vertices\times\vertices$ matrix. The $n$-component \emph{Gaussian measure} $d\mu_C$ on
$\vertices$ with mean $0$ and \emph{covariance} $C$ is defined by the Hamiltonian
\begin{equation}
H_C(\varphi) = \frac{1}{2} \varphi \cdot C^{-1} \varphi.
\end{equation}
This is essentially the simplest choice of non-constant
Hamiltonian\footnote{Note that $e^{-\beta H}$ is not integrable if $H$ is linear.}.
% is the unique measure $\mu$ on $(\R^n)^\vertices$ with characteristic function
% \begin{equation}
% \hat\mu(\xi)
%   \coloneqq
% \int_{(\R^n)^\vertices} \mu(d\varphi) \; e^{i \varphi \cdot \xi}
%   =
% e^{-\frac{1}{2} \xi \cdot C \xi}.
% \end{equation}
% If $\vertices$ is finite, then the Lebesgue measure $d\varphi$ on $(\R^n)^\vertices$
% is well-defined and
The partition function can be computed explicitly, giving\footnote{Here, we have employed our convention of setting $\beta = 1$ when the Hamiltonian depends on a parameter.}
\begin{equation}
d\mu_C(\varphi)
  =
\frac{1}{\sqrt{\det(2\pi C)}} e^{-\frac{1}{2} \varphi\cdot C^{-1}\varphi} d\varphi.
\end{equation}
By \emph{Wick's theorem}, the two-point function is just the covariance $C$:
\begin{equation}
\int \varphi_a \cdot \varphi_b \; d\mu(\varphi) = C_{ab}.
\end{equation}

An important case is the \emph{Gaussian free field} on $\vertices$ with \emph{mass} $m^2 \ge 0$, for which the covariance is equal to the massive Green function: $C = (m^2 + \lap)^{-1}$. It is not hard to show, using the fact that $J$ has nonnegative entries, that the Gaussian free field is ferromagnetic.

%%%%%%%%%%%%%%%%%%%%%%%%%%%%%%%%%%%%%%%%%%%%%%%%%%%%%%%%%%%%%%%%%%%%%%%%%%%%%%%

\subsection{The \texorpdfstring{$|\varphi|^4$}{phi4} spin model}

Let $S = \R^n$ again. The simplest choice of Hamiltonian that is not quadratic is given by a quartic function. The Hamiltonian for the $|\varphi|^4$ spin model is given by
\begin{equation}
H_{g,\nu}(\varphi)
  =
\sum_{x\in\vertices}
\left(
  \frac{1}{4} g |\varphi_x|^4
    +
  \frac{1}{2} \nu |\varphi_x|^2
    +
  \frac{1}{2} \varphi_x \cdot (\lap \varphi)_x
\right),
\end{equation}
where $g > 0$ and $\nu\in\R$. When $\nu > 0$, we have
$H_{g,\nu} = g |\varphi|^4 + H_C$, where $C = (\nu + \lap)^{-1}$ is the covariance of the Gaussian free field with mass $\nu$. Again, this is a ferromagnetic spin system.

%%%%%%%%%%%%%%%%%%%%%%%%%%%%%%%%%%%%%%%%%%%%%%%%%%%%%%%%%%%%%%%%%%%%%%%%%%%%%%%

\subsection{The \texorpdfstring{$O(n)$}{O(n)} spin model}

Let $S = S^{n-1} \subset \R^n$ be the unit $(n-1)$-sphere equipped with the surface measure $d\lambda^0$ (in particular, $S^0 = \{ \pm 1 \}$, which is equipped with the counting measure). The $O(n)$ model is defined by the Hamiltonian
\begin{equation}
H_J(\sigma) = -\frac{1}{2} \sigma \cdot J \sigma,
\end{equation}
which is clearly ferromagnetic.

The corresponding Gibbs measure arises naturally as a limiting case of the $|\varphi|^4$ measure $d\mu_{g,\nu}$ on a regular graph. Indeed, suppose that the diagonal elements of $\lap$ are constant: that is, there exists $d_0$ such that $d_x = d_0$ for all $x$. Then the $|\varphi|^4$ Hamiltonian can be written as
\begin{equation}
H_{g,\nu}(\varphi)
  =
\sum_{x\in\vertices}
\left(
  \frac{1}{4} g |\varphi_x|^4
    +
  \frac{1}{2} (\nu + d_0) |\varphi_x|^2
\right)
  -
\frac{1}{2} \varphi \cdot \jay \varphi
\end{equation}
Thus, $d\mu_{g,\nu} \Rightarrow d\mu_J$ if we take the limit $g\to\infty$ with $\nu = -(d_0 + g / 2)$.

The case $n = 1$ is the celebrated \emph{Ising model}. When $n = 2, 3$, we get the \emph{XY model} and the \emph{classical Heisenberg model}.

%%%%%%%%%%%%%%%%%%%%%%%%%%%%%%%%%%%%%%%%%%%%%%%%%%%%%%%%%%%%%%%%%%%%%%%%%%%%%%%

\subsection{The infinite-volume limit}

The presence of a phase transition in a physical system is signalled by an abrupt (i.e.\ non-analytic) change in an observable quantity as a parameter is varied. In fact, a $p$-th order phase transition is usually said to occur when the free energy has a discontinuous $p$-th derivative (but continuous derivatives of order less than $p$). However, the spin systems we have defined above all have smooth free energy (since the Hamiltonians are smooth functions). Ultimately, the reason we cannot detect phase transitions in these systems is that we have defined them on finite graphs. Thus, we are forced to face the problem of defining spin systems on an infinite volume $\graph$.

A natural way to study spin systems on such a graph is via a procedure known as the \emph{infinite-volume limit}: For any
finite subgraph $\Lambda \subset \graph$, let $H_\Lambda$ be the Hamiltonian
of one of the above spin systems on $\Lambda$ and let $\mu_\Lambda$ be the
corresponding Gibbs measure (which we can view as a measure on the full state
space $\Omega$). Then taking the infinite-volume limit involves studying the
limits
\begin{equation}
\lim_{\Lambda_N\uparrow\graph} \int f \; d\mu_\Lambda,
\end{equation}
where $\Lambda_N \uparrow \graph$ and $f$ varies over a sufficiently rich
class of functions on $\Omega$, e.g.\ all bounded continuous functions. When
these limit exist for a given sequence $\Lambda_N$, they define a measure $\mu$
on $\Omega$, which we call a \emph{Gibbs state} or \emph{infinite-volume Gibbs
measure} on $\Omega$.

We remark that there is a more general approach to the study of spin systems in infinite volume developed by Dobrushin, Lanford, and Ruelle. We do not detail their approach here, but merely mention that, for the examples above, this approach involves defining a Gibbs measure for the collection
$H = (H_\Lambda)$ of Hamiltonians directly as a measure on $\Omega$ satisfying a system of constraints on its conditionals. This is somewhat in the spirit of Kolmogorov's consistency conditions with the importance difference that the resulting collection $\gibbs_\beta(H)$ of Gibbs states at inverse temperature
$\beta$ need consist of only a single element. This is significant due to the interpretation of distinct elements of $\gibbs(H)$ as corresponding to different phases of the system under consideration.

Moreover, for many models, including the Ising and $|\varphi|^4$ models, it is known that there exists a critical inverse temperature $\beta_c$ such that
$|\gibbs(H)| = 1$ if and only if $\beta \le \beta_c$ (\REF?). In this thesis, our main concern is with the behaviour at $\beta_c$ and as
$\beta \uparrow \beta_c$. Thus, we need not concern ourselves with the precise nature of the infinite-volume limit.

\begin{rk}
In fact, we will be studying translation-invariant systems on $\graph = \Zd$, for which a particular choice of finite-volume approximation is convenient. Namely, for $L > 1$ we will let $\Lambda_N = \Zd/L^N\Zd$ be the discrete torus, which we view as a subset of $\Zd$. Strictly speaking, $\Lambda_N$ is not a subgraph of $\Zd$; nevertheless, it can be shown that the infinite-volume limit (if it exists) is a Gibbs measure in the usual sense; see \cite[Example 4.20]{Georgii11} for details.
\end{rk}

%%%%%%%%%%%%%%%%%%%%%%%%%%%%%%%%%%%%%%%%%%%%%%%%%%%%%%%
% COMMENTED OUT: Phase transitions in infinite volume %
%%%%%%%%%%%%%%%%%%%%%%%%%%%%%%%%%%%%%%%%%%%%%%%%%%%%%%%
% \subsection{Phase transitions in infinite volume}

% The presence of a phase transition in a physical system is signalled by an abrupt (i.e.\ non-analytic) change in an observable quantity as a parameter is varied. In fact, a $p$-th order phase transition is usually said to occur when the free energy has a discontinuous $p$-th derivative (but continuous derivatives of order less than $p$). However, the spin systems we have defined above all have smooth free energy (since the Hamiltonians are smooth functions). Ultimately, the reason we cannot detect phase transitions in these systems is that we have defined them on finite graphs. Thus, we are forced to face the problem of defining spin systems on an infinite volume $\graph$. Here we briefly outline the solution to this problem according to the theory of Dobrushin, Lanford, and Ruelle.

% We wish to define a measure corresponding to an ``infinite-volume Hamiltonian''
% \begin{equation}
% H^\Phi(\varphi) = \sum \Phi_A(\varphi),
%   \quad
% \varphi \in \Omega
% \end{equation}
% where the sum is over all finite subsets $A \subset \vertices$ and the $\Phi_A$
% depend only on the values of $\varphi$ in $A$. Although $H^\Phi$ is typically a fictional object, the $\Phi_A$ are well-defined. Thus, given a finite subset $\Lambda\subset\vertices$ and a choice of \emph{boundary condition}
% $\xi\in\Omega$, we can define the \emph{finite-volume} Hamiltonian
% $H^\xi_\Lambda : \Omega \to \R$ on $\Lambda$ by
% \begin{equation}
% H^\xi_\Lambda(\varphi) = \sum \Phi_A(\varphi_{\Lambda} \xi_{\Lambda^c}),
% \end{equation}
% where the sum is over all finite sets $A$ intersecting $\Lambda$,
% $\varphi_{\Lambda} \in S^{\Lambda}$ is the restriction of $\varphi$ to
% $\Lambda$, and $\varphi_{\Lambda} \xi_{\Lambda^c} \in \Omega$ is the concatenation of $\varphi_{\Lambda}$ and $\xi_{\Lambda^c}$. We then say that
% $\mu$ is a \emph{Gibbs measure} if for $\mu$-almost every choice of boundary condition $\xi$, the conditional distribution $\mu(d\varphi \mid \varphi_{\Lambda^c} = \xi_{\Lambda^c})$ is given by the finite-volume Gibbs measure
% \begin{equation}
% \mu^\Phi_\Lambda(\varphi \mid \xi)
%   :=
% \frac{1}{Z^\Phi_\Lambda(\xi)}
% e^{-H^\Phi_\Lambda(\varphi_\Lambda \xi_{\Lambda^c})}
% d\lambda^\Lambda(\varphi),
% \end{equation}
% where $d\lambda^\Lambda(\varphi) = \prod_{x\in\Lambda} d\lambda^0(\varphi_x)$.
% We denote the collection of all such Gibbs measures by $\gibbs(\Phi)$.

% \subsubsection{The infinite-volume limit}

% It can be checked that any finite-volume Gibbs measure (seen as a measure on $\Omega$) is a Gibbs measure in the above sense. More general Gibbs measures can be constructed by an explicit procedure known as the infinite-volume limit: generally speaking, one begins with a sequence $\Lambda_N \subset \vertices$ of finite subsets with $\Lambda_N\uparrow\vertices$. Letting $H_N$ be a Hamiltonian that only depends on the spins in $\Lambda_N$, one proceeds by taking the weak limit of the Gibbs measures $\mu_N$ corresponding to $H_N$. If the sequence $(\Lambda_N, H_N)$ is appropriately chosen, then the weak limit will exist and be an element of $\gibbs(\Phi)$.

% \subsubsection{Periodic boundary conditions}

% When $\graph = \Zd$ and the $\Phi$ are translation-invariant\footnote{That is, $\Phi_A = \Phi_{A+i}$ for any $i\in\Zd$.}, there is a particularly convenient approach to taking the infinite-volume limit. Suppose that $\Phi$ has finite range\footnote{This is a mere convenience. This construction extends to potentials with infinite range.}, i.e.\ $\Phi_A = 0$ whenever $|A| > R$ for some $R$. We let $\Lambda_N = \Zd/L^N\Zd$ be the discrete torus of side $L^N$ (for $L > 1$) and identify the $\Lambda_N$ with an increasing sequence of subsets of $\Zd$ (since $\Phi$ is translation-invariant, it is not important how this identification is made). Then any subset $A \subset \Zd$ of side
% $|A| \le R$ can be identified with a subset of $\Lambda_N$ for $N$ sufficiently large. For such $N$, we let
% \begin{equation}
% H_N(\varphi) = \sum \Phi_A(\varphi),
% \end{equation}
% where the sum is over subsets of $\Lambda_N$ of size at most $R$. \todo{I don't think this is right.} It is shown in \cite[Example 4.20]{Georgii11} that, if $\mu_N$ converges weakly to a measure $\mu$, then
% $\mu \in \gibbs(\Phi)$. The measure $\mu$ is said to have \emph{periodic} boundary conditions.

% \begin{example}
% Let $\graph = \Zd$ and let $\Lambda_N = \Zd/L^N\Zd$ for $L > 1$.
% Let $G_{N,x}$ denote the two-point function of a spin system on $\Lambda_N$.
% We define the two-point function on $\Zd$ by the limit
% \begin{equation}
% G_x = \lim_{N\to\infty} G_{x,N}.
% \end{equation}
% By the previous example, if the limit $\mu$ of the $\mu_N$ exists, then
% \begin{equation}
% G_x = \frac{1}{n} \int d\mu(\varphi) \varphi_0 \cdot \varphi_x.
% \end{equation}
% \end{example}

%%%%%%%%%%%%%%%%%%%%%%%%%%%%%%%%%%%%%%%%%%%%%%%%%%%%%%%%%%%%%%%%%%%%%%%%%%%%%%%
%%%%%%%%%%%%%%%%%%%%%%%%%%%%%%%%%%%%%%%%%%%%%%%%%%%%%%%%%%%%%%%%%%%%%%%%%%%%%%%

\section{Random polymer models}

For simplicity, we take $\graph$ to be a $d_0$-regular vertex-transitive graph and fix a designated vertex $0 \in \vertices$. We let $\interval_T$ denote either of the following choices for all $T \ge 0$:
\begin{equation}
\interval_T
  =
\begin{cases}
\{ 0, \ldots, \lfloor T \rfloor \} \\
[0, T]
\end{cases}.
\end{equation}
For any right-continuous function $\omega : \interval_T \to \vertices$, we define
$\tau_n = \tau_n(\omega)$ inductively by setting $\tau_0 = 0$ and
\begin{equation}
\tau_{n+1} = \inf(t > \tau_n : \omega_t \ne \omega_{\tau_n}).
\end{equation}
We call such a function $\omega$ a \emph{walk} of length $T$ if
$\{ \tau_n : n \in \Z_+ \}$ does not have any cluster points and
$\omega_{\tau_n} \sim \omega_{\tau_{n+1}}$ for all $n$; thus, $\omega$
only jumps between neighbouring vertices. A \emph{discrete-time} (respectively,
\emph{continuous-time}) walk of length $T$ is a walk indexed by
$\{ 0, \ldots, \lfloor T \rfloor \}$ (respectively, $[0, T]$).
We let $\Wcal_T$ denote the set of walks of length $T$ with $\omega_0 = 0$ and set $\Wcal := \bigcup_{T \geq 0} \Wcal_T$.

A model of walks is determined by a choice of finite measure $d\mu_T$ on
$\Wcal_T$ for each $T$. All models we consider will be given by canonical Gibbs measures
\begin{equation}
d\mu_T(\omega) = \frac{1}{c_T} e^{-H_T(\omega)} \; d\lambda_T(\omega),
\end{equation}
with respect to some measure $\lambda_T$ on $\Wcal_T$. Note that we have denoted the partition function by $c_T$. We will assume that models of discrete-time walks satisfy $\mu_T = \mu_{\lfloor T \rfloor}$ for all $T$
(note that both are measures on $\Wcal_{\lfloor T \rfloor}$ in this case).

Given such a model, we write the corresponding grand canonical ensemble as
\begin{equation}
d\mu(\omega) = \frac{1}{\chi(\nu)} e^{-\nu |\omega|} d\mu_{|\omega|}(\omega).
\end{equation}
In this setting, the fugacity $\nu$ is usually referred to as the \emph{killing rate}. The grand canonical partition function, denoted $\chi$, is known as the \emph{susceptibility}. Note that
\begin{equation}
\mu(\cdot \mid \Wcal_T) = \mu_T.
\end{equation}
\todo{When is $\mu$ well-defined?}

For $x \in \vertices$, let $\Wcal_T(x) \subset \Wcal_T$ denote the collection of walks $\omega \in \Wcal_T$ with $\omega_0 = a$ and $\omega_T = x$. The conditional measure $\mu^x_T = \mu_T(\cdot \mid \Wcal_T(x))$ is given by
\begin{equation}
\mu^x_T(d\omega) = \frac{\mu_T(d\omega) \1_{\Wcal_T(x)}(\omega)}{c_T(x)},
\end{equation}
where
\begin{equation}
c_T(x) = \mu_T(\Wcal_T(x)).
\end{equation}
Let $\Wcal(x) = \bigcup_{T \ge 0} \Wcal_T(x)$. Then
\begin{equation}
\mu^x(d\omega) = \frac{\mu(d\omega) \1_{\Wcal(x)}(\omega)}{G_x(\nu)},
\end{equation}
where $G_x$ is the \emph{two-point function}, defined by
\begin{equation}
G_x(\nu) = \mu(\Wcal(x)) = \int_0^\infty dT \; e^{-\nu T} c_T(x).
\end{equation}
Note that
\begin{equation}
\chi(\nu) = \sum_{x\in\vertices} G_x(\nu),
\end{equation}
which is consistent with the analogous relation for spin systems. Later we will discuss the relationship between the two-point function for walks and spin systems.

\todo{We really should not be defining terms for discrete- and continuous-time
walks separately as (for example) $c_n$ in discrete-time is not the same as in
continuous time. We should define discrete-time objects as the skeletons of
continuous-time ones.}

\begin{rk}
In the discrete-time case, we let $z = e^{-\nu}$ and write
\begin{equation}
\mu(\omega)
  =
\frac{1 - e^{-\nu}}{\nu} z^{|\omega|} \mu_{|\omega|}(\omega)
% \mu(f)
  % =
% \frac{1 - e^{-\nu}}{\nu} \sum_{n=0}^\infty z^n \int_{\Wcal_n} d\mu_n(\omega) \; f(\omega).
\end{equation}
with
\begin{equation}
\chi(\nu) = \frac{1 - e^{-\nu}}{\nu} \sum_{n=0}^\infty c_n z^n.
\end{equation}
Similarly,
\begin{align}
G_x(z) = \frac{1 - e^{-\nu}}{\nu} \sum_{n=0}^\infty c_n(x) z^n.
\end{align}
\end{rk}

%%%%%%%%%%%%%%%%%%%%%%%%%%%%%%%%%%%%%%%%%%%%%%%%%%%%%%%%%%%%%%%%%%%%%%%%%%%%%%%

\subsection{Random walk}

\todo{Restrict to $d_0$-regular graph (otherwise, generating function has $z_x$ instead of $z$).
Work towards showing that generating function (for unkilled SRW) is inverse of:
\begin{equation}
1 - z P = 1 - (z / d_0) J = (z / d_0) (m^2 + d_0 - J)
\end{equation}
with $m^2 = d_0 (1 - z) / z$.}

The \emph{simple random walk} on $\graph$ is the Markov chain $X_n$ with transition matrix $P = d_0^{-1} J$. This induces the measure $\lambda_n$ on $\Wcal_n$ defined by
\begin{equation}
\lambda_n(\omega) = \Pr(X_k = \omega_k, \, k \le n \mid X_0 = 0).
\end{equation}
Then
\begin{equation}
c_n(0, x) = \Pr(X_n = x \mid X_0 = 0) = P^n_{0x}
\end{equation}
and
\begin{equation}
G_x = \sum_{n=0}^\infty (z P)^n_{0x} = (1 - z P)^{-1}_{0x},
\end{equation}
which converges for $|z| < 1$. Letting $m^2 = d_0 (1 - z) / z > 0$, we see that
\begin{equation}
1 - z P = (z / d_0) (m^2 + L),
\end{equation}
so the two-point function is just the Green function for the massive Laplacian.

The Laplacian $\lap$ is the generator of the $\vertices$-valued Markov process
$X = (X_t)_{t \ge 0}$ with transition probabilities
\begin{equation}
\Pr(X_t = y \mid X_0 = x) = (e^{-t \lap})_{xy},
\end{equation}
called the \emph{continuous-time simple random walk} on $\graph$.
This process defines measure $\lambda_T$ on $\Wcal_T$ via
\begin{equation}
\lambda_T(d\omega) = \Pr(X_t = d\omega_t, \, t \le T \mid X_0 = 0).
\end{equation}
For the corresponding Gibbs measures (with $H_T \equiv 0$), we have
\begin{equation}
c_T(x) = (e^{-t \lap})_{0x},
  \quad
G_x = (\nu + \lap)^{-1}_{0x}
\end{equation}
for $\nu > 0$. Thus, the two-point function is the Green function for the
massive Laplacian.

The \emph{discrete-time simple random walk} on $\graph$ is the Markov chain
$\hat X = (\hat X_n)_{n\in\Z_+}$ defined by $\hat X_n = X_{\tau_n}$.

% Let $Q$ be the $\vertices\cup\{\partial\}\times\vertices\cup\{\partial\}$ matrix with
% \begin{equation}
% Q_{xy}
%   =
% \begin{cases}
% J_{xy},       & x \ne y, \partial, y \in \vertices \\
% m^2,          & x \ne \partial, y = \partial \\
% -(m^2 + d_x), & x = y \in \vertices \\
% 0,            & x = \partial
% \end{cases}.
% \end{equation}
% Let $X_t$ be the Markov process with generator $Q$. That is, $X$ takes jumps from $x$ at rate
% \begin{equation}
% -Q_{xx}
%   =
% \begin{cases}
% m^2 + d_x,    & x = y \in \vertices \\
% 0,            & x = y = \partial
% \end{cases}
% \end{equation}
% and jumps from $x$ to $y$ with probability (with $0/0 = 1$)
% \begin{equation}
% \hat P_{xy}
%   =
% -Q_{xy} / Q_{xx}
%   =
% \begin{cases}
% z_x P_{xy},             & x \ne \partial, y \in \vertices \\
% \frac{m^2}{m^2 + d_x},  & x \ne \partial, y = \partial \\
% \1_{y=\partial},        & x = \partial
% \end{cases}.
% \end{equation}
% We call $X_t$ the \emph{continuous-time random walk} on $\graph$. In continuous time, we let
% \begin{equation}
% \lambda_T(d\omega) = P_0(d\omega \mid \tau_\partial = T),
% \end{equation}
% where $\tau_\partial = \inf(t : X_T = \partial)$.

% The \emph{discrete-time random walk} on $\graph$ is the Markov chain
% $\hat X_n$ defined by
% \begin{equation}
% \hat X_n = X_{\tau_n},
% \end{equation}
% where $\tau_n$ is the $n$-th jump time of $X$. The transition matrix of
% $\hat X$ is given by $\hat P$.

% For $v \in \vertices$, let $P_v$ and $E_v$ denote the probability and expectation with respect to either the process $X$ or $\hat X$ conditioned so that $X_0 = v$. We have
% \begin{equation}
% c_T(x, y) = P_x (X_T = y) = (e^{T Q})_{xy}.
% \end{equation}
% Thus,
% \begin{equation}
% \int_0^\infty dT \; e^{-\nu T} c_T(x, y)
%   =
% \sum_{n=0}^\infty \frac{1}{n!} Q^n_{xy} \int_0^\infty dT \; e^{-\nu T} T^n
%   =
% \sum_{n=0}^\infty Q^n_{xy} \nu^{-(n+1)}
% \end{equation}

% \begin{example}
% We have
% \begin{equation}
% \int_0^\infty dT \; \1_{X_T=y}
%   =
% \sum_{n=0}^\infty \int_{\tau_n}^{\tau_{n+1}} dT \; \1_{\hat X_n=y}
%   =
% \sum_{n=0}^\infty (\tau_{n+1} - \tau_n) \1_{\hat X_n=y}.
% \end{equation}
% If $\hat X_n = y$, then $\tau_{n+1} - \tau_n$ is a Poisson random variable with rate $d_y$, so taking the expectation of the above yields
% \begin{equation}
% G_{xy} = d_y \sum_{n=0}^\infty c_n(x, y).
% \end{equation}
% \todo{Show that this is}
% \begin{equation}
% C(x, y) = (m^2 + d_x)^{-1} E_x \left(\sum_{n=0}^\infty \1_{X_n=y} \right)
% \end{equation}
% so the sum converges if and only if the expected number of visits to $y$
% made a walk with transition $\hat P$ started at $x$ is finite. Thus, $C$ is
% well-defined if and only if the walk is transient. In particular, it is well-defined for $m^2 > 0$. When $\graph = \Zd$, it is well-defined for $m^2 = 0$ if and only if $d > 2$.
% \end{example}

\begin{example}
\todo{On transitive graphs, SRW is a sum of iid random variables. This gives CLT and invariance principle}
\end{example}

%%%%%%%%%%%%%%%%%%%%%%%%%%%%%%%%%%%%%%%%%%%%%%%%%%%%%%%%%%%%%%%%%%%%%%%%%%%%%%%

\subsection{Self-avoiding walk}

A \emph{self-avoiding walk} of length $n$ on $\graph$ is a discrete walk $\omega\in\Wcal_n$ that has no self intersections, i.e.\ $\omega_x = \omega_y$ if and only if $x = y$. We equip the collection of all self-avoiding walks of length $n$ with the uniform measure $\mu_n$.

These measures do not form a consistent family due to the possibility of ``traps''. That is, the equality
\begin{equation}
\mu_{|\omega|}(\omega) = \sum_{\tilde\omega \supset \omega} \mu_{|\tilde\omega|}(\tilde\omega)
\end{equation}
does not hold for all $\omega\in\Wcal$ (the sum here is over all self-avoiding walks extending $\omega$).
% \begin{wrapfigure}{R}{0.4\textwidth}
% \vspace{-0.5cm}
% \begin{center}
%   \includegraphics[width=0.3\textwidth]{figures/trapped}
%   \caption{A trapped self-avoiding walk}
%   \label{fig:trap}
% \end{center}
% \vspace{-0.5cm}
% \end{wrapfigure}
For instance, consider the self-avoiding walk $\omega\in\Wcal_7$ on $\Zd$ in
Figure~\ref{fig:trap}. This walk has positive probability under $\mu_7$ but,
since there are no self-avoiding walks extending $\omega$, the sum on the right-hand side above is $0$.

As a result, the methods of stochastic processes cannot be directly used to study the self-avoiding walk. The existence of traps also contributes to the combinatorial difficulty of studying self-avoiding walk; for instance, it is not clear how to express $c_{n+1}$ (the number of $(n+1)$-step self-avoiding walks) in terms of $c_n$.

% \begin{figure}[!htb]
% \centering
% \caption{A trapped self-avoiding walk}
% \includegraphics{figures/trapped}
% \end{figure}

%%%%%%%%%%%%%%%%%%%%%%%%%%%%%%%%%%%%%%%%%%%%%%%%%%%%%%%%%%%%%%%%%%%%%%%%%%%%%%%

\subsection{Weakly self-avoiding walk with self-attraction}

Define the \emph{local time} up to time $T$ of $\omega \in \Wcal$ at
$x \in \Zd$ by
\begin{equation}
\lt^x_T(\omega) = \int_0^T \1_{\omega(S)=x} \; dS.
\end{equation}
In the discrete-time case, $\lt^x_n$ is the number of times $\omega$ visits $x$
and is bounded by $n$. In the continuous-time case, $\lt^x_T$ is almost surely
finite for the continuous-time simple random walk.

We define the \emph{intersection local time}
\begin{equation}
\label{e:ITdef}
I_T(\omega) = \sum_{x\in\vertices} (\lt^x_T)^2
  =
\int_0^T \!\! \int_0^T \1_{\omega(S_1)=\omega(S_2)} \; dS_1 dS_2
\end{equation}
and the \emph{contact self-attraction}
\begin{equation}
\label{e:CTdef}
C_T(\omega) =
  \sum_{x \in \vertices} \sum_{y \sim x} \lt_T^x(\omega) \lt_T^y(\omega)
  = \int_0^T ds \int_0^T dt \; \1_{\omega_{s} \sim \omega_{t}}
\end{equation}
up to time $T$.
% Recall that we have set the inverse temperature equal to $1$.
Given a parameter $\gcc > 0$,
and $\gamma \in \R$, we define
\begin{equation}
\label{e:Udef-neg}
U_{\gcc,\gamma}(f)
=
\gcc \sum_{x\in\vertices} f_x^2
- \frac{\gamma}{2d}
\sum_{x\in\vertices} \sum_{y \sim x} f_x f_y
\end{equation}
for $f : \vertices \to \R$.
The \emph{weakly self-avoiding walk with self-attraction} (WSAW-SA) is defined via the Hamiltonian
\begin{equation}
H_T(\omega) = U_{\gcc,\gamma}(L_T(\omega)).
\end{equation}
We denote the canonical partition function by
\begin{equation}
c_T = c_T(\gcc, \gamma) = E_0 \left( e^{-\gcc I(T) + \gamma C(T)} \right),
\end{equation}
where $0 \in \vertices$ is fixed, and the susceptibility by
\begin{equation}
\chi(\gcc, \gamma, \nu) = \int_0^\infty c_T e^{-\nu T} \; dT.
\end{equation}

In the case $\gamma = 0$, the discrete-time version of this model is known as
the \emph{Domb-Joyce model}. In continuous-time, it is the
\emph{continuous-time weakly self-avoiding walk} (WSAW).

%%%%%%%%%%%%%%%%%%%%%%%%%%%%%%%%%%%%%%%%%%%%%%%%%%%%%%%%%%%%%%%%%%%%%%%%%%%%%%%
%%%%%%%%%%%%%%%%%%%%%%%%%%%%%%%%%%%%%%%%%%%%%%%%%%%%%%%%%%%%%%%%%%%%%%%%%%%%%%%

\section{Critical behaviour and universality}

\todo{This description of critical behaviour is too narrow and does not apply to walks}

In this section, we let $\graph = \Zd$.

Let $\gibbs_\beta$ denote the set of all Gibbs measures for some potential at inverse temperature $\beta$. A \emph{phase transition} is said to occur at inverse temperature $\beta$ if $|\gibbs_\beta| > 1$. Many systems possess a unique \emph{critical point} $\beta_c$, such that
\begin{equation}
|\gibbs_\beta|
\begin{cases}
=1,  & \beta < \beta_c \\
> 1, & \beta > \beta_c
\end{cases}
\end{equation}
and it is usually expected that $|\gibbs_{\beta_c}| = 1$. Moreover, such systems tend to exhibit \emph{critical behaviour} when $\beta = \beta_c$:
roughly speaking, this means that a number of observables scale according to a power law (sometimes with logarithmic corrections) at or near $\beta_c$, whereas these same observables exhibit exponential decay away from $\beta_c$.

A strong indicator of critical behaviour is the divergence of the
\emph{correlation length}, defined for any model (of walks or spins) with two-point function $G_x(\mu)$ by
\begin{equation}
\xi(\mu) = \limsup_{k\to\infty} \frac{-k}{\log G_{ke}(\mu)},
\end{equation}
where $e \in \Zd$ is a unit vector. In other words, the correlation length is the exponential rate of decay of the two-point function.
A related quantity is the \emph{correlation length of order $p$}, defined by
\begin{equation}
\xi_p(\mu) = \left(\frac{\sum_{x\in\Zd} |x|^p G_x(\mu)}{\chi(\mu)}\right)^{1/p}.
\end{equation}
\todo{Heuristic relation bewteen $\xi$ and $\xi_p$.}

It is expected that
\begin{align}
G_x       &\sim C_1 |x|^{-(d - 2 + \eta)}, \\
\chi(\nu) &\sim C_2 (\nu - \nu_c)^{-\gamma}, \\
\xi       &\sim C_3 (\nu - \nu_c)^{-\nubar}, \\
\xi_p     &\sim C_4 (\nu - \nu_c)^{-\nubar}
\end{align}
possibly with logarithmic corrections. For walks, it is expected that
\begin{align}
c_T                       &\sim C_5 e^{-\nu_c T} T^{-\gamma}, \\
\langle |X_T|^2 \rangle   &\sim C_6 T^{-\nubar}.
\end{align}

%%%%%%%%%%%%%%%%%%%%%%%%%%%%%%%%%%%%%%%%%%%%%%%%%%%%%%%%%%%%%%%%%%%%%%%%%%%%%%%

\subsection{The DGFF}

The two-point function is just the massive Green function
$(-\Delta + m^2)^{-1}$ which, on $\Zd$, has the well-known Ornstein-Uhlenbeck decay \todo{(show this; use random walks?)}. Moreover,
\begin{equation}
\chi
  =
\sum_{x\in\vertices} (-\Delta + m^2)^{-1}_{0x}
  =
\sum_{x\in\vertices} \sum_{n=0}^\infty z^n P^n_{0x}
  =
\sum_{n=0}^\infty z^n
  =
(1 - z)^{-1}.
\end{equation}
Thus, there is a critical point at $m^2 = 0$ ($z = 1$).

\todo{See candidacy report or preliminary version of it.}

\todo{For the Green function, see Theorem 1.5.4 in Lawler--Intersections of Random Walks}

%%%%%%%%%%%%%%%%%%%%%%%%%%%%%%%%%%%%%%%%%%%%%%%%%%%%%%%%%%%%%%%%%%%%%%%%%%%%%%%

\subsection{Universality}

Critical behaviour should, roughly speaking, only depend on ``tail properties''
(global geometry, range of interaction, and symmetries).

Discuss scaling limits and the RG

%%%%%%%%%%%%%%%%%%%%%%%%%%%%%%%%%%%%%%%%%%%%%%%%%%%%%%%%%%%%%%%%%%%%%%%%%%%%%%%
%%%%%%%%%%%%%%%%%%%%%%%%%%%%%%%%%%%%%%%%%%%%%%%%%%%%%%%%%%%%%%%%%%%%%%%%%%%%%%%

\section{Main results}

%%%%%%%%%%%%%%%%%%%%%%%%%%%%%%%%%%%%%%%%%%%%%%%%%%%%%%%%%%%%%%%%%%%%%%%%%%%%%%%
%%%%%%%%%%%%%%%%%%%%%%%%%%%%%%%%%%%%%%%%%%%%%%%%%%%%%%%%%%%%%%%%%%%%%%%%%%%%%%%

\section{Relations between models}

There are a number of close relationships between models of walks and ferromagnetic spin
systems given by a Gibbs measure.

%%%%%%%%%%%%%%%%%%%%%%%%%%%%%%%%%%%%%%%%%%%%%%%%%%%%%%%%%%%%%%%%%%%%%%%%%%%%%%%

\subsection{The SRW and DGFF}

\todo{See candidacy report or preliminary version of it. See also misc notes}

For the discrete-time simple random walk, the canonical partition function $c_n$,
which is the number of such walks, is just $(2 d)^n$.

%%%%%%%%%%%%%%%%%%%%%%%%%%%%%%%%%%%%%%%%%%%%%%%%%%%%%%%%%%%%%%%%%%%%%%%%%%%%%%%

\subsection{The Kac-Siegert representation}

\todo{See notes on this}

%%%%%%%%%%%%%%%%%%%%%%%%%%%%%%%%%%%%%%%%%%%%%%%%%%%%%%%%%%%%%%%%%%%%%%%%%%%%%%%

\subsection{Self-avoiding walk representations}

\todo{High-temperature expansion of spin system (can be used later to motivate polymer
expansion) to get loop models, De Gennes $n\downarrow0$ limit, McKane/Parisi-Sourlas
and supersymmetry, Grassmann integration and BIS representation (sufficiently general
for WSAW-SA)}