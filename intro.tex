% Parts are the largest structural units, but are optional.
%\part{Thesis}

% Chapters are the next main unit.
\chapter{Introduction}

%%%%%%%%%%%%%%%%%%%%%%%%%%%%%%%%%%%%%%%%%%%%%%%%%%%%%%%%%%%%%%%%%%%%%%%%%%%%%%%
%%%%%%%%%%%%%%%%%%%%%%%%%%%%%%%%%%%%%%%%%%%%%%%%%%%%%%%%%%%%%%%%%%%%%%%%%%%%%%%

Classical thermodynamics deals with the properties of bulk matter. Thus, it seeks
to answer questions regarding the properties of matter that are only well-defined
in terms of large collections of interacting particles. A familiar example of
such a property is density, which is usually defined as the mass of a substance
per unit volume. In order for this to truly be a property of the substance under
consideration (and not of a particular sample), this substance must contain a
sufficiently large number of particles that the ratio defining the density
no longer depends (up to empirical error) on the amount of substance. This is
what is meant by bulk matter.

More generally, such properties of matter are necessarily averages of properties
of many interacting particles. This is the basic idea of statistical mechanics,
as observed by Boltzmann and Gibbs. In its modern formulation, these averages
are seen as arising from a probability measure over possible particle configurations
and different measures are understood to correspond to different \emph{phases}
of matter. In this sense, the phases of matter are a primary object of study
of statistical mechanics and it is not surprising that some of the most interesting
questions deal with phase \emph{transitions}.

There are different kinds of phase transitions. The familiar example of boiling
water is signalled by an abrupt drop in density. From an idealized mathematical
point of view, there is a discontinuity in the density as a function of temperature;
phase transitions of this kind are said to be of first order.

In this example,
we have tacitly assumed that the water is being boiled under normal atmospheric
pressure. At higher pressure, the boiling point of water increases; moreover,
the difference in density between the liquid and gaseous phases \emph{decreases}.
Once the pressure reaches a critical value $P_c$ (see Figure~\ref{fig:liquid-vapour}),
the density difference between
these two phases \emph{vanishes}; if $T_c = T_c(P)$ is the boiling point of water
at pressure $P$, then the point $(T_c, P_c)$ is known as the \emph{critical point}
for water.

% \begin{figure}
% \centering
% \label{fig:liquid-vapour}
% \includegraphics{liquid_vapour}
% \caption{The phase diagram of $\rm H_2 O$}
% \end{figure}

Since statistical mechanics deals with extremely large, complicated systems of
interacting particles, only simplified models of real materials can usually be
studied in detail. These models often serve as a good way to obtain a
\emph{qualitative} understanding of the phases and phase transitions; however,
the simplifications inherent in their definitions means they are usually not
suitable for making quantitative predictions.

A remarkable phenomenon, known as \emph{universality}, is that this is no longer
entirely true at the critical point. At criticality, many observable quantities
behave in a way that is independent of the fine details of the model under
consideration. Thus, the \emph{quantitative} properties of real materials can in
principle be predicted exactly by studying models whose only resemblance to these
materials is in their very coarse properties, such as their symmetries.

In the 1970's, Ken Wilson gave an explanation of universality in terms of an
abstract dynamical system on a space of models, known as the \emph{renormalisation
group}. Wilson won the Nobel Prize for his discovery and his method has become
a standard way amongst physicists to understand critical phenomena.

Nevertheless, a rigorous understanding of the renormalisation group is still
incomplete.

%%%%%%%%%%%%%%%%%%%%%%%%%%%%%%%%%%%%%%%%%%%%%%%%%%%%%%%%%%%%%%%%%%%%%%%%%%%%%%%
%%%%%%%%%%%%%%%%%%%%%%%%%%%%%%%%%%%%%%%%%%%%%%%%%%%%%%%%%%%%%%%%%%%%%%%%%%%%%%%

\section{Asymptotics}

Before we introduce the formal structure of statistical mechanics, we make a
few remarks concerning some of the general mathematical principles underlying
it. We begin with some notation.

\subsection{Asymptotic notation}

Let $F$ and $G$ be a functions on a subset of the real line or the integers.
We write
\begin{equation}
F(x) \sim G(x),
	\quad
x \to a
\end{equation}
if
\begin{equation}
\lim_{x\to a} \frac{F(x)}{G(x)} = 1.
\end{equation}
Here $a$ is allowed to be $\pm\infty$. We write
\begin{equation}
F(x) = O(G(x))
	\text{ or }
F(x) \le O(G(x)),
	\quad
x \to a
\end{equation}
if there is a neighbourhood $N$ of $a$ and a constant $C > 0$ such that
\begin{equation}
F(x) \le C G(x)
\end{equation}
whenever $x \in N$. Again, $a$ can be $\pm\infty$ and a neighbourhood
of $\pm\infty$ is defined in the sense of the extended real line.
Similarly, we write
\begin{equation}
F(x) = o(G(x)),
	\quad
x \to a
\end{equation}
if
\begin{equation}
\lim_{x\to a} \frac{F(x)}{G(x)} = 0.
\end{equation}

\subsection{The generating function}

The \emph{generating function} of a sequence $a_n$ is defined by
\begin{equation}
f(z) = \sum_{n=0}^\infty a_n z^n.
\end{equation}
If the function $f$ is well understood, then the coefficients $a_n$ can be
recovered via
\begin{equation}
a_n = \frac{1}{n!} f^{(n)}(0).
\end{equation}
This is known as the method of generating functions.

In many cases, $f$ cannot be computed exactly. Nevertheless, there is a close
relationship between the asymptotics of the sequence $a_n$
as $n\to\infty$ and the function $f(z)$ near its dominant singularities, i.e.\
its singularities closest to the origin.

For instance, if $a_n \sim r^{-n} n^\alpha$, then the root test implies that
the generating function $f$ has radius of convergence $r$ and it can be shown
(\todo{\REF, see Madras-Slade}) that $f(z) \sim (r - z)^{-\alpha-1}$ as
$z \uparrow r$ (\todo{what about the constant?}). This is an
example of an \emph{Abelian theorem}. The converse does not always hold; a
theorem providing conditions under which the converse is true is known as a
\emph{Tauberian theorem}, and is generally harder to prove than its Abelian
counterpart. \todo{State Tauberian theorem.}

% Let $X$ be a random variable and write $\Ex$ for expectation.
% The generating function $f_X$ of the sequence $a_n = \tfrac{1}{n!} \Ex(X^n)$ is the
% \emph{moment-generating function} of $X$. However, it is often
% more convenient to work with the \emph{cumulant generating function} $\log f_X$.
% Let $X$ be a discrete random variable in $\Z_+$ with $\Pr(X = n) = p_n$. The
% generating function $f_X$ of $p_n$ is known as the probability generating function.

\subsection{The Laplace transform}

Given a sequence $a_T$ parameterized by $T \in \R$, the generating function is
not well-defined. Instead, it is useful to consider the \emph{Laplace transform}
\begin{equation}
\label{e:laplace-transform-def}
F(\nu) = \int a_T e^{-\nu T} \; dT.
\end{equation}
Once again, there is a relationship between the asymptotics of $a_T$ as $T\to\infty$
and of $F(\nu)$ as $\nu$ approaches its nearest singularity.

\begin{theorem}
Suppose that there exist $\nu_0$ such that
the function $F(\nu)$ defined by \eqref{e:laplace-transform-def} converges for
$\nu > \nu_0$. If there exist $A$ and $\alpha \ge 0$ such that
\begin{equation}
a_T \sim \frac{A}{\Gamma(\alpha + 1)} T^\alpha,
\end{equation}
then
\begin{equation}
F(\nu) \sim A \nu_0 (\nu - \nu_0)^{-(\alpha+1)}.
\end{equation}
Moreover, the converse holds if $a_T$ is increasing.
\end{theorem}

The converse is a case of Karamata's theorem; see \cite[Chapter V]{Widd41}.

%%%%%%%%%%%%%%%%%%%%%%%%%%%%%%%%%%%%%%%%%%%%%%%%%%%%%%%%%%%%%%%%%%%%%%%%%%%%%%%
%%%%%%%%%%%%%%%%%%%%%%%%%%%%%%%%%%%%%%%%%%%%%%%%%%%%%%%%%%%%%%%%%%%%%%%%%%%%%%%

\section{Gibbs measure}

Let $(\Omega, \lambda)$ be a measure space. We view $\lambda$ as some ``natural''
measure on $\Omega$.

The dynamics of a physical system with state space $\Omega$
is often determined by a function $H$ on $\Omega$, known as the \emph{Hamiltonian}.
In statistical mechanics, the systems of concern consist of an extraordinarily
large number of particles; typically, the state space $\Omega$ is very
high-dimensional and it is infeasible to study the exact dynamics of such a
system.

A common simplifying assumption is that after a long time has passed, the system
under consideration will settle into a state of \emph{thermal equilibrium}, meaning
that there is no net flow of heat with its surroundings. In this case, the temperature
of the system is constant and we denote by $\beta$ the \emph{inverse temperature}.
The \emph{Gibbs measure} for a system with Hamiltonian $H$ at inverse temperature
$\beta$ is the probability measure on $\Omega$ given by
\begin{equation}
\label{e:gibbs-def}
\mu_\beta(d\omega) = \frac{1}{Z} e^{-\beta H(\omega)} d\lambda(\omega).
\end{equation}
The normalizing constant
\begin{equation}
Z = \int e^{-\beta H} \; d\lambda
\end{equation}
is known as the \emph{partition function}.

An elegant approach to the derivation of this measure was provided by Jaynes
\cite{Jaynes57}: amongst all probability measures $\tilde\mu$ on $\Omega$ absolutely
continuous with respect to $\lambda$, the Gibbs measure maximizes the \emph{entropy}
\begin{equation}
\label{e:entropy-def}
h(\tilde\mu) = -\int \log \frac{d\tilde\mu}{d\lambda} \; d\mu
\end{equation}
with respect to $\lambda$, subject to the fixed average energy constraint
\begin{equation}
\int H \; d\tilde\mu = E.
\end{equation}
This can be shown using the method of Lagrange multipliers; the parameter
$\beta = \beta(E)$ arises as a Lagrange multiplier.

The \emph{free energy} of the system with Hamiltonian $H$ is given by
\begin{equation}
F_\beta = -\frac{1}{\beta} \log Z_\beta.
\end{equation}
A computation of the entropy \eqref{e:entropy-def} of the Gibbs measure
$\mu_\beta$ with inverse temperature $\beta = \beta(E)$ shows that
\begin{equation}
F_\beta = E - \frac{1}{\beta} h_\lambda(\mu_\beta).
\end{equation}
This is the well-known thermodynamic relation between the free energy $F_\beta$,
internal energy $E$, temperature $1/\beta$, and entropy $h_\lambda(\mu_\beta)$.

%%%%%%%%%%%%%%%%%%%%%%%%%%%%%%%%%%%%%%%%%%%%%%%%%%%%%%%%%%%%%%%%%%%%%%%%%%%%%%%
%%%%%%%%%%%%%%%%%%%%%%%%%%%%%%%%%%%%%%%%%%%%%%%%%%%%%%%%%%%%%%%%%%%%%%%%%%%%%%%

\section{Graphs}

Graphs are a useful setting for the definitions of spatially-extended systems of
particles.

An \emph{undirected graph} or simply a \emph{graph} is a pair $\graph = (\vertices, \edges)$,
where $\vertices$
is a set of \emph{vertices} and $\edges$ is a set of
\emph{edges} $\{ x, y \}$ with $x, y \in \vertices$; we will write $x \sim y$ if
$\{ x, y \} \in \edges$.
For simplicity, we will assume that $\vertices$ is countable annd that there are no
\emph{self-loops} $\{ x \} \in \edges$.
We will also assume that $\graph$ is \emph{(vertex-)transitive}: that is, for all pairs
of distinct
vertices $a, b \in \vertices$, there exists a mapping $f : \vertices \to \vertices$
such that $x \sim y$ if and only if $f(x) \sim f(y)$.
We fix a vertex $0\in\vertices$; the assumption of transitivity implies
that the particular choice of $0$ is immaterial.

\begin{example}\mbox{}\\
\smallskip\noindent
(i) We view any set $X \subset \Zd$ as a graph with $\vertices = X$ and
$x\sim y$ if $|x - y| = 1$. In particular, $X = \Zd$ is transitive.

\smallskip\noindent
(ii) Let $L > 1$ be an integer. For $N \ge 0$, let
\begin{equation}
\Lambda_N = \Zd/L^N\Zd.
\end{equation}
We call $\Lambda_N$ the \emph{discrete $d$-dimensional torus} of side $L^N$.
We view $\Lambda_N$ as a graph with $\vertices = \Lambda_N$ and $x \sim y$
if $|x - y| = 1$ modulo $L^N$.
\end{example}

%%%%%%%%%%%%%%%%%%%%%%%%%%%%%%%%%%%%%%%%%%%%%%%%%%%%%%%%%%%%%%%%%%%%%%%%%%%%%%%

\subsection{Functions on graphs}

Let us denote the components of a function $\varphi : \vertices \to \R^n$ by
$\varphi^i_x \in \R$ for $x \in \vertices$ and $i = 1, \ldots, n$.
The Euclidean inner product and norm on the space $(\R^n)^\vertices$ of such functions
are defined by
\begin{align}
\varphi\cdot\tilde\varphi
	&= \sum_{x\in\vertices} \varphi_x \cdot \tilde\varphi_y
  		= \sum_{i=1}^n \sum_{x\in\vertices} \varphi^i_x \tilde\varphi^i_x \\
	&=|\varphi|^2 = \varphi \cdot \varphi.
\end{align}
A $\vertices\times\vertices$ matrix $M = (M_{xy})_{x,y\in\vertices}$ acts on $\varphi$ via
\begin{equation}
(M \varphi)_x = \sum_{y\in\vertices} M_{xy} \varphi_y.
\end{equation}

%%%%%%%%%%%%%%%%%%%%%%%%%%%%%%%%%%%%%%%%%%%%%%%%%%%%%%%%%%%%%%%%%%%%%%%%%%%%%%%

\subsection{The graph Laplacian}

Let us say that a $\vertices\times\vertices$ matrix $M$ is \emph{indexed by} $\edges$
if $M_{xy} = 0$ if and only if $x \not\sim y$.
Let $\jay$ be a matrix indexed by $\edges$.
Throughout this thesis, we will assume that the $\jay$ has nonnegative entries;
thus, $\jay_{xy} \ge 0$ with equality if and only if $x\not\sim y$.
The pair $(\graph, \jay)$ is an example of a \emph{weighted} graph.
We will usually denote this weighted graph simply as $\graph$, with $\jay$
implicit.

Let $\diag$ be a diagonal $\vertices\times\vertices$ matrix with diagonal entries
\begin{equation}
d_x = \diag_{xx} = \sum_{y \sim x} J_{xy}.
\end{equation}
We say that $\graph$ is $d_0$-regular if $d_x = d_y$ for all $x, y$.

The \emph{(massless) graph Laplacian} on $\graph$ is defined by
\begin{equation}
-\lap = \diag - \jay.
\end{equation}
We also define the \emph{massive Laplacian} with squared \emph{mass} $m^2 > 0$
by
\begin{equation}
-\lap + m^2.
\end{equation}
Note that
\begin{equation}
\varphi \cdot (-\lap \varphi)
  =
\frac{1}{2} \sum_{x,y\in\vertices} J_{xy} |\varphi_x - \varphi_y|^2
  \ge
0,
\end{equation}
so $-\lap$ is positive-semidefinite.

\begin{example}
An important case is when $\jay$ has $\{0, 1 \}$-valued entries.
In this case, $d_x$ is the \emph{degree} of $x$ in $\graph$ and we denote $\lap$ by
$\Delta$, which has entries given by
\begin{equation}
\label{e:Deltaxy}
-\Delta_{xy} = d_x \1_{x=y} - \1_{x \sim y}.
\end{equation}
\end{example}

%%%%%%%%%%%%%%%%%%%%%%%%%%%%%%%%%%%%%%%%%%%%%%%%%%%%%%%%%%%%%%%%%%%%%%%%%%%%%%%

\subsection{The Green function}

If $m^2 > 0$, then $-\lap + m^2$ is positive-definite, hence invertible with inverse
\begin{equation}
(-\lap + m^2)^{-1} = (m^2 + D)^{-1} \sum_{n=0}^\infty Z^n P^n,
\end{equation}
where
\begin{align}
Z = (m^2 + D)^{-1} D,
  \quad
P = D^{-1} J.
\end{align}
Let $z_x$ denote the diagonal elements of $Z$. The \emph{Green function} for
of the graph $\graph$ is the kernel of $(-\lap + m^2)^{-1}$, given by
\begin{equation}
\label{e:greendef}
C(x, y)
  =
(m^2 + d_x)^{-1} \sum_{n=0}^\infty z_x^n P^n_{xy}.
\end{equation}

%%%%%%%%%%%%%%%%%%%%%%%%%%%%%%%%%%%%%%%%%%%%%%%%%%%%%%%%%%%%%%%%%%%%%%%%%%%%%%%
%%%%%%%%%%%%%%%%%%%%%%%%%%%%%%%%%%%%%%%%%%%%%%%%%%%%%%%%%%%%%%%%%%%%%%%%%%%%%%%

\section{Spin systems}

\subsection{The Ising model}

Suppose that $|\vertices| < \infty$.
The \emph{Ising model} on $\graph$ is defined by the Gibbs measure on $\Omega = \{ \pm 1 \}^\vertices$
with Hamiltonian
\begin{equation}
H_h(\sigma)
	=
-\frac{1}{2} \sum_{x \sim y} \sigma_x \sigma_y - h \sum_{x\in\vertices} \sigma_x,
\end{equation}
where $h \in \R$ is known as the \emph{external field}.
The first term encourages spins to \emph{align}: an edge $\{ x, y \}$ makes
a negative contribution to the total energy (so a positive contribution to
the probability of a configuration) when $\sigma_x = \sigma_y$.
For a similar reason, the second term encourages spins to adopt the same sign as $h$.

Let $\langle\cdot\rangle_{\beta,h}$ denote the expectation with respect to
this Gibbs measure and let $Z_{\beta,h}$ be the partition function.
The \emph{magnetization} is defined by
\begin{equation}
\langle \sigma_0 \rangle_{\beta,h} = \frac{1}{Z_\beta} \sum_{\sigma\in\Omega} \sigma_0 e^{-\beta H(\sigma)}.
\end{equation}
When $\beta > 0$, it is reasonable to expect that the magnetization has the
same sign as $h$.

To see this, define the \emph{magnetic susceptibility} by
\begin{equation}
\chi(\beta, h)
	=
\frac{1}{\beta} \dd{}{h} \langle \sigma_0 \rangle_{\beta,h}.
\end{equation}
A computation shows that
\begin{equation}
\chi(\beta, h) = \sum_{x\in\vertices} G_x(\beta, h),
\end{equation}
where
\begin{equation}
G_x(\beta, h)
	=
\Big(
	\langle \sigma_0 \sigma_x \rangle_{\beta,h}
		-
	\langle \sigma_0 \rangle_{\beta,h} \langle \sigma_x \rangle_{\beta,h}
\Big)
\end{equation}
is the \emph{two-point function}.
Thus, the susceptibility is positive and so the magnetization is increasing
in $h$. When $h = 0$, the Gibbs measure is invariant under the spin flip
$\sigma \mapsto -\sigma$, and so the magnetization is $0$. It follows that
$\langle \sigma_0 \rangle_{\beta,h} > 0$ if and only if $h > 0$.

%%%%%%%%%%%%%%%%%%%%%%%%%%%%%%%%%%%%%%%%%%%%%%%%%%%%%%%%%%%%%%%%%%%%%%%%%%%%%%%

\subsection{Spin systems on finite graphs}

We continue to assume that $|\vertices| < \infty$.
An $n$-component \emph{field} or \emph{spin configuration} on $\vertices$
with spins in $S \subset \R^n$ is an element of $\Omega = S^\vertices$.
% A \emph{spin system} is a probability measure $d\mu$ on $\Omega$.
Suppose that $S$ is equipped with a measure $d\lambda^0$.
Given a Hamiltonian $H : \Omega \to \R$, we define the measure
\begin{equation}
d\mu_\beta(\varphi)
  =
\frac{1}{Z_\beta} e^{-\beta H(\varphi)} d\lambda(\varphi)
\end{equation}
on $\Omega$, where
\begin{equation}
d\lambda(\varphi) = \prod_{x\in\vertices} d\lambda^0(\varphi_x).
\end{equation}
We denote the expectation with respect to this measure by $\langle\cdot\rangle_\beta$.
% However, there are some problems with this definition when $\vertices$ is infinite.
% For one, $d\lambda$ may not be well-defined (for instance if
% $S = \R$ and $d\lambda^0$ is Lebesgue measure). Another issue is that it may be
% difficult to define a reasonable choice of $H$ on the infinite product space
% $\Omega$.
% 
% For this reason, we temporarily restrict our attention to finite graphs:
% \begin{equation}
% |\vertices| < \infty.
% \end{equation}
% Then the Gibbs measure $\mu = \mu_\beta$ is well-defined and we will denote the
% expectation with respect to this measure by $\langle \cdot \rangle_\mu$.

Following our discussion of the Ising model, we define the two-point function
and susceptibility by
\begin{equation}
G_x(\beta)
  =
\frac{1}{n}
\big(\langle \varphi_0 \cdot \varphi_x \rangle_\beta
  -
\langle \varphi_0 \rangle_\beta \cdot \langle \varphi_x \rangle_\beta\big)
\end{equation}
and
\begin{equation}
\chi(\beta) = \sum_x G_x(\beta).
\end{equation}

We will mainly be concerned with \emph{ferromagnetic} spin systems, for which the
Hamiltonian has the form
\begin{equation}
H(\varphi) = -\varphi \cdot M\varphi,
\end{equation}
where $M_{xy} \ge 0$.
% Thus, $H(\varphi)$ is smaller (and $d\mu_\beta(\varphi)$ larger) when the spins align
% (when $\varphi_x = \varphi_y$ for $x \sim y$).

%%%%%%%%%%%%%%%%%%%%%%%%%%%%%%%%%%%%%%%%%%%%%%%%%%%%%%%%%%%%%%%%%%%%%%%%%%%%%%%

% \subsection{The \texorpdfstring{$O(n)$}{O(n)} spin model}

\begin{example}[The $O(n)$ model]
Let $S = S^{n-1} \subset \R^n$ be the unit $(n-1)$-sphere equipped with the
normalized sphere measure $d\lambda^0$ (in particular, $S^0 = \{ \pm 1 \}$).
The \emph{$O(n)$ spin model} or \emph{$n$-vector model} is the ferromagnetic
spin system with Hamiltonian
\begin{equation}
H_J(\sigma) = -\frac{1}{2} \sigma \cdot J \sigma,
\end{equation}
which is clearly ferromagnetic. When $n = 1$ we recover the Ising model (with
$0$ external field). When $n = 2, 3$, we get the \emph{XY model} and the
\emph{classical Heisenberg model}.
\end{example}

%%%%%%%%%%%%%%%%%%%%%%%%%%%%%%%%%%%%%%%%%%%%%%%%%%%%%%%%%%%%%%%%%%%%%%%%%%%%%%%

% \subsection{The \texorpdfstring{$|\varphi|^4$}{phi4} spin model}

\begin{example}[The $|\varphi|^4$ model]
Let $S = \R^n$. The $|\varphi|^4$ spin model on $\graph$ is the ferromagnetic spin system
defined by the quartic Hamiltonian
\begin{equation}
\label{e:phi4-Ham}
H_{\gcc,\nu}(\varphi)
  =
\sum_{x\in\vertices}
\left(
  \frac{1}{4} \gcc |\varphi_x|^4
    +
  \frac{1}{2} \nu |\varphi_x|^2
    +
  \frac{1}{2} \varphi_x \cdot (-\lap \varphi)_x
\right),
\end{equation}
where $\gcc > 0$ and $\nu\in\R$. Adjusting $\beta$ is equivalent to  rescaling
$\gcc$ and $\nu$, so we set $\beta = 1$ without loss of generality.
% This is a natural generalization of the $O(n)$ model
% in which spins are merely concentrated near a sphere.
When $\graph$ is $d_0$-regular, we can write
\begin{equation}
H_{\gcc,\nu}(\varphi)
  =
\sum_{x\in\vertices} U_{\gcc,\nu}(\varphi_x) - \frac{1}{2} \varphi \cdot \jay \varphi,
\end{equation}
where the \emph{single-spin potential} $U_{\gcc,\nu}$ is defined by
\begin{equation}
U_{\gcc,\nu}(t)
	=
\frac{1}{4} \gcc |t|^4
	+
\frac{1}{2} (\nu + d_0) |t|^2,
	\quad
t \in \R^n.
\end{equation}
\todo{Attach graph.}
We see from this expression that this model is ferromagnetic.
When $\nu + d_0 < 0$, the potential has roots at $0$ and $\pm\sqrt{-2 (\nu + d_0) / \gcc}$.
It follows that the Gibbs measure for $|\varphi|^4$ model converges weakly to the
Gibbs measure for the $O(n)$ model in the limit $\gcc\to\infty$
with $\nu = -(d_0 + \gcc / 2)$.
\end{example}

%%%%%%%%%%%%%%%%%%%%%%%%%%%%%%%%%%%%%%%%%%%%%%%%%%%%%%%%%%%%%%%%%%%%%%%%%%%%%%%

% \subsection{Gaussian measures and the free field}

\begin{example}[The Gaussian free field]
Let $S = \R^n$ and let $C$ be a positive-definite symmetric $\vertices\times\vertices$
matrix. The $n$-component \emph{Gaussian measure} $d\mu_C$ on
$\Omega$ with mean $0$ and \emph{covariance} $C$ is defined by the Hamiltonian
\begin{equation}
H_C(\varphi) = \frac{1}{2} \varphi \cdot C^{-1} \varphi.
\end{equation}
Setting $\beta = 1$ again, the partition function $Z_C$ can be computed explicitly
and the Gibbs measure takes the form
\begin{equation}
\label{e:gauss-density}
\frac{d\varphi}{\sqrt{\det(2\pi C)}}
e^{-\tfrac12 \varphi \cdot A \varphi}.
\end{equation}
\emph{Wick's theorem} gives an expression for the correlations. When $n = 1$,
if $x_1, \ldots, x_{2p} \in \Lambda$, then
% \emph{Wick's theorem}). In particular, the two-point function is the covariance:
\begin{equation}
\label{e:wick}
% \int \varphi_a \cdot \varphi_b \; d\mu(\varphi) = C_{ab}.
\int d\mu_C(\varphi) \prod_{i=1}^{2p} \varphi_{x_i}
	=
\sum_\pi \prod_{ij\in\pi} C_{x_ix_j}
\end{equation}
where the sum is over all pairings $\pi$ of $\{1,\ldots,2p\}$.
Correlations for $n > 1$ can then be computed using the fact that we have defined
Gaussian fields to have independent Gaussian components.

An important case is the \emph{massive Gaussian free field} on $\graph$,
which is the $\gcc = 0$ case of the $|\varphi|^4$ model (with $\nu$ necessarily positive).
Thus, the the covariance is equal to the massive Green function $C = (-\lap + \nu)^{-1}$.
In particular, the only non-zero contributions to the sum in \eqref{e:wick}
come from pairings $\pi$ consisting of edges in $X$.
\end{example}

%%%%%%%%%%%%%%%%%%%%%%%%%%%%%%%%%%%%%%%%%%%%%%%%%%%%%%%%%%%%%%%%%%%%%%%%%%%%%%%

\subsection{Spin systems in infinite volume}

The presence of a phase transition in a physical system is signalled by an abrupt
change in an observable quantity with respect to a parameter;
We make a broad distinction between
\emph{first-order} phase transitions in which the free energy has discontinuous first
derivative (with respect to an external field $h$) and \emph{continuous} phase transitions,
in which the free energy is differentiable but non-analytic.

The systems we have defined above all
have smooth free energy since the Hamiltonians are smooth functions. The reason we cannot
detect a phase transition in these systems is that they have been defined on finite volumes.
This is also the reason that, as we have observered in \REF, the magnetization of the Ising
model in a finite volume is $0$ in the absence of a magnetic field, despite physical evidence
that a ferromagnet can sustain a non-zero magnetization on its own (the magnetization can
be expressed as a derivative of the free energy).
Thus, in order to study phase transitions, we are forced to consider
spin systems on infinite graphs.

A systematic approach to this problem was developed in the work of Dobrushin \cite{Dobrushin68}
and Lanford and Ruelle \cite{LR69}. A comprehensive reference to this subject is \cite{Georgii11}
(see also \cite{LP76} for unbounded spin systems).
Loosely speaking, they take as fundamental not the Hamiltonian
but rather a ``potential'', which is a collection of functions encoding the microscopic interactions
from which the Hamiltonian is to be defined; for instance, for the Ising model,
the Hamiltonian is a sum of contributions of the form $J_{xy} \sigma_x \sigma_y$ whenever
$x \sim y$. Given such a potential, a Hamiltonian can be defined on any
finite subgraph of $\graph$ and a probability measure on $\Omega$ is said to be a Gibbs
measure whenever its finite-volume conditional measures are of the form \eqref{e:gibbs}.
This is somewhat in the spirit of Kolmogorov's consistency conditions with the importance
difference that the resulting collection $\gibbs_\beta$ of Gibbs states at inverse temperature
$\beta$ need not consist of only a single element. This is significant due to the interpretation
of distinct elements of $\gibbs_\beta$ as corresponding to different phases of the system under
consideration.

For many systems of interest there is a critical inverse temperature $\beta_c$ such that
$|\gibbs_\beta| > 1$ if and only if $\beta > \beta_c$. The region $\beta > \beta_c$ is typically
associated with first-order phase transitions whereas continuous phase transitions usually occur
at the critical point $\beta_c$.
In this thesis, our interest lies in the behaviour of spin systems at $\beta = \beta_c$ and in the
high-temperature approach to criticality given by $\beta \uparrow \beta_c$.
Thus, we need not concern ourselves with subtleties concerning the approach to infinite volume.

In fact, we will avoid the entire issue of existence and uniqueness of infinite-volume Gibbs
measures (which is not well understood for multi-component spin systems with unbounded spins
such as the $n$-component $|\varphi|^4$ model) by defining the observable quantities in infinite
volume as limits of their finite-volume counterparts.

% A natural approach to defining such systems is a procedure known as the
% \emph{infinite-volume limit}, which we describe here. For any
% finite subgraph $\Lambda \subset \graph$, let $H_\Lambda$ be the Hamiltonian
% of one of the above spin systems on $\Lambda$ and let $\mu_\Lambda$ be the
% corresponding Gibbs measure (which we can view as a measure on the full state
% space $\Omega = S^\vertices$).
% Now let $\Lambda_N \subset \graph$ be a sequence of subgraphs that exhaust
% $\graph$, i.e.\ $\Lambda_N \uparrow \graph$. If the limits
% \begin{equation}
% \lim_{N\to\infty} \int f \; d\mu_{\Lambda_N}
% \end{equation}
% exist for a sufficiently rich class of functions $f$ (such as all bounded continuous functions), then they define a measure $\mu$ on $\Omega$ (with $\mu(f)$ the above limit), which we call a \emph{Gibbs state} or
% \emph{infinite-volume Gibbs measure} on $\Omega$.

% We remark that there is a more general approach to the study of spin systems in infinite volume developed by Dobrushin, Lanford, and Ruelle. We do not detail their approach here, but merely mention that, for the examples above, this approach involves defining a Gibbs measure for the collection
% $H = (H_\Lambda)$ of Hamiltonians directly as a measure on $\Omega$ satisfying a system of constraints on its conditionals measures. This is somewhat in the spirit of Kolmogorov's consistency conditions with the importance difference that the resulting collection $\gibbs_\beta(H)$ of Gibbs states at inverse temperature
% $\beta$ need not consist of only a single element. This is significant due to the interpretation of distinct elements of $\gibbs(H)$ as corresponding to different phases of the system under consideration.

% For many models, including the $O(n)$ and $n$-component $|\varphi|^4$ models on $\Zd$ with $d > 2$, it is known that there exists a critical inverse temperature $\beta_c < \infty$ such that
% $|\gibbs_\beta(H)| = 1$ if and only if $\beta \le \beta_c$ (\REF). In this thesis, our main concern is with the behaviour at $\beta_c$ and as
% $\beta \uparrow \beta_c$. Thus, we need not concern ourselves with the precise nature of the infinite-volume limit.

\subsubsection{Translation-invariant systems}

Our interest will be in translation-invariant systems on $\Zd$, for which a particular
approach to infinite volume is especially convenient. We begin by letting
\begin{equation}
\Lambda_N = \Zd/L^N\Zd
\end{equation}
be the discrete torus of side $L^N$, where $L \ge 2$. is an integer. We view $\Lambda_N$
as a subset of $\Zd$ approximately centered at the origin (say as
$[-\frac12 L^N+1,\frac12 L^N]^d \cap \Z^d$ if $L^N$ is even
and as $[-\frac12 (L^N-1), \frac12 (L^N-1)]^d \cap \Z^d$ if $L^N$ is odd). This allows us to preserve
translation-invariance of the models that concern us when defining them in finite volume.

In particular, we study a generalization of the $|\varphi|^4$ model whose Hamiltonian
on $\Lambda_N$ is given by
\begin{equation}
\label{e:Vdef1}
V_{\gcc,\gamma,\nu,N}(\varphi)
	=
\sum_{x\in\Lambda_N}
\Big(
	\tfrac{1}{4} (\gcc - \gamma) |\varphi_x|^4
		+
	\tfrac{1}{2} \nu |\varphi_x|^2
		+
	\tfrac{1}{2} \varphi_x \cdot (-\Delta \varphi)_x
		+
	\tfrac{1}{4 d} \gamma (\nabla |\phi_x|^2)^2
\Big),
\end{equation}
where
\begin{equation}
(\nabla |\phi_x|^2)^2
	=
\sum_{|e|=1} (\nabla^e |\phi_x|^2)^2.
\end{equation}
Note that we recover \eqref{e:phi4-Ham} when $\gamma = 0$ and $\graph = \Lambda_N$.
The expectation with respect to the corresponding Gibbs measure will
be denoted $\langle\cdot\rangle_{\gcc,\gamma,\nu,N}$.
We define the infinite-volume two-point function for this model by
\begin{equation}
\label{e:two-point-function-phi4}
G_{x, N}(g,\gamma,\nu; n)
	=
\frac{1}{n} \pair{\varphi_0 \cdot \varphi_x}_{g,\gamma,\nu,N},
	\quad
G_x(g,\gamma,\nu; n)
	=
\lim_{N \to \infty} G_{x, N}(g,\gamma,\nu; n).
\end{equation}
The susceptibility is defined by
\begin{equation}
\label{e:susceptibility-def}
\chi(\gcc, \gamma, \nu; n)
	=
\lim_{N\to\infty} \sum_{x\in\Lambda_N} G_{x,N}(\gcc, \gamma, \nu; n).
\end{equation}

%%%%%%%%%%%%%%%%%%%%%%%%%%%%%%%%%%%%%%%%%%%%%%%%%%%%%%%%%%%%%%%%%%%%%%%%%%%%%%%
%%%%%%%%%%%%%%%%%%%%%%%%%%%%%%%%%%%%%%%%%%%%%%%%%%%%%%%%%%%%%%%%%%%%%%%%%%%%%%%

\section{Critical behaviour and universality}

Many systems exhibit singular behaviour at or near the critical temperature in
the form of power law scaling of various observable quantities. This is known as
\emph{critical behaviour}. For concreteness, let us discuss the $|\varphi|^4$ model
on $\Zd$ with $d > 1$ and with $J_{xy} \in \{ 0, 1 \}$. Thus, $\lap = \Delta$.

\subsection{The critical point}

Since adjust $\beta$ for the $|\varphi|^4$ model is equivalent to rescaling the
parameters $g$ and $\nu$, we fix $\beta = 1$ and vary $\nu$ instead.
The \emph{critical point} is defined by
\begin{equation}
\nu_c = \nu_c(\gcc, \gamma, n) = \inf \{ \nu : \chi(g, \gamma, \nu; n) < \infty \}.
\end{equation}
Since the susceptibility can be written as the second derivative of the free energy
with respect to an external field, there is a continuous phase transition at $\nu_c$.
By \eqref{e:susceptibility-def}, it is
reasonable to expect rapid (i.e.\ summable) decay of $G_x(g, \gamma, \nu; n)$ in $|x|$ for
$\nu > \nu_c$ and much slower decay at $\nu = \nu_c$. In fact, the two-point function
is expected to decay exponentially above $\nu_c$ and sub-exponentially at $\nu_c$.

The \emph{correlation length} $\xi$ is defined to be the reciprocal of the exponential
rate of decay of the two-point function; concretely, we let
\begin{equation}
\xi(g, \gamma, \nu; n) = \limsup_{k\to\infty} \frac{-k}{\log G_{ke}(g, \gamma, \nu; n)},
\end{equation}
where $e \in \Zd$ is a unit vector. Roughly speaking, the correlation length acts as
a ``macroscopic length scale'' of the model; it is a measure of the largest scale at
which spins are strongly correlated.
Based on the above discussion, we expect $\xi$
to diverge as $\nu\downarrow\nu_c$. This divergence is one of the
key features of critical behaviour and is indicative of strong correlations at all
scales. A related quantity is the \emph{correlation length of order $p$}, defined by
\begin{equation}
\label{e:clp-spins}
\xi_p(g, \gamma, \nu; n)
	=
\left(\frac{\sum_{x\in\Zd} |x|^p G_x(g, \gamma, \nu; n)}{\chi(g, \gamma, \nu; n)}\right)^{1/p}.
\end{equation}

\begin{rk}
There is a simple heuristic relationship between $\xi$ and $\xi_p$. Suppose that
the two-point function decays exponentially at rate $1/\xi$, possibly with some
sub-exponential multiplicative correction; for instance, suppose that
\begin{equation}
G_x(g, \gamma, \nu; n) \approx C |x|^{-\alpha} e^{-|x|/\xi(g, \gamma, \nu; n)},
\end{equation}
in some sense, where $\alpha$ and $C$ are positive constants independent of $\nu$.
Then the main contributions to the numerator of
\eqref{e:clp-spins} should come from $|x| \le \xi = \xi(g, \gamma, \nu; n)$. For such $|x|$,
$G_x(g, \gamma, \nu; n) \approx C |x|^{-\alpha}$ and so
\begin{equation}
\sum_{x\in\Zd} |x|^p G_x(g, \gamma, \nu; n)
	\approx
C \sum_{|x| \le \xi} |x|^{-(\alpha-p)}
	\approx
C \xi(g, \gamma, \nu; n)^{-(\alpha-p)}.
\end{equation}
It follows then from the definition that
\begin{equation}
\xi^p_p(g, \gamma, \nu; n) \approx \xi^p(g, \gamma, \nu; n).
\end{equation}
\end{rk}

\subsection{Critical exponents}

For simplicity, let us drop $\gcc$, $\gamma$, and $n$ from the notation.
It is predicted that there exist constants $\eta$, $\gammabar$, and $\nubar$
(unrelated to $\gamma$ and $\nu$), known as \emph{critical exponents}, such that
\begin{align}
G_x(\nu_c)
	&\sim
C_1 |x|^{-(d - 2 + \eta)},
	\qquad
|x|\to\infty \\
\chi(\nu)
	&\sim
C_2 (\nu - \nu_c)^{-\gammabar},
	\qquad
\nu\downarrow\nu_c \\
\xi(\nu)
	&\sim
C_3 (\nu - \nu_c)^{-\nubar},
	\qquad
\nu\downarrow\nu_c \\
\xi_p(\nu)
	&\sim
C_4 (\nu - \nu_c)^{-\nubar},
	\qquad
\nu\downarrow\nu_c.
\end{align}
where $a \sim b$ means that $\lim (a/b) = 1$ and $C_i$ for $i = 1,2,3,4$
are constants that may depend on $g$ and $n$ (and $p$ when $i = 4$).
The critical exponents are expected to be \emph{universal} in the sense that they
only depend on ``large-scale properties'' of the model such as its symmetries and
the global geometry of the underlying graph. In particular,
for the $n$-component $|\varphi|^4$ model on $\Zd$, these exponents should only
depend on $n$ and $d$ and be independent of $g$ and $\gamma$ when $g > 0$ and $\gamma$
is sufficiently small (depending on $g$). In fact,
analogous relations are expected to hold for the $O(n)$ spin model, with the
\emph{same} critical exponents.

These and other relations are all believed to be manifestations of the existence of
a universal \emph{scaling limit} for the $|\varphi|^4$ model and other models in its
\emph{universality class}. That is, any spin system in this class, when appropriately
rescaled, is expected to converge in distribution to a unique continuum random field.
\todo{For instance, the analogue of the conjectured scaling relation for the critical
two-point function of such a scaling limit would follow from scale-invariance of this
limit.}
In this sense, the study of critical behaviour involves a set of far-reaching
generalizations of the central limit theorem.

\begin{example}[The Gaussian free field]
\label{ex:gff-asymp}
On $\Zd$, \eqref{e:greendef} and \eqref{e:wick} imply that
\begin{equation}
G_x(0, 0, m^2; n)
	=
(m^2 + 2 d)^{-1} \sum_{n=0}^\infty z^n P^n_{xy},
\end{equation}
where $z = 2 d / (m^2 + 2 d)$ and $P = (2 d)^{-1} J$
% \begin{equation}
% -\Delta = 2 d (1 - P),
% \end{equation}
is the transition matrix for the simple random walk $X$ on $\Zd$.
Thus, for $m^2 > 0$,
\begin{equation}
\chi(0, 0, m^2; n)
  % =
% \sum_{x\in\vertices} (-\Delta + m^2)^{-1}_{0x}
  =
\sum_{n=0}^\infty z^n \sum_{x\in\Zd} P^n_{0x}
  =
\sum_{n=0}^\infty z^n
  =
(1 - z)^{-1}.
\end{equation}
This diverges linearly at $z = 1$ (corresponding to $m^2 = 0$),
so $\nu_c(0, 0; n) = 0$ and $\gammabar = 1$.

For $d > 2$, the simple random walk is transient. Thus, letting $E_x$ denote the expectation
of $X$ conditioned so that $X_0 = x$,
\begin{equation}
\sum_{n=0}^\infty P^n_{xy} = E_x \sum_{n=0}^\infty \1_{X_n=y} < \infty
\end{equation}
and we can define the \emph{massless} Gaussian free field on $\Zd$ to be the Gaussian
measure with covariance $(-\Delta)^{-1}$. It is a well-known fact that
\begin{equation}
(-\Delta + m^2)^{-1}_{0x} \sim C |x|^{-(d-1)/2} e^{-m |x|}
\end{equation}
and
\begin{equation}
(-\Delta)^{-1}_{0x} \sim C' |x|^{-(d-2)}
\end{equation}
corresponding to $\gcc = \gamma = 0$ critical exponents of $\nubar = 1/2$
and $\eta = 0$, respectively.
% Now if $X_0 = 0$, then $X_n = \sum_{i=1} Y_i$,
% where $Y_i, i \ge 1$ is a sequence of independent random vectors uniformly distributed
% over the set $\Ucal$ of unit vectors in $\Zd$. The central limit theorem states that
% \begin{equation}
% \label{e:clt}
% \frac{1}{\sqrt n} X_n \Rightarrow N,
% \end{equation}
% where $N$ is a normal random variable in $\Rd$. Thus, $P^n_{0x}$ is well-approximated
% \todo{See candidacy report or preliminary version of it. For the Green function, see Theorem 1.5.4 in Lawler--Intersections of Random Walks}
\end{example}

%%%%%%%%%%%%%%%%%%%%%%%%%%%%%%%%%%%%%%%%%%%%%%%%%%%%%%%%%%%%%%%%%%%%%%%%%%%%%%%

\subsection{The effect of dimension}

\subsubsection{Dimensions $d = 2, 3$}

In $d = 2$, the $O(n)$ model and $|\varphi|^4$ model are not expected to possess
a phase transition for $n > 2$. For $n = 2$, the behaviour is quite subtle so
we restrict our attention to $n = 1$.

The $1$-component planar models are
expected to possess \emph{conformally invariant} scaling limits. Recent years
have shown rapid progress in this direction, stimulated
by the identification by Schramm \cite{Schramm00} of a $1$-parameter family of
conformally invariant random planar curves now known as the \emph{Schramm-Loewner
evolution} or \emph{SLE} with parameter $\kappa$. \todo{It was shown in \cite{CDHKS14}
that the interfaces of the Ising model converge to SLE with $\kappa = 3$.}
As a consequence of the conformal symmetries enjoyed by the scaling limits of planar
models, their critical exponents are expected to take on rational; in particular,
it is predicted that $\gammabar = 56/32$, $\nubar = 1$, and $\eta = 1/4$.

% The critical exponents for the $|\varphi|^4$ model (with $g > 0$ and $\gamma$ small)
% and the $O(n)$ spin model are conjectured to take on the following values:
% \begin{equation}
% \eta =
% 	\begin{cases}
% 	\frac{5 + n}{24},		& d = 2, \, n = 1 \\
% 	\approx 0.03,			& d = 3, \\
% 	0,						& d \ge 4
% 	\end{cases}
% \qquad
% \gamma =
% 	\begin{cases}
% 	\frac{43 + 13 n}{32},	& d = 2, \, n = 1 \\
% 	\approx 1,				& d = 3 \\
% 	1,						& d \ge 4
% 	\end{cases}
% \end{equation}
% and
% \begin{equation}
% \nubar =
% 	\begin{cases}
% 	\frac{3 + n}{4},		& d = 2, \, n = 1 \\
% 	\approx 0.6,			& d = 3 \\
% 	\frac{1}{2},			& d \ge 4
% 	\end{cases}
% \end{equation}
% with logarithmic corrections in $d = 4$.
% Thus, it is expected that, when $d > 4$, the critical exponents cease to depend
% on the dimension and $n$. In fact, they are expected to equal the exponents of
% the corresponding non-interacting model, the Gaussian free field.

Three-dimensional models are very poorly understood and only numerical approximations
are available for their critical exponents.

\subsubsection{Dimensions $d > 4$}

If $d > 4$, the critical exponents for the $O(n)$ and $|\varphi|^4$ models are
predicted to become independent of $d$ and $n$ and to take on the values of the
corresponding exponents for the Gaussian free field, i.e.\ $\gammabar = 1$,
$\nubar = 1/2$, and $\eta = 0$. For $n = 1$ and $2$ it is known that $\eta = 0$
for the \emph{continuum limit} of these models \cite{Aiz82,Fro82}.
This phenomenon is known as \emph{mean-field behaviour} and dimension $4$ is
called the \emph{upper-critical dimension} for this class of models.

\subsubsection{Dimension $d = 4$}

In this thesis, we focus on the behaviour in the upper-critical dimension $d = 4$.
It is predicted in dimension $4$ that a number of scales scale according to a
power law with mean-field critical exponents and multiplicative logarithmic
corrections. The behaviour of the $1$-component $|\varphi|^4$ model has been
studied in dimension $4$ by Feldman, Magnen, Rivasseau, and S\'{e}n\'{e}or
\cite{FMRS87} as well as Gawedzki and Kupiainen \cite{GK85}, who showed
that $\eta = 0$. Using the latter method, Hara and Tasaki \cite{HT87} identified
logarithmic corrections to scaling.

%%%%%%%%%%%%%%%%%%%%%%%%%%%%%%%%%%%%%%%%%%%%%%%%%%%%%%%%%%%%%%%%%%%%%%%%%%%%%%%

\subsection{The renormalisation group}
\label{sec:rg-intro}

In \cite{Kada66}, Leo Kadanoff considered a coarse-graining procedure for
studying the Ising model in which disjoint
blocks in $\Zd$ of side $L \ll \xi$ are replaced by single spins. He argued
that spins inside such blocks are so strongly correlated that the model
obtained by making this replacement should behave approximately like an Ising
model with new (renormalized) couplings.

At the critical point, $\xi = \infty$ this transformation
$T$ can be iterated indefinitely resulting in a dynamical system on a space
of models: the one-parameter semigroup $(T^j)_{j\in\Z_+}$ is known as the
\emph{renormalisation group}\footnote{The name
renormalisation \emph{group} is attributed by Wilson in \cite{Wils71I}
to the work of Gell-Mann and Low \cite{GML54}.}. This was basis for Ken Wilson's
generalizations of Wilson's idea in \cite{Wils71I,Wils71II}.

The coarse-graining procedure of Kadanoff can be viewed as an approximate
method for computing integrals with respect to a Boltzmann weight $e^{-\beta H}$
by successively integrating out fluctuations that are ``small'' in the sense
that they are spatially localized. An alternative approach, suggested by
Wilson, is to successively integrate out fluctuations that are localized in
momentum (i.e.\ Fourier) space. \todo{Explain this further. Introduce momentum
space covariance decomposition. Define effective theory.}

Although Wilson did not follow the same coarse-graining procedure as Kadanoff,
his method nevertheless resulted in a dynamical system as above on a space of
models that acts by appropriately integrating out small fluctuations. At the
critical, he argued that this should not affect the long-range behaviour of
the model. Consequently, critical models lying in the same orbit should have
the same universality class. That is, they should possess the same critical
exponents and scaling limit; moreover, this scaling limit, which results from
integrating out all microscopic fluctuations, should arise as a fixed point
of the renormalisation group transformation.

In addition to these rather broad statements regarding the nature of universality
and scaling limits, Wilson demonstrated that critical exponents could be computed
by a careful analysis of the asymptotics of the renormalisation group near its
fixed points. He claimed that such an analysis could be performed by approximating
this infinite-dimensional dynamical system by a \emph{finite-dimensonal} system.

Gaussian behaviour above the upper-critical dimension was explained by the
existence of a unique hyperbolic fixed point in dimensions $4$ and up: the Gaussian
fixed point. In dimension $4$, a bifurcation occurs: when $d = 4$, the Gaussian
fixed point ceases to be hyperbolic and as $d$ is lowered, the fixed point splits
into a stable fixed point and an unstable one (this is an example of a saddle-node
bifurcation). The unstable fixed point corresponds to Gaussian behaviour (this is
why the critical exponents are different for $g = 0$ and $g > 0$ when $d < 4$).

\begin{example}
CLT: Koralov-Sinai, Li-Sinai. See also Jona-Lasinio I think.
\end{example}

%%%%%%%%%%%%%%%%%%%%%%%%%%%%%%%%%%%%%%%%%%%%%%%%%%%%%%%%%%%%%%%%%%%%%%%%%%%%%%%
%%%%%%%%%%%%%%%%%%%%%%%%%%%%%%%%%%%%%%%%%%%%%%%%%%%%%%%%%%%%%%%%%%%%%%%%%%%%%%%

\section{Walks}

\subsection{Discrete-time walks}

Let $[n] = \{0,\ldots,n\}$ and call a function $\omega : [n]\to\vertices$
a \emph{discrete-time walk of length $n$} if $\omega_i \sim \omega_{i+1}$ for all
$i\in[n-1]$.
A model of discrete-time walks can be defined by an assigning of weights
$w_n : \dwalks_n \to \R$ for each $n \in \Z_+$, where $\Wcal_n$ is the
collection of discrete-time walks $\omega$ of length $n$ with $\omega_0 = 0$.
When these weights are nonnegative and summable, this is equivalent to
equipping each $\Wcal_n$ with a probability measure; such \emph{probabilistic}
models of walks are often closely related to ferromagnetic spin systems,
as we will discuss in Section~\REF.

In this thesis, we focus on continuous-time walks, which we define in the next
section. Before proceeding, however, we discuss the following important example.

\begin{example}[Self-avoiding walk]
A discrete-time walk $\omega$ is said to be \emph{self-avoiding} if
$\omega_i \ne \omega_j$
for all $i \ne j$. Let $\Scal_n \subset \dwalks_n$ denote the set of $n$-step
self-avoiding walks starting at $0$.
The \emph{self-avoiding walk} is a model of a linear polymer introduced by
Flory \REF and defined by the weights $w(\omega) = \1_{\omega\in\Scal_n}$.

Equivalently, we equip $\Scal_n$ with the uniform measure $\mu_n$ for each $n$.
These measures do not form a consistent family due to the possibility of ``traps''. That is, the equality
\begin{equation}
\mu_{|\omega|}(\omega) = \sum_{\tilde\omega \supset \omega} \mu_{|\tilde\omega|}(\tilde\omega)
\end{equation}
does not hold for all $\omega\in\dwalks$ (the sum here is over all self-avoiding walks extending $\omega$).
For instance, consider the self-avoiding walk $\omega\in\dwalks_7$
on $\Zd$ in Figure~\REF. % \ref{fig:trap}.
This walk has positive probability under $\mu_7$ but,
since there are no self-avoiding walks extending $\omega$, the sum on the 
right-hand side above is $0$.

As a result, there is no straightforward way to apply the methods of stochastic processes
to study the self-avoiding walk. The existence of traps also contributes to the
combinatorial difficulty of studying self-avoiding walk; for instance, it is
not clear how to express $c_{n+1}$ (the number of $(n+1)$-step self-avoiding walks)
in terms of $c_n$.

Nevertheless, we know that the sequence $c_n$ is sub-multiplicative: $c_{m+n} \le c_m c_n$.
Thus, Fekete's lemma \REF implies that the existence of the \emph{connective constant}
$c(\graph)$ of $\graph$, defined by
\begin{equation}
c(\graph) = \lim_{n\to\infty} n^{-1} \log c_n.
\end{equation}
Note that, by the trivial bounds $1 \le c_n \le d_0^n$ for $n \ge 1$, $c(\graph) \in [0, d_0]$.
% Roughly speaking, this means that $c_n \approx c(\graph)^n$ for large $n$.
By definition, the susceptibility has radius of convergence $c(\graph)^{-1}$.
\begin{figure}[!htb]
\label{fig:trap}
\centering
\caption{A trapped self-avoiding walk}
\includegraphics{trapped}
\end{figure}
\end{example}

%%%%%%%%%%%%%%%%%%%%%%%%%%%%%%%%%%%%%%%%%%%%%%%%%%%%%%%%%%%%%%%%%%%%%%%%%%%%%%%

\subsection{Continuous-time walks}

Let $\omega : [0, T] \to \vertices$ be a right-continuous function and define
$\tau_n = \tau_n(\omega)$ inductively by setting $\tau_0 = 0$ and
\begin{equation}
\tau_{n+1} = \inf (t > \tau_n : \omega_t \ne \omega_{\tau_n}).
\end{equation}
Call $\omega$ a \emph{continuous-time walk of length $T$} if $\{ \tau_n \}$
has no cluster points and $\omega_{\tau_i} \sim \omega_{\tau_{i+1}}$ for all $i$.
Let $\cwalks_T$ denote the space of continuous-time walks $\omega$ of length $T$
with $\omega_0 = 0$.

We can define general models of continuous-time walks in terms of measures on
the $\cwalks_T$. Let us restrict our attention to models defined by Gibbs measures
with base measure given by the continuous-time simple random walk on $\graph$.
That is, given a Hamiltonian $H$ on $\cwalks = \bigcup_{T \ge 0} \cwalks_T$,
we consider the measures of the form
\begin{equation}
\mu_{\beta,H}(A) = E_0 (e^{-\beta H} \1_A)
\end{equation}

%%%%%%%%%%%%%%%%%%%%%%%%%%%%%%%%%%%%%%%%%%%%%%%%%%%%%%%%%%%%%%%%%%%%%%%%%%%%%%%

\subsection{Weakly self-avoiding walk with self-attraction}

\commentbw{Add MCMC figure}

Define the \emph{local time} up to time $T$ of $\omega \in \cwalks$ at
$x \in \vertices$ by
\begin{equation}
\label{e:LTx-def}
\lt^x_T(\omega) = \int_0^T \1_{\omega(S)=x} \; dS.
\end{equation}
In the discrete-time case, $\lt^x_n$ is the number of times $\omega$ visits $x$
and is bounded by $n$. In the continuous-time case, $\lt^x_T$ is almost surely
finite for the continuous-time simple random walk.

We define the \emph{intersection local time}
\begin{equation}
\label{e:ITdef}
I_T(\omega) = \sum_{x\in\vertices} (\lt^x_T)^2
  =
\int_0^T \!\! \int_0^T \1_{\omega(S_1)=\omega(S_2)} \; dS_1 dS_2
\end{equation}
and the \emph{contact self-attraction}
\begin{equation}
\label{e:CTdef}
C_T(\omega)
	=
\sum_{x \in \vertices} \sum_{y \sim x} \lt_T^x(\omega) \lt_T^y(\omega)
	=
\int_0^T ds \int_0^T dt \; \1_{\omega_s \sim \omega_t}
\end{equation}
up to time $T$.
% Recall that we have set the inverse temperature equal to $1$.
Given a parameter $\gcc > 0$, and $\gamma \in \R$, let
\begin{equation}
\label{e:Udef-neg}
U_{\gcc,\gamma}(f)
=
\gcc \sum_{x\in\vertices} f_x^2
- \frac{\gamma}{2d}
\sum_{x\in\vertices} \sum_{y \sim x} f_x f_y
\end{equation}
for $f : \vertices \to \R$.
The \emph{weakly self-avoiding walk with self-attraction} (WSAW-SA) is defined via the Hamiltonian
\begin{equation}
\label{e:V}
U_{\gcc,\gamma,T}
	= U_{\gcc,\gamma} \circ L_T
	= \gcc I_T - \frac{\gamma}{2 d} C_T.
\end{equation}
Thus,
\begin{equation}
\label{e:c}
    c_T = E_a\left(e^{-U_{\gcc,\gamma,T}}\right),
    \quad
    c_T(x) = E_0\left(e^{-U_{\gcc,\gamma,T}}\1_{X_T = x}\right)
\end{equation}
and the two-point function and susceptibility are given by
\begin{align}
\lbeq{Gsa}
G_x(\gcc,\gamma,\nu)
    &=
\int_0^\infty c_T(x) e^{-\nu T} \; dT
\end{align}
and
\begin{equation}
\label{e:suscept-def}
\chi(\gcc, \gamma, \nu)
	=
\int_0^\infty c_T e^{-\nu T} \; dT
	=
\sum_{x\in\Zd} G_x(\gcc,\gamma,\nu)
\end{equation}
There is also a version of the correlation length of order $p$:
\begin{equation}
\label{e:clp-wsawsa}
\xi_p(\gcc, \gamma, \nu; 0)
	=
\left(\frac{\sum_{x\in\Zd} |x|^p G_x(g, \gamma, \nu; 0)}{\chi(g, \gamma, \nu; 0)}\right)^{1/p}.
\end{equation}

In the case $\gamma = 0$, the discrete-time version of this model is known as
the \emph{Domb-Joyce model}. In continuous-time, it is the
\emph{continuous-time weakly self-avoiding walk} (WSAW).

\subsubsection{Alternative representation}

For $f : \Zd \to \R$,
\begin{equation}
\label{e:sbp}
\sum_{x\in\Zd}   f_x \Delta_{\Zd} f_x
=
-\frac{1}{2} |\nabla f|^2.
\end{equation}
It follows that
\begin{equation}
\sum_{x\in\Zd} \sum_{e\in\Ucal} f_x f_{x+e}
=
2 d \sum_{x\in\Zd} f_x^2
+ \sum_{x\in\Zd} f_x \Delta_{\Zd} f_x
=
2 d \sum_{x\in\Zd} f_x^2
- \frac{1}{2} \sum_{x\in\Zd} |\nabla f_x|^2
\end{equation}
and so we get the useful representation:
\begin{equation}
\label{e:Udef-pos}
U_{\gcc,\gamma}(f)
= (\gcc - \gamma) \sum_{x\in\Zd} f_x^2
+ \frac{\gamma}{4d} \sum_{x\in\Zd} \sum_{e\in\Ucal} |\nabla^e f_x|^2.
\end{equation}
In particular,
\begin{equation}
  \label{e:V2}
  U_{\gcc,\gamma,T} =
  (\gcc - \gamma) I_T
  + \frac{\gamma}{4d}
  |\nabla \lt_T|^2
  .
\end{equation}
% A version of \refeq{V2} can be found in \cite{HK01a}.

%%%%%%%%%%%%%%%%%%%%%%%%%%%%%%%%%%%%%%%%%%%%%%%%%%%%%%%%%%%%%%%%%%%%%%%%%%%%%%%

\subsection{Predicted behaviour}

\todo{State expected critical behaviour.}

\todo{
For walks, it is expected that
\begin{align}
c_T                       &\sim C_5 e^{-\nu_c T} T^{-\gammabar}, \\
\langle |X_T|^2 \rangle   &\sim C_6 T^{-\nubar}.
\end{align}
\todo{Motivate the above with an example somewhere.}}

\todo{Discuss phase diagram for WSAW-SA}
