% March 27, 2003  GS    
% June 14, 2000  GS    rk environment added under \DefineTheorems
% August 05, 1999  TH

% File def99b.tex, a version of Latex Macros: for T.Hara. 
% This should work with LaTeX2e, and "amsmath,amssymb,amsfonts,amsthm" packages. 
% Just do something like
%     \documentclass[11pt]{article}
%     \usepackage{amsmath,amssymb,amsfonts,amsthm}
%
% Ver.  3.0:   TH, 1998.03.27, Seattle, % def98a.tex
% Ver.  3.0.1: TH, 1998.03.30, Seattle, 
% Ver.  3.0.2: TH, 1998.04.18, Tokyo,   
% Ver.  3.0.5: TH, 1998.06.07, Tokyo, (changed /cal to /mathcal)
% Ver.  4.0:   TH, 1999.01.05, Tokyo, % def99a.tex
% Ver.  4.1:   TH, 1999.01.30, Tokyo, % def99b.tex
% Ver.  4.2:   TH, 1999.08.05, Tokyo, % def99c.tex

%%%%% Partial History 
% Version 2.0: TH, 1994.10.27, Tokyo.
% This version, 96a, is supposed to be compatible with LaTeX2e. 
%       1996. 11.30, TH. 
% Ver.	1.0: TH, 1997.05.02, Tokyo. % def96d.tex
% 	1.1: TH, 1997.05.07, Tokyo.
% 	1.2: TH, 1997.06.21, Tokyo. % Works also with pLatex2e (Japanese)!
% Ver.  2.1: TH, 1997.11.03, Tokyo, % def97a.tex
%%%%%%%%%%%%%%%%%%%%%%%%%%%%%%%%%%%%%%%%%%%%%%%%%%%%%%%%%%%%%%%%%%%%%%%%%%%%%
%%%%%%%%%%%%%%%%%%%%%%%%%%%%%%%%%%%%%%%%%%%%%%%%%%%%%%%%%%%%%%%%%%%%%%%%%%%%%
%%%%%%%%%%%%%%%%%%%%%%%%%%%%%%%%%%%%%%%%%%%%%%%%%%%%%%%%%%%%%%%%%%%%%%%%%%%%%

% THEOREM, EQN etc. commands

%% In the following, select the desired numbering system for 
%% Eqn, and Thm,etc.--- whether we need subsection in the numbering.
%%%% Macro Counting Scheme Definitions ---------------------------------
\def\UseSection{%%
        \numberwithin{equation}{section}
	\theoremstyle{plain}% default theorem style 
        \newtheorem{theorem}    {Theorem}[section]
        \DefineTheorems % Use this to define other environments to be 
        		% numbered as ``theorem.''
}
\def\UseSubSection{%%
        \numberwithin{equation}{subsection}
	\theoremstyle{plain}% default theorem style 
        \newtheorem{theorem}    {Theorem}[subsection]
        \DefineTheorems % Use this to define other environments to be 
        		% numbered as ``theorem.''
}
%%%% End of Counting Scheme Defs---------------------------------------
\def\DefineTheorems{%%
	\newtheorem{qtheorem}   [theorem] {Q-Theorem}
	\newtheorem{lemma}      [theorem] {Lemma}
	\newtheorem{qlemma}     [theorem] {Q-Lemma}
	\newtheorem{prop}       [theorem] {Proposition}
	\newtheorem{qprop}      [theorem] {Q-Proposition}
	\newtheorem{cor}        [theorem] {Corollary}
	\newtheorem{ass}        [theorem] {Assumption}
	\newtheorem{claim}      [theorem] {Claim}
	\newtheorem{conj}       [theorem] {Conjecture}
	%
	\theoremstyle{definition}% ``defn'' theorem style 
	\newtheorem{defn}       [theorem] {Definition}
	\newtheorem{ques}       [theorem] {Question}
	\newtheorem{example}       [theorem] {Example}
	\newtheorem{rk} 	[theorem] {Remark}
	%
	\theoremstyle{definition}% ``remark'' theorem style 
	\newtheorem*{rem}	{Remark}
	\newtheorem*{note}	{Note}
}

\newcommand{\bt}   {\begin{theorem}}
\newcommand{\et}   {\end  {theorem}}
\newcommand{\bl}   {\begin{lemma}}
\newcommand{\el}   {\end  {lemma}}
\newcommand{\bp}   {\begin{prop}}
\newcommand{\ep}   {\end  {prop}}
\newcommand{\bc}   {\begin{cor}}
\newcommand{\ec}   {\end  {cor}}
\newcommand{\bd}   {\begin{defn}}
\newcommand{\ed}   {\end  {defn}}

\newcommand{\ba}   {\begin{array}}
\newcommand{\ea}   {\end  {array}}
\newcommand{\be}   {\begin{enumerate}}
\newcommand{\ee}   {\end  {enumerate}}
\newcommand{\bi}   {\begin{itemize}}
\newcommand{\ei}   {\end  {itemize}}



%\newcommand{\eq}       {\begin{equation}}%	This does not work with amsmath.
%\newcommand{\en}       {\end{equation}}% 	This does not work with amsmath.
\def\eq#1\en{\begin{equation}#1\end{equation}}  
	% This follows ``technical note'' of AMS-LaTeX.  
%\newcommand{\eqsplit}   {\begin{equation}\begin{split}} 
%\newcommand{\ensplit}   {\end{split}\end{equation}} 
\def\eqsplit#1\ensplit{
	\begin{equation}\begin{split}#1\end{split}\end{equation}
	}
\def\eqalign#1\enalign{
	\begin{align}#1\end{align}
	}
\def\eqmul#1\enmul{
	\begin{multline}#1\end{multline}
	}
\newcommand{\eqarrstar} {\begin{eqnarray*}} 
\newcommand{\enarrstar} {\end{eqnarray*}} 
\newcommand{\eqarray}   {\begin{eqnarray}} 
\newcommand{\enarray}   {\end{eqnarray}} 
\newcommand{\nnb}	{\nonumber \\} 
\newcommand{\diseq}[1]	{$\displaystyle #1$}%


\newcommand{\lbeq}[1]  {\label{e:#1}}
\newcommand{\refeq}[1] {\eqref{e:#1}}    % AMS-LaTeX trick!
%\newcommand{\refeq}[1] {(\ref{eq:#1})}
\newcommand{\lbfg}[1]  {\label{fg: #1}}
\newcommand{\reffg}[1] {\ref{fg: #1}}
\newcommand{\lbtb}[1]  {\label{tb: #1}}
\newcommand{\reftb}[1] {\ref{tb: #1}}

%%%%%%%%%%%%%%%%%%%%%%%%%%%%%%%%%%%%%%%%%%%%%%%%%%%%%%%%%%%%%%%%%%%%%%%%%%%%%
%	Nice labeling scheme (1999.1.30):
%	The following defines a command \labelcounter, which enables us to 
%	refer to an arbitray counter in any(?) environment. 
%  	First argument is ``counter name''.
%  	Second argument is ``label''. 
%  	An example of using it is shown below: 
%  		\newcounter{countC}  
%  			% Defined the counter ``countC''
% 		\setcounter{countC}{0}	
% 			% Initialized the clunter ``countC'' to zero. 
% 		\newcommand{\lbC}[1]{C_{\labelcounter{countC}{C:#1}}} 
% 			%  Now defining the constants C with subscript 
% 			%  (number) labeled by the counter ``countC''. 
% 		\newcommand{\refC}[1]{C_{\ref{C:#1}}}
% 			%  Defining how to refer to the above Constants C_{*}. 
% 
%	
\makeatletter
\newcommand{\labelcounter}[2]{{%
	\stepcounter{#1}%	First, increase the ``countC'' by one.
	\protected@write\@auxout{}%
	{\string\newlabel{#2}{{\csname the#1\endcsname}{\thepage}}}%
		% Then write out the contents of ``countC'' together with 
		% the page number to aux file.  This is what ``label'' 
		% usually does. 
	{\ref{#2}}%	Finally, make sure to refer to this label, 
		%	when defined. 
	}}
\makeatother
%	
%	
%	
%%%%%%%%%%%%%%%%%%%%%%%%%%%%%%%%%%%%%%%%%%%%%%%%%%%%%%%%%%%%%%%%%%%%%%%%%%%%%%%
%  ``Remark, Proof, QED'' etc. 
%\newcommand{\proof} {\noindent {\bf Proof}. \hspace{2mm}}
%\newcommand{\qed}   {\hfill $\Box$}
%\newcommand{\qed}   {\nopagebreak 
%	\begin{flushright} \rule{3mm}{3mm} \end{flushright}}
\newcommand{\REFERENCE}{{\bf REFERENCE !}}
\newcommand{\remark}{\medskip\noindent {\bf Remark}. \hspace{2mm}}
\newcommand{\Remark}{\noindent {\bf Remark}. \hspace{2mm}}
\newcommand{\ssss}  { \scriptstyle } 
\newcommand{\sss}   { \scriptscriptstyle } 

%%%%%%%%%%%%%%%%%%%%%%%%%%%%%%%%%%%%%%%%%%%%%%%%%%%%%%%%%%%%%%%%%%%%%%%%%%
%%%%%%%%%%%%%%%%%%%%%%%%%%%%%%%%%%%%%%%%%%%%%%%%%%%%%%%%%%%%%%%%%%%%%%%%%%
% Bold fonts %%%%%%%%%%%%%%%%%%%%%%%%%%%%%%%%%%%%%%%%%%%%%%%%%%%%%%%%%%%%%
% Implementation of Blackboard fonts by Gord (COMMENT OUT IF UNAVAILABLE)
%\newfam\Bbbfam
%\font\tenBbb=msbm10
%\font\sevenBbb=msbm7
%\font\fiveBbb=msbm5
%\font\twelveBbb=msbm10 scaled\magstep2
%\textfont\Bbbfam=\tenBbb
%\scriptfont\Bbbfam=\sevenBbb
%\scriptscriptfont\Bbbfam=\fiveBbb
%\def\Bbb{\fam\Bbbfam \tenBbb}
% Blackboard fonts inplimentation end %%%%%%%%%%%%%%%%%%%%%%%%%%%%%%%%%%%
%\newcommand{\zd} {{\Bbb Z}^d}

\newcommand{\Abold} {{\mathbb A}}  
\newcommand{\Bbold} {{\mathbb B}}  
\newcommand{\Cbold} {{\mathbb C}}  
\newcommand{\Dbold} {{\mathbb D}}  
\newcommand{\Ebold} {{\mathbb E}}
\newcommand{\Fbold} {{\mathbb F}}
\newcommand{\Gbold} {{\mathbb G}}
\newcommand{\Hbold} {{\mathbb H}}
\newcommand{\Ibold} {{\mathbb I}}
\newcommand{\Jbold} {{\mathbb J}}
\newcommand{\Kbold} {{\mathbb K}}
\newcommand{\Lbold} {{\mathbb L}}
\newcommand{\Mbold} {{\mathbb M}}
\newcommand{\Nbold} {{\mathbb N}}
\newcommand{\Obold} {{\mathbb O}}
\newcommand{\Pbold} {{\mathbb P}}
\newcommand{\Qbold} {{\mathbb Q}}
\newcommand{\Rbold} {{\mathbb R}}
\newcommand{\Sbold} {{\mathbb S}}
\newcommand{\Tbold} {{\mathbb T}}
\newcommand{\Ubold} {{\mathbb U}}
\newcommand{\Vbold} {{\mathbb V}}
\newcommand{\Wbold} {{\mathbb W}}
\newcommand{\Xbold} {{\mathbb X}}
\newcommand{\Ybold} {{\mathbb Y}}
\newcommand{\Zbold} {{\mathbb Z}}

% Vector fonts %%%%%%%%%%%%%%%%%%%%%%%%%%%%%%%%%%%%%%%%%%%%%%%%%%%%%%%%%%%%%
\newcommand{\Avec} {{\bf A}}  
\newcommand{\Bvec} {{\bf B}}  
\newcommand{\Cvec} {{\bf C}}  
\newcommand{\Dvec} {{\bf D}}  
\newcommand{\Evec} {{\bf E}}
\newcommand{\Fvec} {{\bf F}}
\newcommand{\Gvec} {{\bf G}}
\newcommand{\Hvec} {{\bf H}}
\newcommand{\Ivec} {{\bf I}}
\newcommand{\Jvec} {{\bf J}}
\newcommand{\Kvec} {{\bf K}}
\newcommand{\Lvec} {{\bf L}}
\newcommand{\Mvec} {{\bf M}}
\newcommand{\Nvec} {{\bf N}}
\newcommand{\Ovec} {{\bf O}}
\newcommand{\Pvec} {{\bf P}}
\newcommand{\Qvec} {{\bf Q}}
\newcommand{\Rvec} {{\bf R}}
\newcommand{\Svec} {{\bf S}}
\newcommand{\Tvec} {{\bf T}}
\newcommand{\Uvec} {{\bf U}}
\newcommand{\Vvec} {{\bf V}}
\newcommand{\Wvec} {{\bf W}}
\newcommand{\Xvec} {{\bf X}}
\newcommand{\Yvec} {{\bf Y}}
\newcommand{\Zvec} {{\bf Z}}
\newcommand{\avec}  {\boldsymbol{a}}
\newcommand{\bvec}  {\boldsymbol{b}}
\newcommand{\cvec}  {\boldsymbol{c}}
\newcommand{\dvec}  {\boldsymbol{d}}
\newcommand{\evec}  {\boldsymbol{e}}
\newcommand{\fvec}  {\boldsymbol{f}}
\newcommand{\gvec}  {\boldsymbol{g}}
\newcommand{\hvec}  {\boldsymbol{h}}
\newcommand{\ivec}  {\boldsymbol{i}}
\newcommand{\jvec}  {\boldsymbol{j}}
\newcommand{\kvec}  {\boldsymbol{k}}
\newcommand{\lvec}  {\boldsymbol{l}}
\newcommand{\mvec}  {\boldsymbol{m}}
\newcommand{\nvec}  {\boldsymbol{n}}
\newcommand{\ovec}  {\boldsymbol{o}}
\newcommand{\pvec}  {\boldsymbol{p}}
\newcommand{\qvec}  {\boldsymbol{q}}
\newcommand{\rvec}  {\boldsymbol{r}}
\newcommand{\svec}  {\boldsymbol{s}}
\newcommand{\tvec}  {\boldsymbol{t}}
\newcommand{\uvec}  {\boldsymbol{u}}
\newcommand{\vvec}  {\boldsymbol{v}}
\newcommand{\wvec}  {\boldsymbol{w}}
\newcommand{\xvec}  {\boldsymbol{x}}
\newcommand{\yvec}  {\boldsymbol{y}}
\newcommand{\zvec}  {\boldsymbol{z}}
\newcommand{\zerovec} {\boldsymbol{0}}
\newcommand{\onevec}  {\boldsymbol{1}}

% Caligraph fonts %%%%%%%%%%%%%%%%%%%%%%%%%%%%%%%%%%%%%%%%%%%%%%%%%%%%%
\newcommand{\Acal}   {\mathcal{A}} 
\newcommand{\Bcal}   {\mathcal{B}} 
\newcommand{\Ccal}   {\mathcal{C}} 
\newcommand{\Dcal}   {\mathcal{D}} 
\newcommand{\Ecal}   {\mathcal{E}} 
\newcommand{\Fcal}   {\mathcal{F}} 
\newcommand{\Gcal}   {\mathcal{G}} 
\newcommand{\Hcal}   {\mathcal{H}} 
\newcommand{\Ical}   {\mathcal{I}} 
\newcommand{\Jcal}   {\mathcal{J}} 
\newcommand{\Kcal}   {\mathcal{K}} 
\newcommand{\Lcal}   {\mathcal{L}} 
\newcommand{\Mcal}   {\mathcal{M}} 
\newcommand{\Ncal}   {\mathcal{N}} 
\newcommand{\Ocal}   {\mathcal{O}} 
\newcommand{\Pcal}   {\mathcal{P}}
\newcommand{\Qcal}   {\mathcal{Q}}
\newcommand{\Rcal}   {\mathcal{R}}
\newcommand{\Scal}   {\mathcal{S}} 
\newcommand{\Tcal}   {\mathcal{T}} 
\newcommand{\Ucal}   {\mathcal{U}} 
\newcommand{\Vcal}   {\mathcal{V}} 
\newcommand{\Wcal}   {\mathcal{W}} 
\newcommand{\Xcal}   {\mathcal{X}}
\newcommand{\Ycal}   {\mathcal{Y}} 
\newcommand{\Zcal}   {\mathcal{Z}} 

% Hatted fonts, Upper case %%%%%%%%%%%%%%%%%%%%%%%%%%%%%%%%%%%%%%%%%%%%%%%%
\newcommand{\Ahat} {{\hat{A} }}
\newcommand{\Bhat} {{\hat{B} }}
\newcommand{\Chat} {{\hat{C} }}  
\newcommand{\Dhat} {{\hat{D} }}  
\newcommand{\Ehat} {{\hat{E} }}  
\newcommand{\Fhat} {{\hat{F} }}  
\newcommand{\Ghat} {{\hat{G} }}  
\newcommand{\Hhat} {{\hat{H} }}  
\newcommand{\Ihat} {{\hat{I} }}  
\newcommand{\Jhat} {{\hat{J} }}
\newcommand{\Khat} {{\hat{K} }}  
\newcommand{\Lhat} {{\hat{L} }}  
\newcommand{\Mhat} {{\hat{M} }}  
\newcommand{\Nhat} {{\hat{N} }}  
\newcommand{\Ohat} {{\hat{O} }}  
\newcommand{\Phat} {{\hat{P} }}  
\newcommand{\Qhat} {{\hat{Q} }} 
\newcommand{\Rhat} {{\hat{R} }} 
\newcommand{\Shat} {{\hat{S} }}  
\newcommand{\That} {{\hat{T} }}  
\newcommand{\Uhat} {{\hat{U} }}
\newcommand{\Vhat} {{\hat{V} }}  
\newcommand{\What} {{\hat{W} }}
\newcommand{\Xhat} {{\hat{X} }}  
\newcommand{\Yhat} {{\hat{Y} }}  
\newcommand{\Zhat} {{\hat{Z} }}  

% Hatted fonts, Lower case %%%%%%%%%%%%%%%%%%%%%%%%%%%%%%%%%%%%%%%%%%%%%%
\newcommand{\ahat}  {{ \hat{a}  }}
\newcommand{\bhat}  {{ \hat{b}  }}
\newcommand{\chat}  {{ \hat{c}  }}
\newcommand{\dhat}  {{ \hat{d}  }}
\newcommand{\ehat}  {{ \hat{e}  }}
\newcommand{\fhat}  {{ \hat{f}  }}
\newcommand{\ghat}  {{ \hat{g}  }}
\newcommand{\hhat}  {{ \hat{h}  }}
\newcommand{\ihat}  {{ \hat{i}  }}
\newcommand{\jhat}  {{ \hat{j}  }}
\newcommand{\khat}  {{ \hat{k}  }}
\newcommand{\lhat}  {{ \hat{l}  }}
\newcommand{\mhat}  {{ \hat{m}  }}
\newcommand{\nhat}  {{ \hat{n}  }}
\newcommand{\ohat}  {{ \hat{o}  }}
\newcommand{\phat}  {{ \hat{p}  }}
\newcommand{\qhat}  {{ \hat{q}  }}
\newcommand{\rhat}  {{ \hat{r}  }}
\newcommand{\shat}  {{ \hat{s}  }}
\newcommand{\that}  {{ \hat{t}  }}
\newcommand{\uhat}  {{ \hat{u}  }}
\newcommand{\vhat}  {{ \hat{v}  }}
\newcommand{\what}  {{ \hat{w}  }}
\newcommand{\xhat}  {{ \hat{x}  }}
\newcommand{\yhat}  {{ \hat{y}  }}
\newcommand{\zhat}  {{ \hat{z}  }}

% Hatted fonts, Greek Letters %%%%%%%%%%%%%%%%%%%%%%%%%%%%%%%%%%%%%%%%%%%
\newcommand{\tauhat}{{ \hat{\tau}  }}
\newcommand{\Pihat} {\mbox{${\hat{\Pi}}$}}  

% Bar fonts %%%%%%%%%%%%%%%%%%%%%%%%%%%%%%%%%%%%%%%%%%%%%%%%%%%%%%%%%%%%%
\newcommand{\nbar}	{\bar{n}}
\newcommand{\ubar}	{\bar{u}}
% Overlined fonts %%%%%%%%%%%%%%%%%%%%%%%%%%%%%%%%%%%%%%%%%%%%%%%%%%%%%%%
\newcommand{\Gbar}  { \overline{\overline{G} }  } 

% Lattice, Space
\newcommand{\Rd}    {{ {\Rbold}^d}}
\newcommand{\Zd}    {{ {\Zbold}^d }}

%%%%%%%%%%%%%%%%%%%%%%%%%%%%%%%%%%%%%%%%%%%%%%%%%%%%%%%%%%%%%%%%%%%%%%%%%%%%%
%%%%%%%%%%%%%%%%%%%%%%%%%%%%%%%%%%%%%%%%%%%%%%%%%%%%%%%%%%%%%%%%%%%%%%%%%%%%%
% Mathematical symbols (can be used in other papers also) : 

%\ltapprox and \gtapprox produce > and < signs with twiddle underneath
\newcommand{\spose}[1] {{\hbox to 0pt{#1\hss}} }
\newcommand{\ltapprox} {\mathrel{\spose{\lower 3pt\hbox{$\mathchar"218$}}
 \raise 2.0pt\hbox{$\mathchar"13C$}}}
\newcommand{\gtapprox} {\mathrel{\spose{\lower 3pt\hbox{$\mathchar"218$}}
 \raise 2.0pt\hbox{$\mathchar"13E$}}}

\newcommand{\smfrac}[2]{\textstyle{#1\over #2}}

\newcommand {\asympp}{\mbox{${\approx}$}} 
\newcommand{\abs}[1]    {  {  \left | #1 \right |  } } 
\newcommand{\set}[1]    {  {  \left \{ #1 \right \}  } } 
\newcommand{\crit}	{{\textnormal{crit}}}


\newcommand{\combination}[2]{ { \left ( 
	\begin{array}{c}    {#1} \\ {#2} 
	\end{array} 
	\right  )   }} 
\newcommand{\cupd} {\stackrel{\cdot}{\cup}} 
\newcommand{\ddk}  {\frac{d^d k}{(2\pi)^d}}
\newcommand{\dist} {{  \rm dist }}
\newcommand{\kx}   {{  k \cdot x }}
\newcommand{\ika}  {{ i k \cdot a }}
\newcommand{\ikx}  {{ i k \cdot x }}
\newcommand{\iky}  {{ i k \cdot y }}
\newcommand{\Ind}  { {\rm I} } 
\newcommand{\intsub}{\int_{[-\pi,\pi]^d}}
%\newcommand{\nexists} {{ \not\exists }}
\newcommand{\nin}  {{ \not\in }}
\newcommand{\nni}  {{ \not\ni }} 
\newcommand{\Prob} {{\rm Prob}}
\newcommand{\prodtwo}[2]{
	\prod_{ \mbox{ \scriptsize 
		$\begin{array}{c} 
		{#1} \\ 
		{#2}  
		\end{array} $ } 
		} 
	} 
%\newcommand {\Real}   {{  {\rm Re} }} 
\newcommand{\Imaginary}{{ {\rm Im} }}
% \newcommand{\sumtwo}[2]{\displaystyle \sum_{ \mbox{ \scriptsize 
% 	$\begin{array}{c}
%                         {#1} \\ {#2}
%                         \end{array} $ } 
% 	}
% }
\newcommand{\toinf} {\rightarrow \infty}

%%%%%%%%%%%%%%%%%%%%%%%%%%%%%%%%%%%%%%%%%%%%%%%%%%%%%%%%%%%%%%%%%%%%%%%%%%%%%
%Derivatives 
\newcommand{\pder}[2] { \frac{\partial #1}{\partial #2} } 
\newcommand{\der}[2] { \frac{d #1}{d #2} } 
\newcommand{\derder}[2]	{\frac{d^2 #1}{d {#2}^2 }} 
\newcommand{\pderder}[2]	{\frac{\partial^2 #1}{\partial {#2}^2 }} 

%%%%%%%%%%%%%%%%%%%%%%%%%%%%%%%%%%%%%%%%%%%%%%%%%%%%%%%%%%%%%%%%%%%%%%%%%%%%%
%For Quantum Mechanics
\newcommand{\ket}[1] 	{\left | #1 \right \rangle} 
\newcommand{\bra}[1] 	{\left \langle #1 \right |} 
\newcommand{\commutator}[2]	{\left [ #1, #2 \right  ]_{-} } 
\newcommand{\expec}[1]	{\left \langle #1 \right \rangle} 

