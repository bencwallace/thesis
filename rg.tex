\chapter{Renormalisation group method}

\renewcommand{\pm}{+}					% only consider \gamma \ge 0

In this chapter, we outline the renormalisation group method of Bauerschmidt,
Brydges, and Slade developed in the series of papers
\cite{BS-rg-norm,BS-rg-loc,BBS-rg-pt,BS-rg-IE,BS-rg-step} and
applied in \cite{BBS-phi4-log,BBS-saw4-log,BBS-saw4,ST-phi4}. We will often
state results from these papers without proof.

The main contribution of this
thesis, which is based on the work in \cite{BSTW-clp,BSW-saw-sa}, is the
improvement of the estimates in Theorem~\ref{thm:step-mr-fv} and the extension
to $\gamma_0 \ne 0$ in Theorem~\ref{thm:rhatflow}.

%%%%%%%%%%%%%%%%%%%%%%%%%%%%%%%%%%%%%%%%%%%%%%%%%%%%%%%%%%%%%%%%%%%%%%%%%%%%%%%
%%%%%%%%%%%%%%%%%%%%%%%%%%%%%%%%%%%%%%%%%%%%%%%%%%%%%%%%%%%%%%%%%%%%%%%%%%%%%%%

\section{Unified notation}

To unify our treatment of the two models,
we define the forms $\tau_x, \tau_{\Delta,x}, |\nabla\tau_x|^2$ according
to \eqref{e:taudef}--\eqref{e:nablatau} if $n = 0$ and
\begin{equation}
\label{e:tauphi}
\tau_x = \tfrac{1}{2} |\varphi_x|^2,
	\quad
\tau_{\Delta,x} = \tfrac{1}{2} \varphi_x \cdot (-\Delta \varphi)_x,
	\quad
|\nabla\tau_x|^2 = \sum_{|e|=1} (\nabla^e |\phi_x|^2)^2
\end{equation}
if $n \ge 1$.

Then by \eqref{e:Vdef1}, \eqref{e:Udef-pos}, and \eqref{e:Vdef2},
% $V_{g,\gamma,\nu,N} \in \Ncal^\varnothing$ is given by
\begin{equation}
V_{g,\gamma,\nu,N}
	=
\sum_{x\in\Lambda_N}
\Big(
	(g - \gamma) \tau_x^2 + \nu \tau_x + \tau_{\Delta,x} + \tfrac{1}{4 d} \gamma |\nabla\tau_x|^2
\Big)
\end{equation}
for any choice of $n$.
We write
\begin{equation}
\langle F \rangle_{g,\gamma,\nu,N}
	=
\begin{cases}
\displaystyle \int F e^{-U_{g,\gamma,\nu,N}},				& n = 0 \\
\displaystyle \frac{1}{Z_{g,\gamma,\nu,N}}
	\int F(\varphi) e^{-U_{g,\gamma,\nu,N}} \; d\varphi,	& n \ge 1.
\end{cases}
\end{equation}
The action $S_A$ is defined by \eqref{e:action} if $n = 0$ and
\begin{equation}
S_A = \frac12 \sum_{x\in\Lambda} \varphi_x \cdot (A \varphi)_x
\end{equation}
if $n \ge 1$. In either case, if $A = -\Delta + m^2$, then
\begin{equation}
\label{e:SAtauDelta}
S_A = \sum_{x\in\Lambda} (\tau_{\Delta,x} + m^2 \tau_x).
\end{equation}
Thus, if $\Ex_C \theta$ is the super-expectation \eqref{e:ExCF} for $n = 0$
and Gaussian integration over $(\R^n)^\Lambda$ if $n \ge 1$ (recall \eqref{e:gauss-density}),
then for $\nu > 0$,
\begin{equation}
\label{e:phi4-gauss}
\langle F \rangle_{0,0,m^2,N}
	=
\Ex_C F,
	\qquad
C = (-\Delta + m^2)^{-1}.
\end{equation}

By \eqref{e:two-point-function-phi4}, \eqref{e:Givlc}, and \eqref{e:Grep-pos-bis},
\begin{equation}
G_x(g, \gamma, \nu; n) = \lim_{N\to\infty} G_{x,N}(g, \gamma, \nu; n),
\end{equation}
where
\begin{equation}
G_{x,N}(g, \gamma, \nu; n)
	=
\begin{cases}
\langle \phib_0 \phi_x \rangle_{g,\gamma,\nu,N},      & n = 0 \\
\langle \varphi_0 \cdot \varphi_x \rangle_{g,\gamma,\nu,N}  & n \ge 1.
\end{cases}
\end{equation}

%%%%%%%%%%%%%%%%%%%%%%%%%%%%%%%%%%%%%%%%%%%%%%%%%%%%%%%%%%%%%%%%%%%%%%%%%%%%%%%
%%%%%%%%%%%%%%%%%%%%%%%%%%%%%%%%%%%%%%%%%%%%%%%%%%%%%%%%%%%%%%%%%%%%%%%%%%%%%%%

\section{Reformulation of the problem}

Given $m^2>0$ and $z_0 >-1$, let
\begin{equation}
\label{e:gg0}
g_0 = (\gcc - \gamma) (1 + z_0)^2,
	\quad
\nu_0 = \nu (1 + z_0) - m^2,
	\quad
\gamma_0 = \frac{1}{4d} \gamma (1 + z_0)^2.
\end{equation}
We fix the two points $0,x\in \Lambda$
and introduce \emph{observable fields} $\sigmaa, \sigmab \in \R$.
In summary, we distinguish between the following kinds of fields:
real ($\varphi$) and complex ($\phi$, $\phib$) bosonic fields,
observable fields ($\sigma$), and fermionic fields ($\psi$, $\psib$).

For any $y\in\Lambda$, we define the polynomials
\begin{equation}
\label{e:V0def}
\Vp^+_{0,y}
	=
g_0\tau_y^2 + \nu_0 \tau_y + z_0 \tau_{\Delta,y}
- f_0 \sigma_0 \1_{y=x}
- f_x \sigma_x \1_{y=x},
	\quad
U^+_y
	=
|\nabla \tau_y|^2
\end{equation}
where
\begin{equation}
\label{e:obs-couple}
f_u =
\begin{cases}
\phib_0,		& n = 0, u = 0 \\
\phi_x,			& n = 0, u = x \\
\varphi^1_u,	& n \ge 1.
\end{cases}
\end{equation}
These are examples of local polynomials, which are polynomials
in the fields and their derivatives at a point $y\in\Lambda$. For any such local
polynomial $V_y$, we will usually write
\begin{equation}
\label{e:VX}
V(X) = \sum_{y\in X} V_y.
\end{equation}

Let
\begin{equation}
\label{e:Z0def}
Z_0 = \prod_{x\in \Lambda} e^{-(\Vp^+_{0,x} + \gamma_0 U^+_x)}
\end{equation}
and
\begin{equation}
\label{e:ZNdef}
Z_N = \Ex_C \theta Z_0
\end{equation}
where the covariance is given by $C = (-\Delta + m^2)^{-1}$.
In particular,
\begin{equation}
\label{e:exp-conv}
\Ex_C Z_0 = Z_{N,\varnothing}(0).
\end{equation}
Recall here that $Z_{N,\varnothing}$ denotes the $0$-degree part of $Z_N$;
this is a function of the bosonic fields, which we have set to $0$ on the
right-hand side.

Recall that the Gaussian convolution operator $\Ex_C\theta$ was defined in
Section~\ref{sec:forms}; recall also from this section that $Z_{N,\varnothing}$
denotes the $0$-degree part of $Z_N$ (when $n \ge 1$, $Z_{N,\varnothing} = Z_N$).
When $n = 0$, this is a function of the fields $(\phi, \bar\phi)$ and when $n \ge 1$
it is a function of $\varphi$. We define a test function $\1: \Lambda_N \to \R$ by
$\1_y=1$ for all $y$. If $F$ is a sufficiently smooth form, let
% and write $D^2 Z_{N,\varnothing}(0; \1, \1)$ for the directional derivative of
% $Z_{N,\varnothing}$ evaluated with all fields equal to $0$ and
% % at $(\phi, \bar\phi) = (0, 0)$,
% with both directions equal to $\1$. That is,
\begin{equation}
D^2 F(0; \1, \1)
  =
\ddp{^2}{s\partial t}\Big|_0
\begin{cases}
F(s \1, t\1), & n = 0 \\
F(s \1 + t\1), & n \ge 1
\end{cases}
\end{equation}
where the derivative is evaluated with all fields (bulk and observable) and $s, t$ set to $0$.
We will also denote by $D^2_{\sigma_0\sigma_x} F(0)$ denote the second partial derivative
with respect to $\sigma_0$ and $\sigma_x$ evaluated with all fields $0$.

\begin{prop}
\label{prop:intrep}
% from saw-sa
Let $d > 0$, $\gamma, \nu \in \R$, $\gcc >0$ and $\gamma <\gcc$.
If the relations \eqref{e:gg0} hold, then
\begin{align}
% from clp
\label{e:generating-fn}
G_{x,N}(g, \gamma, \nu; n)
	&=
(1+z_0)
D^2_{\sigma_0\sigma_x}
\log \Ex_C Z_0
\end{align}
% \begin{equation}
% \label{e:GG2}
% G_{x,N}(\gcc,\gamma,\nu)
%     =
% (1+z_0)
% \Ex_C (Z_0 \bar\phi_a \phi_b),
% \end{equation}
and
\begin{equation}
\label{e:chichibar}
\chi_N\left(\gcc,\gamma,\nu; n\right)
	=
(1+z_0)\hat\chi_N(m^2, g_0, \gamma_0, \nu_0, z_0; n),
\end{equation}
with
\begin{equation}
\label{e:chibarm}
\hat\chi_N(m^2, g_0, \gamma_0, \nu_0, z_0; n)
	=
\frac{1}{m^2}
	+
\frac{1}{m^4} \frac{1}{|\Lambda|} \frac{D^2 Z_{N,\varnothing}(0; \1, \1)}{Z_N(0)}.
\end{equation}
\end{prop}

\begin{proof}
We prove the case $n = 0$ and drop the parameter $n$ from the notation; note that
$Z_N(0)\big|_{\sigma_0=\sigma_x=0} = 1$ in this case. The proof for $n \ge 1$ is similar
and involves only ordinary integration with to real boson fields.

We make the change of variables $\phi_x \mapsto (1 + z_0)^{1/2} \phi_x$
and likewise for $\bar\phi_x, \psi_x, \bar\psi_x$ in \eqref{e:Grep-pos-bis}, and obtain
\begin{equation}
\label{e:Grep-pos}
G_{x,N}(\gcc,\gamma,\nu)
	=
(1+z_0) \int e^{-\sum_{x\in\Lambda}
\left(
	g_0 \tau_x^2 + \gamma_0 |\nabla \tau_x|^2 + \nu (1+z_0) \tau_x + (1+z_0)\tau_{\Delta,x}
\right)}
\bar\phi_a \phi_b.
\end{equation}
\todo{(Note here that there is no Jacobian factor due to the change of variables in $\psi, \psib$.)}
For any $m^2 \in\R$, it follows that
\begin{equation}
\lbeq{GNint}
G_{x,N}(\gcc,\gamma,\nu)
    =
(1 + z_0) \int
e^{-\sum_{x\in\Lambda} (\tau_{\Delta,x} + m^2 \tau_x)}
Z_0 \phib_0 \phi_x
\end{equation}
($m^2$ simply cancels with $\nu_0$ on the right-hand side).
We use this with $m^2>0$, so that the inverse matrix $C=(-\Delta+m^2)^{-1}$ exists and
\begin{equation}
\label{e:G-gauss}
G_{x,N}(g, \gamma, \nu)
	=
(1 + z_0) \Ex_C (Z_0 \phib_0 \phi_x)
\end{equation}
% and \eqref{e:GNint} is a Gaussian expectation
by \eqref{e:phi4-gauss}
% By symmetry of the matrix $\Delta$, \refeq{action} gives
% \begin{equation}
% \label{e:SAtauDelta}
% S_{(-\Delta+m^2)}
% =
% \sum_{x\in\Lambda} \left( \tau_{\Delta,x}
% + m^2  \tau_x \right).
% \end{equation}
and \eqref{e:generating-fn} follows.

Summation of \eqref{e:G-gauss} over $x\in \Lambda_N$ gives the formula
$\chi_N(\gcc,\gamma,\nu) = (1+z_0)\sum_{x\in \Lambda} \Ex_C (Z_0\phib_0\phi_x)$.
Call the right-hand side $\hat\chi_N(\gcc,\gamma,\nu)$. To show that this is
consistent with \eqref{e:chibarm}, begin by noting that
\begin{equation}
\hat\chi_N(\gcc,\gamma,\nu)
	=
|\Lambda|^{-1} \frac{D^2 \Sigma(0; \1, \1)}{Z_N(0)},
\end{equation}
where
\begin{equation}
\Sigma(J, \bar J) = \Ex_C (Z_0 e^{J \cdot \phib + \phi \cdot \bar J}).
\end{equation}
Completing the square yields
\begin{equation}
\Sigma(J, \bar J)
	=
e^{J \cdot C \bar J} Z_{N,\varnothing}(C J, C \bar J)
\end{equation}
and differentiating this expression gives
\begin{equation}
D^2 \Sigma(0; \1, \1)
	=
(\1, C \1) + D^2 Z_{N,\varnothing}(0; C\1, C\1)
\end{equation}
The result then follows from the fact that
\begin{equation}
C \1 = A^{-1} \1 = m^{-2} \1.
\end{equation}
\end{proof}

%%%%%%%%%%%%%%%%%%%%%%%%%%%%%%%%%%%%%%%%%%%%%%%%%%%%%%%%%%%%%%%%%%%%%%%%%%%%%%%
%%%%%%%%%%%%%%%%%%%%%%%%%%%%%%%%%%%%%%%%%%%%%%%%%%%%%%%%%%%%%%%%%%%%%%%%%%%%%%%

\section{Progressive integration}
% Based on both
\label{sec:prog}

By Proposition~\ref{prop:intrep}, our task is to understand the Gaussian expectation
$Z_N = \Ex_C Z_0$ and its derivatives to leading order, uniformly in the volume
$\Lambda_N$ and the mass $m^2$ near $0$.

We proceed using the covariance decomposition
\begin{equation}
\label{e:NCj}
C = C_1 + \cdots + C_{N-1} + C_{N,N}
\end{equation}
constructed in \cite{Baue13a}; a similar decomposition was also constructed in \cite{BGM04}.
The covariances $C_1, \ldots, C_{N-1}$ are independent of the volume $\Lambda_N$. The final
covariance $C_{N,N}$ \emph{does} depend on the volume; so, for instance, $C_{N,N} \ne C_{N,N+1}$.
Nevertheless, we will often write $C_N \coloneqq C_{N,N}$ for simplicity when the volume is implicit.

The covariances $C_j$ have the following important \emph{finite-range property}:
\begin{equation}
C_{j;xy} = 0 \text{ if } |x - y| \ge \tfrac12 L^j.
\end{equation}
Thus, $\zeta$ is a Gaussian field with covariance $C_j$, then $\zeta_x$ is independent
of $\zeta_y$ whenever $|x - y| \ge \tfrac12 L^j$. In particular,
if $F_x, F_y$ are functions of the fields in $x, y$, respectively, then
\begin{equation}
\label{e:uncorr}
\Ex_{C_{j+1}} (F_x F_y) = (\Ex_{C_{j+1}} F_x) (\Ex_{C_{j+1}} F_y).
\end{equation}
In addition, we have the following covariance bounds (this is a restatement of
\cite[Proposition~\ref{pt-prop:Cdecomp}(a)]{BBS-rg-pt}).

\begin{prop}
\label{prop:Cdecomp}
  Let $d >2$, $L\geq 2$, $j \ge 1$, $\bar m^2 >0$.
  For multi-indices $\alpha,\beta$ with
  $\ell^1$ norms $|\alpha|_1,|\beta|_1$ at most
  some fixed value $p$,
  and for any $k$, and for $m^2 \in [0,\bar m^2]$,
  \begin{equation}
    \label{e:scaling-estimate}
    |\nabla_x^\alpha \nabla_y^\beta C_{j;x,y}|
    \leq c(1+m^2L^{2(j-1)})^{-k}
    L^{-(j-1)(d-2 +|\alpha|_1+|\beta|_1)},
  \end{equation}
  where $c=c(p,k,\bar m^2)$ is independent of $m^2,j,L$.
  The same bound holds for $C_{N,N}$ if
  $m^2L^{2(N-1)} \ge \varepsilon$ for some $\varepsilon >0$,
  with $c$ depending on $\varepsilon$ but independent of $N$.
\end{prop}

% By Proposition~\ref{prop:Cdecomp}, the decomposition \eqref{e:NCj} is a multiscale decomposition,
% in the sense that $C_{j+1} = O(L^{-1} C_j)$. Moreover, the covariances are approximately
% constant on blocks in the sense that $\nabla C_j = O(L^{-1} C_j)$.

It is a basic property  of the Gaussian distribution that a sum of independent Gaussian random
variables with covariances
$C'$ and $C''$ is itself Gaussian with covariance $C' + C''$. It follows that for any
boson field $F$,
\begin{equation}
\Ex_{C'+C''}\theta F = \Ex_{C'}\theta \circ \Ex_{C''}\theta F.
\end{equation}
% When $C' = s C_0$ and $C'' = t C_0$ for some covariance $C_0$ and $s, t > 0$,
% this is essentially a restatement of the semigroup property of the heat kernel.
This extends to any sufficiently smooth form $F$ (see \cite{BS-rg-norm}).
It follows that
\begin{equation}
\label{e:prog-int}
Z_N =
\Ex_{C_N}\theta \circ \Ex_{C_{N-1}}\theta \circ \ldots \circ \Ex_{C_1}\theta Z_0.
\end{equation}
We define the \emph{renormalisation group map} $Z_j \mapsto Z_{j+1}$ by
\begin{equation}
\label{e:rgmapZ}
Z_{j+1} = \Ex_{C_{j+1}} \theta Z_j, \quad j < N.
\end{equation}

\begin{rk}
This is truly a ($j$-dependent) map and not just a fixed sequence of field
functionals.  Indeed, recall that the initial condition $Z_0$ is not fixed:
it depends on the two free parameters $m^2$ and $z_0$. We think of the family
of possible sequences $Z_j$ as a (non-autonomous) dynamical system.

The key to understanding
$Z_N$ for large $N$ is the careful choice of \emph{critical} initial conditions
$(m^2, z_0)$. In a sense, these initial conditions, which are functions of
$(g, \gamma, \nu)$, define the stable manifold for the renormalisation group
and the fixed point for this stable manifold is the Gaussian measure with covariance
$(1 + z_0) (-\Delta + m^2)^{-1}$.

We have scaled out the factor $1 + z_0$
in the change of variables performed in the proof of Proposition~\ref{prop:intrep}.
This factor plays a similar role to the standard deviation $\sigma$ in the central
limit theorem: rather than a unique fixed point, there is a one-parameter family
of fixed points.
\end{rk}

%%%%%%%%%%%%%%%%%%%%%%%%%%%%%%%%%%%%%%%%%%%%%%%%%%%%%%%%%%%%%%%%%%%%%%%%%%%%%%%
%%%%%%%%%%%%%%%%%%%%%%%%%%%%%%%%%%%%%%%%%%%%%%%%%%%%%%%%%%%%%%%%%%%%%%%%%%%%%%%

\section{The space of field functionals}
\label{sec:Ncal}

For the analysis of the dynamical system \eqref{e:rgmapZ}, we require a suitable
space on which this system evolves.

Let $\Ncal^\varnothing$ be defined as in Section~\ref{sec:intrep} if $n = 0$ and
\begin{equation}
\label{e:Ncaldef}
\Ncal^\varnothing
	= \Ncal^\varnothing(\Lambda)
	= C^{p_\Ncal}((\R^n)^\Lambda,\R)
\end{equation}
if $n \ge 1$. Recall that $p_\Ncal$ is the smoothness parameter discussed in
Section~\ref{sec:intrep}.
% In Section~\ref{sec:newnorm},
% we explain that $p_\Ncal$ must be chosen in a way that depends on the
% parameter $p$ in Theorem~\ref{thm:mr}(iii).

We extend $\Ncal^\varnothing$ to a space $\Ncal$
that includes functions of the observable fields $\sigma_0$ and $\sigma_x$.
Recalling \eqref{e:generating-fn}, this space is defined in such a way that
functions of the observable fields are identified to second order. A precise
definition is given in
in \cite[Section~\ref{phi4-sec:phi4observables_representation}]{ST-phi4}.
The upshot is that every $F \in \Ncal$ has the form
\begin{equation}
\label{e:obs-decomp}
F = F_\varnothing + F_a + F_b + F_{ab},
	\quad
F_\alpha \in \Ncal^\varnothing.
\end{equation}
There are natural projections $\pi_\alpha : \Ncal \to \Ncal_\alpha$ with
$\alpha = \varnothing, a, b, ab$ such that $\pi_\alpha F = F_\alpha$.
For $X \subset \Lambda$, we let $\Ncal(X)$ denote the subset of $\Ncal$ consisting
of field functionals that only depend on fields in $X$.

In order to control the evolution of $Z_j$ on $\Ncal$, we make use of a family
$\|\cdot\|_{T_{\phi,j}(\h_j)}$ of scale-dependent dependent seminorms depending on a
sequence of weights $\h_j > 0$; the field $\phi$ lies in $\C^\Lambda$ if $n = 0$ and
$(\R^n)^\Lambda$ if $n \ge 1$. For convenience,
we will simply write $\|\cdot\|_{T_\phi(\h_j)}$ with the scale $j$ implied by the
choice of parameter $\h_j$. The precise definitions are given below.

%%%%%%%%%%%%%%%%%%%%%%%%%%%%%%%%%%%%%%%%%%%%%%%%%%%%%%%%%%%%%%%%%%%%%%%%%%%%%%%

\subsection{Test functions}

We define the $T_\phi$ seminorm for $n = 0$. The case $n \ge 1$ involves only
minor changes, which we describe in Remark~\ref{rk:Tphi-n}.

Recall the notation introduced in Section~\ref{sec:forms}.
A \emph{test function} $g$ is defined to be a function $(\vec x, \vec y) \mapsto g_{\vec x,\vec y}$,
where $\vec x$ and $\vec y$ are finite sequences of elements in $\Lambda \sqcup \bar\Lambda$.
When $\vec x$ or $\vec y$ is the empty sequence $\varnothing$,
we drop it from the notation as long as this causes no confusion;
e.g., we may write $g_{\vec x} = g_{\vec x,\varnothing}$.
The length of a sequence $\vec x$ is denoted $|\vec x|$.
Gradients of test functions are defined component-wise.
Thus, if $\vec x = (x_1, \ldots, x_m)$
and $\alpha = (\alpha_1, \ldots, \alpha_m)$
with each $\alpha_i \in \N_0^\Ucal$, and similarly for $\vec y=(y_1,\ldots,y_n)$ and
$\beta=(\beta_1,\ldots,\beta_n)$,
then
\begin{equation}
\nabla^{\alpha,\beta}_{\vec x,\vec y} g_{\vec x,\vec y}
  =
\nabla^{\alpha_1}_{x_1} \ldots \nabla^{\alpha_m}_{x_m}
\nabla^{\beta_1}_{y_1} \ldots \nabla^{\beta_n}_{y_n}  g_{x_1,\ldots,x_m,y_1,\ldots,y_n}.
\end{equation}

We fix a positive constant $p_\Phi\ge 4$ and restrict our attention to test functions
that vanish when $|\vec x|  +|\vec y| > p_\Ncal$.
% For Theorem~\ref{thm:suscept}(i-ii), any choice of $p_\Ncal \ge 10$ is sufficient,
% whereas for Theorem~\ref{thm:suscept}(iii) it is necessary to choose $p_\Ncal$ large
% depending on $p$ \cite{BSTW-clp}.
The $\Phi_j = \Phi(\h_j)$ norm on such test functions is defined by
\begin{equation}
\|g\|_{\Phi_j}
	=
\sup_{\vec x, \vec y} \h_j^{-(|\vec x| +|\vec y|)}
	\shift\shift
\sup_{\alpha,\beta: |\alpha|_1+|\beta|_1 \le p_\Phi}
L^{j (|\alpha|_1 + |\beta|_1)}
|\nabla^{\alpha,\beta} g_{\vec x, \vec y}|,
\end{equation}
where $|\alpha|_1$ denotes the total order of the differential operator $\nabla^\alpha$.
Thus, for any test function $g$ and for sequences
$\vec x, \vec y$ with $|\vec x| +|\vec y| \leq p_\Ncal$ and
corresponding $\alpha, \beta$ with $|\alpha|_1 + |\beta|_1 \leq p_\Phi$,
\begin{equation}
\label{e:testfcnbd}
|\nabla^{\alpha,\beta} g_{\vec x,\vec y}|
	\leq
\h_j^{|\vec x| + |\vec y|} L^{-j (|\alpha|_1 + |\beta|_1)} \|g\|_{\Phi_j}.
\end{equation}

%%%%%%%%%%%%%%%%%%%%%%%%%%%%%%%%%%%%%%%%%%%%%%%%%%%%%%%%%%%%%%%%%%%%%%%%%%%%%%%

\subsection{The \texorpdfstring{$T_\phi$}{Tphi} seminorm}
\label{sec:Tphi}

If $n = 0$, then for any $F \in \Ncal^\varnothing$, there are
\emph{unique} functions $F_{\vec y}$ of $(\phi, \bar\phi)$
that are anti-symmetric under permutations of $\vec y$, such that
\begin{equation}
F = \sum_{\vec y} \frac{1}{|\vec y|!} F_{\vec y}(\phi, \bar\phi) \psi^{\vec y}.
\end{equation}
Given a sequence $\vec{x}$ with $|\vec{x}| = m$, we define
\begin{equation}
F_{\vec x, \vec y} = \ddp{^m F_{\vec y}}{\phi_{x_1} \ldots \partial\phi_{x_m}}.
\end{equation}
We define a $\phi$-dependent pairing of elements of $\Ncal$ with test functions, by
\begin{equation}
\label{e:pairing}
\langle F, g \rangle_\phi
  =
\sum_{\vec x, \vec y}
\frac{1}{|\vec x|! |\vec y|!}
F_{\vec x,\vec y}(\phi, \bar\phi)
g_{\vec x,\vec y}.
\end{equation}

Let $B(\Phi)$ denote the unit $\Phi$-ball in the space of test functions. Then the
$T_\phi = T_\phi(\h_j)$ semi-norm on $\Ncal^\varnothing$ is defined by
\begin{equation}
\label{e:Tphi-def}
\|F\|_{T_\phi} = \sup_{g\in B(\Phi_j)} |\langle F, g \rangle_\phi|.
\end{equation}

\begin{rk}
\label{rk:Tphi-n}
If $n \ge 1$, a test function is a function $g$ on sequences over $\Lambda\times\{1,\ldots,n\}$.
For any such sequence $\vec x = ((x_1, i_1), \ldots, (x_m, i_m))$, we write $|\vec x| = m$
and set
\begin{equation}
F_{\vec x}
	=
\ddp{^m F}{\varphi^{i_1}_{x_1} \ldots \partial\varphi^{i_m}_{x_m}}
\end{equation}
and
\begin{equation}
\langle F, g \rangle_\varphi
	=
\sum_{\vec x} \frac{1}{|\vec x|!} F_{\vec x}(\varphi) g_{\vec x}.
\end{equation}
Then the $T_\varphi$ seminorm can be defined as in \eqref{e:Tphi-def}.
\end{rk}

To extend the $T_\phi$ seminorm to $\Ncal$, we make use of an additional sequence
of parameters $\h_{\sigma,j}$. For any $F \in \Ncal$ of the form \eqref{e:obs-decomp},
we let
\begin{equation}
\|F\|_{T_\phi}
	=
\|F_\varnothing\|_{T_\phi}
	+ (\|F_a\|_{T_\phi} + \|F_b\|_{T_\phi}) \h_\sigma
	+ \|F_{ab}\|_{T_\phi} \h_\sigma^2.
\end{equation}

By its definition, the $T_\phi$ seminorm controls the values of $F$ and its derivatives
(up to order $p_\Ncal$) at $\phi$. For instance, we will make use of the following facts.

\begin{lemma}
\label{lem:deriv-norm-bds}
For $F \in \Ncal$, we have $|F_\varnothing(0)| \le \|F\|_{T_0}$ and
\begin{equation}
\label{e:deriv-norm-bd}
|D^2 F_\varnothing(0; \1, \1)|
	\le
2 \|F\|_{T_0(\h_j)} \|\1\|^2_{\Phi_N(\h_j)}
	=
2 \|F\|_{T_0(\h_j)} \h_j^{-1}
\end{equation}
and
\begin{equation}
|D^2_{\sigma_0\sigma_x} F_\varnothing|
	\le
\h_{\sigma,j}^{-2} \|F\|_{T_0}.
\end{equation}
\end{lemma}

An essential property of the $T_\phi$ seminorm is the following \emph{product property},
which is essential to fully take advantage of the factorization property \eqref{e:uncorr}.

\begin{prop}
\label{prop:prod}
If $F, G \in \Ncal$, then $\|F G\|_{T_\phi} \le \|F\|_{T_\phi} \|G\|_{T_\phi}$.
\end{prop}

\begin{rk}
This follows essentially from the fact that
the series expansion of the product of two functions is the product of their
respective series expansions (see \cite{BS-rg-norm}). This is why the $T_\phi$
seminorm was defined in terms of the pairing \eqref{e:pairing}.
\end{rk}

%%%%%%%%%%%%%%%%%%%%%%%%%%%%%%%%%%%%%%%%%%%%%%%%%%%%%%%%%%%%%%%%%%%%%%%%%%%%%%%

\subsection{Norm parameters}

Control of the $T_\phi$ seminorm is needed for all values of
$\phi$ in order to control the Gaussian expectation in \eqref{e:rgmapZ}.
This will be discussed further in Section~\REF.

For now, we turn our attention to the special case of the $T_0$ seminorm. Recalling
\eqref{e:exp-conv}, it is natural to choose the weights $\h_j$ so that
$\Ex_{C_{j+1}} F$ is of order $\|F\|_{T_0(\h_j)}$.
By Wick's theorem \eqref{e:wick}, for a $1$-component field $\varphi$,
\begin{equation}
\Ex_{C_{j+1}} \varphi_x^{2p} = (2p - 1)!! C_{j+1;00}^p.
\end{equation}
On the other hand, by definition of the $T_0$ seminorm,
\begin{equation}
\label{e:gauss-moments}
\|\varphi_x^{2p}\|_{T_0(\h_j)} \asymp \h_j^p.
\end{equation}
This suggests defining $\h_j$ so that $|C_{j+1;00}| \le O(\h_j)$.
% Bounds on the covariance were stated in \eqref{e:scaling-estimate}.
% For instance, with $k = 0$, these become
% \begin{equation}
% \label{e:massless-cov-bd}
% |C_{j;xy}| \le O(L^{-j (d - 2)}).
% \end{equation}

The key to our analysis of the correlation length is that we make a choice of norm
weights that takes full advantage of the
$k$-dependence in the covariance bounds \eqref{e:scaling-estimate}.
With $k = s + 1$, this estimate together with the elementary bound
\begin{equation}
\label{e:mass-decay}
(1 + m^2 L^{2j})^{-k} \le c_L L^{-2(s+1)(j - j_m)_+}
\end{equation}
imply that
\begin{equation}
|C_{j;xy}| \le O(L^{-j (d - 2) - s (j - j_m)_+}),
\end{equation}
where $j_m$ is the \emph{mass scale}, defined by
\begin{equation}
\label{e:jmdef}
j_m	= \lfloor\log_{L} m^{-1}\rfloor.
\end{equation}
Based on this, we define the following weights:
\begin{align}
\label{e:elldef-zz}
\ell_j &= \ell_0 L^{-j - s (j - j_m)_+}, \quad
\ell_{\sigma,j}
=
\ell_{j \wedge j_{x}}^{-1} 2^{(j - j_{x})_+} \ggen_j,
\end{align}
where
\begin{equation}
\label{e:jxdef}
j_x = \max\{0,\lfloor \log_{L} (2 |x|)\rfloor\}
\end{equation}
is the \emph{coalescence scale}
and the sequence $\ggen_j = \ggen_j(m^2,g_0)$ will be discussed in Section~\ref{sec:step}.
% For now, we remark only that it is bounded above and below by constant multiples of
% the sequence $\gbar$ defined in
% \eqref{e:gbar}, by \cite[Lemma~\ref{log-lem:gbarmcomp}]{BBS-saw4-log}.
We will discuss the origin of the definition \refeq{elldef-zz} in detail
in Section~\ref{sec:Rpf1}.

We will set $\h_j = \ell_j$ to estimate ``small'' fields. These are fields which
are assumed not to deviate too much from their expected value. A different norm
parameter $\h_j = h_j$ will be used to control ``large'' fields.
This will be discussed in \REF.

%%%%%%%%%%%%%%%%%%%%%%%%%%%%%%%%%%%%%%%%%%%%%%%%%%%%%%%%%%%%%%%%%%%%%%%%%%%%%%%

\subsection{Symmetries}

It is useful to restrict our attention to field functionals $F \in \Ncal$ that
obey certain symmetry conditions preserved by Gaussian expectation (and which
are obeyed by $V_0$).

We let any automorphism $E$ of $\Lambda$ act on $\Ncal$ by $EF(\varphi) = F(E\varphi)$.
We say that $F\in\Ncal$ is \emph{Euclidean invariant} if $EF = F$ for all such automorphisms.

Let $n = 0$.
Define the \emph{supersymmetry generator}
\begin{equation}
Q = (2\pi i)^{1/2} \sum_{x\in\Lambda}
\left(
	\psi_x \ddp{}{\phi_x} + \psib_x \ddp{}{\phib_x}
		-
	\phi_x \ddp{}{\psi_x} + \phib_x \ddp{}{\psib_x}.
\right)
\end{equation}
A form $F \in \Ncal$ is said to be \emph{supersymmetric} if $Q F = 0$.
Such a form is said to be \emph{gauge invariant} if it is invariant under the
\emph{gauge flow} $(q, \bar q) \mapsto (e^{-2\pi it} q, 2^{2\pi it} \bar q)$
for $q = \phi_x, \psi_x$ and all $x\in\Lambda$.

Let $n \ge 1$.
An $n \times n$ matrix $T$ acts on $\Ncal$ via $T F(\varphi) = F(T \varphi)$.
We say that $F\in\Ncal$ is \emph{$O(n)$ invariant} if $TF = F$ for all
orthogonal matrices $T$.

%%%%%%%%%%%%%%%%%%%%%%%%%%%%%%%%%%%%%%%%%%%%%%%%%%%%%%%%%%%%%%%%%%%%%%%%%%%%%%%
%%%%%%%%%%%%%%%%%%%%%%%%%%%%%%%%%%%%%%%%%%%%%%%%%%%%%%%%%%%%%%%%%%%%%%%%%%%%%%%

\section{Perturbative coordinate}

As mentioned in Section~\ref{rg-intro}, one of Wilson's key insights was that the renormalisation
group could be well-approximated by a finite-dimensional dynamical system. In this
section, we reformulate Wilson's insights in terms of the covariance decomposition
and define a subspace on which this finite-dimensional system will evolve.

% The dynamical system is analysed via a perturbative part which is tracked accurately
% to second order in $g$, together with a third-order non-perturbative part whose study
% forms the main part of our effort.  For the perturbative part, we first introduce
% an appropriate space of local field polynomials.

%%%%%%%%%%%%%%%%%%%%%%%%%%%%%%%%%%%%%%%%%%%%%%%%%%%%%%%%%%%%%%%%%%%%%%%%%%%%%%%

\subsection{Dimensional analysis}

% We define the \emph{scaling dimension}
% \begin{equation}
% [\varphi] = \frac{d - 2}{2}.
% \end{equation}
We call $M_x \in \Ncal$ a local monomial if it is a monomial in $\varphi_x$ and
its (discrete) gradients, i.e.\ if $M_x$ has the form
\begin{equation}
M_x = (\nabla^{\alpha_1} \varphi_x) \ldots (\nabla^{\alpha_p} \varphi_x).
\end{equation}
The $T_0$ seminorm of a local monomial $M_x$ essentially
just counts the number of fields and derivatives in $M_x$. For instance, let
$\varphi$ be a $1$-component Gaussian field with covariance $C_j$. Then for $M_x$
as above,
\begin{equation}
\|M_x\|_{T_0(\ell_j)}
	=
O(L^{-j (|\alpha| + p [\varphi])})
\end{equation}
where $|\alpha| = |\alpha_1| + \cdots + |\alpha_p|$ and
\begin{equation}
[\varphi] = \frac{d - 2}{2}
\end{equation}
is the \emph{scaling dimension}. Based on this observation, we define the
\emph{dimension} of $M_x$ by
\begin{equation}
[M_x] = |\alpha| + p [\varphi].
\end{equation}
Note here that we have neglected the rapid decay of fields above the mass scale.

By \eqref{e:scaling-estimate}, $\varphi$ is approximately constant on blocks of side $L^j$. In a sense,
the fields on a block act as a unit and this contributes to a volume factor $L^{jd}$.
This leads us to compare the dimension of a monomial with the dimension $d$ of the
lattice. We say that $M_x$ is \emph{relevant} if $[M_x] < d$, \emph{marginal} if
$[M_x] = d$, and \emph{irrelevant} if $[M_x] > d$.

%%%%%%%%%%%%%%%%%%%%%%%%%%%%%%%%%%%%%%%%%%%%%%%%%%%%%%%%%%%%%%%%%%%%%%%%%%%%%%%

\subsection{Local field polynomials}
% Based on clp

For $y \in \Lambda$, we supplement \eqref{e:taudef}--\eqref{e:nablatau} and \refeq{tauphi}
by defining
\begin{equation}
\label{e:tauphi2}
\quad \tau_{\nabla\nabla,y}
	=
\begin{cases}
\frac 12 \sum_{e \in \units}
\left(
	(\nabla^e \phi)_y (\nabla^e \bar\phi)_y +
	(\nabla^e \psi)_y (\nabla^e \bar\psi)_y
\right),
	& n = 0 \\
\frac{1}{4} \sum_{|e| = 1} \nabla^e \varphi_y \cdot \nabla^e \varphi_y,
	& n \ge 1.
\end{cases}
\end{equation}

When $n = 0$, it can be shown that the only marginal and relevant local monomials
that are Euclidean invariant and supersymmetric are constant multiples of
\begin{equation}
1, \quad \tau_x, \quad \tau_x^2, \quad \tau_{\Delta,x}, \quad \tau_{\nabla\nabla,x}.
\end{equation}
When $n \ge 1$, these are the only marginal and relevant monomials that are Euclidean
invariant and $O(n)$-invariant (see \cite{BBS-rg-pt}).

The marginal and relevant contributions to the evolution of the renormalisation group
will be tracked by a \emph{local polynomial} (a sum of local monomials) of the form
$\sum_{x\in\Lambda} \Vc_y$, where (recall \eqref{e:obs-couple})
\begin{align}
\lbeq{Vy}
\Vc_y
	&=
g \tau_y^2 + \nu \tau_y + z \tau_{\Delta,y}
	% + y \tau_{\nabla\nabla,y}
	+ u
		\nnb&\quad
	- \1_{y=\pp}\lambda_{\pp} f_0 \sigma_0
	- \1_{y=\qq}\lambda_{\qq} f_x \sigma_x
		\nnb&\quad
	- \textstyle{\frac 12} (\1_{y=\pp} q_\pp + \1_{y=\qq}q_\qq )\sigma_\pp\sigma_\qq.
\end{align}
We have omitted $\tau_{\nabla\nabla}$ as summation by parts gives
\begin{equation}
\label{e:nabla-delta}
\sum_{x\in\Lambda} \tau_{\nabla\nabla,x} = \sum_{x\in\Lambda} \tau_{\Delta,x}.
\end{equation}

We define $\Vcalc$ to be the space of all polynomials of the form $\Vc_y$.
Given $X \subset \Lambda$, we let
\begin{equation}
\label{e:Vcalesig}
\Vcalc(X) = \{\Vc(X) : \Vc \in \Vcalc \},
\end{equation}
where $\Vc(X)$ is defined as in \eqref{e:VX}.
We also make use of the % subspaces $\Vcalc^{(1)} \subseteq \Vcalc$ consisting
% of polynomials with $y = 0$, as well as the
subspace $\Vcalp$ of polynomials with
$u = y = q_\pp=q_\qq = 0$.
We will usually denote an element of $\Vcalp$ as $\Vp$.
For $\Vc \in \Vcalc$, we define % maps $\Vc \mapsto \Vc^{(1)} \in \Vcalc^{(1)}$ and
the map $\Vc \mapsto \Vc^{(0)} \in \Vcalp$, which sets
% Both maps replace $z\tau_{\Delta}+y\tau_{\nabla\nabla}$
% by $(z+y)\tau_{\Delta}$, and the latter additionally
$u = q_\pp = q_\qq = 0$.

We define the $\Vcalc = \Vcalc_j$ norm by
\begin{equation}
\label{e:Vnormdef}
\begin{aligned}
\|\Vc\|_{\Vcalc} &=
\max\Big\{
|g|, L^{2j}|\nu|, |z|, |y|,  L^{4j}|u|,
\ell_j\ell_{\sigma,j}(|\lambda_\pp|\vee|\lambda_\qq|),\;
%\\
%& \qquad\qquad\qquad
 \ell_{\sigma,j}^{2} (|q_\pp|\vee|q_\qq|)
\Big\}
\end{aligned}
\end{equation}
on $\Vc \in \Vcalc$, which depends on the parameters $\ell_j$ and $\ell_{\sigma,j}$.
The $\Vcalc = \Vcalc_j$ norm is equivalent to the $T_0(\ell_j)$ seminorm on $\Vcalc(B)$
when $|B| = L^{jd}$:
\begin{equation}
\|\Vc\|_\Vcalc \asymp \|\Vc(B)\|_{T_0(\ell_j)} = L^{jd} \|\Vc_y\|_{T_0(\ell_j)}.
\end{equation}

%%%%%%%%%%%%%%%%%%%%%%%%%%%%%%%%%%%%%%%%%%%%%%%%%%%%%%%%%%%%%%%%%%%%%%%%%%%%%%%

\subsection{Perturbative flow}
\label{sec:pt}

\commentbw{Work on motivating $\Vpt$ and $I$}

In \cite{BBS-rg-pt}, a
map\footnote{In \cite{BBS-rg-pt}, $\Vpt$ is defined on a larger space including
$\tau_{\nabla\nabla}$. Here, following \eqref{e:nabla-delta}, we define $\Vpt$ by composing that
map with the map that replaces $z \tau_\Delta + y \tau_{\nabla\nabla}$ by
$(z + y) \tau_\Delta$.}
$\Vpt : \Vcalp\to\Vcalc$
is defined so as to maintain the approximate form
\begin{equation}
Z_j \approx e^{-\Vc_j} (1 + O(\Vp_j^2)).
\end{equation}
Precisely, $\Vpt$ depends on a covariance $C$ and satisfies
\begin{equation}
\Ex_C\theta e^{-V(\Lambda)} (1 + W(\Vp))
	=
e^{-\Vpt(\Lambda)} (1 + W(\Vpt^{(0)})) + O(\Vp^3),
\end{equation}
where $W = W(V)$ is an explicit polynomial that is quadratic in $V\in\Vcalp$; see
\cite[\eqref{pt-e:WLTF}]{BBS-rg-pt} for its definitin, which also depends on $C$.
\todo{Explain $O(V^3)$.}

To motivate the definition of $\Vpt$, note that, by the \emph{cumulant expansion},
\begin{equation}
\Ex_C \theta e^{-V}
	\approx
e^{-\Ex_C \theta V + \tfrac12 \Ex_C (\theta V; \theta V)}.
\end{equation}
The polynomial in the exponent on the right-hand side is not local.
In Section~\REF, we will discuss an operator $\Loc_x$ for approximating a non-local
polynomial by a local polynomial in $\Vcal$ at $x$.

In practice, we set $C = C_{j+1}$ and obtain a sequence $\Vpt = \Vc_{\mathrm{pt},j+1}$
of maps and $W_j$ of polynomials. By \cite[\eqref{IE-e:W-logwish}]{BS-rg-IE},
\todo{(check this)},
\begin{equation}
\label{e:Wbilinbd}
\|W_j\|_{T_0(\ell_j)}
	\le
O(\chicCov_j) \|V\|_\Vcal^2,
\end{equation}
where $\chicCov_j$ is a parameter that decays exponentially above the mass scale
and will be discussed in Section~\ref{sec:step}.
The maps $\Vp \mapsto \Vpt$
generate a sequence of couplings constants that we refer to as the
\emph{perturbative flow}. The equations defining this flow can be
computed exactly by way of Feynman diagrams.

\subsubsection{The flow of \texorpdfstring{$g$}{g}}

Following an approximate change of coordinates with $O(\Vp^3)$ errors, the perturbative
flow of $g$ takes the form
\begin{equation}
\label{e:gbar}
\gbar_{j+1}
	=
\gbar_j - \beta_j  \gbar_j^{2}, \qquad \gbar_0
	=
g_0
\end{equation}
where
\begin{equation}
\beta_j = (8 + n) \sum_{x\in\Zd} (w_{j+1;0x}^2 - w_{j;0x}^2),
	\quad
w_j = \sum_{i=1}^j C_i.
\end{equation}
The sequence $\beta_j$ is closely related to the free bubble diagram,
which is the $\ell_2(\Zd)$ norm of the Green function $C$ (this is
the expected number of intersections of two independent, killed simple
random walks).

The recursion \eqref{e:gbar} was analyzed in \cite{BBS-rg-pt}. It was shown in
\cite[Proposition~\ref{log-prop:approximate-flow}]{BBS-saw4-log}
that
\begin{equation}
\label{e:gjxgjmbd}
\gbar_{j}
	=
O((\log m^{-1})^{-1}) \;\; \text{for $j \geq j_m$},
	\quad
\gbar_{j_x}
	=
O((\log |x|)^{-1}) \;\; \text{for $j_x \leq j_m$.}
\end{equation}

\begin{rk}
A heuristic argument is as follows: Using Proposition~\ref{prop:Cdecomp}, it is
straightforward to show that
\begin{equation}
\beta_j = O(L^{-j (d - 4)}).
\end{equation}
Thus, a crude approximation to the flow of $\gbar$ is the recursion
\begin{equation}
x_{j+1} = x_j - c \1_{j \le j_m} x_j^2,
	\quad
c > 0.
\end{equation}
Comparing this to the differential equation $\dot x = - c x^2$, which has solutions
of the form $x(t) = (C + c t)^{-1}$, it is reasonable to expect that $x_j \approx (c j)^{-1}$
for $j \le j_m$ and $x_j \approx x_{j_m}$ for $j > j_m$. The relations \eqref{e:gjxgjmbd}
follow easily if $g_j$ behaves in a similar way.
\end{rk}

\subsubsection{The flow of \texorpdfstring{$\lambda$ and $q$}{lambda and q}}

It was shown in \cite[\eqref{pt-e:lambdapt2}--\eqref{pt-e:qpt2}]{BBS-rg-pt} (for $n = 0$)
and \cite[Proposition~\ref{phi4-prop:pt}]{ST-phi4} (for $n \ge 1$) that,
with $C = C_{j+1}$ and $u = 0, x$,
\begin{align}
\label{e:lampt}
\lambda_{u,\pt}
	&=
\begin{cases}
(1 - \delta[\nu w^{(1)}]) \lambda_u,
	& j + 1 < j_x \\
\lambda_u,
	& j + 1 \ge j_x
\end{cases}
	\\
\label{e:qpt}
q_\pt
	&=
q + \lambda_0 \lambda_x C_{0x},
\end{align}
where
\begin{equation}
\label{e:deltanuw1}
\delta[\nu w^{(1)}] = (\nu + 2 g C_{00}) w^{(1)}_{j+1} - \nu w^{(1)}_j,
	\qquad
w^{(1)}_j = \sum_{x\in\Lambda} \sum_{i=1}^j C_{i;0x}.
\end{equation}
Note that $q_\pt = q$ for $j + 1 < j_x$.
% \todo{It's strange that $w$ appears in the flow of $\lambda$ when $\Vpt$ only
% depends on $C$.}

%%%%%%%%%%%%%%%%%%%%%%%%%%%%%%%%%%%%%%%%%%%%%%%%%%%%%%%%%%%%%%%%%%%%%%%%%%%%%%%
%%%%%%%%%%%%%%%%%%%%%%%%%%%%%%%%%%%%%%%%%%%%%%%%%%%%%%%%%%%%%%%%%%%%%%%%%%%%%%%

\section{Non-perturbative coordinate}
% Based on clp
\label{sec:rgcoord}

Let $\volume$ denote either $\Lambda_N$ or $\Zd$. We allow $\Ncal$ to depend on
$\volume$. If $\volume = \Lambda$, then $\Ncal = \Ncal(\Lambda)$ was defined in
Section~\ref{sec:Ncal}. Otherwise, we set
\begin{equation}
\Ncal(\Zd) = \bigcup_{\text{finite } X \subset \volume} \Ncal(X).
\end{equation}

We set $N(\volume) = N$ if
$\volume = \Lambda_N$ and $N(\volume) = \infty$ if $\volume = \Zd$.
For $j \le N(\volume)$ (meaning $j < \infty$ if $N(\volume) = \infty$), we partition
$\volume$ into disjoint
\emph{scale-$j$ blocks} of side length $L^j$, each of which is a translate of
the block $\{ x \in \Lambda : 0 \le x_i < L^j, i = 1, \ldots, d\}$.
A scale-$j$ \emph{polymer} is a union of scale-$j$ blocks.
Given a scale-$j$ polymer $X$ and $k \le j$, we let $\Bcal_k(X)$
(respectively, $\Pcal_k(X)$)
denote the set of all scale-$k$ blocks (respectively, scale-$k$ polymers) in $X$.
We sometimes write $\Bcal_j = \Bcal_j(\volume)$ and $\Pcal_j = \Pcal_j(\volume)$
when the volume $\volume$ is implicit.

Given maps $F, G : \Pcal_j(\Lambda) \to \Ncal$ (sometimes called \emph{polymer activities}),
we define the \emph{circle product} $F \circ G : \Pcal_j(\Lambda) \to \Ncal$ by
\begin{equation}
(F \circ G)(X) = \sum_{Y\in\Pcal_j(X)} F(X \setminus Y) G(Y).
\end{equation}
We track $Z_j$ using \emph{renormalisation group coordinates}
$u_j, q_{0,j}, q_{x,j} \in \R$,
$I_j, K_j : \Pcal_j \to \Ncal$, defined such that
\begin{equation}
\label{e:IcircKnew}
	Z_j = e^{\zeta_j}(I_j\circ K_j)(\Lambda),
	\qquad
	\zeta_j= - u_j|\Lambda|
	+ \textstyle{\frac 12} (q_{\pp,j} + q_{\qq,j}) \sigma_\pp\sigma_\qq
	.
\end{equation}
The coordinate $I_j = I_j(\Vp, \cdot)$ is defined by
\begin{equation}
I_j(\Vp, X)
	=
\prod_{B \in \Bcal_j(X)} e^{-\Vp(B)} (1 + W_j(B, \Vp)), \quad X \in \Pcal_j,
	\quad
\Vp \in \Vcalp.
\end{equation}

Before defining the space in which $K_j$ lies, we need the following notions:
\begin{itemize}
\item
We call a nonempty polymer $X\in \Pcal_j$ \emph{connected}
if for any $x, x' \in X$, there is a sequence
$x = x_0, \ldots, x_n = x' \in X$ such that
$|x_{i+1} - x_i|_\infty = 1$ for $i = 0, \ldots, n - 1$.
Let $\Ccal_0 = \Ccal_0(\volume)$ denote the set of connected polymers.

\item
For $X \in \Pcal_j$, let $|X|_j$ denote the number of scale-$j$ blocks in $X$.
We call a connected polymer $X\in\Ccal_j$ a \emph{small set} if $|X|_j \le 2^d$.
Let $\Scal_j = \Scal_j(\volume)$ denote the collection of small sets.
The \emph{small set neighbourhood} $X^\square$ of a polymer $X$ is defined by
\begin{equation}
\label{e:ssn}
X^\Box = \bigcup_{Y\in\Scal_j : Y \cap X \ne \varnothing} Y.
\end{equation}

\item
Two polymers $X, Y$ \emph{do not touch} if $\min(|x - y|_\infty : x \in X, y \in Y) > 1$.
We let ${\rm Comp}(X)$ denote the set of maximal connected components that do not touch
in $X$.
\end{itemize}

\begin{defn}
For $j \le N(\volume)$, let $\CKspace_j = \CKspace_j(\volume)$ denote the complex
vector space of maps $K : \C_j(\volume) \to \Ncal(\volume)$ satisfying the following
properties:
\begin{itemize}
\item
Field Locality: If $X \in \Ccal_j$, then $K(X) \in \Ncal(X^\square)$.
Also: (i) $\pi_\alpha K(X) = 0$ unless $\alpha \in X$ for $\alpha = a, b$;
(ii) $\pi_{ab} K(X) = 0$ unless $a\in X$ and $b \in X^\square$ or vice-versa;
and (iii) $\pi_{ab} K(X) = 0$ if $X \in \Scal_j$ and $j < j_x$.

\item
Symmetry: (i) $K$ is Euclidean invariant and, if $n = 0$, $K$ is gauge-invariant;
(ii) $\pi_\varnothing K$ is supersymmetric and has no constant part if $n = 0$
or $O(n)$ invariant if $n \ge 1$;
and (iii) $\pi_\varnothing K$ is Euclidean covariant.
\end{itemize}
We let $\Kspace_j = \Kspace_j(\volume)$ denote the complex vector space of functions
$K \in \CKspace_j$ with the following additional property:
\begin{itemize}
\item
Component factorization: If $X \in \Pcal_j$, then $K(X) = \prod_{Y\in{\rm Comp}(X)} K(Y)$.
\end{itemize}
\end{defn}

Addition in $\CKspace_j$ is defined by $(F_1 + F_2)(X) = F_1(X) + F_2(X)$.
We extend any $F \in \CKspace_j$ to $\Kspace_j$ by defining
$F(X) = \prod_{Y\in{\rm Comp}(X)} F(Y)$.

\subsection{Initial coordinates}

At scale $j = 0$, we are given $\Vp^+_0$ as defined in \eqref{e:V0def}
and we set $\zeta_0 = 0$. In particular,
the initial values of $u$, $q_0$, $q_x$ are zero, and the initial values of $\lambda_0$, $\lambda_x$
are $1$. By definition, $W_0 = 0$.
For $X \subset \Lambda$, we define
\begin{equation}
\label{e:IK0def}
I_0^+(X) = I_0(\Vp^+_0, X) = \prod_{x\in X} e^{-\Vp^+_{0,x}},
	\qquad
K_0^+(X) = \prod_{x \in X} I_{0,x}^+ (e^{-\gamma_0 U^{+}_{x}} - 1).
\end{equation}
It is straightforward to verify that $K_0 \in \Kcal_0$.
With these choices, $Z_0$ (recall \refeq{Z0def})
takes the form \eqref{e:IcircKnew}, and we seek
$(u_j, q_{0,j}, q_{x,j}, \Vp_j, K_j)$ such that this continues to hold as the scale advances.

Equivalently, given $(\Vp_j, K_j)$, we must define
$(\delta u_{j+1}, \delta q_{0,j+1}, \delta q_{x,j+1}, V_{j+1}, K_{j+1})$ so that
\begin{equation} \label{e:IcircKdu}
	\Ex_{j+1}\theta(I_j \circ K_j)(\Lambda)
	=
	e^{-\delta \zeta_{j+1}}(I_{j+1} \circ K_{j+1})(\Lambda).
\end{equation}
Moreover, we need $K_j$ to contract as the scale advances, under an appropriate norm.
The construction of (scale-dependent) maps $\Vc_+$ and $K_+$ such that
\eqref{e:IcircKdu} holds with
\begin{equation}
(\delta u_{j+1}, \delta q_{0,j+1}, \delta q_{x,j+1}, \Vp_{j+1})
	=
\Vc_+(\Vp_j, K_j),
	\quad
K_{j+1} =  K_+(\Vp_j, K_j)
\end{equation}
is the main accomplishment of \cite{BS-rg-step}.

%%%%%%%%%%%%%%%%%%%%%%%%%%%%%%%%%%%%%%%%%%%%%%%%%%%%%%%%%%%%%%%%%%%%%%%%%%%%%%%
%%%%%%%%%%%%%%%%%%%%%%%%%%%%%%%%%%%%%%%%%%%%%%%%%%%%%%%%%%%%%%%%%%%%%%%%%%%%%%%

\section{Renormalisation group step}
% from saw-sa and clp
\label{sec:step}

For fixed $(\mgen^2, \ggen_0) \in [0, \delta) \times (0, \delta)$,
we define a sequence $\ggen_j = \ggen_j(\mgen^2, \ggen_0)$ by
\begin{equation} \label{e:ggendef}
  \ggen_j(m^2,g_0) =
  \gbar_j(0,g_0) \1_{j \le j_m} + \gbar_{j_m}(0,g_0) \1_{j > j_m},
\end{equation}
where $\gbar_j = \gbar_j(m^2, g_0)$ is the sequence discussed in Section~\ref{sec:pt};
in particular, $\ggen_0(\mgen^2, \ggen_0) = \ggen_0$.
In \cite[Section~\ref{step-sec:Knorms}]{BS-rg-step},
a sequence of norms $\|\cdot\|_{\Wcal_j} = \|\cdot\|_{\Wcal_j(\mgen^2, \ggen_j, \Lambda)}$
parameterised by $(\mgen^2, \ggen_j)$ is defined on $\Kcal_j$.
These are defined in terms of the $T_\phi(\h_j)$ norms with parameters $\h_j = \ell_j, h_j$.
In order to make use of the improved norm parameters with $s > 0$,
we must modify the definition of $\Wcal_j$ when $j$ is above the mass scale.
This will be discussed in Chapter~\ref{sec:RGstep}.
We note here only the fact that (for any $s \ge 0$) the $\Wcal_j(\Lambda)$
norm dominates the $T_0(\ell_j)$ norm in the following sense:
\begin{equation}
\label{e:T0dom}
\|F(\Lambda)\|_{T_0(\ell_j)} \le \|F\|_{\Wcal_j}.
\end{equation}
We let $\Wcal_j = \Wcal_j(\volume)$ denote the space of $K\in\Kcal_j(\volume)$ with
finite $\Wcal_j$ norm and
denote the ball of radius $r$ in the normed space $\Wcal_j$ by $B_{\Wcal_j}(r)$.

In \cite[\eqref{log-e:mass-scale}--\eqref{log-e:chidef}]{BBS-saw4-log},
a function $\chicCov_j = \chicCov_j(m^2)$ (denoted $\chi_j$ in \cite{BBS-saw4-log})
is defined in such a way that $\chicCov_j$ decays exponentially
when $j$ is sufficiently large depending on $m$. We write $\chicCovgen_j = \chicCov_j(\mgen^2)$.
Given constants $\alpha > 0$ and $C_\DV > 0$, we define the (finite-volume)
renormalisation group domains
\begin{align}
\label{e:DVdef}
\DV_j
	&=
\{ \Vp\in \Vcalp :
	g > C_{\DV}^{-1} \ggen_j, \; \|\Vp\|_{\Vcalc} < C_{\DV} \ggen_j \}, \\
\label{e:domRG}
\domRG_j
	&= \domRG_j(\volume)
	= \DV_j \times B_{\Wcal_j}(\alpha \chicCovgen_j \ggen_j^3).
\end{align}
The domain $\DV_j$ is independent of the volume $\volume$ while $\domRG_j$
depends on $\volume$ through $\Wcal_j$.

In the statement of the following theorem, we fix the scale $j$ and
consider maps $\Vc_+ = \Vc_{j+1}$ and $K_+ = K_{j+1}$ that act on the domain
$\domRG_j$ and map into $\Vcalc_{j+1}$, $\Kcal_{j+1}$, respectively.
We will drop the scale $j$ from the notation for objects at scale $j$
and replace $j + 1$ with $+$.
The deviation of the map $\Vc_+$ from the perturbative map $\Vpt$
is denoted by $R_+$:
\begin{equation}
\label{e:Rplusdef}
    R_+(\Vp,K) = \Vc_+(\Vp,K) -\Vpt(\Vp).
\end{equation}

The renormalisation group map depends also on the mass $m^2$ through its
dependence on the covariance $C_{j+1}$.
We let $\Igen_j(\mgen^2)$ be the neighbourhood of $\mgen^2$ defined by
\begin{equation}
\lbeq{Itilint}
    \Igen_j = \Igen_j(\mgen^2) =
    \begin{cases}
    [\frac 12 \mgen^2, 2 \mgen^2] \cap \Iint_j & (\mgen^2 \neq 0)
    \\
    [0,L^{-2(j-1)}] \cap \Iint_j & (\mgen^2 =0)
    \end{cases},
\end{equation}
where $\Iint_j = [0, \delta]$ if $j < N$ and $\Iint_N = [\delta L^{-2 (N - 1)}, \delta]$.

\begin{theorem}
\label{thm:step-mr-fv}
Let $d = 4$ and let $n \ge 0$. Fix $s > 0$. Let $C_\DV$ and $L$
be sufficiently large. There exist $M>0$, $\delta >0$,
and $\kappa = O(L^{-1})$											% added
such that for $\ggen \in (0,\delta)$
and $\mgen^2 \in \Iint_+$,											% added
and with the domain $\domRG$ defined using any $\DVa> M$, the maps
\begin{equation}
\label{e:RKplusmaps}
R_+:\domRG \times \Igen_+ % (\mgen^2)
	\to \Vcal,
		\quad
K_+:\domRG \times \Igen_+ % (\mgen^2)
	\to \Wcal_{+}%(\sgen_+)
\end{equation}
are analytic in $(V, K)$											% added
and satisfy the estimates
\begin{equation}
\label{e:RKplus}
\|R_+\|_{\Vcal}
\le
M\chigen_+\ggen_+^{3}
, \qquad
\|K_+\|_{\Wcal_+}
\le
M\chigen_+ \ggen_+^{3}
\end{equation}
and
\begin{equation}
\label{e:DKkappa}
\|D_K K_+\|_{L(\Wcal,\Wcal_+)} \le \kappa.
\end{equation}
When $\volume = \Lambda$, these maps define $(\Vp,K)\mapsto (\Vc_+,K_+)$
obeying \eqref{e:IcircKdu}.
\end{theorem}

With $s = 0$ in the choice of weights $\ell_j$ and $\ell_{\sigma,j}$,
this theorem was the main achievement of \cite{BS-rg-step}. With $s > 0$
arbitrary, the bounds \eqref{e:RKplus} are greatly improved beyond the
mass scale. The statement in \cite{BS-rg-step} with $s = 0$ contains
bounds on the derivatives of the maps $R_+$ and $K_+$. Our improvements
apply to these bounds as well, but we do not state them here as we will
not make direct use of these bounds\footnote{These derivative bounds are
indirectly used to prove Theorem~\ref{thm:rhatflow}.}.

Note that the maps $R_+$ and $K_+$ are \emph{independent} of $s$.
The proof of Theorem~\ref{thm:step-mr-fv} involves showing that the
inductive estimates \eqref{e:RKplus} hold for any $s$. In some cases,
we will make use of these estimates both with $s > 0$ and $s = 0$.

The proof of Theorem~\ref{thm:step-mr-fv} is identical to the proof of
\cite[Theorem~\ref{phi4-thm:step-mr-fv}]{ST-phi4} \todo{(if $n \ge 1$)},
via a version of
\cite[Theorems~\ref{step-thm:mr-R}--\ref{step-thm:mr}]{BS-rg-step} that
uses the norm parameters \eqref{e:elldef-zz} with $s > 0$.
The proof of the latter results with these new norm parameters is an
adaptation of the proof of the $s = 0$ case contained in \cite{BS-rg-IE,BS-rg-step}.
Some steps in this proof continue to hold unchanged whereas others require
some modification. As mentioned above, a major change that is required is
a new definition of $\Wcal_j$ above the mass scale. A detailed verification
that the proof holds for $s > 0$ is carried out in Chapter~\ref{sec:RGstep}.
% \ref{sec:Rpf2} below.

Theorem~\ref{thm:step-mr-fv} expresses a contractive property of the map $K_+$,
as it takes $K$ in a ball whose radius involves $\alpha=4M$ at scale $j$ to
an image which lies in a ball whose radius involves the smaller number $M$
at scale $j+1$.  The fact that $K_+$ is a contraction is used in
Theorem~\ref{thm:rhatflow}
% \cite[Proposition~\ref{log-prop:KjNbd}]{BBS-saw4-log} (for $n=0$) and
% \cite[Theorem~\ref{phi4-log-thm:flow-flow}]{BBS-phi4-log}
% (for $n \ge 1$)
to prove that, for $m^2$ and $g_0$ sufficiently small, there exist
\emph{critical} initial conditions
$\nu_0 = \nu_0^c(m^2, g_0, \gamma_0; n)$ and $z_0 = z_0^c(m^2, g_0,\gamma_0; n)$
such that for any $N$,
iteration of the maps $(\Vc_+,K_+)$ defines a sequence $(\Vp_j, K_j)$
which lies in the domain $\domRG_j$ and obeys the estimates \refeq{RKplus}
\emph{for all} $j = 1, \ldots, N$.
% This construction of critical initial conditions uses the $s=0$ version
% of \refeq{elldef-zz}.

% The case with observable fields included is handled in \cite{ST-phi4}.
% Because we have increased $\ell_{\sigma,j}$ beyond the mass scale, the
% estimates on $q_0,q_x$ given by the bound on $R_+$ in \refeq{RKplus} are
% significantly improved compared to their versions with $s=0$ in \cite{ST-phi4}.
% As is discussed in detail in  \cite[Section~\ref{phi4-sec:pfmr1}]{ST-phi4},
% $V_j^{(0)}$ remains in the domain $\domRG_j$ for all $j$
% (also concerning $\lambda_{0,j}, \lambda_{x,j}$).
% Moreover,
% $\lambda_{0,j}, \lambda_{x,j},q_{0,j},q_{x,j}$, are independent of the volume parameter $N$ and
% so can be extended to infinite sequences, and the following limits exist:
% \begin{equation}
% q_{u,\infty} = \lim_{j\to\infty} q_{u,j}, \quad u = 0, x.
% \end{equation}

%%%%%%%%%%%%%%%%%%%%%%%%%%%%%%%%%%%%%%%%%%%%%%%%%%%%%%%%%%%%%%%%%%%%%%%%%%%%%%%
%%%%%%%%%%%%%%%%%%%%%%%%%%%%%%%%%%%%%%%%%%%%%%%%%%%%%%%%%%%%%%%%%%%%%%%%%%%%%%%

\section{Renormalisation group flow}
% mainly from saw-sa

Theorem~\ref{thm:step-mr-fv} allows us to perform a single renormalisation group
step. In \cite[Proposition~\ref{log-prop:flow-flow}]{BBS-saw4-log} it was shown
that there exist \emph{critical} initial conditions
$\nu_0^c, z_0^c$ depending on $(m^2, g_0)$ such that this map can be iterated
indefinitely, i.e.\ for $j = 1, \ldots, N$ and any $N$. This results in a
sequence $(\Vc_j, K_j)$ generated by the renormalisation group map, hence
whose elements lie in the domains $\domRG_j$. This was proved with $s = 0$,
but the sequence itself is independent of $s$ and continues to exist in our
setting. However, Theorem~\ref{thm:step-mr-fv} shows that this sequence satisfies
improved estimates. Thus, there is no difficulty in extending
\cite[Proposition~\ref{log-prop:flow-flow}]{BBS-saw4-log} to the $s$-dependent
domains used here.

However, \cite{BBS-saw4-log} studied the WSAW ($\gamma = 0$). In order to study
the WSAW-SA, we extend \cite[Proposition~\ref{log-prop:flow-flow}]{BBS-saw4-log}
by taking advantage of the additional generality in \cite{BBS-rg-flow}, whose
main result is the basis for the proof of this proposition. The result, which we
state as Theorem~\ref{thm:rhatflow}, is the
construction of critical initial conditions $\hat\nu_0^c, \hat z_0^c$ that now
depend on $\gamma_0$ as well as $(m^2, g_0)$. Thus, this theorem is an extension
to $s > 0$ and $\gamma_0 \ne 0$ of \cite[Proposition~\ref{log-prop:flow-flow}]{BBS-saw4-log}.

Let $\delta > 0$ and suppose $r : [0, \delta] \to [0, \infty)$
is a continuous \emph{positive-definite}\footnote{Note that our usage of this term is
different from that in the theory of quadratic forms.} function; by this we
mean that $r(x) > 0$ if $x > 0$ and $r(0) = 0$.
We define
\begin{equation}
\lbeq{Ddef}
D(\delta, r)
	=
\{ (w, x, y) \in [0, \delta]^2 \times (-\delta, \delta) : |y| \leq r(x) \}
\end{equation}
and we let $C^{0,1,\pm}(D(\delta, r))$ denote the space of continuous functions
$f = f(w, x, y)$ on $D(\delta, r)$
that are $C^1$ in $(x, y)$ away from $y = 0$, $C^1$ in $x$ everywhere,
and have left- and right-derivatives in $y$ at $y = 0$.
In our applications, we take $w = m^2$, $x = g_0$ or $\gcc$,
and $y = \gamma_0$ or $\gamma$.

\begin{theorem}
\label{thm:rhatflow}
There exists a domain $D(\delta, \hat r)$ (with $\delta > 0$ and $\hat r$
positive-definite) and functions $\hat\nu_0^c, \hat z_0^c \in C^{0,1,\pm}(D(\delta, \hat r))$
such that for any $(m^2, g_0, \gamma_0) \in D(\delta, \hat r)$
with $g_0 > 0$ and $m^2 \in [\delta L^{-2 (N - 1)}, \delta)$, the following holds:
if $(\Vc_0, K_0) = (\Vp^+_0, K^+_0)$ with $(\nu_0, z_0) = (\hat\nu_0^c, \hat z_0^c)$,
then for any $N \in \N$, there exists a sequence $(\Vc_j, K_j) \in \domRG_j$ such that
\begin{equation}
\label{e:VjKjDj-hat}
(\Vc_{j+1},K_{j+1})
	=
(\Vc_{j+1}(\Vp_j, K_j), K_{j+1}(\Vp_j, K_j)) \text{ for all } j < N
\end{equation}
and \eqref{e:IcircKdu} is satisfied.
Moreover, the sequence $\Vc_j, j = 1, \ldots, N$ is independent of the volume $\Lambda$ and
\begin{equation}
\label{e:hat-est}
\hat\nu_0^c = O(g_0),
\quad
\hat z_0^c = O(g_0)
\end{equation}
uniformly in $(m^2, \gamma_0)$.
% Moreover, the second-order evolution equation for $V_j$ is independent of $\gamma_0$.
\end{theorem}

\commentbw{Define infinite sequence of couplign constants. Mention that different
versions of $g$ can be interchanged.}