\chapter{Renormalisation group method}
\label{sec:rg}

\setcounter{footnote}{0}

\renewcommand{\pm}{+}					% only consider \gamma \ge 0

This chapter introduces the elements of the renormalisation group method of
developed in the series of papers
\cite{BS-rg-norm,BS-rg-loc,BBS-rg-pt,BS-rg-IE,BS-rg-step} and
applied in \cite{BBS-phi4-log,BBS-saw4-log,BBS-saw4,ST-phi4}. We will often
state results from these papers without proof.

The main contribution of this
thesis, which is based on the work in \cite{BSTW-clp,BSW-saw-sa}, is the
improvement of the estimates in Theorem~\ref{thm:step-mr-fv} and the extension
to $\gamma_0 \ne 0$ in Theorem~\ref{thm:rhatflow}.

% \commentbw{p.34 para 2:  this is not very clear since we have no hint of what
% those Thms say, and it does not sound as exciting as it is.  Can you
% maybe work on the marketing aspect a bit here?  And explain more about
% what is after all the main contribution of the thesis.}

%%%%%%%%%%%%%%%%%%%%%%%%%%%%%%%%%%%%%%%%%%%%%%%%%%%%%%%%%%%%%%%%%%%%%%%%%%%%%%%
%%%%%%%%%%%%%%%%%%%%%%%%%%%%%%%%%%%%%%%%%%%%%%%%%%%%%%%%%%%%%%%%%%%%%%%%%%%%%%%

\section{Notation}

To unify our treatment of the two models,
we define the forms $\tau_x, \tau_{\Delta,x}, |\nabla\tau_x|^2$ according
to \eqref{e:taudef}--\eqref{e:nablatau} if $n = 0$ and
\begin{equation}
\label{e:tauphi}
\tau_x = \tfrac{1}{2} |\varphi_x|^2,
	\quad
\tau_{\Delta,x} = \tfrac{1}{2} \varphi_x \cdot (-\Delta \varphi)_x,
	\quad
|\nabla\tau_x|^2 = \sum_{|e|=1} (\nabla^e |\phi_x|^2)^2
\end{equation}
if $n \ge 1$.
Then by \eqref{e:Vdef1}, \eqref{e:Udef-pos}, and \eqref{e:Vdef2},
% $V_{g,\gamma,\nu,N} \in \Ncal^\bulk$ is given by
\begin{equation}
V_{g,\gamma,\nu,N}
	=
\sum_{x\in\Lambda_N}
\Big(
	(g - \gamma) \tau_x^2 + \nu \tau_x + \tau_{\Delta,x} + \tfrac{1}{4 d} \gamma |\nabla\tau_x|^2
\Big)
\end{equation}
for any choice of $n$.
We write
\begin{equation}
\langle F \rangle_{g,\gamma,\nu,N}
	=
\begin{cases}
\displaystyle \int F e^{-U_{g,\gamma,\nu,N}},				& n = 0 \\
\displaystyle \frac{1}{Z_{g,\gamma,\nu,N}}
	\int F(\varphi) e^{-U_{g,\gamma,\nu,N}} \; d\varphi,	& n \ge 1.
\end{cases}
\end{equation}
The action $S_A$ is defined by \eqref{e:action} if $n = 0$ and
\begin{equation}
S_A = \frac12 \sum_{x\in\Lambda} \varphi_x \cdot (A \varphi)_x
\end{equation}
if $n \ge 1$. In either case, if $A = -\Delta + m^2$, then
\begin{equation}
\label{e:SAtauDelta}
S_A = \sum_{x\in\Lambda} (\tau_{\Delta,x} + m^2 \tau_x).
\end{equation}
Thus, if $\Ex_C \theta$ is the super-expectation \eqref{e:ExCF} for $n = 0$
and Gaussian integration over $(\R^n)^\Lambda$ if $n \ge 1$,
% (recall \eqref{e:gauss-density}),
then for $\nu > 0$,
\begin{equation}
\label{e:phi4-gauss}
\langle F \rangle_{0,0,m^2,N}
	=
\Ex_C F,
	\qquad
C = (-\Delta + m^2)^{-1}.
\end{equation}

By \eqref{e:two-point-function-phi4}, \eqref{e:Givlc}, and \eqref{e:Grep-pos-bis},
\begin{equation}
G_x(g, \gamma, \nu; n) = \lim_{N\to\infty} G_{x,N}(g, \gamma, \nu; n),
\end{equation}
where
\begin{equation}
G_{x,N}(g, \gamma, \nu; n)
	=
\begin{cases}
\langle \phib_0 \phi_x \rangle_{g,\gamma,\nu,N},      & n = 0 \\
\langle \varphi_0 \cdot \varphi_x \rangle_{g,\gamma,\nu,N}  & n \ge 1.
\end{cases}
\end{equation}
\new{By Proposition~\ref{prop:finvol} and \eqref{e:susceptibility-def}},
for any integer $n \ge 0$,
\begin{align}
\label{e:chilim}
\chi(\gcc, \gamma, \nu; n)
	&= \lim_{N\to\infty} \chi_N(\gcc, \gamma, \nu; n) \\
\label{e:chiNdef}
\chi_N(\gcc, \gamma, \nu; n)
	&= \sum_{x\in\Lambda_N} G_{x,N}(\gcc, \gamma, \nu; n). \\
\label{e:clp}
\xi_p(g, \gamma, \nu; n)
	&= \left(\frac{\sum_{x\in\Zd} |x|^p G_x(g, \gamma, \nu; n)}{\chi(g, \gamma, \nu; n)}\right)^{1/p} \\
\label{e:nuc-def}
\nu_c
	= \nu_c(\gcc, \gamma; n) &= \inf \{ \nu : \chi(g, \gamma, \nu; n) < \infty \}.
\end{align}

%%%%%%%%%%%%%%%%%%%%%%%%%%%%%%%%%%%%%%%%%%%%%%%%%%%%%%%%%%%%%%%%%%%%%%%%%%%%%%%
%%%%%%%%%%%%%%%%%%%%%%%%%%%%%%%%%%%%%%%%%%%%%%%%%%%%%%%%%%%%%%%%%%%%%%%%%%%%%%%

\section{Reformulation of the problem}
\label{sec:reformulation}

In preparation for our application of the renormalisation group, we write the
two-point function and susceptibility in terms of appropriate perturbations of
Gaussian measures.

Given $m^2>0$ and $z_0 > -1$, let
\begin{equation}
\label{e:gg0}
g_0 = (\gcc - \gamma) (1 + z_0)^2,
	\quad
\nu_0 = \nu (1 + z_0) - m^2,
	\quad
\gamma_0 = \frac{1}{4d} \gamma (1 + z_0)^2.
\end{equation}
We discuss the role of $(m^2, z_0)$ some more in Remark~\ref{rk:crit-params}.
% \begin{rk}
% The parameters $(m^2, z_0)$ are arbitrary for now, but a careful choice of
% parameters must be made in order to study the stable manifold of the
% renormalisation group. We will construct such parameters for $\gamma_0 \ne 0$
% as perturbations of their $\gamma_0 = 0$ counterparts in Section~\ref{sec:nuztilde}.
% \end{rk}

We fix two points $0,x\in \Lambda$
and introduce \emph{observable fields} $\sigmaa, \sigmab \in \R$.
We distinguish these from the \emph{bulk} fields $\varphi$, $\phi$,
$\phib$, $\psi$, $\psib$. We also make a distinction between bosonic
fields $\varphi$, $\phi$, $\phib$, $\sigma$, and fermionic fields
$\psi$, $\psib$.
% We now distinguish between the following kinds of fields:
% real bosonic ($\varphi$), complex bosonic ($\phi$, $\phib$),
% observable bosonic ($\sigma$), and fermionic ($\psi$, $\psib$).

For any $y\in\Lambda$, we define the polynomials
\begin{equation}
\label{e:V0def}
\Vp^+_{0,y}
	=
g_0\tau_y^2 + \nu_0 \tau_y + z_0 \tau_{\Delta,y}
- f_0 \sigma_0 \1_{y=x}
- f_x \sigma_x \1_{y=x},
	\quad
U^+_y
	=
|\nabla \tau_y|^2
\end{equation}
where
\begin{equation}
\label{e:obs-couple}
f_u =
\begin{cases}
\phib_0,		& n = 0, u = 0 \\
\phi_x,			& n = 0, u = x \\
\varphi^1_u,	& n \ge 1.
\end{cases}
\end{equation}
These are examples of local polynomials, which are polynomials
in the fields and their derivatives at a point $y\in\Lambda$. For any such local
polynomial $V_y$, we will usually write
\begin{equation}
\label{e:VX}
V(X) = \sum_{y\in X} V_y.
\end{equation}

Let
\begin{equation}
\label{e:Z0def}
Z_0 = \prod_{y\in \Lambda} e^{-(\Vp^+_{0,y} + \gamma_0 U^+_y)}
\end{equation}
and
\begin{equation}
\label{e:ZNdef}
Z_N = \Ex_C \theta Z_0
\end{equation}
where the covariance is given by $C = (-\Delta + m^2)^{-1}$
as in \eqref{e:phi4-gauss}. In particular,
\begin{equation}
\label{e:exp-conv}
\Ex_C Z_0 = Z_N^\zero(0).
\end{equation}
Recall here that $Z_N^\zero$ denotes the $0$-degree part of $Z_N$
(when $n \ge 1$, $Z_N^\zero = Z_N$).
This is a function of the bulk bosonic fields, which we have set to $0$ on the
right-hand side of \eqref{e:exp-conv}.

Recall that the Gaussian convolution operator $\Ex_C\theta$ was defined in
Section~\ref{sec:forms}. We define a test function $\1: \Lambda_N \to \R$ by
$\1_y=1$ for all $y$. If $F$ is a sufficiently smooth function \new{of the
bosonic fields} (a $0$-degree form), let
% and write $D^2 Z_N^\zero(0; \1, \1)$ for the directional derivative of
% $Z_N^\zero$ evaluated with all fields equal to $0$ and
% % at $(\phi, \bar\phi) = (0, 0)$,
% with both directions equal to $\1$. That is,
\begin{equation}
D^2 F(0; \1, \1)
  =
\ddp{^2}{s\partial t}\Big|_0
\begin{cases}
F(s \1, t\1), & n = 0 \\
F(s \1 + t\1), & n \ge 1
\end{cases}
\end{equation}
where the derivative is evaluated with all fields (bulk and observable) and $s, t$ set to $0$.
\new{Let $F(0)$ denote $F$ evaluated at $0$ bulk field.
We denote by $D^2_{\sigma_0\sigma_x} F(0)$ the second partial derivative of $F(0)$
with respect to the observable fields $\sigma_0, \sigma_x$ evaluated at $\sigma_0=\sigma_x=0$.}

\begin{prop}
\label{prop:intrep}
% from saw-sa
Let $d > 0$, $\gamma, \nu \in \R$, $\gcc >0$ and $\gamma <\gcc$.
If the relations \eqref{e:gg0} hold, then
\begin{equation}
% from clp
\label{e:generating-fn}
G_{x,N}(g, \gamma, \nu; n)
	=
(1+z_0)
D^2_{\sigma_0\sigma_x}
\log \Ex_C Z_0
\end{equation}
% \begin{equation}
% \label{e:GG2}
% G_{x,N}(\gcc,\gamma,\nu)
%     =
% (1+z_0)
% \Ex_C (Z_0 \bar\phi_a \phi_b),
% \end{equation}
and
\begin{equation}
\label{e:chichibar}
\chi_N\left(\gcc,\gamma,\nu; n\right)
	=
(1+z_0)\hat\chi_N(m^2, g_0, \gamma_0, \nu_0, z_0; n),
\end{equation}
with
\begin{equation}
\label{e:chibarm}
\hat\chi_N(m^2, g_0, \gamma_0, \nu_0, z_0; n)
	=
\frac{1}{m^2}
	+
\frac{1}{m^4} \frac{1}{|\Lambda|} \frac{D^2 Z_N^\zero(0; \1, \1)}{Z^0_N(0)}.
\end{equation}
\end{prop}

\begin{proof}
We prove the case $n = 0$ and drop the parameter $n$ from the notation. Note that by
\eqref{e:tau-iso},
$Z_N(0)\big|_{\sigma_0=\sigma_x=0} = 1$ in this case. The proof for $n \ge 1$ is similar
and involves only ordinary integration with respect to real boson fields.

We make the change of variables $\phi_x \mapsto (1 + z_0)^{1/2} \phi_x$
and likewise for $\bar\phi_x, \psi_x, \bar\psi_x$ in \eqref{e:Grep-pos-bis}, and obtain
\begin{equation}
\label{e:Grep-pos}
G_{x,N}(\gcc,\gamma,\nu)
	=
(1+z_0) \int e^{-\sum_{x\in\Lambda}
\left(
	g_0 \tau_x^2 + \gamma_0 |\nabla \tau_x|^2 + \nu (1+z_0) \tau_x + (1+z_0)\tau_{\Delta,x}
\right)}
\bar\phi_a \phi_b.
\end{equation}
Note here that the Jacobian factor is automatically accounted for by the change of variables
in the fermionic fields.
For any $m^2 \in\R$, it follows that
\begin{equation}
\lbeq{GNint}
G_{x,N}(\gcc,\gamma,\nu)
    =
(1 + z_0) \int
e^{-\sum_{x\in\Lambda} (\tau_{\Delta,x} + m^2 \tau_x)}
Z_0 \phib_0 \phi_x
\end{equation}
($m^2$ simply cancels with $\nu_0$ on the right-hand side).
We use this with $m^2>0$, so that the inverse matrix $C=(-\Delta+m^2)^{-1}$ exists and
\begin{equation}
\label{e:G-gauss}
G_{x,N}(g, \gamma, \nu)
	=
(1 + z_0) \Ex_C (Z_0 \phib_0 \phi_x)
\end{equation}
% and \eqref{e:GNint} is a Gaussian expectation
by \eqref{e:phi4-gauss}.
% By symmetry of the matrix $\Delta$, \refeq{action} gives
% \begin{equation}
% \label{e:SAtauDelta}
% S_{(-\Delta+m^2)}
% =
% \sum_{x\in\Lambda} \left( \tau_{\Delta,x}
% + m^2  \tau_x \right).
% \end{equation}
The identity \eqref{e:generating-fn} follows by the standard procedure of writing
the moments of an integral as a derivative of a moment-generating function.

Summation of \eqref{e:G-gauss} over $x\in \Lambda_N$ gives the formula
$\chi_N(\gcc,\gamma,\nu) = (1+z_0)\sum_{x\in \Lambda} \Ex_C (Z_0\phib_0\phi_x)$.
Call the right-hand side $\hat\chi_N(\gcc,\gamma,\nu)$. To show that this is
consistent with \eqref{e:chibarm}, begin by noting that
\begin{equation}
\hat\chi_N(\gcc,\gamma,\nu)
	=
|\Lambda|^{-1} \frac{D^2 \Sigma(0; \1, \1)}{Z_N(0)},
\end{equation}
where
\begin{equation}
\Sigma(J, \bar J) = \Ex_C (Z_0 e^{J \cdot \phib + \phi \cdot \bar J}).
\end{equation}
Completing the square yields
\begin{equation}
\Sigma(J, \bar J)
	=
e^{J \cdot C \bar J} Z_N^\zero(C J, C \bar J)
\end{equation}
and differentiating this expression gives
\begin{equation}
D^2 \Sigma(0; \1, \1)
	=
(\1, C \1) + D^2 Z_N^\zero(0; C\1, C\1)
\end{equation}
The result then follows from the fact that
\begin{equation}
C \1 = A^{-1} \1 = m^{-2} \1.
\end{equation}
\end{proof}

%%%%%%%%%%%%%%%%%%%%%%%%%%%%%%%%%%%%%%%%%%%%%%%%%%%%%%%%%%%%%%%%%%%%%%%%%%%%%%%
%%%%%%%%%%%%%%%%%%%%%%%%%%%%%%%%%%%%%%%%%%%%%%%%%%%%%%%%%%%%%%%%%%%%%%%%%%%%%%%

\section{Progressive integration}
% Based on both
\label{sec:prog}

By Proposition~\ref{prop:intrep}, our task is to understand the Gaussian expectation
$Z_N = \Ex_C Z_0$ and its derivatives to leading order, uniformly in the volume
$\Lambda_N$ and the mass $m^2$ near $0$.

We proceed using the covariance decomposition
\begin{equation}
\label{e:NCj}
C = C_1 + \cdots + C_{N-1} + C_{N,N}
\end{equation}
constructed in \cite{Baue13a}; a similar decomposition was also constructed in \cite{BGM04}.
The covariances $C_1, \ldots, C_{N-1}$ are independent of the volume $\Lambda_N$. The final
covariance $C_{N,N}$ \emph{does} depend on the volume; so, for instance, $C_{N,N} \ne C_{N,N+1}$.
Nevertheless, we will often write $C_N \coloneqq C_{N,N}$ when the volume is implicit.

The covariances $C_j$ have the following important \emph{finite-range property}:
\begin{equation}
C_{j;xy} = 0 \text{ if } |x - y| \ge \tfrac12 L^j.
\end{equation}
Thus, if $\zeta$ is a Gaussian field with covariance $C_j$, then $\zeta_x$ is independent
of $\zeta_y$ whenever $|x - y| \ge \tfrac12 L^j$. In particular,
if $F_x, F_y$ are functions of the fields at $x, y$, respectively, then
\begin{equation}
\label{e:uncorr}
\Ex_{C_{j+1}} (F_x F_y) = (\Ex_{C_{j+1}} F_x) (\Ex_{C_{j+1}} F_y).
\end{equation}
In addition, we have the following covariance bounds (this is a restatement of
\cite[Proposition~\ref{pt-prop:Cdecomp}(a)]{BBS-rg-pt}).

\begin{prop}
\label{prop:Cdecomp}
  Let $d >2$, $L\geq 2$, $j \ge 1$, $\bar m^2 >0$.
  For multi-indices $\alpha,\beta$ with
  $\ell^1$ norms $|\alpha|_1,|\beta|_1$ at most
  some fixed value $p$,
  for any $k$, and for $m^2 \in [0,\bar m^2]$,
  \begin{equation}
    \label{e:scaling-estimate}
    |\nabla_x^\alpha \nabla_y^\beta C_{j;x,y}|
    \leq c(1+m^2L^{2(j-1)})^{-k}
    L^{-(j-1)(d-2 +|\alpha|_1+|\beta|_1)},
  \end{equation}
  where $c=c(p,k,\bar m^2)$ is independent of $m^2,j,L$.
  The same bound holds for $C_{N,N}$ if
  $m^2L^{2(N-1)} \ge \varepsilon$ for some $\varepsilon >0$,
  with $c$ depending on $\varepsilon$ but independent of $N$.
\end{prop}

% By Proposition~\ref{prop:Cdecomp}, the decomposition \eqref{e:NCj} is a multiscale decomposition,
% in the sense that $C_{j+1} = O(L^{-1} C_j)$. Moreover, the covariances are approximately
% constant on blocks in the sense that $\nabla C_j = O(L^{-1} C_j)$.

It is a basic property  of the Gaussian distribution that a sum of independent Gaussian random
variables with covariances
$C'$ and $C''$ is itself Gaussian with covariance $C' + C''$. It follows that for any
boson field $F$,
\begin{equation}
\Ex_{C'+C''}\theta F = \Ex_{C'}\theta \circ \Ex_{C''}\theta F.
\end{equation}
% When $C' = s C_0$ and $C'' = t C_0$ for some covariance $C_0$ and $s, t > 0$,
% this is essentially a restatement of the semigroup property of the heat kernel.
This extends to any sufficiently smooth form $F$ (see \cite{BS-rg-norm}).
It follows that
\begin{equation}
\label{e:prog-int}
Z_N =
\Ex_{C_N}\theta \circ \Ex_{C_{N-1}}\theta \circ \ldots \circ \Ex_{C_1}\theta Z_0.
\end{equation}
We define the \emph{renormalisation group map} $Z_j \mapsto Z_{j+1}$ by
\begin{equation}
\label{e:rgmapZ}
Z_{j+1} = \Ex_{C_{j+1}} \theta Z_j, \quad j < N.
\end{equation}

\begin{rk}
\label{rk:crit-params}
% This is truly a ($j$-dependent) map and not just a fixed sequence of field
% functionals.  Indeed, recall that the initial condition $Z_0$ is not fixed:
% it depends on the two free parameters $m^2$ and $z_0$. We think of the family
% of possible sequences $Z_j$ as a (non-autonomous) dynamical system.
% 
The key to understanding
$Z_N$ for large $N$ is the careful choice of \emph{critical} initial conditions
$(m^2, z_0)$ \new{in \eqref{e:gg0}. % In a sense, these initial conditions, which are functions of
Viewed as functions of $(g, \gamma, \nu)$, these define a stable manifold for the
dynamical system induced by the renormalisation group map
and the fixed point for this stable manifold is the Gaussian measure with covariance
$(1 + z_0) (-\Delta + m^2)^{-1}$. However, we have scaled out the factor $1 + z_0$}
in the change of variables performed in the proof of Proposition~\ref{prop:intrep}.
% This factor plays a similar role to the standard deviation $\sigma$ in the central
% limit theorem: rather than a unique fixed point, there is a one-parameter family
% of fixed points.

The construction of the critical parameters for $\gamma \ne 0$ will be carried
out in Section~\ref{sec:nuztilde} and is a key step in the proof of Theorem~\ref{thm:mr}.
\end{rk}

% \commentbw{Rk 2.3.2: don't understand the point here.  Doesn't (2.3.7) simply define
% a map on any $F$ by Gaussian convolution (so any integrable $Z_j$ is a suitable
% input?)?  It sounds like you are saying that there is a map because
% $m^2$ and $z_0$ are free parameters, and this seems to be not at all the point.
% I find the subsequent paragraphs a bit confusing because you say that
% $z_0$ has to be carefully (uniquely) fixed but then you say it indexes a
% 1-parameter family of fixed points which makes it sound like you can vary
% it if you wish.}

%%%%%%%%%%%%%%%%%%%%%%%%%%%%%%%%%%%%%%%%%%%%%%%%%%%%%%%%%%%%%%%%%%%%%%%%%%%%%%%
%%%%%%%%%%%%%%%%%%%%%%%%%%%%%%%%%%%%%%%%%%%%%%%%%%%%%%%%%%%%%%%%%%%%%%%%%%%%%%%

\section{The space of field functionals}
\label{sec:Ncal}

For the analysis of the dynamical system \eqref{e:rgmapZ}, we require a suitable
space on which this system evolves.

Let $\Ncal^\bulk$ be defined as in Section~\ref{sec:intrep} if $n = 0$ and
\begin{equation}
\label{e:Ncaldef}
\Ncal^\bulk
	= \Ncal^\bulk(\Lambda)
	= C^{p_\Ncal}((\R^n)^\Lambda,\R)
\end{equation}
if $n \ge 1$. Recall that $p_\Ncal$ is the smoothness parameter discussed in
Section~\ref{sec:intrep}.
% In Section~\ref{sec:newnorm},
% we explain that $p_\Ncal$ must be chosen in a way that depends on the
% parameter $p$ in Theorem~\ref{thm:mr}(iii).

We extend $\Ncal^\bulk$ to a space $\Ncal$
that includes functions of the observable fields $\sigma_0$ and $\sigma_x$,
\new{which we identify to order $1, \sigma_0, \sigma_x, \sigma_0\sigma_x$
(this is sufficient for computing the derivative in \eqref{e:generating-fn}).
Formally, we let $\Ncal'$ denote the extension of $\Ncal^\bulk$
whose elements may depend smoothly on $\sigma_0$, $\sigma_x$. In other words,
if $n \ge 1$, then $\Ncal'$ consists of functions of $(\varphi, \sigma_0, \sigma_x$)
that are $C^{p_\Ncal}$ in $\varphi$ and $C^\infty$ in $\sigma_0, \sigma_x$.
Likewise, for $n = 0$, a similar statement is true of the coefficients $F_{\vec y}$
in \eqref{e:FinNcal}. Letting $\Ical \subset \Ncal'$ be the ideal consisting
of elements whose formal expansion to order $1, \sigma_0, \sigma_x, \sigma_0\sigma_x$
is $0$, we define $\Ncal = \Ncal'/\Ical$. Then $\Ncal$ has the direct sum
decomposition
\begin{equation}
\Ncal = \Ncal^\bulk \oplus \Ncal^a \oplus \Ncal^b \oplus \Ncal^{ab},
\end{equation}
where $\Ncal^a$ consists of elements of the form $\sigma_a F$ with $F\in\Ncal^\bulk$
and a similar statement is true of $\Ncal^b, \Ncal^{ab}$.
Thus,}
% Recalling \eqref{e:generating-fn}, this space is defined in such a way that
% functions of the observable fields are identified to second order. A precise
% definition is given in
% in \cite[Section~\ref{phi4-sec:phi4observables_representation}]{ST-phi4}.
% The upshot is that
every $F \in \Ncal$ has the form
\begin{equation}
\label{e:obs-decomp}
F = \new{F_\bulk + \sigma_0 F_0 + \sigma_x F_x + \sigma_0 \sigma_x F_{0x}},
	\quad
\new{F_\bulk, F_0, F_x, F_{0x}} \in \Ncal^\bulk.
\end{equation}
% \commentbw{(2.4.2): I think the ``upshot'' is too brief.  It is true that every F has
% that form, but this is only after taking the quotient with an identification.
% Without this explanation, $e^{\sigma \varphi}$ is not in $\Ncal$.
% I think you need to go into some detail here.}
There are natural projections $\pi_\alpha : \Ncal \to \Ncal_\alpha$ with
$\alpha = \bulk, 0, x, 0x$ such that $\pi_\alpha F = F_\alpha$.
For $X \subset \Lambda$, we let $\Ncal(X)$ denote the subspace of $\Ncal$ consisting
of field functionals that only depend on fields in $X$.

In order to control the evolution of $Z_j$ on $\Ncal$, we make use of a family
$\|\cdot\|_{T_{\phi,j}(\h_j)}$ of scale-dependent seminorms defined in terms of a
sequence of weights $\h_j > 0$; the field $\phi$ lies in $\C^\Lambda$ if $n = 0$ and
$(\R^n)^\Lambda$ if $n \ge 1$. For convenience,
we will simply write $\|\cdot\|_{T_\phi(\h_j)}$ with the scale $j$ implied by the
choice of parameter $\h_j$.

We given the precise definitions below for $n = 0$. The case $n \ge 1$ involves only
minor changes, which we describe in Remark~\ref{rk:Tphi-n}.

%%%%%%%%%%%%%%%%%%%%%%%%%%%%%%%%%%%%%%%%%%%%%%%%%%%%%%%%%%%%%%%%%%%%%%%%%%%%%%%

\subsection{Test functions}

Recall the notation introduced in Section~\ref{sec:forms}.
A \emph{test function} $g$ is defined to be a function $(\vec x, \vec y) \mapsto g_{\vec x,\vec y}$,
where $\vec x$ and $\vec y$ are finite sequences of elements in $\Lambda \sqcup \bar\Lambda$.
When $\vec x$ or $\vec y$ is the empty sequence $\varnothing$,
we drop it from the notation as long as this causes no confusion;
e.g., we may write $g_{\vec x} = g_{\vec x,\varnothing}$.
The length of a sequence $\vec x$ is denoted $|\vec x|$.
Gradients of test functions are defined component-wise.
Thus, if $\vec x = (x_1, \ldots, x_m)$
and $\alpha = (\alpha_1, \ldots, \alpha_m)$
with each $\alpha_i \in \N_0^\Ucal$, and similarly for $\vec y=(y_1,\ldots,y_n)$ and
$\beta=(\beta_1,\ldots,\beta_n)$,
then
\begin{equation}
\nabla^{\alpha,\beta}_{\vec x,\vec y} g_{\vec x,\vec y}
  =
\nabla^{\alpha_1}_{x_1} \ldots \nabla^{\alpha_m}_{x_m}
\nabla^{\beta_1}_{y_1} \ldots \nabla^{\beta_n}_{y_n}  g_{x_1,\ldots,x_m,y_1,\ldots,y_n}.
\end{equation}

We fix a positive constant $p_\Phi\ge 4$ and restrict our attention to test functions
that vanish when $|\vec x|  +|\vec y| > p_\Ncal$.
% For Theorem~\ref{thm:suscept}(i-ii), any choice of $p_\Ncal \ge 10$ is sufficient,
% whereas for Theorem~\ref{thm:suscept}(iii) it is necessary to choose $p_\Ncal$ large
% depending on $p$ \cite{BSTW-clp}.
The $\Phi_j = \Phi(\h_j)$ norm on such test functions is defined by
\begin{equation}
\|g\|_{\Phi_j}
	=
\sup_{\vec x, \vec y} \h_j^{-(|\vec x| +|\vec y|)}
	\shift\shift
\sup_{\alpha,\beta: |\alpha|_1+|\beta|_1 \le p_\Phi}
L^{j (|\alpha|_1 + |\beta|_1)}
|\nabla^{\alpha,\beta} g_{\vec x, \vec y}|,
\end{equation}
where $|\alpha|_1$ denotes the total order of the differential operator $\nabla^\alpha$.
Thus, for any test function $g$ and for sequences
$\vec x, \vec y$ with $|\vec x| +|\vec y| \leq p_\Ncal$ and
corresponding $\alpha, \beta$ with $|\alpha|_1 + |\beta|_1 \leq p_\Phi$,
\begin{equation}
\label{e:testfcnbd}
|\nabla^{\alpha,\beta} g_{\vec x,\vec y}|
	\leq
\h_j^{|\vec x| + |\vec y|} L^{-j (|\alpha|_1 + |\beta|_1)} \|g\|_{\Phi_j}.
\end{equation}

%%%%%%%%%%%%%%%%%%%%%%%%%%%%%%%%%%%%%%%%%%%%%%%%%%%%%%%%%%%%%%%%%%%%%%%%%%%%%%%

\subsection{The \texorpdfstring{$T_\phi$}{Tphi} seminorm}
\label{sec:Tphi}

If $n = 0$, then for any $F \in \Ncal^\bulk$, there are
\emph{unique} functions $F_{\vec y}$ of $(\phi, \bar\phi)$
that are anti-symmetric under permutations of $\vec y$, such that
\begin{equation}
F = \sum_{\vec y} \frac{1}{|\vec y|!} F_{\vec y}(\phi, \bar\phi) \psi^{\vec y}.
\end{equation}
Given a sequence $\vec{x}$ with $|\vec{x}| = m$, we define
\begin{equation}
F_{\vec x, \vec y} = \ddp{^m F_{\vec y}}{\phi_{x_1} \ldots \partial\phi_{x_m}}.
\end{equation}
We define a $\phi$-dependent pairing of elements of $\Ncal$ with test functions by
\begin{equation}
\label{e:pairing}
\langle F, g \rangle_\phi
  =
\sum_{\vec x, \vec y}
\frac{1}{|\vec x|! |\vec y|!}
F_{\vec x,\vec y}(\phi, \bar\phi)
g_{\vec x,\vec y}.
\end{equation}

Let $B(\Phi)$ denote the unit $\Phi$-ball in the space of test functions. Then the
$T_\phi = T_\phi(\h_j)$ seminorm on $\Ncal^\bulk$ is defined by
\begin{equation}
\label{e:Tphi-def}
\|F\|_{T_\phi} = \sup_{g\in B(\Phi_j)} |\langle F, g \rangle_\phi|.
\end{equation}

\begin{rk}
\label{rk:Tphi-n}
If $n \ge 1$, a test function is a function $g$ on sequences over $\Lambda\times\{1,\ldots,n\}$.
For any such sequence $\vec x = ((x_1, i_1), \ldots, (x_m, i_m))$, we write $|\vec x| = m$
and set
\begin{equation}
F_{\vec x}
	=
\ddp{^m F}{\varphi^{i_1}_{x_1} \ldots \partial\varphi^{i_m}_{x_m}}
\end{equation}
and
\begin{equation}
\label{e:pairing-boson}
\langle F, g \rangle_\varphi
	=
\sum_{|\vec x| \le p_\Ncal} \frac{1}{|\vec x|!} F_{\vec x}(\varphi) g_{\vec x}.
\end{equation}
Then the $T_\varphi$ seminorm can be defined as in \eqref{e:Tphi-def}.
\end{rk}

To extend the $T_\phi$ seminorm to $\Ncal$, we make use of an additional sequence
of parameters $\h_{\sigma,j}$. For any $F \in \Ncal$ of the form \eqref{e:obs-decomp},
we let
\begin{equation}
\label{e:Tphi-obs}
\|F\|_{T_\phi}
	=
\|F_\bulk\|_{T_\phi}
	+ (\|F_0\|_{T_\phi} + \|F_x\|_{T_\phi}) \h_\sigma
	+ \|F_{0x}\|_{T_\phi} \h_\sigma^2.
\end{equation}

By its definition, the $T_\phi$ seminorm controls the values of $F$ and its derivatives
(up to order $p_\Ncal$) at $\phi$. For instance, we will make use of the following facts.

\begin{lemma}
\label{lem:deriv-norm-bds}
\new{If $F\in\Ncal^\bulk$, then $|F^\zero(0)| \le \|F\|_{T_0}$.
For $F \in \Ncal$,}
% we have $|F^\zero_\bulk(0)| \le \|F\|_{T_0}$} and
\begin{equation}
\label{e:deriv-norm-bd}
\new{|D^2 F^\zero(0; \1, \1)|}
	\le
2 \|F\|_{T_0(\h_j)} \|\1\|^2_{\Phi_N(\h_j)}
	=
2 \|F\|_{T_0(\h_j)} \h_j^{-1}
\end{equation}
and
\begin{equation}
\new{|D^2_{\sigma_0\sigma_x} F^\zero(0)|}
	\le
\h_{\sigma,j}^{-2} \|F\|_{T_0}.
\end{equation}
\end{lemma}

% \commentbw{Second inequality: LHS should be $F$ not $F_\varnothing$.  (This inequality is also 
% inconsistent with your version of \eqref{e:Tphi-obs} but is correct if you change \eqref{e:Tphi-obs} 
% to the usual thing --- in fact is it not an equality rather than an inequality?)}

An essential property of the $T_\phi$ seminorm is the following \emph{product property},
which is essential to fully take advantage the factorization property \eqref{e:uncorr}
that follows from the finite-range property of the covariance decomposition.

\begin{prop}
\label{prop:prod}
If $F, G \in \Ncal$, then $\|F G\|_{T_\phi} \le \|F\|_{T_\phi} \|G\|_{T_\phi}$.
\end{prop}

\begin{rk}
This follows essentially from the fact that
the series expansion of the product of two functions is the product of their
respective series expansions (see \cite{BS-rg-norm}). This is part of the reason
the $T_\phi$ seminorm was defined in terms of the pairing \eqref{e:pairing}.
\end{rk}

%%%%%%%%%%%%%%%%%%%%%%%%%%%%%%%%%%%%%%%%%%%%%%%%%%%%%%%%%%%%%%%%%%%%%%%%%%%%%%%

\subsection{Norm weights}
\label{sec:weights}

Control of the $T_\phi$ seminorm is needed for all values of
$\phi$ in order to obtain control of the convolution \eqref{e:rgmapZ} sufficient
for iteration of the renormalisation group map.
This will be discussed further in Section~\ref{sec:newnorm}.

For now, we turn our attention to the special case of the $T_0$ seminorm. Recalling
\eqref{e:exp-conv}, it is natural to choose the weights $\h_j$ so that
$\Ex_{C_{j+1}} F$ is of order $\|F\|_{T_0(\h_j)}$.
By Wick's theorem \eqref{e:wick}, for a $1$-component field $\varphi$,
\begin{equation}
\Ex_{C_{j+1}} \varphi_x^{2p} = (2p - 1)!! C_{j+1;00}^p
\end{equation}
and similar statements hold for complex and fermionic fields by the analogues of
Wick's theorem for such fields.
On the other hand, by definition of the $T_0$ seminorm,
\begin{equation}
\label{e:gauss-moments}
\|\varphi_x^{2p}\|_{T_0(\h_j)} \asymp \h_j^{2p}.
\end{equation}
This suggests defining $\h_j$ so that $|C_{j+1;00}| \le O(\h^2_j)$.
% Bounds on the covariance were stated in \eqref{e:scaling-estimate}.
% For instance, with $k = 0$, these become
% \begin{equation}
% \label{e:massless-cov-bd}
% |C_{j;xy}| \le O(L^{-j (d - 2)}).
% \end{equation}

The key to our analysis of the correlation length is that we make a choice of norm
weights that takes full advantage of the
$k$-dependence in the covariance bounds \eqref{e:scaling-estimate}.
With $k = s + 1$, this estimate together with the elementary bound
\begin{equation}
\label{e:mass-decay}
(1 + m^2 L^{2j})^{-k} \le c_L L^{-2(s+1)(j - j_m)_+}
\end{equation}
imply that
\begin{equation}
|C_{j;xy}| \le O(L^{-j (d - 2) - s (j - j_m)_+}),
\end{equation}
where $j_m$ is the \emph{mass scale}, defined by
\begin{equation}
\label{e:jmdef}
j_m	= \lfloor\log_{L} m^{-1}\rfloor.
\end{equation}
Based on this, when $d = 4$, we define the following weights:
\begin{equation}
\label{e:elldef-zz}
\ell_j
	=
\ell_0 L^{-j - s (j - j_m)_+},
	\quad
\ell_{\sigma,j}
	=
\ell_{j \wedge j_{x}}^{-1} 2^{(j - j_{x})_+} \ggen_j,
\end{equation}
where
\begin{equation}
\label{e:jxdef}
j_x = \max\{0,\lfloor \log_{L} (2 |x|)\rfloor\}
\end{equation}
is the \emph{coalescence scale} and the sequence $\ggen_j = \ggen_j(m^2,g_0)$ will
be discussed in Section~\ref{sec:pt}.
% For now, we remark only that it is bounded above and below by constant multiples of
% the sequence $\gbar$ defined in
% \eqref{e:gbar}, by \cite[Lemma~\ref{log-lem:gbarmcomp}]{BBS-saw4-log}.
The origin of the definition of $\ell_{\sigma,j}$ is discussed in
\cite[Remark~\ref{IE-rk:hsigmot}]{BS-rg-IE}.

We will set $\h_j = \ell_j$ to estimate ``small'' fields. These are fields which
are assumed not to deviate too much from their expected value. A different norm
parameter $\h_j = h_j$ will be used to control ``large'' fields.
This will be discussed in Section~\ref{sec:newnorm}.

\begin{rk}
The parameter $\ggen_j$ is used to overcome what \cite{Abde07} refers to as the
``fibred norm problem''. Briefly, the norms used to control the renormalisation
group trajectory must be decoupled from the initial parameter $g_0$. Ultimately,
we will set $\ggen_0 = g_0$ (see Remark~\ref{rk:ggen}).
\end{rk}

%%%%%%%%%%%%%%%%%%%%%%%%%%%%%%%%%%%%%%%%%%%%%%%%%%%%%%%%%%%%%%%%%%%%%%%%%%%%%%%

\subsection{Symmetries}

It is useful to restrict our attention to field functionals $F \in \Ncal$ that
obey certain symmetry conditions preserved by Gaussian expectation (and which
are obeyed by $V^+_0$).

We let any automorphism $E$ of $\Lambda$ act on $\Ncal$ by $EF(\varphi) = F(E\varphi)$
\new{with $(E\varphi)_x = \varphi_{Ex}$.}
We say that $F\in\Ncal$ is \emph{Euclidean-invariant} if $EF = F$ for all such automorphisms.

If $n = 0$, \new{we define the
\emph{gauge flow} $(q, \bar q) \mapsto (e^{-2\pi it} q, 2^{2\pi it} \bar q)$,
where $q = \phi_x, \psi_x, \sigma$ with $\sigma_0 = \sigma$ and $\sigma_x = \bar\sigma$
for all $x\in\Lambda$. A form $F\in\Ncal$ is said to be \emph{gauge-invariant} if it is
invariant under the gauge flow.
We also} define the \emph{supersymmetry generator}
\begin{equation}
Q = (2\pi i)^{1/2} \sum_{x\in\Lambda}
\left(
	\psi_x \ddp{}{\phi_x} + \psib_x \ddp{}{\phib_x}
		-
	\phi_x \ddp{}{\psi_x} + \phib_x \ddp{}{\psib_x}.
\right)
\end{equation}
A form $F \in \Ncal$ is said to be \emph{supersymmetric} if $Q F = 0$.
% Such a form is said to be \emph{gauge-invariant} if it is invariant under the
% \emph{gauge flow} $(q, \bar q) \mapsto (e^{-2\pi it} q, 2^{2\pi it} \bar q)$
% for $q = \phi_x, \psi_x$ and all $x\in\Lambda$.

If $n \ge 1$, we let
an $n \times n$ matrix $T$ act on $\Ncal$ by $T F(\varphi) = F(T \varphi)$,
where $(T\varphi)_x = T(\varphi_x)$.
We say that $F\in\Ncal$ is \emph{$O(n)$-invariant} if $TF = F$ for all
orthogonal matrices $T$.

% \commentbw{Sec 2.4.4: I think you are too fast with automorphism because you don't say
% how an automorphism of $\Lambda$ acts on a field.
% And it sounds backwards what you say about gauge invariance because supersymmetric
% forms are gauge invariant but you seem to be defining the latter only for
% supersymmetric forms?}

%%%%%%%%%%%%%%%%%%%%%%%%%%%%%%%%%%%%%%%%%%%%%%%%%%%%%%%%%%%%%%%%%%%%%%%%%%%%%%%
%%%%%%%%%%%%%%%%%%%%%%%%%%%%%%%%%%%%%%%%%%%%%%%%%%%%%%%%%%%%%%%%%%%%%%%%%%%%%%%

\section{Perturbative coordinate}

As mentioned in Section~\ref{sec:rg-intro}, one of Wilson's key insights was that the renormalisation
group could be well-approximated by a finite-dimensional dynamical system. In this
section, we reformulate Wilson's insights in terms of the covariance decomposition
and define a subspace on which this finite-dimensional system will evolve.

% The dynamical system is analysed via a perturbative part which is tracked accurately
% to second order in $g$, together with a third-order non-perturbative part whose study
% forms the main part of our effort.  For the perturbative part, we first introduce
% an appropriate space of local field polynomials.

%%%%%%%%%%%%%%%%%%%%%%%%%%%%%%%%%%%%%%%%%%%%%%%%%%%%%%%%%%%%%%%%%%%%%%%%%%%%%%%

\subsection{Dimensional analysis}
\label{sec:dimensional}

% We define the \emph{scaling dimension}
% \begin{equation}
% [\varphi] = \frac{d - 2}{2}.
% \end{equation}
We call $M_x \in \Ncal$ a local monomial if it is a monomial in $\varphi_x$ and its
(discrete) gradients. For instance, for a $1$-component field, such $M_x$ has the form
\begin{equation}
\label{e:field-mon}
M_x = (\nabla^{\alpha_1} \varphi_x) \ldots (\nabla^{\alpha_p} \varphi_x).
\end{equation}
The $T_0$ seminorm of a local monomial $M_x$ essentially
just counts the number of fields and derivatives in $M_x$. For instance, for $M_x$
as above,
\begin{equation}
\|M_x\|_{T_0(\ell_j)}
	=
O\big(L^{-j (|\alpha| + p [\varphi])}\big)
\end{equation}
where $|\alpha| = |\alpha_1| + \cdots + |\alpha_p|$ and
\begin{equation}
[\varphi] = \frac{d - 2}{2}
\end{equation}
is the \emph{scaling dimension} of the field.
Based on this observation, we define the
\emph{dimension} of $M_x$ by
\begin{equation}
\label{e:mon-dim}
[M_x] = |\alpha| + p [\varphi].
\end{equation}
Note here that we have neglected the rapid decay of fields above the mass scale.

By \eqref{e:scaling-estimate}, $\varphi$ is approximately constant on blocks of side $L^j$. In a sense,
the fields on a block $B$ act as a unit and this contributes to a volume factor $|B| = L^{jd}$.
This leads us to compare the dimension of a monomial with the dimension $d$ of the
lattice. We say that $M_x$ is \emph{relevant} if $[M_x] < d$, \emph{marginal} if
$[M_x] = d$, and \emph{irrelevant} if $[M_x] > d$.

%%%%%%%%%%%%%%%%%%%%%%%%%%%%%%%%%%%%%%%%%%%%%%%%%%%%%%%%%%%%%%%%%%%%%%%%%%%%%%%

\subsection{Local field polynomials}
% Based on clp

For $y \in \Lambda$, we supplement \eqref{e:taudef}--\eqref{e:nablatau} and \refeq{tauphi}
by defining
\begin{equation}
\label{e:tauphi2}
\quad \tau_{\nabla\nabla,y}
	=
\begin{cases}
\frac 12 \sum_{e \in \units}
\left(
	(\nabla^e \phi)_y (\nabla^e \bar\phi)_y +
	(\nabla^e \psi)_y (\nabla^e \bar\psi)_y
\right),
	& n = 0 \\
\frac{1}{4} \sum_{|e| = 1} \nabla^e \varphi_y \cdot \nabla^e \varphi_y,
	& n \ge 1.
\end{cases}
\end{equation}

When $n = 0$, it can be shown that the only marginal and relevant local monomials
that are Euclidean-invariant and supersymmetric are constant multiples of
\begin{equation}
1, \quad \tau_x, \quad \tau_x^2, \quad \tau_{\Delta,x}, \quad \tau_{\nabla\nabla,x}.
\end{equation}
When $n \ge 1$, these are the only marginal and relevant monomials that are
Euclidean-invariant and $O(n)$-invariant (see \cite{BBS-rg-pt}).

The marginal and relevant contributions to the evolution of the renormalisation group
will be tracked by a \emph{local polynomial} (a sum of local monomials) of the form
$\sum_{y\in\Lambda} \Vc_y$, where (recall \eqref{e:obs-couple})
\begin{align}
\lbeq{Vy}
\Vc_y
	&=
g \tau_y^2 + \nu \tau_y + z \tau_{\Delta,y}
	% + y \tau_{\nabla\nabla,y}
	+ u
		\nnb&\quad
	- \1_{y=\pp}\lambda_{\pp} f_0 \sigma_0
	- \1_{y=\qq}\lambda_{\qq} f_x \sigma_x
		\nnb&\quad
	- \textstyle{\frac 12} (\1_{y=\pp} q_\pp + \1_{y=\qq}q_\qq )\sigma_\pp\sigma_\qq.
\end{align}
We have omitted $\tau_{\nabla\nabla}$ as \eqref{e:sbp-gen} gives
\begin{equation}
\label{e:nabla-delta}
\sum_{x\in\Lambda} \tau_{\nabla\nabla,x} = \sum_{x\in\Lambda} \tau_{\Delta,x}.
\end{equation}

\begin{rk}
\label{rk:noconst}
When $n = 0$, we can also omit $u$ since constant terms are not produced by the 
Gaussian super-expectation. For example, $\Ex_C\theta \tau_x$ has constant part
$0$ by \eqref{e:wick-complex2} and \eqref{e:wick-fermion2}. More generally,
this is a consequence of the supersymmetry identity \eqref{e:tau-iso}.
\end{rk}

We define $\Vcalc$ to be the space of all polynomials of the form $\Vc_y$.
Given $X \subset \Lambda$, we let
\begin{equation}
\label{e:Vcalesig}
\Vcalc(X) = \{\Vc(X) : \Vc \in \Vcalc \},
\end{equation}
where $\Vc(X)$ is defined as in \eqref{e:VX}.
We also make use of the % subspaces $\Vcalc^{(1)} \subseteq \Vcalc$ consisting
% of polynomials with $y = 0$, as well as the
subspace $\Vcalp$ of polynomials with
$u = y = q_\pp=q_\qq = 0$.
We will usually denote an element of $\Vcalp$ as $\Vp$.
For $\Vc \in \Vcalc$, we define % maps $\Vc \mapsto \Vc^{(1)} \in \Vcalc^{(1)}$ and
the map $\Vc \mapsto \Vc^{(0)} \in \Vcalp$, which sets
% Both maps replace $z\tau_{\Delta}+y\tau_{\nabla\nabla}$
% by $(z+y)\tau_{\Delta}$, and the latter additionally
$u = q_\pp = q_\qq = 0$.

We define the $\Vcalc = \Vcalc_j$ norm by
\begin{equation}
\label{e:Vnormdef}
\|\Vc\|_{\Vcalc}
	=
\max
\Big\{
	|g|, L^{2j}|\nu|, |z|, L^{4j}|u|,
	\ell_j\ell_{\sigma,j}(|\lambda_\pp|\vee|\lambda_\qq|), \;
	\ell_{\sigma,j}^{2} (|q_\pp|\vee|q_\qq|)
\Big\}
\end{equation}
on $\Vc \in \Vcalc$, which depends on the parameters $\ell_j$ and $\ell_{\sigma,j}$.
The $\Vcalc = \Vcalc_j$ norm is equivalent to the $T_0(\ell_j)$ seminorm on $\Vcalc(B)$
when $|B| = L^{jd}$:
\begin{equation}
\|\Vc\|_\Vcalc \asymp \|\Vc(B)\|_{T_0(\ell_j)} = L^{jd} \|\Vc_y\|_{T_0(\ell_j)}.
\end{equation}

%%%%%%%%%%%%%%%%%%%%%%%%%%%%%%%%%%%%%%%%%%%%%%%%%%%%%%%%%%%%%%%%%%%%%%%%%%%%%%%

\subsection{Perturbative flow}
\label{sec:pt}

Here we discuss how to maintain the form $Z_j \approx e^{-\Vp_j(\Lambda)}$ to
second order with $\Vp_j\in\Vcalp$. The basic idea begins with the \emph{cumulant
expansion}
\begin{equation}
\Ex_C \theta e^{-\Vp(\Lambda)}
	\approx
e^{-\Ex_C \theta \Vp(\Lambda) + \scriptstyle{\frac12} \Ex_C (\theta \Vp(\Lambda); \theta \Vp(\Lambda))},
\end{equation}
where
\begin{equation}
\Ex_C (F; G) = \Ex_C (FG) - (\Ex_C F) (\Ex_C G)
\end{equation}
is the \emph{truncated expectation}.
In \cite{BS-rg-loc} an operator $\Loc_x$ is defined so that $\Loc_x F$ is an
approximation of $F$ by a local polynomial at $x$. We make the split
\begin{equation}
\frac12 \Ex_C (\theta \Vp(\Lambda); \theta \Vp(\Lambda))
	=
\frac12 \Loc_x \Ex_C (\theta \Vp(\Lambda); \theta \Vp(\Lambda))
	+
\frac12 (1 - \Loc_x) \Ex_C (\theta \Vp(\Lambda); \theta \Vp(\Lambda))
\end{equation}
With
% $\Vp(\Lambda) = \sum_{x\in\Lambda}$
$e^F \approx 1 + F$, we get
\begin{equation}
\Ex_C \theta e^{-\Vp(\Lambda)}
	\approx
e^{-\Ex_C \theta \Vp(\Lambda)
	+
\scriptstyle{\frac12} \Loc_x \Ex_C (\theta \Vp(\Lambda); \theta \Vp(\Lambda))}
\Big(1 + \tfrac12 (1 - \Loc_x) \Ex_C (\theta \Vp(\Lambda); \theta \Vp(\Lambda))\Big).
\end{equation}

Based on this idea, in \cite{BBS-rg-pt}
a map\footnote{In \cite{BBS-rg-pt}, $\Vpt$ maps into a
larger space including $\tau_{\nabla\nabla}$. Here, following \eqref{e:nabla-delta},
we define $\Vpt$ by composing that map with the map that replaces
$z \tau_\Delta + y \tau_{\nabla\nabla}$ by $(z + y) \tau_\Delta$.}
$\Vpt : \Vcalp\to\Vcalc$ of the form
\begin{equation}
\label{e:Vpt-def}
\Vpt(\Vp) = \Ex_C\theta\Vp - P(\Vp)
\end{equation}
is defined so as to maintain the approximation
\begin{equation}
\label{e:Zapprox}
Z_j \approx e^{-\Vc_j(\Lambda)} (1 + W_j),
\end{equation}
where $P(\Vp)$ is a local polynomial and $W_j = W_j(\Vp)$ is non-local.
Both are explicitly defined and second-order in $V_j$;
in fact, by \cite[\eqref{IE-e:W-logwish}]{BS-rg-IE},
\begin{equation}
\label{e:Wbilinbd}
\|W_j\|_{T_0(\ell_j)}
	\le
O(\chicCov_j) \|V\|_\Vcal^2,
\end{equation}
where $\chicCov_j$ is a parameter that decays exponentially above the mass scale
and will be discussed in Section~\ref{sec:step}. We will elaborate on the  meaning
of \eqref{e:Zapprox} in Section~\ref{sec:rgcoord}.

The map $\Vpt$ depends on the covariance $C$ and
in practice we set $C = C_{j+1}$ and obtain a sequence $\Vpt = \Vc_{\mathrm{pt},j+1}$.
By successively iterating these maps, we generate a sequence of coupling constants that
we refer to as the \emph{perturbative flow}. The equations defining this flow can be
computed exactly by way of Feynman diagrams or with a computer program \cite{BBS-rg-ptsoft}.
In \cite{BBS-rg-pt}, these flow equations are summarized and it is shown that a change of
variables can be used to triangularize the resulting system of equations up to third-order
errors. Below, we summarize these transformed flow equations for $g$, $\lambda$, and $q$.

\subsubsection{The flow of \texorpdfstring{$g$}{g}}

The (transformed) perturbative flow of $g$ takes the form
\begin{equation}
\label{e:gbar}
\gbar_{j+1}
	=
\gbar_j - \beta_j  \gbar_j^{2}, \qquad \gbar_0
	=
g_0
\end{equation}
where
\begin{equation}
\beta_j = (n + 8) \sum_{x\in\Zd} (w_{j+1;0x}^2 - w_{j;0x}^2),
	\quad
w_j = \sum_{i=1}^j C_i.
\end{equation}
The sequence $\beta_j$ is closely related to the free bubble diagram
\eqref{e:bubble}. Indeed, using the telescope nature of $\sum_j \beta_j$,
we can show that
\begin{equation}
\sum_{j=1}^\infty \beta_j
	=
{\sf B}_{m^2}.
\end{equation}
The logarithmic divergence of the bubble diagram in \eqref{e:bubble-log} is
reflected in the behaviour of $g$ and, ultimately, in the appearance of logarithmic
corrections in Theorem~\ref{thm:mr}. Precisely, the results of \cite{BBS-rg-flow},
were used to show in
\cite[Proposition~\ref{log-prop:approximate-flow}]{BBS-saw4-log}
that
\begin{equation}
\label{e:gjxgjmbd}
\gbar_{j}
	=
O((\log m^{-1})^{-1}) \;\; \text{for $j \geq j_m$},
	\quad
\gbar_{j_x}
	=
O((\log |x|)^{-1}) \;\; \text{for $j_x \leq j_m$.}
\end{equation}

\begin{rk}
A heuristic argument is as follows:
Using Proposition~\ref{prop:Cdecomp}, it is straightforward to show that
\begin{equation}
\label{e:beta-bd}
\beta_j = O(L^{-j (d - 4) - s (j - j_m)_+}).
\end{equation}
Thus, a crude approximation to the flow of $\gbar$ is the recursion
\begin{equation}
y_{j+1} = y_j - c \1_{j \le j_m} y_j^2,
	\quad
c > 0.
\end{equation}
Comparing this to the differential equation $\dot y = - c y^2$, which has solutions
of the form $y(t) = (C + c t)^{-1}$, it is reasonable to expect that $y_j \approx (c j)^{-1}$
for $j \le j_m$. By definition, $y_j = y_{j_m}$ for $j > j_m$. Thus,
$y_j \approx (c \log m^{-1})^{-1}$ for $j \ge j_m$. A similar argument can be used
to study $y_{j_x}$.
% The relations \eqref{e:gjxgjmbd} follow easily if $g_j$ behaves in a similar way.
\end{rk}

Following \cite[\eqref{log-e:ggendef}]{BBS-saw4-log}, we define the parameter
$\ggen_j$ in \eqref{e:elldef-zz} as a function of two variables $(\mgen^2, \ggen_0)$ by
\begin{equation}
\label{e:ggendef}
\ggen_j(\mgen^2,\ggen_0)
	=
\gbar_j(0,\ggen_0) \1_{j \le j_{\mgen}} + \gbar_{j_{\mgen}}(0,\ggen_0) \1_{j > j_{\mgen}}.
\end{equation}
These parameters play an important role in Section~\ref{sec:step} and in the proof of
Theorem~\ref{thm:rhatflow}.

\subsubsection{The flow of \texorpdfstring{$\lambda$ and $q$}{lambda and q}}

It was shown in \cite[\eqref{pt-e:lambdapt2}--\eqref{pt-e:qpt2}]{BBS-rg-pt} (for $n = 0$)
and \cite[Proposition~\ref{phi4-prop:pt}]{ST-phi4} (for $n \ge 1$) that,
with $C = C_{j+1}$ and $u = 0, x$,
\begin{align}
\label{e:lampt}
\lambda_{u,\pt}
	&=
\begin{cases}
(1 - \delta[\nu w^{(1)}]) \lambda_u,
	& j + 1 < j_x \\
\lambda_u,
	& j + 1 \ge j_x
\end{cases}
	\\
\label{e:qpt}
q_\pt
	&=
q + \lambda_0 \lambda_x C_{0x},
\end{align}
where
\begin{equation}
\label{e:deltanuw1}
\delta[\nu w^{(1)}] = (\nu + 2 g C_{00}) w^{(1)}_{j+1} - \nu w^{(1)}_j,
	\qquad
w^{(1)}_j = \sum_{x\in\Lambda} \sum_{i=1}^j C_{i;0x}.
\end{equation}
Note that $q_\pt = q$ for $j + 1 < j_x$.

%%%%%%%%%%%%%%%%%%%%%%%%%%%%%%%%%%%%%%%%%%%%%%%%%%%%%%%%%%%%%%%%%%%%%%%%%%%%%%%
%%%%%%%%%%%%%%%%%%%%%%%%%%%%%%%%%%%%%%%%%%%%%%%%%%%%%%%%%%%%%%%%%%%%%%%%%%%%%%%

\section{Non-perturbative coordinate}
% Based on clp
\label{sec:rgcoord}

Let $\volume$ denote either $\Lambda_N$ or $\Zd$. We allow $\Ncal$ to depend on
$\volume$. If $\volume = \Lambda$, then $\Ncal = \Ncal(\Lambda)$ was defined in
Section~\ref{sec:Ncal}. Otherwise, we set
\begin{equation}
\Ncal(\Zd) = \bigcup_{\text{finite } X \subset \volume} \Ncal(X).
\end{equation}

We set $N(\volume) = N$ if
$\volume = \Lambda_N$ and $N(\volume) = \infty$ if $\volume = \Zd$.
For $j \le N(\volume)$ (meaning $j < \infty$ if $N(\volume) = \infty$), we partition
$\volume$ into disjoint
\emph{scale-$j$ blocks} of side length $L^j$, each of which is a translate of
the block $\{ x \in \Lambda : 0 \le x_i < L^j, i = 1, \ldots, d\}$.
A scale-$j$ \emph{polymer} is a union of scale-$j$ blocks.
Given a scale-$j$ polymer $X$ and $k \le j$, we let $\Bcal_k(X)$
(respectively, $\Pcal_k(X)$)
denote the set of all scale-$k$ blocks (respectively, scale-$k$ polymers) in $X$.
We sometimes write $\Bcal_j = \Bcal_j(\volume)$ and $\Pcal_j = \Pcal_j(\volume)$
when the volume $\volume$ is implicit.

Any map $F : \Bcal_j \to \Ncal$ can be extended to a map on $\Pcal_j$ by \emph{block-factorization}:
\begin{equation}
\label{e:block-fact}
F(X) = F^X \coloneqq \prod_{B\in\Bcal_j(X)} F(B).
\end{equation}
Given maps $F, G : \Pcal_j(\Lambda) \to \Ncal$ (sometimes called \emph{polymer activities}),
we define the \emph{circle product} $F \circ G : \Pcal_j(\Lambda) \to \Ncal$ by
\begin{equation}
\label{e:circ}
(F \circ G)(X) = \sum_{Y\in\Pcal_j(X)} F(X \setminus Y) G(Y).
\end{equation}
The circle product is commutative, associative, and has the identity element
\begin{equation}
\1_\varnothing(X) =
\begin{cases}
1,	& X = \varnothing \\
0,	& X \ne \varnothing
\end{cases}.
\end{equation}
We track $Z_j$ using \emph{renormalisation group coordinates}
$u_j, q_{0,j}, q_{x,j} \in \R$,
$I_j, K_j : \Pcal_j \to \Ncal$ such that
\begin{equation}
\label{e:IcircKnew}
Z_j = e^{\zeta_j}(I_j\circ K_j)(\Lambda),
	\qquad
\zeta_j
	=
- u_j|\Lambda|
+ \textstyle{\frac 12} (q_{\pp,j} + q_{\qq,j}) \sigma_\pp\sigma_\qq
.
\end{equation}
The coordinate $I_j = I_j(\Vp, \cdot)$ is defined by setting
\begin{equation}
\label{e:Idef}
I_j(\Vp, B)
	=
% \prod_{B \in \Bcal_j(X)}
e^{-\Vp(B)} (1 + W_j(B, \Vp)), \quad X \in \Pcal_j,
	\quad
\Vp \in \Vcalp
\end{equation}
and extending this by block-factorization.
Thus, \eqref{e:IcircKnew} gives a rigorous implementation of \eqref{e:Zapprox}.

Before defining the space in which $K_j$ lies, we need the following notions:
\begin{itemize}
\item
We call a nonempty polymer $X\in \Pcal_j$ \emph{connected}
if for any $x, x' \in X$, there is a sequence
$x = x_0, \ldots, x_n = x' \in X$ such that
$|x_{i+1} - x_i|_\infty = 1$ for $i = 0, \ldots, n - 1$.
Let $\Ccal_0 = \Ccal_0(\volume)$ denote the set of connected polymers.

\item
For $X \in \Pcal_j$, let $|X|_j$ denote the number of scale-$j$ blocks in $X$.
We call a connected polymer $X\in\Ccal_j$ a \emph{small set} if $|X|_j \le 2^d$.
Let $\Scal_j = \Scal_j(\volume)$ denote the collection of small sets.
The \emph{small set neighbourhood} $X^\square$ of a polymer $X$ is defined by
\begin{equation}
\label{e:ssn}
X^\Box = \bigcup_{Y\in\Scal_j : Y \cap X \ne \varnothing} Y.
\end{equation}

\item
Two polymers $X, Y$ \emph{do not touch} if $\min(|x - y|_\infty : x \in X, y \in Y) > 1$.
We let ${\rm Comp}(X)$ denote the set of maximal connected components that do not touch
in $X$.
\end{itemize}

\new{We say that a map $F : \Pcal_j \to \Ncal$ is \emph{Euclidean-covariant} if
$E(F(X)) = F(EX)$ for all $X\in\Pcal_j$ and all automorphisms $E$ of $\volume$.
We also say that $F$ is \emph{gauge-invariant}, \emph{supersymmetric}, or
\emph{$O(n)$-invariant} if $F(X)$ is gauge-invariant, supersymmetric, or $O(n)$-invariant,
respectively.}

\begin{defn}
For $j \le N(\volume)$, let $\CKspace_j = \CKspace_j(\volume)$ denote the real
vector space of maps $K : \Ccal_j(\volume) \to \Ncal(\volume)$ satisfying the following
properties:
\begin{itemize}
\item
Field Locality: If $X \in \Ccal_j$, then $K(X) \in \Ncal(X^\square)$.
Also: (i) $\pi_\alpha K(X) = 0$ unless $\alpha \in X$ for $\alpha = 0, x$;
(ii) $\pi_{0x} K(X) = 0$ unless $a\in X$ and $x \in X^\square$ or vice versa;
and (iii) $\pi_{0x} K(X) = 0$ if $X \in \Scal_j$ and $j < j_x$.

\item
Symmetry: (i) \new{$\pi_\bulk K$ is Euclidean-covariant;
(ii) if $n = 0$, then $K$ is gauge-invariant and $\pi_\bulk K$ is supersymmetric
and has no constant part; if $n \ge 1$, then $\pi_\bulk K$ is $O(n)$-invariant.}
% and (iii) $\pi_\bulk K$ is supersymmetric and has no constant part if $n = 0$
% or $O(n)$-invariant if $n \ge 1$.
% and (iii) $\pi_\bulk K$ is Euclidean-covariant.
\end{itemize}
We let $\Kspace_j = \Kspace_j(\volume)$ denote the real vector space of functions
$K \in \CKspace_j$ with the following additional property:
\begin{itemize}
\item
Component factorization: If $X \in \Pcal_j$, then $K(X) = \prod_{Y\in{\rm Comp}(X)} K(Y)$.
\end{itemize}
\end{defn}

Addition in $\CKspace_j$ is defined by $(F_1 + F_2)(X) = F_1(X) + F_2(X)$.
We extend any $F \in \CKspace_j$ to $\Kspace_j$ by defining
$F(X) = \prod_{Y\in{\rm Comp}(X)} F(Y)$.

\subsection{Initial coordinates}
\label{sec:initIK}

At scale $j = 0$, we are given $\Vp^+_0$ as defined in \eqref{e:V0def}
and we set $\zeta_0 = 0$. In particular,
the initial values of $u$, $q_0$, $q_x$ are zero, and the initial values of $\lambda_0$, $\lambda_x$
are $1$. By definition, $W_0 = 0$.
For $X \subset \Lambda$, we define
\begin{equation}
\label{e:IK0def}
I_0^+(X) = I_0(\Vp^+_0, X) = \prod_{y\in X} e^{-\Vp^+_{0,y}},
	\qquad
K_0^+(X) = \prod_{y \in X} I_{0,y}^+ (e^{-\gamma_0 U^{+}_{y}} - 1),
\end{equation}
\new{
where $I^+_{0,y} = I^+_0(\{y\})$.
It is straightforward to verify that $K_0 \in \Kcal_0$.
Moreover, by \eqref{e:Z0def},
\begin{equation}
Z_0
	=
\prod_{y\in\Lambda} \Big(I^+_{0,y} + I^+_{0,y} (e^{-\gamma_0 U^+_y} - 1)\Big)
	=
(I^+_0 \circ K^+_0)(\Lambda).
\end{equation}
The second equality here follows from the binomial expansion formula
\begin{equation}
\label{e:binom}
\prod_{y\in\Lambda} (F_y + G_y)
	=
\sum_{X\subset\Lambda}
\Big(\prod_{y\in\Lambda\setminus X} F_y\Big)
\Big(\prod_{z\in X} G_z\Big).
\end{equation}
Thus, $Z_0$ takes the form \eqref{e:IcircKnew}}
% It is straightforward to verify that $K_0 \in \Kcal_0$.
% With these choices, $Z_0$ (recall \refeq{Z0def}) takes the form \eqref{e:IcircKnew},
and we seek
$(u_j, q_{0,j}, q_{x,j}, \Vp_j, K_j)$ such that this continues to hold as the scale advances.

% \commentbw{under (2.6.8): it seems too fast to say that (2.6.5) holds, I think it should
% be spelled out.}

Equivalently, given $(\Vp_j, K_j)$, we must define
$(\delta u_{j+1}, \delta q_{0,j+1}, \delta q_{x,j+1}, V_{j+1}, K_{j+1})$ so that
\begin{equation} \label{e:IcircKdu}
	\Ex_{j+1}\theta(I_j \circ K_j)(\Lambda)
	=
	e^{-\delta \zeta_{j+1}}(I_{j+1} \circ K_{j+1})(\Lambda).
\end{equation}
Moreover, we need $K_j$ to contract as the scale advances, under an appropriate norm.
The construction of (scale-dependent) maps $\Vc_+$ and $K_+$ such that
\eqref{e:IcircKdu} holds with
\begin{equation}
(\delta u_{j+1}, \delta q_{0,j+1}, \delta q_{x,j+1}, \Vp_{j+1})
	=
\Vc_+(\Vp_j, K_j),
	\quad
K_{j+1} =  K_+(\Vp_j, K_j)
\end{equation}
is the main accomplishment of \cite{BS-rg-step}.

% \begin{rk}
% We will sometimes view $K^+_0$ as a function of parameters $(g_0, \gamma_0, \nu_0, z_0)$.
% In \cite{BSW-saw-sa} we use a \emph{different} choice of $K^+_0$ to handle the
% case $\gamma_0 < 0$. Since we will not deal with this case in this thesis, we
% \emph{define}
% \begin{equation}
% g_0 = g (1 + z_0)^2
% 	\text{ and }
% K^+_0 = 0
% 	\text{ if }
% \gamma_0 < 0.
% \end{equation}
% \commentbw{Rk 2.6.2: Why make definitions for $\gamma_0 < 0$ when you say we will not
% consider this case?  Better just to insist that $\gamma_0 \ge 0$ always in
% the thesis?}
% \end{rk}

%%%%%%%%%%%%%%%%%%%%%%%%%%%%%%%%%%%%%%%%%%%%%%%%%%%%%%%%%%%%%%%%%%%%%%%%%%%%%%%
%%%%%%%%%%%%%%%%%%%%%%%%%%%%%%%%%%%%%%%%%%%%%%%%%%%%%%%%%%%%%%%%%%%%%%%%%%%%%%%

\section{Renormalisation group step}
% from saw-sa and clp
\label{sec:step}

% For fixed $(\mgen^2, \ggen_0) \in [0, \delta) \times (0, \delta)$,
% we define a sequence $\ggen_j = \ggen_j(\mgen^2, \ggen_0)$ by
% \begin{equation} \label{e:ggendef}
%   \ggen_j(m^2,g_0) =
%   \gbar_j(0,g_0) \1_{j \le j_m} + \gbar_{j_m}(0,g_0) \1_{j > j_m},
% \end{equation}
% where $\gbar_j = \gbar_j(m^2, g_0)$ is the sequence discussed in Section~\ref{sec:pt};
% in particular, $\ggen_0(\mgen^2, \ggen_0) = \ggen_0$.
In \cite[Section~\ref{step-sec:Knorms}]{BS-rg-step},
a sequence of norms $\|\cdot\|_{\Wcal_j} = \|\cdot\|_{\Wcal_j(\mgen^2, \ggen_j, \Lambda)}$
parameterized by $(\mgen^2, \ggen_j)$ is defined on $\Kcal_j$.
These are defined in terms of the $T_\phi(\h_j)$ norms with parameters $\h_j = \ell_j, h_j$.
In order to make use of the improved norm parameters with $s > 1$,
we must modify the definition of $\Wcal_j$ when $j$ is above the mass scale.
\new{The precise definition of the $\Wcal_j$ norm adapted to our current setting
will be discussed in Section~\ref{sec:KWcal}}.
We note here only the fact that the $\Wcal_j(\Lambda)$
norm dominates the $T_0(\ell_j)$ norm in the following sense:
\begin{equation}
\label{e:T0dom}
\|F(\Lambda)\|_{T_0(\ell_j)} \le \|F\|_{\Wcal_j}.
\end{equation}
We let $\Wcal_j = \Wcal_j(\volume)$ denote the space of $K\in\Kcal_j(\volume)$ with
finite $\Wcal_j$ norm and
denote the ball of radius $r$ in the normed space $\Wcal_j$ by $B_{\Wcal_j}(r)$.

% \commentbw{Sec 2.7: Are you never going to define the $\Wcal$ norm?  I think you need
% that defn in the thesis. Same for $\vartheta_j$.  If you do define it later
% you should say where here.}

\new{
Let
\begin{equation}
j_\Omega
	=
j_\Omega(m^2)
	=
\inf \{ k \ge 0 : |\beta_j| \le 2^{-(j - k)} \|\beta\|_\infty \text{ for all } j \}
\end{equation}
and note that, by \eqref{e:beta-bd}, $j_\Omega < \infty$ for $m^2 > 0$.
We define
\begin{equation}
\label{e:chicCov-def}
\chicCov_j = \chicCov_j(m^2) = 2^{-(j - j_\Omega)_+}
\end{equation}
and}
% In \cite[\eqref{log-e:mass-scale}--\eqref{log-e:chidef}]{BBS-saw4-log},
% a function $\chicCov_j = \chicCov_j(m^2)$ (denoted $\chi_j$ in \cite{BBS-saw4-log})
% is defined in such a way that $\chicCov_j$ decays exponentially
% when $j$ is sufficiently large depending on $m$. We
write $\chicCovgen_j = \chicCov_j(\mgen^2)$.
Given constants $\alpha > 0$ and $C_\DV > 0$, we define the (finite-volume)
renormalisation group domains
\begin{align}
\label{e:DVdef}
\DV_j
	&=
\{ \Vp\in \Vcalp :
	g > C_{\DV}^{-1} \ggen_j, \; \|\Vp\|_{\Vcalc} < C_{\DV} \ggen_j \}, \\
\label{e:domRG}
\domRG_j
	&= \domRG_j(\volume)
	= \DV_j \times B_{\Wcal_j}(\alpha \chicCovgen_j \ggen_j^3).
\end{align}
The domain $\DV_j$ is independent of the volume $\volume$ while $\domRG_j$
depends on $\volume$ through $\Wcal_j$.

In the statement of the following theorem, we fix the scale $j$ and
consider maps $\Vc_+ = \Vc_{j+1}$ and $K_+ = K_{j+1}$ that act on the domain
$\domRG_j$ and map into $\Vcalc$, $\Kcal_{j+1}$, respectively.
We will drop the scale $j$ from the notation for objects at scale $j$
and replace $j + 1$ with $+$.
\new{When $\volume=\Lambda$, we take $j < N$.}
The deviation of the map $\Vc_+$ from the perturbative map $\Vpt$
is denoted by $R_+$:
\begin{equation}
\label{e:Rplusdef}
    R_+(\Vp,K) = \Vc_+(\Vp,K) -\Vpt(\Vp).
\end{equation}

The renormalisation group map depends also on the mass $m^2$ through its
dependence on the covariance $C_{j+1}$.
We let $\Igen_j(\mgen^2)$ be the neighbourhood of $\mgen^2$ defined by
\begin{equation}
\lbeq{Itilint}
    \Igen_j = \Igen_j(\mgen^2) =
    \begin{cases}
    [\frac 12 \mgen^2, 2 \mgen^2] \cap \Iint_j & (\mgen^2 \neq 0)
    \\
    [0,L^{-2(j-1)}] \cap \Iint_j & (\mgen^2 =0)
    \end{cases},
\end{equation}
where $\Iint_j = [0, \delta]$ if $j < N$ and $\Iint_N = [\delta L^{-2 (N - 1)}, \delta]$.

% \commentbw{Thm 2.7.1: you need a restriction on $j$ if $\volume=\Lambda$ and should say in the
% thm that $V$ can be $\Lambda$ or $\Zd$.}

\begin{theorem}
\label{thm:step-mr-fv}
Let $d = 4$, $n \ge 0$, and $\volume = \Lambda$ or $\Zd$.
Fix $s > 1$ (or $s = 0$).
Let $C_\DV$ and $L$ be sufficiently large.
There exist $M>0$, $\delta >0$, and $\kappa = O(L^{-1})$			% added
such that for $\ggen \in (0,\delta)$ and $\mgen^2 \in \Iint_+$,		% added
and with the domain $\domRG$ defined using any $\DVa> M$,
the maps
\begin{equation}
\label{e:RKplusmaps}
R_+:\domRG \times \Igen_+ % (\mgen^2)
	\to \Ucal,
		\quad
K_+:\domRG \times \Igen_+ % (\mgen^2)
	\to \Wcal_{+}%(\sgen_+)
\end{equation}
are analytic in $(V, K)$											% added
and satisfy the estimates
\begin{equation}
\label{e:RKplus}
\|R_+\|_{\Ucal}
	\le
M\chigen_+\ggen_+^{3},
	\qquad
\|K_+\|_{\Wcal_+}
	\le
M\chigen_+ \ggen_+^{3}
\end{equation}
and
\begin{equation}
\label{e:DKkappa}
\|D_K K_+\|_{L(\Wcal,\Wcal_+)} \le \kappa.
\end{equation}
When $\volume = \Lambda$, these maps define $(\Vp,K)\mapsto (\Vc_+,K_+)$
obeying \eqref{e:IcircKdu}.
\end{theorem}

\new{
\begin{rk}
When the improved weights are used, a new norm must be employed above the
mass scale. This will be discussed in Section~\ref{sec:newnorm}. A technical
requirement of this new norm is that we set $s > 1$ rather than $s > 0$ as
Lemma~\ref{lem:mart} fails with $s \in (0, 1)$. This issue is absent when
$s = 0$ as we do not change the norms in this case. Since we are ultimately
interested only in the cases $s = 0$ and $s$ large, we have not attempted to
handle the case $s \in (0, 1)$.
\end{rk}}

With $s = 0$ in the choice of weights $\ell_j$ and $\ell_{\sigma,j}$,
this theorem was the main achievement of \cite{BS-rg-step}.
The statement in \cite{BS-rg-step} with $s = 0$ additionally contains
bounds on the derivatives of the maps $R_+$ and $K_+$. Our improvements
apply to these bounds as well, but we do not state them here as we will
not make direct use of these bounds.
One of the main novelties in this thesis is the case $s > 1$,
for which the bounds on the observables derived from \eqref{e:RKplus}
are greatly improved beyond the mass scale.

% \commentbw{p.51 last para of Sec 2.7: you make the distinction $s>1$ vs $s=0$.  Do you mean
% $s>0$ vs $s=0$?  Also you say $s>1$ near bottom p.51 and maybe other places too.}

Note that the maps $R_+$ and $K_+$ themselves are \emph{independent} of $s$.
The proof of Theorem~\ref{thm:step-mr-fv} involves showing that the
inductive estimates \eqref{e:RKplus} hold for any $s$. In some cases,
we will make use of these estimates both with $s > 1$ and $s = 0$.
The proof for $s > 1$ is an adaptation of the proof of the $s = 0$ case contained
in\footnote{For $n \ge 1$, there is an additional step to deal with observables.
This is dealt with in the proof of \cite[Theorem~\ref{phi4-thm:step-mr-fv}]{ST-phi4}
and is unchanged in the present context.}
\cite{BS-rg-IE,BS-rg-step}.
Some steps in this proof continue to hold unchanged whereas others require
some modification. As mentioned above, a major change that is required is
a new definition of $\Wcal_j$ above the mass scale. A detailed verification
that the proof holds for $s > 1$ is carried out in Chapter~\ref{sec:RGstep}.

%%%%%%%%%%%%%%%%%%%%%%%%%%%%%%%%%%%%%%%%%%%%%%%%%%%%%%%%%%%%%%%%%%%%%%%%%%%%%%%
%%%%%%%%%%%%%%%%%%%%%%%%%%%%%%%%%%%%%%%%%%%%%%%%%%%%%%%%%%%%%%%%%%%%%%%%%%%%%%%

\section{Renormalisation group flow}
% mainly from saw-sa

% \commentbw{Sec 2.8: you make reference to \cite{BBS-saw4-log} but this is for
% $n=0$ yet you seem to be claiming things here for all $n \ge 0$?
% Concerning Thm 2.8.1, it is not clear from reading Sec 2.8 if you are going
% to prove this theorem or claim that it is already proved somewhere else (where?,
% I hope you are going to discuss the proof!).
% I understand from the last sentence of Sec 2.7 that you will say why Thm 2.7.1
% is true.  According to p.34 Thms 2.7.1 and 2.8.1 are the main results of the thesis.
% I think you should be repeating this loud and clear and also saying where the proof is.}

Theorem~\ref{thm:step-mr-fv} allows us to perform a single renormalisation group step.
% The estimate \eqref{e:DKkappa} expresses a contractive property of the map $K_+$.
The fact that $K_+$ is a contraction, \new{as expressed by the estimate \eqref{e:DKkappa},
was used in \cite[Proposition~\ref{log-prop:flow-flow}]{BBS-saw4-log} to construct}
\emph{critical} initial conditions
$\nu_0^c, z_0^c$ depending on $(m^2, g_0, n)$ such that the renormalisation
group map can be iterated indefinitely \new{(this was shown for $n = 0$ in
\cite{BBS-saw4-log} but extends without difficulty to $n \ge 1$ as discussed
in \cite{BBS-phi4-log})}. This
results in a sequence $(\Vc_j, K_j)$ generated by the renormalisation group map, hence
whose elements lie in the domains $\domRG_j$. This was proved with $s = 0$,
but the sequence itself is independent of $s$ and continues to exist in our
setting. \new{In particular}, Theorem~\ref{thm:step-mr-fv} shows that this sequence satisfies
improved estimates. Thus, there is no difficulty in extending
\cite[Proposition~\ref{log-prop:flow-flow}]{BBS-saw4-log} to the $s$-dependent
domains used here.

\new{However, in order to study the WSAW-SA and the generalized $|\varphi|^4$ model,
we must extend \cite[Proposition~\ref{log-prop:flow-flow}]{BBS-saw4-log} to $\gamma_0\ne0$.
We state this extension as Theorem~\ref{thm:rhatflow} below, which is one of the main
contributions of this thesis. Its proof, which depends on the results of \cite{BBS-rg-flow}
together with a specially tailored version of the implicit function theorem, is the
subject of Chapter~\ref{sec:RGflow}. We note that, for $n = 0$, this proof first
appeared in \cite{BSW-saw-sa}; the proof for $n \ge 1$ is new to this thesis.}
% In summary, Theorem~\ref{thm:rhatflow} is an extension
% % constructs critical initial conditions} $\hat\nu_0^c, \hat z_0^c$ that now
% % depend on $\gamma_0$ as well as $(m^2, g_0, n)$. Thus, this theorem is an extension
% to $s > 1$ and $\gamma_0 \ne 0$ of \cite[Proposition~\ref{log-prop:flow-flow}]{BBS-saw4-log}.}

% is used, in Theorem~\ref{thm:rhatflow},
% to prove that, for $m^2$ and $g_0$ sufficiently small, there exist
% \emph{critical} initial conditions
% $\nu_0 = \nu_0^c(m^2, g_0, \gamma_0; n)$ and $z_0 = z_0^c(m^2, g_0,\gamma_0; n)$
% such that for any $N$,
% iteration of the maps $(\Vc_+,K_+)$ defines a sequence $(\Vp_j, K_j)$
% which lies in the domain $\domRG_j$ and obeys the estimates \refeq{RKplus}
% \emph{for all} $j = 1, \ldots, N$.
% % This construction of critical initial conditions makes use of the $s=0$ version
% % of \refeq{elldef-zz}. However, the resulting sequence is independent of $s$
% % and can therefore be bounded with any $s$.

% In \cite[Proposition~\ref{log-prop:flow-flow}]{BBS-saw4-log}, it was shown
% that for $\gamma_0 = 0$ there exist \emph{critical} initial conditions
% $\nu_0^c, z_0^c$ depending on $(m^2, g_0, n)$ such that the renormalisation
% group map can be iterated indefinitely \new{(this was shown for $n = 0$ in
% \cite{BBS-saw4-log} but extends without difficulty to $n \ge 1$ as discussed
% in \cite{BBS-phi4-log})}. This
% results in a sequence $(\Vc_j, K_j)$ generated by the renormalisation group map, hence
% whose elements lie in the domains $\domRG_j$. This was proved with $s = 0$,
% but the sequence itself is independent of $s$ and continues to exist in our
% setting. However, Theorem~\ref{thm:step-mr-fv} shows that this sequence satisfies
% improved estimates. Thus, there is no difficulty in extending
% \cite[Proposition~\ref{log-prop:flow-flow}]{BBS-saw4-log} to the $s$-dependent
% domains used here.

% % However, \cite{BBS-saw4-log} studied the WSAW ($\gamma = 0$).
% \new{However, in order to study the WSAW-SA and the generalized $|\varphi|^4$ model,
% we must extend \cite[Proposition~\ref{log-prop:flow-flow}]{BBS-saw4-log} to $\gamma_0\ne0$.
% We do so} by taking
% advantage of the additional generality in \cite{BBS-rg-flow}, whose
% main result is the basis for the proof of \cite[Proposition~\ref{log-prop:flow-flow}]{BBS-saw4-log}. The result, which we state as Theorem~\ref{thm:rhatflow}, is the
% construction of critical initial conditions $\hat\nu_0^c, \hat z_0^c$ that now
% depend on $\gamma_0$ as well as $(m^2, g_0)$. Thus, this theorem is an extension
% to $s > 1$ and $\gamma_0 \ne 0$ of \cite[Proposition~\ref{log-prop:flow-flow}]{BBS-saw4-log}.
% \new{and \cite[Theorem~\ref{phi4-log-thm:flow-flow}]{BBS-phi4-log}.}

Let $\delta > 0$ and suppose $r : [0, \delta] \to [0, \infty)$
is a continuous \emph{positive-definite} function; by this we
mean\footnote{Note that our usage of this term is
different from that in the theory of quadratic forms.}
that $r(x) > 0$ if $x > 0$ and $r(0) = 0$.
We define
\begin{equation}
\lbeq{Ddef}
D(\delta, r)
	=
\new{\{ (w, x, y) \in [0, \delta]^3 : y \leq r(x) \}}
\end{equation}
and we let $C^{0,1,\pm}(D(\delta, r))$ denote the space of continuous functions
$f = f(w, x, y)$ on $D(\delta, r)$
that are $C^1$ in $(x, y)$ away from $y = 0$, $C^1$ in $x$ everywhere,
and whose right-derivative in $y$ at $y = 0$ exists.
In our applications, we take $w = m^2$, $x = g_0$ or $\gcc$,
and $y = \gamma_0$ or $\gamma$.

\begin{theorem}
\label{thm:rhatflow}
There exists a domain $D(\delta, \hat r)$ (with $\delta > 0$ and $\hat r$
positive-definite) and functions $\hat\nu_0^c, \hat z_0^c \in C^{0,1,\pm}(D(\delta, \hat r))$
such that for any $(m^2, g_0, \gamma_0) \in D(\delta, \hat r)$
with $g_0 > 0$ and $m^2 \in [\delta L^{-2 (N - 1)}, \delta)$, the following holds:
if $(\Vc_0, K_0) = (\Vp^+_0, K^+_0)$ with $(\nu_0, z_0) = (\hat\nu_0^c, \hat z_0^c)$,
then for any $N \in \N$, there exists a sequence $(\Vc_j, K_j) \in \domRG_j(m^2, g_0)$ such that
\begin{equation}
\label{e:VjKjDj-hat}
(\Vc_{j+1},K_{j+1})
	=
(\Vc_{j+1}(\Vp_j, K_j), K_{j+1}(\Vp_j, K_j)) \text{ for all } j < N
\end{equation}
and \eqref{e:IcircKdu} is satisfied.
Moreover, the sequence $\Vc_j, j = 1, \ldots, N$ is independent of the volume $\Lambda$ and
\begin{equation}
\label{e:hat-est}
\hat\nu_0^c = O(g_0),
\quad
\hat z_0^c = O(g_0)
\end{equation}
uniformly in $(m^2, \gamma_0)$.
% Moreover, the second-order evolution equation for $V_j$ is independent of $\gamma_0$.
\end{theorem}

Note that in the statement of Theorem~\ref{thm:rhatflow} flow, we have evaluated
the domains $\domRG_j$ at $(\mgen^2, \ggen_0) = (m^2, g_0)$, where $m^2$ is the
mass in the covariance $C$ and $g_0 = g (1 + z_0^c)^2$.

%%%%%%%%%%%%%%%%%%%%%%%%%%%%%%%%%%%%%%%%%%%%%%%%%%%%%%%%%%%%%%%%%%%%%%%%%%%%%%%

\section{Bibliographic remarks}

The notion of a polymer used in Section~\ref{sec:rgcoord} was introduced in
\cite{GK71}. The utility of multi-scale decompositions of a singular covariance
as a sum of regular covariances in the context of the renormalisation group was
probably first clearly articulated in \cite{BCGNOPS78}. The use of expansions of
the form \eqref{e:circ} together with carefully weighted norms to achieve rigorous
control of the renormalisation group map goes back to Brydges and Yau \cite{BY90}.
This method was extended by Dimock and Hurd, see e.g.\ \cite{DH92,DH00}. Finite-range
decompositions were first used with this method to study a continuum model in
\cite{MS00}, following a suggestion of Brydges. \emph{Lattice} covariance
decompositions were constructed in \cite{BGM04} and used in \cite{MS08} to study
the renormalisation group flow for the supersymmetric field theory corresponding
to WSAW; however, critical exponents were not computed. Critical exponents for
a version of weakly self-avoiding walk on a hierarchical lattice were computed
by a renormalisation group method in \cite{BI03c,BI03d} (such hierarchical models
go back to \cite{Dyso69}).
