\chapter{Renormalisation group method}

\todo{Somewhere should include a discussion of the flow of
$g$, $\lambda$, and $q$.}

% Discuss RG method with improved norms and non-trivial $K_0$

%%%%%%%%%%%%%%%%%%%%%%%%%%%%%%%%%%%%%%%%%%%%%%%%%%%%%%%%%%%%%%%%%%%%%%%%%%%%%%%
%%%%%%%%%%%%%%%%%%%%%%%%%%%%%%%%%%%%%%%%%%%%%%%%%%%%%%%%%%%%%%%%%%%%%%%%%%%%%%%

\section{Unified notation}

To unify our treatment of the two models,
we define the forms $\tau_x, \tau_{\Delta,x}, |\nabla\tau_x|^2 \in \Ncal^\varnothing$ according
to \eqref{e:taudef}--\eqref{e:nablatau} if $n = 0$ and
\begin{equation}
\label{e:tauphi}
\tau_x = \tfrac{1}{2} |\varphi_x|^2,
	\quad
\tau_{\Delta,x} = \tfrac{1}{2} \varphi_x \cdot (-\Delta \varphi)_x,
	\quad
|\nabla\tau_x|^2 = \sum_{|e|=1} (\nabla^e |\phi_x|^2)^2
\end{equation}
if $n \ge 1$.

Then by \eqref{e:Vdef1}, \eqref{e:Udef-pos}, and \eqref{e:Vdef2},
% $V_{g,\gamma,\nu,N} \in \Ncal^\varnothing$ is given by
\begin{equation}
V_{g,\gamma,\nu,N}
	=
\sum_{x\in\Lambda_N}
\Big(
	(g - \gamma) \tau_x^2 + \nu \tau_x + \tau_{\Delta,x} + \tfrac{1}{4 d} \gamma |\nabla\tau_x|^2
\Big)
\end{equation}
for any choice of $n$.
We write
\begin{equation}
\langle F \rangle_{g,\gamma,\nu,N}
	=
\begin{cases}
\displaystyle \int F e^{-U_{g,\gamma,\nu,N}},           & n = 0 \\
\displaystyle \frac{1}{Z_{g,\gamma,\nu,N}}
	\int F(\varphi) e^{-U_{g,\gamma,\nu,N}} \; d\varphi,  & n \ge 1.
\end{cases}
\end{equation}
By \eqref{e:two-point-function-phi4}, \eqref{e:Givlc}, and \eqref{e:Grep-pos-bis},
\begin{equation}
G_x(g, \gamma, \nu; n) = \lim_{N\to\infty} G_{x,N}(g, \gamma, \nu; n),
\end{equation}
where
\begin{equation}
G_{x,N}(g, \gamma, \nu; n)
	=
\begin{cases}
\langle \phib_0 \phi_x \rangle_{g,\gamma,\nu,N},      & n = 0 \\
\langle \varphi_0 \cdot \varphi_x \rangle_{g,\gamma,\nu,N}  & n \ge 1.
\end{cases}
\end{equation}
By \eqref{e:susceptibility-def}, \eqref{e:chiNdef-pre}, and \eqref{e:chilim-pre},
\begin{equation}
\label{e:chilim}
\chi(\gcc, \gamma, \nu; n)
	=
\lim_{N\to\infty} \chi_N(\gcc, \gamma, \nu; n)
\end{equation}
with
\begin{equation}
\label{e:chiNdef}
\chi_N(\gcc, \gamma, \nu; n)
	=
\sum_{x\in\Lambda_N} G_{x,N}(\gcc, \gamma, \nu; n).
\end{equation}
Lastly, we note that
\begin{equation}
\label{e:clp}
\xi_p(g, \gamma, \nu; n)
	=
\left(\frac{\sum_{x\in\Zd} |x|^p G_x(g, \gamma, \nu; n)}{\chi(g, \gamma, \nu; n)}\right)^{1/p}
\end{equation}
for all $n \ge 0$.
For both models, we have
\begin{equation}
\label{e:nuc-def}
\nu_c = \nu_c(\gcc, \gamma, n) = \inf \{ \nu : \chi(g, \gamma, \nu; n) < \infty \}.
\end{equation}

%%%%%%%%%%%%%%%%%%%%%%%%%%%%%%%%%%%%%%%%%%%%%%%%%%%%%%%%%%%%%%%%%%%%%%%%%%%%%%%

\section{Reformulation of the problem}

Given $m^2>0$ and $z_0 >-1$, let
\begin{equation}
\label{e:gg0}
g_0 = (\gcc - \gamma) (1 + z_0)^2,
	\quad
\nu_0 = \nu (1 + z_0) - m^2,
	\quad
\gamma_0 = \frac{1}{4d} \gamma (1 + z_0)^2.
\end{equation}
Let ${\sf h} = n^{-1/2}(1, \ldots, 1) \in \R^n$.
With the two points $0,x\in \Lambda$ fixed,
we introduce \emph{observable fields} $\sigmaa, \sigmab \in \R$, and define
\begin{equation}
\label{e:V0def}
	V^+_{0,x}
	= g_0\tau_x^2 + \nu_0 \tau_x + z_0 \tau_{\Delta,x} -
	\sigma_\pp (\varphi_{\pp} \cdot {\sf h})
	- \sigma_\qq (\varphi_{\qq} \cdot {\sf h}),
	\quad
	U^+_x = |\nabla \tau_x|^2.
\end{equation}
Let
\begin{equation}
\label{e:Z0def}
Z_0 = \prod_{x\in \Lambda} e^{-(V^+_{0,x} + \gamma_0 U^+_x)}
\end{equation}
and
\begin{equation}
\label{e:ZNdef}
Z_N = \Ex_C \theta Z_0,
\end{equation}
where the expectation is given by \refeq{ExCF} with $C = (-\Delta + m^2)^{-1}$.
Recall that $Z_{N,\varnothing}$ denotes the $0$-degree part of $Z_N$.
We define a test function $\1: \Lambda_N \to \R$ by $\1_x=1$ for all $x$,
and write $D^2 Z_{N,\varnothing}(0; \1, \1)$ for the directional derivative of
$Z_{N,\varnothing}$ evaluated with all fields equal to $0$ and
% at $(\phi, \bar\phi) = (0, 0)$,
with both directions equal to $\1$.
That is,
\begin{equation}
D^2 Z_{N,\varnothing}(0; \1, \1)
  =
\ddp{^2}{s\partial t} Z_{N,\varnothing}(s \1, t\1)\big|_{s=t=0}.
\end{equation}

\begin{prop}
\label{prop:intrep}
% from saw-sa
Let $d > 0$, $\gamma, \nu \in \R$, $\gcc >0$ and $\gamma <\gcc$.
If the relations \eqref{e:gg0} hold, then
\begin{align}
% from clp
\label{e:generating-fn}
G_{x,N}(g, \nu; n)
	&=
(1+z_0)
\frac{\partial^2 }{\partial \sigma_\pp  \partial \sigma_\qq}\Big|_{0}
\log \Ex_C  Z_0
\end{align}
% \begin{equation}
% \label{e:GG2}
% G_{x,N}(\gcc,\gamma,\nu)
%     =
% (1+z_0)
% \Ex_C (Z_0 \bar\phi_a \phi_b),
% \end{equation}
and
\begin{equation}
\label{e:chichibar}
  \chi_N\left(\gcc,\gamma,\nu\right)
  = (1+z_0)\hat\chi_N(m^2, g_0, \gamma_0, \nu_0, z_0)
  ,
\end{equation}
with
\begin{equation}
\label{e:chibarm}
\hat\chi_N(m^2, g_0, \gamma_0, \nu_0, z_0)
	=
\frac{1}{m^2}
	+
\frac{1}{m^4} \frac{1}{|\Lambda|} \frac{D^2 Z_{N,\varnothing}(0; \1, \1)}{Z_N(0)}.
\end{equation}
\end{prop}

\begin{proof}
\todo{We prove the case $n = 0$.}
We make the change of variables
$\varphi_x \mapsto (1 + z_0)^{1/2} \varphi_x$ (with $\varphi = \phi, \bar\phi, \psi, \bar\psi$)
in \eqref{e:Grep-pos-bis}, and obtain
\begin{align}
    G_{x,N}(\gcc,\gamma,\nu)
    &=  (1+z_0)
    \int
    e^{-\sum_{x\in\Lambda}
    \left(
    g_0 \tau_x^2 + \gamma_0 |\nabla \tau_x|^2
    + \nu (1+z_0) \tau_x + (1+z_0)\tau_{\Delta,x}\right)} \bar\phi_a \phi_b
    .
    \label{e:Grep-pos}
\end{align}
Then, for any $m^2 \in\R$, we have
\begin{equation}
\lbeq{GNint}
G_{x,N}(\gcc,\gamma,\nu)
    =
(1 + z_0) \int
e^{-\sum_{x\in\Lambda} (\tau_{\Delta,x} + m^2 \tau_x)}
Z_0 \bar\phi_a \phi_b
\end{equation}
($m^2$ simply cancels with $\nu_0$ on the right-hand side).
We use this with $m^2>0$, so that the inverse matrix $C=(-\Delta+m^2)^{-1}$ exists.
By symmetry of the matrix $\Delta$, \refeq{action} gives
\begin{equation}
\label{e:SAtauDelta}
S_{(-\Delta+m^2)}
=
\sum_{x\in\Lambda} \left( \tau_{\Delta,x}
+ m^2  \tau_x \right).
\end{equation}
Then \eqref{e:generating-fn} follows from \refeq{GNint}--\eqref{e:SAtauDelta} and \refeq{ExCF}.  Summation
over $x\in \Lambda_N$ gives the formula $\chi_N(\gcc,\gamma,\nu) = (1+z_0)\sum_{x\in \Lambda} \Ex_C
(Z_0\phib_0\phi_x)$.  Then \refeq{chichibar}, with \refeq{chibarm}, follows
by an elementary computation as in \cite[Section~\ref{log-sec:ga}]{BBS-saw4-log}.
\end{proof}

%%%%%%%%%%%%%%%%%%%%%%%%%%%%%%%%%%%%%%%%%%%%%%%%%%%%%%%%%%%%%%%%%%%%%%%%%%%%%%%
%%%%%%%%%%%%%%%%%%%%%%%%%%%%%%%%%%%%%%%%%%%%%%%%%%%%%%%%%%%%%%%%%%%%%%%%%%%%%%%

\section{Progressive integration}
% Based on both
\label{sec:prog}

We evaluate the Gaussian integral $\Ex_C Z_0$ progressively, via
the covariance decomposition
\begin{equation}
\label{e:NCj}
C = C_1 + \cdots + C_{N-1} + C_{N,N}
\end{equation}
constructed in \cite{Baue13a} (see also \cite{BGM04}). For simplicity, we write $C_N = C_{N,N}$.
% For $F \in \Ncal$, we let
% $\Ex_{w}\theta F$ be the convolution of $F$ with the Gaussian measure of covariance $w$, i.e.,
% $(\Ex_w\theta F)(\varphi) = \Ex_wF(\varphi + \zeta)$ where the expectation integrates the
% variable $\zeta$.
It is a property of Gaussian integration (see \cite{BS-rg-norm}) that
\begin{equation}
	(\Ex_C \theta F)(\varphi)
	=
	(\Ex_{C_N}\theta \circ \Ex_{C_{N-1}}\theta \circ \ldots \circ \Ex_{C_1}\theta F)
	(\varphi).
\end{equation}
Then
\begin{equation}
% \label{e:ZNdef}
Z_N
% = \Ex_C \theta Z_0
=
\Ex_{C_N}\theta \circ \Ex_{C_{N-1}}\theta \circ \ldots \circ \Ex_{C_1}\theta Z_0.
\end{equation}
In particular,
\begin{equation}
\label{e:exp-conv}
\Ex_C Z_0 = Z_N(0).
\end{equation}
This allows us to evaluate the integral $\Ex_C Z_0$ by studying the
dynamical system $Z_j \mapsto Z_{j+1}$ defined by
\begin{equation}
\label{e:rgmapZ}
Z_{j+1} = \Ex_{C_{j+1}} \theta Z_j, \quad j < N.
\end{equation}

%%%%%%%%%%%%%%%%%%%%%%%%%%%%%%%%%%%%%%%%%%%%%%%%%%%%%%%%%%%%%%%%%%%%%%%%%%%%%%%
%%%%%%%%%%%%%%%%%%%%%%%%%%%%%%%%%%%%%%%%%%%%%%%%%%%%%%%%%%%%%%%%%%%%%%%%%%%%%%%

\section{The space of field functionals}

For the analysis of the dynamical system \eqref{e:rgmapZ}, we require a suitable
space on which this dynamical system acts.

We define an $n$-dependent space of forms
$\Ncal^\varnothing$, which is defined as in Section~\ref{sec:intrep} if $n = 0$ and
\begin{equation}
\label{e:Ncaldef}
\Ncal^\varnothing
	= \Ncal^\varnothing(\Lambda)
	= C^{p_\Ncal}((\R^n)^\Lambda,\R).
\end{equation}
if $n \ge 1$, where $p_\Ncal$ is the smoothness parameter discussed in Section~\ref{sec:intrep}.

We define a space $\Ncal$ of $\Ncal^\varnothing$ that includes functions
of the observable fields $\sigma_0$ and $\sigma_x$. Its definition is given
in \cite[Section~\ref{phi4-sec:phi4observables_representation}]{ST-phi4}.
The finite smoothness parameter $p_\Ncal$ is discussed in Section~\ref{sec:newnorm}
below, where it is explained that $p_\Ncal$ must be chosen in a way that depends on the
parameter $p$ in Theorem~\ref{thm:mr}(iii).
The part of $\Ncal$ involving the observable fields contains some subtleties that
need not concern us here; see \cite[Section~\ref{phi4-sec:phi4observables_representation}]{ST-phi4}
for details.

In order to control the evolution of $Z_j$, we make use of a family
$\|\cdot\|_{T_{\phi,j}(\h_j)}$ of scale-dependent dependent seminorms depending on a
sequence of weights $\h_j > 0$; the field $\phi$ lies in $\C^\Lambda$ if $n = 0$ and
$(\R^n)^\Lambda$ if $n \ge 1$. For convenience,
we will simply write $\|\cdot\|_{T_\phi(\h_j)}$ with the scale $j$ implied by the
choice of parameter $\h_j$.

\subsection{Test functions}

Recall the notation introduced in Section~\ref{sec:forms}.
A \emph{test function} $g$ is defined to be a function $(\vec x, \vec y) \mapsto g_{\vec x,\vec y}$,
where $\vec x$ and $\vec y$ are finite sequences of elements in $\Lambda \sqcup \bar\Lambda$.
When $\vec x$ or $\vec y$ is the empty sequence $\varnothing$,
we drop it from the notation as long as this causes no confusion;
e.g., we may write $g_{\vec x} = g_{\vec x,\varnothing}$.
The length of a sequence $\vec x$ is denoted $|\vec x|$.
Gradients of test functions are defined component-wise.
Thus, if $\vec x = (x_1, \ldots, x_m)$
and $\alpha = (\alpha_1, \ldots, \alpha_m)$
with each $\alpha_i \in \N_0^\Ucal$, and similarly for $\vec y=(y_1,\ldots,y_n)$ and
$\beta=(\beta_1,\ldots,\beta_n)$,
then
\begin{equation}
\nabla^{\alpha,\beta}_{\vec x,\vec y} g_{\vec x,\vec y}
  =
\nabla^{\alpha_1}_{x_1} \ldots \nabla^{\alpha_m}_{x_m}
\nabla^{\beta_1}_{y_1} \ldots \nabla^{\beta_n}_{y_n}  g_{x_1,\ldots,x_m,y_1,\ldots,y_n}.
\end{equation}

We fix a positive constant $p_\Phi\ge 4$ and recall $p_\Ncal$
and assume that all test functions
vanish when $|\vec x|  +|\vec y| > p_\Ncal$.
% For Theorem~\ref{thm:suscept}(i-ii), any choice of $p_\Ncal \ge 10$ is sufficient,
% whereas for Theorem~\ref{thm:suscept}(iii) it is necessary to choose $p_\Ncal$ large
% depending on $p$ \cite{BSTW-clp}.
The $\Phi_j = \Phi(\h_j)$ norm on such test functions is defined by
\begin{equation}
\|g\|_{\Phi_j}
	=
\sup_{\vec x, \vec y} \h_j^{-(|\vec x| +|\vec y|)}
	\shift\shift
\sup_{\alpha,\beta: |\alpha|_1+|\beta|_1 \le p_\Phi}
L^{j (|\alpha|_1 + |\beta|_1)}
|\nabla^{\alpha,\beta} g_{\vec x, \vec y}|,
\end{equation}
where $|\alpha|_1$ denotes the total order of the differential operator $\nabla^\alpha$.
Thus, for any test function $g$ and for sequences
$\vec x, \vec y$ with $|\vec x| +|\vec y| \leq p_\Ncal$ and
corresponding $\alpha, \beta$ with $|\alpha|_1 + |\beta|_1 \leq p_\Phi$,
\begin{equation}
\label{e:testfcnbd}
|\nabla^{\alpha,\beta} g_{\vec x,\vec y}|
	\leq
\h_j^{|\vec x| + |\vec y|} L^{-j (|\alpha|_1 + |\beta|_1)} \|g\|_{\Phi_j}.
\end{equation}

\subsection{The \texorpdfstring{$T_\phi$}{Tphi} seminorm}

\todo{Define for observables.}

For any $F \in \Ncal^\varnothing$,
there exist \emph{unique} functions $F_{\vec y}$ of $(\phi, \bar\phi)$
that are anti-symmetric under permutations of $\vec y$, such that
\begin{equation}
F = \sum_{\vec y} \frac{1}{|\vec y|!} F_{\vec y}(\phi, \bar\phi) \psi^{\vec y}.
\end{equation}
Given a sequence $\vec{x}$ with $|\vec{x}| = m$, we define
\begin{equation}
F_{\vec x, \vec y} = \ddp{^m F_{\vec y}}{\phi_{x_1} \ldots \partial\phi_{x_m}}.
\end{equation}

We define a $\phi$-dependent pairing of elements of $\Ncal$ with test functions, by
\begin{equation}
\langle F, g \rangle_\phi
  =
\sum_{\vec x, \vec y}
\frac{1}{|\vec x|! |\vec y|!}
F_{\vec x,\vec y}(\phi, \bar\phi)
g_{\vec x,\vec y}.
\end{equation}
Let $B(\Phi)$ denote the unit $\Phi$-ball in the space of test functions. Then the
$T_\phi = T_\phi(\h_j)$ semi-norm on $\Ncal^\varnothing$ is defined by
\begin{equation}
\|F\|_{T_\phi} = \sup_{g\in B(\Phi_j)} |\langle F, g \rangle_\phi|.
\end{equation}

The $T_\phi$ norm is defined in greater generality in \cite{BS-rg-norm}.
In particular, for $F \in \Ncal$ an additional sequence $\h_{\sigma,j}$
of parameters is used and the $T_\phi$ norm is defined so that for
\begin{equation}
F = F_\varnothing + F_a + F_b + F_{ab},
	\quad
F_\alpha \in \Ncal^\varnothing
\end{equation}
we have
\begin{equation}
\|F\|_{T_\phi}
	=
\|F_\varnothing\|_{T_\phi}
	+ (\|F_a\|_{T_\phi} + \|F_b\|_{T_\phi}) \h_\sigma
	+ \|F_{ab}\|_{T_\phi} \h_\sigma^2.
\end{equation}

By its definition, the $T_\phi$ seminorm controls the values of $F$ and its derivatives
(up to order $p_\Ncal$) at $\phi$. For instance, we will make use of the following facts.

\begin{lemma}
\label{lem:deriv-norm-bds}
For $F \in \Ncal$, we have $|F_\varnothing(0)| \le \|F\|_{T_0}$ and
\begin{equation}
\label{e:deriv-norm-bd}
|D^2 F_\varnothing(0; \1, \1)|
	\le
2 \|F\|_{T_0(\h_j)} \|\1\|^2_{\Phi_N(\h_j)}
	=
2 \|F\|_{T_0(\h_j)} \h_j^{-1}
\end{equation}
Letting $F_{\varnothing;\sigma_0\sigma_x}(0)$ denote the second derivative of $F$ with
respect to $\sigma_0$ and $\sigma_x$ evaluated at $0$ field,
\begin{equation}
|F_{\varnothing;\sigma_0\sigma_x}|
	\le
\h_{\sigma,j}^{-2} \|F\|_{T_0}.
\end{equation}
\end{lemma}

\subsection{Norm parameters}

Control of the $T_\phi$ seminorm is needed for all values of
$\phi$ in order to control the Gaussian expectation in \eqref{e:rgmapZ}. This will
be discussed further in the sequel.

For now, we turn our attention to the special case of the $T_0$ seminorm. Recalling
\eqref{e:exp-conv}, the value of $\|F\|_{T_0(\h_j)}$ should reflect the size the
Gaussian expectation $\Ex_{C_{j+1}} F$. For a monomial $\phi_x^{2p}$, we have
\begin{equation}
\|\phi_x^{2p}\|_{T_0(\h_j)} \asymp \h_j^p
\end{equation}
and
\begin{equation}
\Ex_{C_{j+1}} \phi_x^{2p} = \mathrm{const}_p C_{j+1;00}^p
\end{equation}
and this suggests letting $\h_j \approx |C_{j+1;00}|$. Together with the covariance
bounds \REF, this leads us to define the weights $\ell_j$ by
\begin{align}
\label{e:elldef-zz}
\ell_j &= \ell_0 L^{-j - s (j - j_m)_+}, \quad
\ell_{\sigma,j}
=
\ell_{j \wedge j_{x}}^{-1} 2^{(j - j_{x})_+} \ggen_j,
\end{align}
where the \emph{mass scale} $j_m$ and \emph{coalescence scale} $j_x$
are defined by
\begin{align}
\label{e:jmdef}
j_m		&= \lfloor\log_{L} m^{-1}\rfloor
	\\
\label{e:jxdef}
j_x 	&= \max\{0,\lfloor \log_{L} (2 |x|)\rfloor\}.
\end{align}
The sequence $\ggen_j = \ggen_j(m^2,g_0)$ is defined in
\cite[\eqref{log-e:ggendef}]{BBS-saw4-log};
it is bounded above and below by constant multiples of
the sequence $\gbar$ defined in
\eqref{e:gbar},
by
\cite[Lemma~\ref{log-lem:gbarmcomp}]{BBS-saw4-log}.
We discuss the origin of the definition \refeq{elldef-zz} in detail
in Section~\ref{sec:Rpf1}.

%%%%%%%%%%%%%%%%%%%%%%%%%%%%%%%%%%%%%%%%%%%%%%%%%%%%%%%%%%%%%%%%%%%%%%%%%%%%%%%
%%%%%%%%%%%%%%%%%%%%%%%%%%%%%%%%%%%%%%%%%%%%%%%%%%%%%%%%%%%%%%%%%%%%%%%%%%%%%%%

\section{Perturbative coordinate}

The dynamical system is analysed via a perturbative part which is tracked accurately
to second order in $g$, together with a third-order non-perturbative part whose study
forms the main part of our effort.  For the perturbative part, we first introduce
an appropriate space of local field polynomials.

%%%%%%%%%%%%%%%%%%%%%%%%%%%%%%%%%%%%%%%%%%%%%%%%%%%%%%%%%%%%%%%%%%%%%%%%%%%%%%%

\subsection{Local field polynomials}
% Based on clp

For $y \in \Lambda$, we supplement \eqref{e:taudef}--\eqref{e:nablatau} and \refeq{tauphi}
by defining
\begin{equation}
\label{e:tauphi2}
\quad \tau_{\nabla\nabla,y}
	=
\begin{cases}
\frac 12 \sum_{e \in \units}
\left(
	(\nabla^e \phi)_y (\nabla^e \bar\phi)_y +
	(\nabla^e \psi)_y (\nabla^e \bar\psi)_y
\right),
	& n = 0 \\
\frac{1}{4} \sum_{|e| = 1} \nabla^e \varphi_y \cdot \nabla^e \varphi_y,
	& n \ge 1.
\end{cases}
\end{equation}
Now for fixed $x \in \Lambda$,
we define $\Vcal$ to be the space of functions $V=V_y$ of the form 
\begin{align}
\lbeq{Vy}
V_y
	&=
g \tau_y^2 + \nu \tau_y + z \tau_{\Delta,y} + y \tau_{\nabla\nabla,y} + u
	\nnb&\quad
- \1_{y=\pp}\lambda_{\pp}(\varphi_{\pp} \cdot {\sf h})\sigma_\pp
- \1_{y=\qq}\lambda_{\qq}(\varphi_{\qq} \cdot {\sf h})\sigma_\qq
	\nnb&\quad
- \textstyle{\frac 12} (\1_{y=\pp} q_\pp + \1_{y=\qq}q_\qq )\sigma_\pp\sigma_\qq.
\end{align}
Given $X \subset \Lambda$, we also define
\begin{equation}
\label{e:Vcalesig}
\Vcal(X) = \{V(X) = \textstyle{\sum_{y\in X}} V_y : V \in \Vcal \}.
\end{equation}
We also make use of the subspaces $\Vcal^{(1)} \subseteq \Vcal$ consisting of polynomials with $y = 0$, as well as the subspace
$\Vcal^{(0)} \subseteq \Vcal^{(1)}$ of polynomials with
$u = y =   q_\pp=q_\qq = 0$.
For $V \in \Vcal$, we define maps $V \mapsto V^{(1)} \in \Vcal^{(1)}$
and $V \mapsto V^{(0)} \in \Vcal^{(0)}$. Both maps replace
$z\tau_{\Delta}+y\tau_{\nabla\nabla}$ by
$(z+y)\tau_{\Delta}$, and the latter
additionally sets
$u = q_\pp = q_\qq = 0$.

We define the $\Vcal = \Vcal_j$ norm by
\begin{equation}
\label{e:Vnormdef}
\begin{aligned}
\|V\|_{\Vcal} &=
\max\Big\{
|g|, L^{2j}|\nu|, |z|, |y|,  L^{4j}|u|,
\ell_j\ell_{\sigma,j}(|\lambda_\pp|\vee|\lambda_\qq|),\;
%\\
%& \qquad\qquad\qquad
 \ell_{\sigma,j}^{2} (|q_\pp|\vee|q_\qq|)
\Big\}
\end{aligned}
\end{equation}
on $V \in \Vcal$, which depends on the parameters $\ell_j$ and $\ell_{\sigma,j}$.
The $\Vcal = \Vcal_j$ norm is equivalent to the $T_0(\ell_j)$ seminorm.

%%%%%%%%%%%%%%%%%%%%%%%%%%%%%%%%%%%%%%%%%%%%%%%%%%%%%%%%%%%%%%%%%%%%%%%%%%%%%%%

\subsection{Perturbative flow}

In \cite{BBS-rg-pt}, a map $\Vpt : \Vcal\to\Vcal$
is defined so as to maintain the approximation form
\begin{equation}
Z_j \approx e^{-V_j} (1 + O(V_j^2)).
\end{equation}
Precisely, $\Vpt$ depends on a covariance $C$ and satisfies
\begin{equation}
\Ex_C\theta e^{-V(\Lambda)} (1 + W(V))
	=
e^{-\Vpt(\Lambda)} (1 + W(\Vpt)) + O(V^3),
\end{equation}
where $W = W(V)$ is an explicit polynomial defined
in \cite[\eqref{pt-e:WLTF}]{BBS-rg-pt}. \todo{Explain $O(V^3)$.}
In practice, we set $C = C_{j+1}$ and obtain a sequence $\Vpt = V_{\mathrm{pt},j+1}$
of maps and $W_j$ of polynomials. By \cite[\eqref{IE-e:W-logwish}]{BS-rg-IE},
\todo{(check this)},
\begin{equation}
\label{e:Wbilinbd}
\|W_j\|_{T_0(\ell_j)}
	\le
O(\chicCov_j) \|V\|_\Vcal^2.
\end{equation}
The maps $V \mapsto \Vpt$
generate a sequence of couplings constants that we refer to as the
\emph{perturbative flow}. The equations defining this flow can be
computed exactly by way of Feynman diagrams.

\subsubsection{The flow of \texorpdfstring{$g$}{g}}

\todo{Find a good way to introduce the ``nice'' perturbative flow of $g$
following the approximate change of variables.}
We have
\begin{equation} \label{e:gbar}
\gbar_{j+1}
	=
\gbar_j - \beta_j  \gbar_j^{2}, \qquad \gbar_0
	=
g_0,
\end{equation}
It follows from \cite[Proposition~\ref{log-prop:approximate-flow}]{BBS-saw4-log}
that
\begin{equation}
\label{e:gjxgjmbd}
\gbar_{j}
	=
O((\log m^{-1})^{-1}) \;\; \text{for $j \geq j_m$},
	\quad
\gbar_{j_x}
	=
O((\log |x|)^{-1}) \;\; \text{for $j_x \leq j_m$.}
\end{equation}

%%%%%%%%%%%%%%%%%%%%%%%%%%%%%%%%%%%%%%%%%%%%%%%%%%%%%%%%%%%%%%%%%%%%%%%%%%%%%%%

\subsubsection{The flow of \texorpdfstring{$\lambda$ and $q$}{lambda and q}}

It is shown in \cite[\eqref{pt-e:lambdapt2}--\eqref{pt-e:qpt2}]{BBS-rg-pt} (for $n = 0$)
and \cite[Proposition~\ref{phi4-prop:pt}]{ST-phi4} (for $n \ge 1$) that,
with $C = C_{j+1}$ and $u = 0, x$,
\begin{align}
\label{e:lampt}
\lambda_{u,\pt}
	&=
\begin{cases}
(1 - \delta[\nu w^{(1)}]) \lambda_u,
	& j + 1 < j_x \\
\lambda_u,
	& j + 1 \ge j_x
\end{cases}
	\\
\label{e:qpt}
q_\pt
	&=
q + \lambda_0 \lambda_x C_{0x},
\end{align}
where
\begin{equation}
\label{e:deltanuw1}
\delta[\nu w^{(1)}] = (\nu + 2 g C_{00}) w^{(1)}_{j+1} - \nu w^{(1)}_j,
	\qquad
w^{(1)}_j = \sum_{x\in\Lambda} \sum_{i=1}^j C_{i;0x}.
\end{equation}
Note that $q_\pt = q$ for $j + 1 < j_x$.
% \todo{It's strange that $w$ appears in the flow of $\lambda$ when $\Vpt$ only
% depends on $C$.}


%%%%%%%%%%%%%%%%%%%%%%%%%%%%%%%%%%%%%%%%%%%%%%%%%%%%%%%%%%%%%%%%%%%%%%%%%%%%%%%
%%%%%%%%%%%%%%%%%%%%%%%%%%%%%%%%%%%%%%%%%%%%%%%%%%%%%%%%%%%%%%%%%%%%%%%%%%%%%%%

\section{Non-perturbative coordinate}
% Based on clp
\label{sec:rgcoord}

For $j=0,\ldots, N$, we partition $\Lambda$ into $L^{N-j}$ disjoint
\emph{scale-$j$ blocks} of side length $L^j$.
A scale-$j$ \emph{polymer} is a union of scale-$j$ blocks.
The set of all scale-$j$ blocks is denoted $\Bcal_j$, and
the set of all scale-$j$ polymers is denoted $\Pcal_j$.
For $X \in \Pcal_j$, we write $\Bcal_j(X)$ for the set of scale-$j$ blocks in $X$.
For $F, G : \Pcal_j \to \Ncal$, we define the \emph{circle product} $F \circ G : \Pcal_j \to \Ncal$ by
\begin{equation}
(F \circ G)(X) = \sum_{Y\in\Pcal_j(X)} F(X \setminus Y) G(Y).
\end{equation}

The evolution of $Z_j$ can be tracked in the \emph{renormalisation group coordinates}
$\zeta_j \in \R$ and
$I_j, K_j : \Pcal_j \to \Ncal$, defined such that
\begin{equation}
\label{e:IcircKnew}
	Z_j = e^{\zeta_j}(I_j\circ K_j)(\Lambda),
	\qquad
	\zeta_j= - u_j|\Lambda|
	+ \textstyle{\frac 12} (q_{\pp,j} + q_{\qq,j}) \sigma_\pp\sigma_\qq
	.
\end{equation}
The coordinate $I_j$ tracks the evolution of the
\emph{relevant} and \emph{marginal} directions.  It
is determined by a local polynomial
$U\in \Vcal^{(0)}$,
and takes the form
\begin{equation}
I_j(X; V)
	=
\prod_{B \in \Bcal_j(X)} e^{-V(B)} (1 + W_j(B, V)), \quad X \in \Pcal_j.
\end{equation}
The evolution of $(\zeta, V)$ to second order is called the \emph{perturbative flow} and is
given by the explicit map $\Vpt : \Vcal \to \Vcal$ defined in
\cite[\eqref{pt-e:Vptdef}]{BBS-rg-pt}.

At scale $j = 0$, we are given $V^+_0$ as defined in \eqref{e:V0def}
and we set $\zeta_0 = 0$. In particular,
the initial values of $u$, $q_0$, $q_x$ are zero, and the initial values of $\lambda_0$, $\lambda_x$
are $1$. By definition, $W_0 = 0$.
For $X \subset \Lambda$, we define
\begin{equation}
\label{e:IK0def}
I_0^+(X) = I_0(X; V^+_0) = \prod_{x\in X} e^{-V^+_{0,x}},
	\qquad
K_0^+(X) = \prod_{x \in X} I_{0,x}^+ (e^{-\gamma_0 U^{+}_{x}} - 1).
\end{equation}
With these choices, $Z_0$ (recall \refeq{Z0def})
takes the form \eqref{e:IcircKnew}, and we seek $(\zeta_j, V_j, K_j)$ such that
this continues to hold as the scale advances.

Equivalently, given $(V_j, K_j)$, we must define $(\delta\zeta_{j+1}, V_{j+1}, K_{j+1})$ so that
\begin{equation} \label{e:IcircKdu}
	\Ex_{j+1}\theta(I_j \circ K_j)(\Lambda)
	=
	e^{-\delta \zeta_{j+1}}(I_{j+1} \circ K_{j+1})(\Lambda),
\end{equation}
where $\delta\zeta_{j+1} = \zeta_{j+1} - \zeta_j$.
Moreover, we need $K_j$ to contract as the scale advances, under an appropriate norm.
The construction of (scale-dependent) maps $V_+$ and $K_+$ such that
\eqref{e:IcircKdu} holds with
\begin{equation}
(\delta\zeta_{j+1}, V_{j+1}) = V_+(V_j, K_j),
	\quad
K_{j+1} =  K_+(V_j, K_j)
\end{equation}
is the main accomplishment of \cite{BS-rg-step} and is summarised in Section~\ref{sec:step}
below, in a form adapted to our current setting.

%%%%%%%%%%%%%%%%%%%%%%%%%%%%%%%%%%%%%%%%%%%%%%%%%%%%%%%%%%%%%%%%%%%%%%%%%%%%%%%
%%%%%%%%%%%%%%%%%%%%%%%%%%%%%%%%%%%%%%%%%%%%%%%%%%%%%%%%%%%%%%%%%%%%%%%%%%%%%%%

\section{Renormalisation group step}
% from saw-sa and clp
\label{sec:step}

% In \cite{BS-rg-step}, maps $V_+,K_+$ are defined which map a pair $(V,K)$ at scale $j$
% to $(V_+(V,K),K_+(V,K))$ at scale $j+1$, and which preserve the circle product
% $I\circ K$ under expectation as in \refeq{IcircKdu}.
In \cite[Definition~\ref{step-def:Kspace}]{BS-rg-step},
a space $\Kcal_j = \Kcal_j(\Lambda)$ of maps $\Pcal_j \to \Ncal$ required to satisfy
several properties is defined.
The coordinate $K_j$ is constructed in \cite{BS-rg-step} as an element of $\Kcal_j$.
% A norm has already been defined on the space $\Vcal$ in \refeq{Vnormdef}.
In \cite[Section~\ref{step-sec:Knorms}]{BS-rg-step},
a sequence of norms $\|\cdot\|_{\Wcal_j} = \|\cdot\|_{\Wcal_j(\mgen^2, \ggen_j, \Lambda)}$
parametrised by $(\mgen^2, \ggen_j)$ is defined on $\Kcal_j$.
We note here only the fact that the $\Wcal_j$ norm dominates the $T_0(\ell_j)$ norm
in the sense that
\begin{equation}
\label{e:T0dom}
\|F(\Lambda)\|_{T_0(\ell_j)} \le \|F\|_{\Wcal_j}.
\end{equation}
We denote the ball of radius $r$ in the normed space $\Wcal_j$ by $B_{\Wcal_j}(r)$.

In \cite[\eqref{log-e:mass-scale}--\eqref{log-e:chidef}]{BBS-saw4-log},
a function $\chicCov_j = \chicCov_j(m^2)$ (denoted $\chi_j$ in \cite{BBS-saw4-log})
is defined in such a way that $\chicCov_j$ decays exponentially
when $j$ is sufficiently large depending on $m$.
We write $\chicCovgen_j = \chicCov_j(\mgen^2)$.
Given constants $\alpha > 0$ and $C_\DV > 0$,
we define the (finite-volume)
renormalisation group domains
% $\domRG_j \subset \R^3 \oplus \Wcal_j$ by
% Given $C_\DV>0$ and $\alpha>0$, we define the domains
\begin{align}
\label{e:DVdef}
\DV_j
	&=
\{ V\in \Vcal^{(0)} :
	g> C_{\DV}^{-1} \ggen_j  , \;  \|V\|_{\Vcal} < C_{\DV} \ggen_j \}, \\
\label{e:domRG}
\domRG_j
	&=
\DV_j \times B_{\Wcal_j}(\alpha \chicCovgen_j \ggen_j^3).
\end{align}
% In particular, if $(V_j, K_j) \in \domRG_j$, then Lemma~\ref{lem:deriv-norm-bds}
% and \eqref{e:T0dom} imply that
% \begin{equation}
% \label{e:VKbds}
% \|V_j\|_{\Vcal} \le O(\ggen_j),
% 	\quad
% |D^2 K_{j,\varnothing}(0; \1, \1)|
% 	\le
% \ell_j^{-2} |K_{j,\varnothing}(0)|
% 	\le
% \ell_j^{-2} O(\chicCovgen_j \ggen_j^3).
% \end{equation}

At scale $j$, the maps $V_+,K_+$ act on the domain $\domRG_j$
and map into $\Vcal_{j+1}^{(1)},\Kcal_{j+1}$, respectively.
The deviation of the map $V_+$ from the perturbative map $\Vpt$
% (mentioned above \refeq{qpt})
is denoted by $R_+$:
\begin{equation}
\label{e:Rplusdef}
    R_+(V,K) = V_+(V,K) -\Vpt^{(1)}(V).
\end{equation}
The following theorem is applied with $\alpha =4M$ as a convenient choice.

\begin{theorem}
\label{thm:step-mr-fv}
Let $d = 4$ and let $n \ge 0$. Fix $s > 0$.
Let $C_\DV$ and $L$ be sufficiently large.
There exist $M>0$ and $\delta >0$ such that
for $\ggen \in (0,\delta)$, % and $\mgen^2 \in \Iint_+$,
and with the domain
$\domRG$ defined using any $\DVa> M$, the maps
\begin{equation}
\label{e:RKplusmaps}
R_+:\domRG %\times \Igen_+(\mgen^2)
\to \Vcal^{(1)},
\quad
K_+:\domRG %\times \Igen_+(\mgen^2)
\to \Wcal_{+}%(\sgen_+)
\end{equation}
define $(U,K)\mapsto (V_+,K_+)$ obeying \eqref{e:IcircKdu},
and satisfy the estimates
\begin{equation}
\label{e:RKplus}
\|R_+\|_{\Vcal}
\le
M\chigen_+\ggen_+^{3}
, \qquad
\|K_+\|_{\Wcal_+}
\le
M\chigen_+ \ggen_+^{3}
.
\end{equation}
\end{theorem}

The proof of Theorem~\ref{thm:step-mr-fv} is identical to the proof of
\cite[Theorem~\ref{phi4-thm:step-mr-fv}]{ST-phi4}, via a version of
\cite[Theorems~\ref{step-thm:mr-R}--\ref{step-thm:mr}]{BS-rg-step} that
uses the norm parameters \eqref{e:elldef-zz} with $s > 0$.
The proof of the latter results with these new norm parameters amounts to
checking that the proof of the $s = 0$ case contained in \cite{BS-rg-IE,BS-rg-step}
continues to hold with $s > 0$. A verification of this fact is carried
out in Section~\REF % \ref{sec:Rpf2}
below.

Theorem~\ref{thm:step-mr-fv} expresses a contractive property of the map $K_+$,
as it takes $K$ in a ball whose radius involves $\alpha=4M$ at scale $j$ to
an image which lies in a ball whose radius involves the smaller number $M$
at scale $j+1$.  The fact that $K_+$ is a contraction is used in
Theorem~\ref{thm:rhatflow}
% \cite[Proposition~\ref{log-prop:KjNbd}]{BBS-saw4-log} (for $n=0$) and
% \cite[Theorem~\ref{phi4-log-thm:flow-flow}]{BBS-phi4-log}
% (for $n \ge 1$)
to prove that, for $m^2$ and $g_0$ sufficiently small, there exist
\emph{critical} initial conditions
$\nu_0 = \nu_0^c(m^2, g_0)$ and $z_0 = z_0^c(m^2, g_0)$ such that
iteration of the maps $(V_+,K_+)$ defines a sequence $(V_j^{(0)}, K_j)$
which lies in the domain $\domRG_j$ and obeys the estimates \refeq{RKplus}
\emph{for all} $j = 1, \ldots, N$.
This construction of critical initial conditions uses the $s=0$ version
of \refeq{elldef-zz}.

The case with observable fields included is handled in \cite{ST-phi4}.
Because we have increased $\ell_{\sigma,j}$ beyond the mass scale, the
estimates on $q_0,q_x$ given by the bound on $R_+$ in \refeq{RKplus} are
significantly improved compared to their versions with $s=0$ in \cite{ST-phi4}.
As is discussed in detail in  \cite[Section~\ref{phi4-sec:pfmr1}]{ST-phi4},
$V_j^{(0)}$ remains in the domain $\domRG_j$ for all $j$
(also concerning $\lambda_{0,j}, \lambda_{x,j}$).
Moreover,
$\lambda_{0,j}, \lambda_{x,j},q_{0,j},q_{x,j}$, are independent of the volume parameter $N$ and
so can be extended to infinite sequences, and the following limits exist:
\begin{equation}
q_{u,\infty} = \lim_{j\to\infty} q_{u,j}, \quad u = 0, x.
\end{equation}

%%%%%%%%%%%%%%%%%%%%%%%%%%%%%%%%%%%%%%%%%%%%%%%%%%%%%%%%%%%%%%%%%%%%%%%%%%%%%%%
%%%%%%%%%%%%%%%%%%%%%%%%%%%%%%%%%%%%%%%%%%%%%%%%%%%%%%%%%%%%%%%%%%%%%%%%%%%%%%%

\section{Main theorem}
% mainly from saw-sa

Suppose $\delta > 0$ and suppose $r : [0, \delta] \to [0, \infty)$
is a continuous positive-definite\footnote{Note that our usage of this term is
different from that in the theory of quadratic forms.} function; the latter
means that $r(x) > 0$ if $x > 0$ and $r(0) = 0$.
We define
\begin{equation}
\lbeq{Ddef}
D(\delta, r)
	=
\{ (w, x, y) \in [0, \delta]^2 \times (-\delta, \delta) : |y| \leq r(x) \}
\end{equation}
and we let $C^{0,1,\pm}(D(\delta, r))$ denote the space of continuous functions on $D(\delta, r)$
that are $C^1$ in $(x, y)$ away from $y = 0$, $C^1$ in $x$ everywhere,
and have left- and right-derivatives in $y$ at $y = 0$.
In our applications, we take $w = m^2$, $x = g_0$ or $\gcc$,
and $y = \gamma_0$ or $\gamma$.

\begin{theorem}
\label{thm:rhatflow}
There exists a domain $D(\delta, \hat r)$ (with $\delta > 0$ and $\hat r$
positive-definite) and functions $\hat\nu_0^c, \hat z_0^c \in C^{0,1,\pm}(D(\delta, \hat r))$
such that for any $(m^2, g_0, \gamma_0) \in D(\delta, \hat r)$
with $g_0 > 0$ and $m^2 \in [\delta L^{-2 (N - 1)}, \delta)$, the following holds:
if $(V_0, K_0) = (V^+_0, K^+_0)$ with $(\nu_0, z_0) = (\hat\nu_0^c, \hat z_0^c)$,
% and $\sigma_0 = \sigma_x = 0$,
then for any $N \in \N$,
there exists a sequence $(V_j, K_j) \in \domRG_j(m^2, g_0, \Lambda)$ such that
\begin{equation}
	\label{e:VjKjDj-hat}
	(V_{j+1},K_{j+1}) = (V_{j+1}(V_j, K_j), K_{j+1}(V_j, K_j)) \text{ for all } j < N
\end{equation}
and \eqref{e:IcircKnew} is satisfied.
These functions satisfy the bounds
\begin{equation}
\label{e:hat-est}
\hat\nu_0^c = O(g_0),
\quad
\hat z_0^c = O(g_0)
\end{equation}
uniformly in $(m^2, \gamma_0)$.
Moreover, the second-order evolution equation for $V_j$ is independent of $\gamma_0$.
\end{theorem}
