\chapter{Renormalisation group method}

\commentbw{Outline:
\begin{itemize}
\item
extend map to new norms: independent of $K_0$, do $n \ge 1$ ($n = 0$ similar)

\item
extend bulk flow to general $K_0$
(done for $n = 0$, but extension to $n \ge 1$ should be straightforward)

\item
do observable flow
\end{itemize}}

\commentbw{We are presently considering $n \ge 1$ throughout
Sections~\ref{sec:prog}--\ref{sec:rgcoord}}

% Discuss RG method with improved norms and non-trivial $K_0$

%%%%%%%%%%%%%%%%%%%%%%%%%%%%%%%%%%%%%%%%%%%%%%%%%%%%%%%%%%%%%%%%%%%%%%%%%%%%%%%
%%%%%%%%%%%%%%%%%%%%%%%%%%%%%%%%%%%%%%%%%%%%%%%%%%%%%%%%%%%%%%%%%%%%%%%%%%%%%%%

\section{Reformulation of the problem}

Let
\begin{equation}
\tau_x = \tfrac{1}{2} |\varphi_x|^2,
  \quad
\tau_{\Delta,x} = \tfrac{1}{2} \varphi_x \cdot (-\Delta \varphi)_x,
  \quad
|\nabla\tau_x|^2 = \sum_e (\nabla^e |\phi_x|^2)^2.
\end{equation}
Given $g>0$, $\nu\in \R$, and  given $m^2>0$ and $z_0 >-1$, let
\begin{equation}
  \label{e:gg0}
  g_0 = g(1+z_0)^2, \quad \quad
  \nu_0 = (1+z_0)\nu-m^2.
\end{equation}
Let ${\sf h} = n^{-1/2}(1, \ldots, 1) \in \R^n$.
With the two points $0,x\in \Lambda$ fixed,
we introduce \emph{observable fields} $\sigmaa, \sigmab \in \R$, and define
\begin{equation}
\label{e:V0def}
  V^+_{0,x}
  = g_0\tau_x^2 + \nu_0 \tau_x + z_0 \tau_{\Delta,x} -
    \sigma_\pp (\varphi_{\pp} \cdot {\sf h})
    - \sigma_\qq (\varphi_{\qq} \cdot {\sf h}),
  \quad
  U^+_x = |\nabla \tau_x|^2.
\end{equation}
Then we can show that
\begin{align}
\label{e:generating-fn}
    G_{x,N}(g, \nu; n)
    &=
    (1+z_0)
    \frac{\partial^2 }{\partial \sigma_\pp  \partial \sigma_\qq }
    \Big|_{0}
    \log
    \Ex_C  Z_0
\end{align}
with
\begin{equation}
\label{e:Z0def}
  Z_0
  =
  \prod_{x\in \Lambda} e^{-(V^+_{0,x} + \gamma_0 U^+_x)}.
\end{equation}
Here, the expectation is given by \refeq{ExCF}, with $C = (-\Delta + m^2)^{-1}$.

%%%%%%%%%%%%%%%%%%%%%%%%%%%%%%%%%%%%%%%%%%%%%%%%%%%%%%%%%%%%%%%%%%%%%%%%%%%%%%%
%%%%%%%%%%%%%%%%%%%%%%%%%%%%%%%%%%%%%%%%%%%%%%%%%%%%%%%%%%%%%%%%%%%%%%%%%%%%%%%

\section{Progressive integration}
% Based on both
\label{sec:prog}

We evaluate the Gaussian integral $\Ex_C Z_0$ progressively, via
the covariance decomposition
\begin{equation}
\label{e:NCj}
C = C_1 + \cdots + C_{N-1} + C_{N,N}
\end{equation}
constructed in \cite{Baue13a} (see also \cite{BGM04}). For simplicity, we write $C_N = C_{N,N}$.
For an integrable function $F$ of the spin field $\varphi$, we let
$\Ex_{w}\theta F$ be the convolution of $F$ with the Gaussian measure of covariance $w$, i.e.,
$(\Ex_w\theta F)(\varphi) = \Ex_wF(\varphi + \zeta)$ where the expectation integrates the
variable $\zeta$.
It is a property of Gaussian integration (see \cite{BS-rg-norm}) that
\begin{equation}
    (\Ex_C \theta F)(\varphi)
    =
    (\Ex_{C_N}\theta \circ \Ex_{C_{N-1}}\theta \circ \ldots \circ \Ex_{C_1}\theta F)
    (\varphi).
\end{equation}
Let
\begin{equation}
\label{e:ZNdef}
Z_N = \Ex_C \theta Z_0
=
\Ex_{C_N}\theta \circ \Ex_{C_{N-1}}\theta \circ \ldots \circ \Ex_{C_1}\theta Z_0.
\end{equation}
In particular,
\begin{equation}
\Ex_C Z_0 = Z_N(0).
\end{equation}
This allows us to evaluate the integral $\Ex_C Z_0$ by studying the
dynamical system $Z_j \mapsto Z_{j+1}$ defined by
\begin{equation}
Z_{j+1} = \Ex_{C_{j+1}} \theta Z_j, \quad j < N.
\end{equation}

For its analysis, we require
a suitable space $\Ncal$ of functions of the spin
and observable fields, on which the dynamical system acts.  The space $\Ncal$ is
discussed in detail in \cite[Section~\ref{phi4-sec:phi4observables_representation}]{ST-phi4}.
The part of $\Ncal$ which does not involve the observable fields $\sigmaa,\sigmab$ is
given by
\begin{equation}
\label{e:Ncaldef}
    \Ncal^\varnothing = \Ncal^\varnothing(\Lambda) = C^{p_\Ncal}((\R^n)^\Lambda,\R).
\end{equation}
The finite smoothness parameter $p_\Ncal$ is discussed in Section~\ref{sec:newnorm}
below,
where it is explained that $p_\Ncal$ must be chosen in a way that depends on the
parameter $p$ in Theorem~\ref{thm:mr}.
The part of $\Ncal$ involving the observable fields contains some subtleties that
need not concern us here; see \cite[Section~\ref{phi4-sec:phi4observables_representation}]{ST-phi4}
for details.

%%%%%%%%%%%%%%%%%%%%%%%%%%%%%%%%%%%%%%%%%%%%%%%%%%%%%%%%%%%%%%%%%%%%%%%%%%%%%%%
%%%%%%%%%%%%%%%%%%%%%%%%%%%%%%%%%%%%%%%%%%%%%%%%%%%%%%%%%%%%%%%%%%%%%%%%%%%%%%%

\section{Local field polynomials}
% Based on clp

The dynamical system is analysed via a perturbative part which is tracked accurately
to second order in $g$, together with a third-order non-perturbative part whose study
forms the main part of our effort.  For the perturbative part, we first introduce
an appropriate space of local field polynomials.

For $y \in \Lambda$, we supplement \refeq{tauphi} by defining
% \begin{align}
% &\tau_y = |\varphi_y|^2 \\
% &\tau_{\Delta,y} = \\
% &\tau_{\nabla\nabla,y} = \frac{1}{4} \sum_e \nabla^e \varphi_y \cdot \nabla^e \varphi_y.
% \end{align}
\begin{equation}
\label{e:tauphi2}
\quad \tau_{\nabla\nabla,y}
= \frac{1}{4} \sum_{e\in\Z^d:|e| = 1} \nabla^e \varphi_y \cdot \nabla^e \varphi_y.
\end{equation}
With $x \in \Lambda$ fixed, and
given $g,\nu,z,y,u,\lambda_0,\lambda_x,q_0,q_x \in \R$, we extend
% \refeq{Vtil0def}--\refeq{V0def}
\refeq{V0def}
by defining the polynomial
\begin{align}
% U_y = \frac{1}{4} g |\varphi_y|^4 + \frac{1}{2} \nu |\varphi_y|^2 + \frac{1}{2} z \varphi_y \cdot (-\Delta \varphi)_y + \frac{1}{2} y \tau_{\nabla\nabla,y} + u,
    V_y &= g \tau_y^2 + \nu \tau_y + z \tau_{\Delta,y} + y \tau_{\nabla\nabla,y} + u
    \nnb
    & \quad
    - \1_{y=\pp}\lambda_{\pp}(\varphi_{\pp} \cdot {\sf h})\sigma_\pp
    - \1_{y=\qq}\lambda_{\qq}(\varphi_{\qq} \cdot {\sf h})\sigma_\qq
     \nnb
    & \quad
    - \textstyle{\frac 12} (\1_{y=\pp} q_\pp + \1_{y=\qq}q_\qq )\sigma_\pp\sigma_\qq .
\lbeq{Vy}
\end{align}
Then we define $\Vcal$ to be the space of functions $V=V_y$ of the form \refeq{Vy}.
Given $X \subset \Lambda$, we also define
\begin{equation}
\label{e:Vcalesig}
    \Vcal(X) = \{V(X) = \textstyle{\sum_{y\in X}} V_y : V \in \Vcal \}.
\end{equation}

We also make use of the subspaces $\Vcal^{(1)} \subseteq \Vcal$ consisting of polynomials with $y = 0$, as well as the subspace
$\Vcal^{(0)} \subseteq \Vcal^{(1)}$ of polynomials with
$u = y =   q_\pp=q_\qq = 0$.
For $V \in \Vcal$, we define maps $V \mapsto V^{(1)} \in \Vcal^{(1)}$
and $V \mapsto V^{(0)} \in \Vcal^{(0)}$. Both maps replace
$z\tau_{\Delta}+y\tau_{\nabla\nabla}$ by
$(z+y)\tau_{\Delta}$, and the latter
additionally sets
$u = q_\pp = q_\qq = 0$.

%%%%%%%%%%%%%%%%%%%%%%%%%%%%%%%%%%%%%%%%%%%%%%%%%%%%%%%%%%%%%%%%%%%%%%%%%%%%%%%
%%%%%%%%%%%%%%%%%%%%%%%%%%%%%%%%%%%%%%%%%%%%%%%%%%%%%%%%%%%%%%%%%%%%%%%%%%%%%%%

\section{Renormalisation group coordinates}
% Based on clp
\label{sec:rgcoord}

For $j=0,\ldots, N$,
we partition $\Lambda$ into $L^{N-j}$ disjoint scale-$j$ blocks of side length $L^j$.
A scale-$j$ \emph{polymer} is a union of scale-$j$ blocks.
The set of all scale-$j$ blocks is denoted $\Bcal_j$, and
the set of all scale-$j$ polymers is denoted $\Pcal_j$.
For $X \in \Pcal_j$, we write $\Bcal_j(X)$ for the set of scale-$j$ blocks in $X$.
For $F, G : \Pcal_j \to \Ncal$, we define the \emph{circle product} $F \circ G : \Pcal_j \to \Ncal$ by
\begin{equation}
(F \circ G)(X) = \sum_{Y\in\Pcal_j(X)} F(X \setminus Y) G(Y).
\end{equation}

The evolution of $Z_j$ can be tracked in the \emph{renormalisation group coordinates}
$\zeta_j \in \R$,
$I_j, K_j : \Pcal_j \to \Ncal$, defined such that
\begin{equation}
\label{e:IcircKnew}
    Z_j = e^{\zeta_j}(I_j\circ K_j)(\Lambda),
    \qquad
    \zeta_j= - u_j|\Lambda|
    + \textstyle{\frac 12} (q_{\pp,j} + q_{\qq,j}) \sigma_\pp\sigma_\qq
    .
\end{equation}
The coordinate $I_j$ tracks the evolution of the
\emph{relevant} and \emph{marginal} directions.  It
is determined by a local polynomial
$U\in \Vcal^{(0)}$,
and takes the form
\begin{equation}
I_j(X) = \prod_{B \in \Bcal_j(X)} e^{-U(B)} (1 + W_j(B, U)), \quad X \in \Pcal_j,
\end{equation}
with $W_j$ an explicit quadratic term in $U$ (defined in \cite[\eqref{pt-e:WLTF}]{BBS-rg-pt}).
The evolution of $(\zeta, U)$ to second order is called the \emph{perturbative flow} and is
given by the explicit map $\Vpt : \Vcal \to \Vcal$ defined in
\cite[\eqref{pt-e:Vptdef}]{BBS-rg-pt}.
In particular, it is shown in \cite[Proposition~\ref{phi4-prop:pt}]{ST-phi4}
that the perturbative flow of $q$ is given by
\begin{align}
\label{e:qpt}
q_\pt = q + \lambda_0 \lambda_x C_{j+1;0x},
\end{align}
and that the perturbative flow of $\lambda_0$ and $\lambda_x$ becomes the identity map
once $j$ exceeds the coalescence scale $j_x$.

At scale $j = 0$, we are given $V^+_0$ as defined in \eqref{e:V0def}
and we set $\zeta_0 = 0$. In particular,
the initial values of $u,q_0,q_x$ are zero, and the initial values of $\lambda_0,\lambda_x$
are $1$. By definition, $W_0 = 0$.
For $X \subset \Lambda$, we define
\begin{equation}
  \label{e:IK0def}
    I_0^+(X) = \prod_{x\in X} e^{-V^+_{0,x}},
    \quad\quad
    K_0^+(X) = \prod_{x \in X} I_{0,x}^+ (e^{-\gamma_0 U^{+}_{x}} - 1).
\end{equation}
With these choices, $Z_0$ of \refeq{Z0def}
takes the form \eqref{e:IcircKnew}, and we seek $(\zeta_j, U_j, K_j)$ such that
this continues to hold as the scale advances.

Equivalently, given $(U_j, K_j)$, we define $(\delta\zeta_{j+1}, U_{j+1}, K_{j+1})$ so that
\begin{equation} \label{e:IcircKdu}
  \Ex_{j+1}\theta(I_j \circ K_j)(\Lambda)
  =
  e^{-\delta \zeta_{j+1}}(I_{j+1} \circ K_{j+1})(\Lambda),
\end{equation}
where $\delta\zeta_{j+1} = \zeta_{j+1} - \zeta_j$.
Moreover, we need $K_j$ to contract as the scale advances, under an appropriate norm.
The construction of (scale-dependent) maps $V_+$ and $K_+$ such that
\eqref{e:IcircKdu} holds with
\begin{equation}
    (\delta\zeta_{j+1}, U_{j+1}, K_{j+1}) = (V_+(U_j, K_j), K_+(U_j, K_j))
\end{equation}
is the main accomplishment of \cite{BS-rg-step} and is summarised in Section~\ref{sec:step}
below, in a form adapted to our current setting.

%%%%%%%%%%%%%%%%%%%%%%%%%%%%%%%%%%%%%%%%%%%%%%%%%%%%%%%%%%%%%%%%%%%%%%%%%%%%%%%
%%%%%%%%%%%%%%%%%%%%%%%%%%%%%%%%%%%%%%%%%%%%%%%%%%%%%%%%%%%%%%%%%%%%%%%%%%%%%%%

\section{Renormalisation group domains}
% from saw-sa and clp

We define a scale-dependent norm
\begin{equation}
\label{e:Vnormdef}
\begin{aligned}
\|V\|_{\Vcal} &=
\max\Big\{
|g|, L^{2j}|\nu|, |z|, |y|,  L^{4j}|u|,
\ell_j\ell_{\sigma,j}(|\lambda_\pp|\vee|\lambda_\qq|),\;
%\\
%& \qquad\qquad\qquad
 \ell_{\sigma,j}^{2} (|q_\pp|\vee|q_\qq|)
\Big\}
\end{aligned}
\end{equation}
on $V \in \Vcal$, which depends on parameters $\ell_j$ and $\ell_{\sigma,j}$.
An innovation from \cite{BSTW-clp} is that we define these parameters by
\begin{align}
\label{e:elldef-zz}
\ell_j &= \ell_0 L^{-j - s (j - j_m)_+}, \quad
\ell_{\sigma,j}
=
\ell_{j \wedge j_{x}}^{-1} 2^{(j - j_{x})_+} \ggen_j,
\end{align}
where the mass scale $j_m$ is defined in \eqref{e:jmdef},
the coalescence scale
$j_x$ is defined in \eqref{e:Phi-def-jc},
and $s$ is the parameter appearing in Proposition~\ref{prop:R}.
The sequence $\ggen_j = \ggen_j(m^2,g_0)$ is defined in
\cite[\eqref{log-e:ggendef}]{BBS-saw4-log};
it is bounded above and below by constant multiples of
the sequence $\gbar$ defined in
\eqref{e:gbar},
by
\cite[Lemma~\ref{log-lem:gbarmcomp}]{BBS-saw4-log}.
We discuss the origin of the definition \refeq{elldef-zz} in detail
in Section~\ref{sec:Rpf1}.

% In \cite{BS-rg-step}, maps $V_+,K_+$ are defined which map a pair $(U,K)$ at scale $j$
% to $(V_+(U,K),K_+(U,K))$ at scale $j+1$, and which preserve the circle product
% $I\circ K$ under expectation as in \refeq{IcircKdu}.
In \cite[Definition~\ref{step-def:Kspace}]{BS-rg-step},
a space $\Kcal_j = \Kcal_j(\Lambda)$ of maps $\Pcal_j \to \Ncal$ required to satisfy
several properties is defined.
The coordinate $K_j$ is constructed in \cite{BS-rg-step} as an element of $\Kcal_j$.
% A norm has already been defined on the space $\Vcal$ in \refeq{Vnormdef}.
In \cite[Section~\ref{step-sec:Knorms}]{BS-rg-step},
a sequence of norms $\|\cdot\|_{\Wcal_j} = \|\cdot\|_{\Wcal_j(\mgen^2, \ggen_j, \Lambda)}$
parametrised by $(\mgen^2, \ggen_j)$ is defined on $\Kcal_j$.
We denote the ball of radius $r$ in
the normed space $\Wcal_j$ by $B_{\Wcal_j}(r)$.
In \cite[\eqref{log-e:mass-scale}--\eqref{log-e:chidef}]{BBS-saw4-log},
a function $\chicCov_j = \chicCov_j(m^2)$ (denoted $\chi_j$ in \cite{BBS-saw4-log})
is defined in such a way that $\chicCov_j$ decays exponentially
when $j$ is sufficiently large depending on $m$.
We write $\chicCovgen_j = \chicCov_j(\mgen^2)$.
Given constants $\alpha > 0$ and $C_\DV > 0$,
we define the (finite-volume)
renormalisation group domains $\domRG_j \subset \R^3 \oplus \Wcal_j$ by
% Given $C_\DV>0$ and $\alpha>0$, we define the domains
\begin{align}
\label{e:DVdef}
    \DV_j &= \{U\in \Vcal^{(0)} :
    g> C_{\DV}^{-1} \ggen_j  , \;  \|U\|_{\Vcal} < C_{\DV} \ggen_j \},
\\
\domRG_j &= \DV_j \times B_{\Wcal_j}(\alpha \chigen_j \ggen_j^3).
\end{align}

%%%%%%%%%%%%%%%%%%%%%%%%%%%%%%%%%%%%%%%%%%%%%%%%%%%%%%%%%%%%%%%%%%%%%%%%%%%%%%%
%%%%%%%%%%%%%%%%%%%%%%%%%%%%%%%%%%%%%%%%%%%%%%%%%%%%%%%%%%%%%%%%%%%%%%%%%%%%%%%

\section{Renormalisation group flow}
% from saw-sa

Suppose $\delta > 0$ and suppose $r : [0, \delta] \to [0, \infty)$
is a continuous positive-definite function; the latter
means that $r(x) > 0$ if $x > 0$ and $r(0) = 0$.
We define
\begin{equation}
\lbeq{Ddef}
D(\delta, r)
    =
\{ (w, x, y) \in [0, \delta]^2 \times (-\delta, \delta) : |y| \leq r(x) \}
\end{equation}
and we let $C^{0,1,\pm}(D(\delta, r))$ denote the space of continuous functions on $D(\delta, r)$
that are $C^1$ in $(x, y)$ away from $y = 0$, $C^1$ in $x$ everywhere,
and have left- and right-derivatives in $y$ at $y = 0$.
In our applications, we take $w = m^2$, $x = g_0$ or $\beta$,
and $y = \gamma_0$ or $\gamma$.

\begin{theorem}
\label{thm:rhatflow}
Fix $(m^2, g_0, \gamma_0) \in D(\delta, \hat r)$ with $g_0 > 0$ and
$m^2 \in [\delta L^{-2 (N - 1)}, \delta)$ and
set $(V_0, K_0) = (V^\pm_0, K^\pm_0)$
with $(\nu_0, z_0) = (\hat\nu_0^c, \hat z_0^c)$.
Then for any $N \in \N$,
there exists a sequence $(V_j, K_j) \in \domRG_j(m^2, g_0, \Lambda)$
such that
\begin{equation}
  \label{e:VjKjDj-hat}
  (V_{j+1},K_{j+1}) = (V_{j+1}(V_j, K_j), K_{j+1}(V_j, K_j)) \text{ for all } j < N
\end{equation}
and \eqref{e:ZjIjKj} is satisfied.
Moreover, the second-order evolution equation for $V_j$ is independent of $\gamma_0$.
\end{theorem}