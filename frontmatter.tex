%% This starts numbering in Roman numerals as required for the thesis style and is mandatory.
\frontmatter

\maketitle

\begin{abstract}
The central concern of this thesis is the study of critical behaviour
in models of statistical physics in the upper-critical dimension. We
study a generalized $n$-component lattice $|\varphi|^4$ model and a
model of weakly self-avoiding walk with nearest-neighbour contact
self-attraction on the Euclidean lattice $\Zd$. By utilizing a
supersymmetric integral representation involving boson and fermion
fields, the two models are studied in a unified manner.

Our main result, \new{which is contingent on a small coupling hypothesis,}
% \new{which is an extension of work of Bauerschmidt, Brydges, and Slade,}
identifies the precise leading-order asymptotics of the two-point function,
susceptibility, and finite-order correlation length of both models in $d = 4$.
In particular, we show that the critical two-point function satisfies mean-field
scaling whereas the near-critical susceptibility and finite-order correlation
length exhibit logarithmic corrections to mean-field behaviour. The proof employs
a renormalisation group method \new{of Bauerschmidt, Brydges, and Slade} based on
a finite-range covariance decomposition and requires two extensions to this method.

The first extension, which is required for the computation of the finite-order
correlation length, is an improvement of the norms used to control the evolution
of the renormalisation group. This allows us to obtain improved error estimates
in the massive regime of the renormalisation group flow.

The second extension involves the identification of critical parameters for models
initialized with a non-zero error coordinate coupled to a marginal/relevant coordinate.
This allows us, for example,
% to realize the renormalisation group flow of the weakly
% self-avoiding walk with self-attraction as a small perturbation of the flow of the
% ordinary weakly self-avoiding walk.
\new{to realize the the two-point function and susceptibility for the walk with
self-attraction as a small perturbation of the corresponding quantities without
self-attraction, thereby establishing a form of universality.
% whose asymptotic behaviour was determined by Bauerschmidt, Brydges, and Slade.
}
\end{abstract}

\cleardoublepage

\chapter{Preface}

Sections~\ref{sec:intro}--\ref{sec:walks} are a general introduction to the
subject matter of this thesis and we do not claim any originality here. Section~\ref{sec:mr}
states our main result, which combines and slightly extends results from the following:
\begin{itemize}
\item
the article \cite{BSTW-clp}, written jointly with
Roland Bauerschmidt, Gordon Slade, and Alexandre Tomberg and
published in \textit{Annales Henri Poincar\'{e}};
and

\item
the article \cite{BSW-saw-sa}, written jointly with
Roland Bauerschmidt and Gordon Slade and published
in \textit{Journal of Statistical Physics}.
\end{itemize}
Section~\ref{sec:spin-walk} includes part of \cite{BSW-saw-sa}.

Chapters~\ref{sec:rg}--\ref{sec:RGflow} are based on \cite{BSTW-clp,BSW-saw-sa}:
\begin{itemize}
\item
Chapter~\ref{sec:rg} includes part of \cite{BSTW-clp,BSW-saw-sa},
and discusses the general theory developed in
\cite{BS-rg-norm,BS-rg-loc,BBS-rg-pt,BS-rg-IE,BS-rg-step}
and applied and extended in
\cite{BBS-saw4-log,BBS-saw4,ST-phi4};

\item
Chapter~\ref{sec:chi-G-xi} includes part of \cite{BSTW-clp,BSW-saw-sa}
and Section~\ref{sec:suscept} includes an additional discussion regarding an argument
of \cite{BBS-saw4-log};

\item
Chapter~\ref{sec:RGstep} includes part of \cite{BSTW-clp}; in addition,
Sections~\ref{sec:rgmech} and \ref{sec:cc} include discussions regarding some of the ideas in
\cite{BS-rg-step} and \cite{BS-rg-loc}, respectively; and

\item
Chapter~\ref{sec:RGflow} includes part of \cite{BSW-saw-sa}.
\end{itemize}

\cleardoublepage

\tableofcontents                  %% Mandatory

\cleardoublepage

% \listoffigures                  %% Mandatory if thesis has figures
{%								  %% For new thesis format
\let\oldnumberline\numberline%
\renewcommand{\numberline}{\figurename~\oldnumberline}%
\listoffigures%
}

\cleardoublepage

%% Any other lists should come here, i.e.
%% Abbreviation schemes, definitions, lists of formulae, list of
%% schemes, glossary, list of symbols etc.

% LIST OF SYMBOLS GOES HERE
% http://latex.org/know-how/latex/55-latex-general/263-glossaries-nomenclature-lists-of-symbols-and-acronyms

\chapter{Acknowledgements}      %% Optional
I would like to thank my advisor Gordon Slade for his invaluable guidance and support
during my time at UBC. I cannot overstate how much I have benefited
from his patience and generosity, without which this thesis would certainly not have
been possible. I would also like to thank Joel Feldman and Ed Perkins for serving on my
thesis committee. The work discussed in this thesis resulted from collaborations with
Gord, Roland Bauerschmidt, and Alex Tomberg and I am grateful towards all of them for
fruitful discussions as well as for a lot of helpful advice. I am also grateful
to David Brydges for a number of enlightening and inspiring conversations as well as for
suggesting a wealth of promising research directions. I thank Maxime Bergeron and
Tom Hutchcroft for helpful comments on Chapter~\ref{sec:intro} of this thesis and I am
thankful towards the UBC probability group as a whole for providing such a stimulating
and friendly research environment.

Lastly, I am grateful for the unwavering support I have received from my family and friends
throughout the course of my studies.
% Gordon Slade
% Committee: Joel Feldman and Ed Perkins
% Collaborators: Roland Bauerschmidt, Martin Lohmann, Gordon Slade, Alex Tomberg
% Special thanks to David Brydges: conversations, advice, suggestions, inspirations
% Brian Marcus: conversations and chairing candidacy
% Maxime Bergeron and Tom Hutchcroft: reading intro and conversations
% Laszlo?
% UBC/Probability: stimulating research environment
% Chanel/friends/family
