%% This starts numbering in Roman numerals as required for the thesis
%% style and is mandatory.
\frontmatter

%%% The order of the following components should be preserved.  The order
%%% listed here is the order currently required by FoGS:        \\
%%% Title (Mandatory)                                           \\
%%% Preface (Manditory if any collaborator contributions)       \\
%%% Abstract (Mandatory)                                        \\
%%% List of Contents, Tables, Figures, etc. (As appropriate)    \\
%%% Acknowledgements (Optional)                                 \\
%%% Dedication (Optional)                                       \\

\maketitle                      %% Mandatory

\begin{abstract}                %% Mandatory
The central concern of this thesis is the study of critical behaviour in models of
statistical physics in the upper-critical dimension. We study a generalized lattice
$|\varphi|^4$ model and a model of weakly self-avoiding walk with self-attraction.
By utilizing a supersymmetric integral representation,
% of the latter's two-point function,
we are able to study the two models in a unified manner. Our main result
involves the identification of leading-order asymptotics of the two-point function,
susceptibility, and finite-order correlation length of these models.

The proof proceeds via a renormalisation group method, and requires two extensions to this
method. The first extension, which is required for the computation of the finite-order
correlation length, is an improvement of the norms used to control the evolution of the
renormalisation group. This allows us to obtain better error estimates in the massive regime
of the renormalisation group flow.

The second extension involves the identification of critical parameters for models initialized
with a non-zero error coordinate coupled to the marginal/relevant coordinate. This allows us,
for example, to realize the renormalisation group flow of the weakly self-avoiding walk with self-attraction as a small perturbation of the flow of the ordinary weakly self-avoiding walk.
\end{abstract}

\cleardoublepage

\chapter{Preface}

Sections~\ref{sec:intro}--\ref{sec:walks} are a general introduction to the
subject matter of this thesis and we do not claim any originality here. Section~\ref{sec:mr}
states our main results, which combine and slightly extend results from the following:
\begin{itemize}
\item
the article \cite{BSTW-clp}, written jointly with
Roland Bauerschmidt, Gordon Slade, and Alexandre Tomberg and
published in \textit{Annales Henri Poincar\'{e}};
and

\item
the article \cite{BSW-saw-sa}, written jointly with
Roland Bauerschmidt and Gordon Slade and published
in \textit{Journal of Statistical Physics}.
\end{itemize}
Section~\ref{sec:spin-walk} includes part of \cite{BSW-saw-sa}.

Chapters~\ref{sec:rg}--\ref{sec:RGflow} are based on \cite{BSTW-clp,BSW-saw-sa}:
\begin{itemize}
\item
Chapter~\ref{sec:rg} includes part of \cite{BSTW-clp,BSW-saw-sa},
and discusses the general theory developed in
\cite{BS-rg-norm,BS-rg-loc,BBS-rg-pt,BS-rg-IE,BS-rg-step}
and applied and extended in
\cite{BBS-saw4-log,BBS-saw4,ST-phi4};

\item
Chapter~\ref{sec:chi-G-xi} includes part of \cite{BSTW-clp,BSW-saw-sa},
and Section~\ref{sec:suscept} includes an additional discussion regarding an argument
of \cite{BBS-saw4-log};

\item
Chapter~\ref{sec:RGstep} includes part of \cite{BSTW-clp}, and
Section~\ref{sec:rgmech} includes an additional discussion regarding some of the ideas in
\cite{BS-rg-step}; and

\item
Chapter~\ref{sec:RGflow} includes part of \cite{BSW-saw-sa}.
\end{itemize}


% The Preface must include a statement indicating the student's contribution to the following:

% Identification and design of the research program,
% Performance of the various parts of the research, and
% Analysis of the research data.
% Certain additional elements may also be required, as specified below.

% If any of the work presented in the thesis has led to any publications or submissions, all of these must be listed in the Preface. Bibliographic details should include the title of the article and the name of the publisher (if the article has been accepted or published), and the chapter(s) of the thesis in which the associated work is located.
% If the work includes publications or material submitted for publication, the statement described above must detail the relative contributions of all collaborators and co-authors (including supervisors and members of the supervisory committee) and state the proportion of research and writing conducted by the student. For further details, see “Including Published Material in a Thesis or Dissertation”.
% If the work includes other scholarly artifacts (such as film and other audio, visual, and graphic representations, and application-oriented documents such as policy briefs, curricula, business plans, computer and web tools, pages, and applications, etc.), all of these must be listed in the Preface (with bibliographical information, if applicable).
% If ethics approval was required for the research, the Preface must name the responsible UBC Research Ethics Board, and report the project title(s) and the Certificate Number(s) of the Ethics Certificate(s) applicable to the project.
% In a thesis where the research was not subject to ethics review, produced no publications, and was designed, carried out, and analyzed by the student alone, the text of the Preface may be very brief. Samples are available on this website and in the University Library's online repository of accepted theses.

% The content of the Preface must be verified by the student's supervisor, whose endorsement must appear on the final Thesis/Dissertation Approval form.

% Acknowledgements, introductory material, and a list of publications do not belong in the Preface. Please put them respectively in the Acknowledgements section, the first section of the thesis, and the appendices.

% %%% Sections and subsections etc. in the Preface should in general
% %%% not be listed in the table of contents, so use the starred form
% %%% of \section etc.
% \section*{Examples}
% Chapter~\ref{cha:apple_ref} is based on work conducted in UBC's Maple
% Syrup Laboratory by Dr. A.  Apple, Professor B. Boat, and Michael
% McNeil Forbes. I was responsible for tapping the trees in forests X
% and Z, conducted and supervised all boiling operations, and performed
% frequent quality control tests on the product.

% A version of chapter~\ref{cha:apple_ref} has been
% published~\cite{Apple:2010}. I conducted all the testing and wrote
% most of the manuscript. The section on ``Testing Implements'' was
% originally drafted by Boat, B.  Check the first pages of this
% chapter to see footnotes with similar information.

% Note that this preface must come before the table of contents.  Note
% also that this section ``Examples'' should not be listed in the table
% of contents, so we have used the starred form: \verb|\section*{Example}|.

\cleardoublepage

\tableofcontents                %% Mandatory

\cleardoublepage

% \listoftables                 %% Mandatory if thesis has tables

% \listoffigures                  %% Mandatory if thesis has figures
{%
\let\oldnumberline\numberline%
\renewcommand{\numberline}{\figurename~\oldnumberline}%
\listoffigures%
}

\cleardoublepage

% \listof{Program}{List of Programs} %% Optional

%% Any other lists should come here, i.e.
%% Abbreviation schemes, definitions, lists of formulae, list of
%% schemes, glossary, list of symbols etc.

% LIST OF SYMBOLS GOES HERE
% http://latex.org/know-how/latex/55-latex-general/263-glossaries-nomenclature-lists-of-symbols-and-acronyms

\chapter{Acknowledgements}      %% Optional
To come...
% Acknowledge: Gord, David, Roland, Alex, committee, family, etc.

% \chapter{Dedication} %% Optional
% Dedication...

% Any other unusual prefactory material should come here before the
% main body.
