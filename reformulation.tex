\chapter{Reformulation of the problem}

\commentbw{Think of a better title}

\section{Finite-volume approximation}
% Based on saw-sa
\label{sec:finvol}

The first step in the proof of Theorem~\ref{thm:suscept}
is an approximation of $G_x(\gcc, \gamma, \nu)$
by finite-volume analogues of these quantities.
This is the content of Proposition~\ref{prop:finvol}.

Before proving the proposition, we require some preliminaries.
Let $P^n$ be the projection
of $\Zd$ onto the discrete torus of side $n$,
which we denote $\Z_n^d$.
Then $P^n$ has a natural action
on the path space $(\Zd)^{[0,\infty)}$. We let
$X^n = P^n(X)$ be the projection of $X$
and note that $X^n$ is a simple random walk on $\Z^d_n$.

We call $f = (f_x)_{x\in\Zd}$ a \emph{field of path functionals} if
$f_x : (\Zd)^{[0,\infty)} \to \R$ is a function on continuous-time paths
for each $x \in \Zd$;
a simple example is given by the local time functional.
We assume that the \emph{random} field $f(X) = (f_x(X))_{x\in\Zd}$
has finite support almost surely, i.e.,
with probability $1$, $f_x(X) = 0$ for all but finitely many $x$.
Denote by $f(X^n)$ the corresponding random field for $X^n$, i.e., for $x \in \Z_n^d$,
\begin{equation}
f_x(X^n) = \sum_{y\in\Zd} f_{x+ny}(X).
\end{equation}
Then
\begin{equation}
\label{e:ffold1}
\sum_{\|y\|_\infty < k} f_{x+ny}(X^{kn})
  = \sum_{\|y\|_\infty < k} \sum_{z\in\Zd} f_{x+ny+knz}(X)
  = \sum_{y\in\Zd} f_{x+ny}(X)
  = f_x(X^n),
\end{equation}
and it follows by summation over $x \in \Z^d_n$ that
\begin{equation}
\label{e:ffold2}
\sum_{x\in\Z^d_{kn}} f_x(X^{kn})
  =
\sum_{x\in\Z^d_n} f_x(X^n).
\end{equation}

\begin{lemma}
\label{lem:mono}
Let $n,k \ge 1$ and let $f$ and $h$ be nonnegative fields of path functionals
with finite support almost surely.
Then
\begin{equation}
\sum_{x\in\Z^d_{kn}} f_x(X^{kn}) h_x(X^{kn})
  \leq
\sum_{x\in\Z^d_n} f_x(X^n) h_x(X^n).
\end{equation}
\end{lemma}

\begin{proof}
By \eqref{e:ffold2} and \eqref{e:ffold1},
\begin{equation}
\sum_{x\in\Z_{kn}^d} f_x(X^{kn}) h_x(X^{kn})
  =
\sum_{x\in\Z_n^d}
\sum_{\|y\|_\infty < k}
  f_{x+ny}(X^{kn}) h_{x+ny}(X^{kn}).
\lbeq{mono}
\end{equation}
By nonnegativity and two more applications of \eqref{e:ffold1},
\begin{align}
\sum_{x\in\Z_n^d}
\sum_{\|y\|_\infty < k}
f_{x+ny}(X^{kn}) h_{x+ny}(X^{kn})
  &\le \sum_{x\in\Z_n^d}
      \left(\sum_{\|y\|_\infty < k} f_{x+ny}(X^{kn})\right)
      \left(\sum_{\|y\|_\infty < k} h_{x+ny}(X^{kn})\right) \nonumber \\
  &= \sum_{x\in\Z_n^d} f_x(X^n) h_x(X^n).
\end{align}
This completes the proof.
\end{proof}

Fix $L \geq 2$ and $N \geq 1$.
For $F_T = F_T(X)$ any one of the functions $\lt^x_T,I_T,C_T$
of $X$ defined in \eqref{e:LTx-def}--\eqref{e:CTdef},
we write $F_{N,T} = F_T(X^{L^N})$. For instance, with $n=L^N$,
\begin{equation}
\lt^x_{N,T} = \int_0^T \1_{X^{n}_t=\;x} \; dt.
\end{equation}
We apply Lemma~\ref{lem:mono} with $k = L$ and $n = L^N$
for three choices of $f,h$:
\begin{alignat}{2}
\label{e:ILT-mon}
I_{N+1,T} &\leq I_{N,T}\quad &&(f_x=h_x=\lt^x_T),
\\
\label{e:CSA-mon}
C_{N+1,T} &\leq C_{N,T} \quad &&(f_x=\textstyle{\sum_{e\in \Ucal}\lt^{x+e}_T},\; h_x=\lt^x_T) ,
\\
\sum_{x\in\Lambda_{N+1}} |\nabla^e \lt^x_{N+1,T}|^2
  &\leq
\sum_{x\in\Lambda_N} |\nabla^e \lt^x_{N,T}|^2
\quad &&
(f_x = h_x = \left|\nabla^e \lt^x_T\right|).
\lbeq{nabL}
\end{alignat}
Summation of \refeq{nabL} over $e\in\Ucal$ also gives
\begin{align}
\label{e:gradLT-mon}
\sum_{x\in\Lambda_{N+1}} |\nabla \lt^x_{N+1,T}|^2
  \leq
\sum_{x\in\Lambda_N} |\nabla \lt^x_{N,T}|^2.
\end{align}

Recall that $\Lambda_N$ denotes the discrete torus of period $L^N$. % added
For $N \ge 1$, we identify the vertices of $\Lambda_N$ with nested subsets of $\Zd$,
centred at the origin (approximately if $L$ is even),
with $\Lambda_{N+1}$ paved by $L^d$ translates of $\Lambda_N$.
We can thus define $\partial \Lambda_N$ to be the inner vertex boundary of $\Lambda_N$.
We denote the expectation of $X^{L^N}$ started from $a \in \Lambda_N$ by $E^{\Lambda_N}_a$
and define
\begin{align}
\label{e:cN}
&c_{N,T}(a, b)
    = E^{\Lambda_N}_a \left( e^{-U_{\gcc,\gamma,T}} \1_{X(T)=b} \right)
    \quad (a, b \in \Lambda_N) \\
&c_{N,T}
    = E^{\Lambda_N}_0 \left( e^{-U_{\gcc,\gamma,T}} \right).
\end{align}
The finite-volume two-point function and susceptibility
are defined by
\begin{align}
&G_{x,N}(\gcc, \gamma, \nu)
    = \int_0^\infty c_{N,T}(a, b) e^{-\nu T} \; dT, \\
&\chi_N(\gcc, \gamma, \nu)
    = \int_0^\infty c_{N,T} e^{-\nu T} \; dT
    .
    \label{e:chiNdef}
\end{align}

\begin{prop}
\label{prop:finvol}
Let $d >0$, $\gcc >0$ and $\gamma < \gcc$. For all $\nu \in \R$,
\begin{equation}
\label{e:Givlc}
\lim_{N \to \infty}
G_{x,N}(\gcc, \gamma, \nu)
=
G_x(\gcc, \gamma, \nu)
\end{equation}
and
\begin{equation}
\label{e:chilim}
\lim_{N\to\infty}\chi_N(\gcc,\gamma,\nu)=   \chi(\gcc,\gamma,\nu).
\end{equation}
\end{prop}

\begin{proof}
Fix $x \in \Zd$, and consider $N$ sufficiently large that $x$ can be identified
with a point in $\Lambda_N$.
By \eqref{e:V2} and \eqref{e:gradLT-mon} (if $0 \le \gamma <\gcc$)
or \eqref{e:V} and \eqref{e:CSA-mon} (if $\gamma < 0$),
\begin{equation}
\label{e:ctmon}
c_{N,T}(a, b) \leq c_{N+1,T}(a, b).
\end{equation}
Thus, \eqref{e:Givlc} follows by monotone convergence, once we show that
\begin{equation}
\lim_{N\to\infty} c_{N,T}(a, b) = c_T(a, b).
\end{equation}
This follows as in \cite[(2.8)]{BBS-saw4}.
That is, first we define
\begin{equation}
c_T^*(a, b)
  =
E^{\Lambda_N}_a
\left(
  e^{-U_{\gcc,\gamma,T}} \1_{X(T)=b} \1_{\{X([0, T]) \cap \partial \Lambda_N \neq \varnothing\}}
\right).
\end{equation}
Since walks which do not reach $\partial \Lambda_N$ make equal contributions to both
$c_T(a,b)$ and $c_{N,T}(a,b)$,
we have
\begin{equation}
c_T(a, b) - c_T^*(a, b) = c_{N,T}(a, b) - c_{N,T}^*(a, b).
\end{equation}
Thus,
\begin{align}
|c_T(a, b) - c_{N,T}(a, b)|
= |c_T^*(a, b) - c_{N,T}^*(a, b)|
\leq c_T^*(a, b) + c_{N,T}^*(a, b).
\end{align}
Let $P^{\Lambda_N}_a$ be the measure associated with $E^{\Lambda_N}_a$.
With $Y_t$ a rate-$2d$ Poisson process with measure ${\sf P}$,
\begin{align}
  c_T^*(a, b) + c_{N,T}^*(a, b)
  &\leq P^{\Lambda_N}_a (X([0, T]) \cap \partial\Lambda_N \neq \varnothing)
    + P^{\Lambda_N}_a (X([0, T]) \cap \partial\Lambda_N \neq \varnothing) \nonumber \\
  &\leq 2 {\sf P} (Y_T \geq \diam{\Lambda_N}) \to 0
\end{align}
as $N\to\infty$.
This completes the proof of \eqref{e:Givlc}.
Moreover, by monotone convergence of $G_N$ to $G$,
for $\nu \in \R$,
\begin{equation}
\lim_{N\to\infty} \chi_N(\gcc, \gamma, \nu)
    = \sum_{b\in\Zd} \lim_{N\to\infty} G_{x,N}(\gcc, \gamma, \nu) \1_{b\in\Lambda_N}
    = \chi(\gcc, \gamma, \nu),
\end{equation}
which proves \eqref{e:chilim}.
\end{proof}

\section{Integral representation}
% Based on saw-sa
In this section, we reformulate the % model
weakly self-avoiding walk with self-attraction in terms of a perturbation of a supersymmetric
Gaussian integral, in order to prepare for the application of the renormalisation group.
The integral representation, which is a special case of a result from \cite{BIS09},
makes use of the Grassmann integral. We begin by recalling the definition of the Grassmann
integral in Section~\ref{sec:forms} and state the integral representation in Section~\ref{sec:Gintrep}.
In Section~\ref{sec:Gauss-approx}, we split the integral into a Gaussian part and a perturbation.
The basic idea underlying the renormalisation group is the progressive evaluation of this
Gaussian integral via a multi-scale decomposition of its covariance, which we introduce in Section~\ref{sec:prog}.

%%%%%%%%%%%%%%%%%%%%%%%%%%%%%%%%%%%%%%%%%%%%%%%%%%%%%%%%%%%%%%%%%%%%%%%%%%%%%%%%

\subsection{Boson and fermion fields}
\label{sec:forms}

We fix $N$ and write $\Lambda = \Lambda_N$.
Given complex variables $\phi_x, \bar\phi_x$
(the boson field) for $x \in \Lambda$,
we define the differentials (the fermion field)
\begin{equation}
\psi_x = \frac{1}{\sqrt{2\pi i}} d\phi_x,
\quad
\bar\psi_x = \frac{1}{\sqrt{2\pi i}} d\bar\phi_x,
\end{equation}
where we fix a choice of complex square root.
The fermion fields are multiplied with each other
via the anti-commutative wedge product,
though we suppress this in our notation.

A differential form that is the
product of a function of $(\phi, \bar\phi)$
with $p$ differentials is said to have degree $p$.
A sum of forms of even degree is said to be \emph{even}.
We introduce a copy $\bar\Lambda$ of $\Lambda$
and we denote the copy of $X \subset \Lambda$ by $\bar X \subset \bar\Lambda$.
We also denote the copy of $x \in \Lambda$
by $\bar x \in \bar\Lambda$ and define $\phi_{\bar x} = \bar\phi_x$ and $\psi_{\bar x} = \bar\psi_x$.
Then any differential form $F$ can be written
\begin{equation}
\lbeq{FinNcal}
F
=
\sum_{\vec y}
F_{\vec y} (\phi, \bar\phi)
\psi^{\vec y}
\end{equation}
where the sum is over finite sequences $\vec y$ over $\Lambda\sqcup\bar\Lambda$
and $\psi^{\vec y} = \psi_{y_1} \ldots \psi_{y_p}$ when $\vec y = (y_1, \ldots, y_p)$.
When $\vec y = \varnothing$ is the empty sequence,
$F_\varnothing$ denotes the $0$-degree (bosonic) part of $F$.

In order to apply the results of \cite{BBS-saw4-log,BBS-saw4,BSTW-clp}, we require
smoothness of the coefficients $F_{\vec y}$ of $F$.  For Theorem~\ref{thm:suscept}(i,ii),
we need these coefficients to be $C^{10}$, and for Theorem~\ref{thm:suscept}(iii) we require
a $p$-dependent number of derivatives for the analysis of % $\xi_p$
the finite-order correlation length,
as discussed in \cite{BSTW-clp}.
We let $\Ncal$ be the algebra of even forms with sufficiently smooth coefficients
and we let $\Ncal(X) \subset \Ncal$ be the sub-algebra of even forms only depending on fields
in $X$. Thus, for $F \in \Ncal(X)$, the sum in \eqref{e:FinNcal} runs over sequences $\vec y$
over $X \sqcup \bar X$.
Note that $\Ncal = \Ncal(\Lambda)$.


Now let $F = (F_j)_{j \in J}$ be a finite collection of even forms
indexed by a set $J$
and write $F_\varnothing = (F_{\varnothing,j})_{j \in J}$.
Given a $C^\infty$ function $f : \R^J \to \C$, we define
$f(F)$ by its Taylor expansion about $F_\varnothing$:
\begin{equation}
f(F) = \sum_\alpha \frac{1}{\alpha!} f^{(\alpha)}(F_\varnothing) (F - F_\varnothing)^\alpha.
\end{equation}
The summation terminates as a finite sum,
since $\psi_x^2 = \bar\psi_x^2 = 0$ due to the anti-commut\-ative product.

We define the integral
$\int F$
of a differential form $F$ in the usual way
as the Riemann integral of its top-degree part
(which may be regarded as a function
of the boson field).
In particular, given a positive-definite
$\Lambda \times \Lambda$ symmetric matrix $C$
with inverse $A = C^{-1}$,
we define the \emph{Gaussian expectation}
(or \emph{super-expectation}) of $F$ by
\begin{equation}
\lbeq{ExCF}
\Ex_C F = \int e^{-S_A} F,
\end{equation}
where
\begin{equation}
\label{e:action}
S_A = \sum_{x\in\Lambda} \Big(\phi_x (A\bar\phi)_x + \psi_x (A \bar\psi)_x\Big).
\end{equation}

Finally, for $F = f(\phi, \bar\phi) \psi^{\vec y}$,
we let
\begin{equation}
\theta F = f(\phi + \xi, \bar\phi + \bar\xi) (\psi + \eta)^{\vec y},
\end{equation}
where $\xi$ is a new boson field, $\eta = (2\pi i)^{-1/2} d\xi$ a new fermion field,
and $\bar\xi, \bar\eta$ are the corresponding conjugate fields.
We extend $\theta$ to all $F \in \Ncal$ by linearity
and define the convolution operator $\Ex_C\theta$ by letting
$\Ex_C\theta F \in \Ncal$ denote the Gaussian expectation of $\theta F$ with respect
to $(\xi, \bar\xi, \eta, \bar\eta)$, with $\phi,\phib,\psi,\psib$ held fixed.

%%%%%%%%%%%%%%%%%%%%%%%%%%%%%%%%%%%%%%%%%%%%%%%%%%%%%%%%%%%%%%%%%%%%%%%%%%%%%%%%

\subsection{Integral representation of the two-point function}
\label{sec:Gintrep}

An integral representation formula applying to general local time functionals
is given in \cite{BEI92,BIS09}; see also \cite[Appendix~A]{ST-phi4}.
We state the result we need in the proposition below.

% Let $\Delta$ denote the Laplacian on $\Lambda$, i.e.\ $\Delta_{xy}$ is given by
% the right-hand side of \eqref{e:Deltaxy} for $x, y \in \Lambda$.
Recall that $\Delta$ denotes the Laplacian on $\Lambda$, defined in \eqref{e:DeltaLambda}.
We define the differential forms:
\begin{align}
\tau_x
&= \phi_x \bar\phi_x
+ \psi_x   \bar\psi_x
= |\phi_x|^2 + |\psi_x|^2 \\
\label{e:addDelta}
\tau_{\Delta,x}
&=
\frac 12 \Big(
\phi_{x} (- \Delta \bar{\phi})_{x} + (- \Delta \phi)_{x} \bar{\phi}_{x} +
\psi_{x}  (- \Delta \bar{\psi})_{x} + (- \Delta \psi)_{x}  \bar{\psi}_{x}
\Big) \\
|\nabla \tau_x|^2
&= \sum_{e\in\Ucal} (\nabla^e \tau)_x^2.
\end{align}

\begin{prop}
Let $d > 0$ and $\gcc > 0$. For $\gamma < \gcc$ and $\nu \in \R$,
\begin{align}
G_{x,N}(\gcc, \gamma, \nu)
&=  \int
    e^{-\sum_{x\in\Lambda}
    \left(
    U_{\gcc,\gamma}(\tau)
    + \nu \tau_x + \tau_{\Delta,x}\right)} \bar\phi_a \phi_b
    .
    \label{e:Grep-pos-bis}
\end{align}
\end{prop}

\begin{proof}
The proof is identical to the proof of the $p = 1$ case of
\cite[Proposition~\ref{phi4-prop:Integral-Representation}]{ST-phi4}
when, in the notation used in \cite{ST-phi4}, we set
\begin{equation}
F(S) = e^{-U_{\gcc,\gamma}(S) - (\nu - 1) \sum_{x\in\Lambda} S_x}
\end{equation}
in \cite[\eqref{phi4-e:intrep1}]{ST-phi4}.
\end{proof}

%%%%%%%%%%%%%%%%%%%%%%%%%%%%%%
% COMMENTED OUT: OLD VERSION %
%%%%%%%%%%%%%%%%%%%%%%%%%%%%%%
% \begin{theorem}
% \label{thm:BIS}
% Let $F : [0, \infty)^\Lambda \to \R$ be a $C^\infty$ function and
% assume that for any $\epsilon > 0$ and multi-index $\alpha$,
% there exists a constant $C_{\epsilon,\alpha}$ such that
% \begin{equation}
% \label{e:BIS-expbd}
% |F^{(\alpha)}(t)| \leq C_{\epsilon,\alpha}
%   \exp\left(\epsilon \sum_{x\in\Lambda} t_x\right).
% \end{equation}
% Then
% % for any $\delta > 0$,
% \begin{equation}
% \label{e:BIS}
% \int_0^\infty E^\Lambda_a \left(F(L_T) \1_{X(T)=b} \right) \; dT
% =
% \int e^{-\sum_{x\in\Lambda} \tau_{\Delta,x}} F(\tau) \bar\phi_a \phi_b.
% \end{equation}
% \end{theorem}
%
% \begin{proof}
% Fix $\delta > 0$ and define the $\Lambda \times \Lambda$ matrices
% $D = 2 d + \delta$ and $J = \Delta + 2 d$.
% Set $A = D - J = -\Delta + \delta$.
% Define $\tilde F(t) = F(t) e^{\delta \sum_{x\in\Lambda} t_x}$.
% We begin by applying \cite[(4.26)]{BIS09} to $\tilde F$
% with the choices of $D, J$ above,
% which \chgs{do} %are easily seen to
%  satisfy the necessary hypotheses.
% This yields
% \begin{equation}
% \label{e:BIS1}
% \frac{1}{(2d + \delta) \pi_{b\partial}}
% E_a^* (\tilde F(L) \1_{X(\zeta^-)=b})
%     =
% \int e^{-S_A} F(\tau) e^{\delta \sum_{x\in\Lambda} \tau_x} \bar\phi_a \phi_b,
% \end{equation}
% where $E_a^*$ is the expectation of a continuous-time Markov chain
% with state space $\Lambda \cup \{ \partial \}$
% (where $\partial$ is a ``cemetery state''),
% $\operatorname{Exp}(2d + \delta)$ holding times,
% and transition probabilities
% $\pi_{xy} = (2d + \delta)^{-1} J_{xy}$ if $y \in \Lambda$
% and $\pi_{x\partial} = 1 - \sum_{y\in\Lambda} \pi_{xy}$.
% We denote its field of local times by $L$
% and the hitting time at $\partial$ by $\zeta$.

% The right-hand side of \eqref{e:BIS1} can be simplified using \eqref{e:SAtauDelta}
% to get the right-hand side of \eqref{e:BIS}.
% By \cite[(2.36)]{BIS09}, the left-hand side of \eqref{e:BIS1} is equal to
% \begin{equation}
% \int_0^\infty E^\Lambda_a (\tilde F(L_T) e^{-\sum_{x\in\Lambda} (2d + \delta - \bar d_x) L^x_T} \1_{X(T)=b}) \; dT,
% \end{equation}
% with $\bar d_x = \sum_{y\in\Lambda} J_{xy} = 2 d$.
% Simplifying this expression yields the left-hand side of \eqref{e:BIS}.
% \end{proof}
%
% We apply Theorem~\ref{thm:BIS} with
% \begin{equation}
% F(t) = e^{-U_{\gcc,\gamma}(t) - \nu \sum_{x\in\Lambda} t_x},
% \end{equation}
% and obtain
% \begin{align}
% G_{x,N}(\gcc, \gamma, \nu)
% &=  \int
%     e^{-\sum_{x\in\Lambda}
%     \left( (\gcc - \gamma) \tau_x^2 + \frac{\gamma}{4d} |\nabla \tau_x|^2
%     + \nu \tau_x + \tau_{\Delta,x}\right)} \bar\phi_a \phi_b
%     .
%     \label{e:Grep-pos}
% \end{align}
% The hypothesis \eqref{e:BIS-expbd} is seen to be verified,
% from \eqref{e:Udef-pos} when $\gamma \geq 0$
% and from \eqref{e:Udef-neg} when $\gamma < 0$.

\section{Gaussian approximation}
% Based on both
\label{sec:Gauss-approx}

\commentbw{Extend to $n > 0$}

We divide
the integral in \eqref{e:Grep-pos-bis} into
a Gaussian part and a perturbation.  Although the division is arbitrary here,
a careful choice of the division must be made, and it is made in Theorem~\ref{thm:flow-flow}.
We require several definitions.
Let $z_0>-1$ and $m^2 >0$. We set
\begin{equation}
\label{e:gg0}
g_0 = (\gcc - \gamma) (1 + z_0)^2,
\quad
\nu_0 = \nu (1 + z_0) - m^2,
\quad
\gamma_0 = \frac{1}{4d} \gamma (1 + z_0)^2,
\end{equation}
and define
\begin{equation}
\label{e:V0def}
  V^+_{0,x}
  = g_0\tau_x^2 + \nu_0 \tau_x + z_0 \tau_{\Delta,x},
  \quad
  U^+_x = |\nabla \tau_x|^2.
\end{equation}
The monomial $U^+_x$ should not be confused with
the potential $U_{\gcc,\gamma}$.
We define
\begin{equation}
\label{e:Z0def}
  Z_0
  =
  \prod_{x\in \Lambda} e^{-(V^+_{0,x} + \gamma_0 U^+_x)},
\end{equation}
and, with $C = (-\Delta + m^2)^{-1}$ and with the expectation given by \refeq{ExCF},
\begin{equation}
\label{e:ZNdef}
Z_N = \Ex_C \theta Z_0.
\end{equation}
Recall that $Z_{N,\varnothing}$ denotes the $0$-degree part of $Z_N$.
We define a test function $\1: \Lambda_N \to \R$ by $\1_x=1$ for all $x$,
and write $D^2 Z_{N,\varnothing}(0, 0; \1, \1)$ for the directional derivative of
$Z_{N,\varnothing}$
at $(\phi, \bar\phi) = (0, 0)$, with both directions equal to $\1$.

\begin{prop}
Let $d > 0$, $\nu \in \R$, $\gcc >0$ and $\gamma <\gcc$.
If the relations \eqref{e:gg0} hold, then
\label{prop:intrep}
\begin{equation}
\label{e:GG2}
G_{x,N}(\gcc, \gamma, \nu)
    =
(1+z_0)
\Ex_C (Z_0 \bar\phi_a \phi_b),
\end{equation}
and
\begin{equation}
\label{e:chichibar}
  \chi_N\left(\gcc,\gamma,\nu\right)
  = (1+z_0)\hat\chi_N(m^2, g_0, \gamma_0, \nu_0, z_0)
  ,
\end{equation}
with
\begin{equation}
  \label{e:chibarm}
  \hat\chi_N(m^2, g_0, \gamma_0, \nu_0, z_0)
  = \frac{1}{m^2}  + \frac{1}{m^4} \frac{1}{|\Lambda|} D^2 Z_{N,\varnothing}(0, 0; \1, \1).
\end{equation}
\end{prop}

\begin{proof}
We make the change of variables
$\varphi_x \mapsto (1 + z_0)^{1/2} \varphi_x$ (with $\varphi = \phi, \bar\phi, \psi, \bar\psi$)
in \eqref{e:Grep-pos-bis}, and obtain
\begin{align}
    G_{x,N}(\gcc, \gamma, \nu)
    &=  (1+z_0)
    \int
    e^{-\sum_{x\in\Lambda}
    \left(
    g_0 \tau_x^2 + \gamma_0 |\nabla \tau_x|^2
    + \nu (1+z_0) \tau_x + (1+z_0)\tau_{\Delta,x}\right)} \bar\phi_a \phi_b
    .
    \label{e:Grep-pos}
\end{align}
Then, for any $m^2 \in\R$, we have
\begin{equation}
\lbeq{GNint}
G_{x,N}(\gcc, \gamma, \nu)
    =
(1 + z_0) \int
e^{-\sum_{x\in\Lambda} (\tau_{\Delta,x} + m^2 \tau_x)}
Z_0 \bar\phi_a \phi_b
\end{equation}
($m^2$ simply cancels with $\nu_0$ on the right-hand side).
We use this with $m^2>0$, so that the inverse matrix $C=(-\Delta+m^2)^{-1}$ exists.
By symmetry of the matrix $\Delta$, \refeq{action} gives
\begin{equation}
\label{e:SAtauDelta}
S_{(-\Delta+m^2)}
=
\sum_{x\in\Lambda} \left( \tau_{\Delta,x}
+ m^2  \tau_x \right).
\end{equation}
Then \eqref{e:GG2} follows from \refeq{GNint}--\eqref{e:SAtauDelta} and \refeq{ExCF}.  Summation
over $b\in \Lambda_N$ gives the formula $\chi_N(\gcc,\gamma,\nu) = (1+z_0)\sum_{x\in \Lambda} \Ex_C
(Z_0\phib_0\phi_x)$.  Then \refeq{chichibar}, with \refeq{chibarm}, follows
by an elementary computation as in \cite[Section~\ref{log-sec:ga}]{BBS-saw4-log}.
\end{proof}