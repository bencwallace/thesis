\chapter{Critical initial conditions}
\label{sec:RGflow}

In this chapter, we prove Theorem~\ref{thm:rhatflow}. We begin in
Section~\ref{sec:norms} by introducing
norms used to control the evolution of the coordinates $(I, K)$ and showing
that $K^+_0$ satisfies the inductive assumption required by
Theorem~\ref{thm:step-mr-fv}. In Section~\ref{sec:flow}, we discuss a general
version of this theorem for a parameter $K_0$ that is independent of the
coupling constants. This theorem is then applied with $K_0 = K^+_0$ by solving
a set of implicit equations in Section~\ref{sec:nu0z0c}.

Throughout this chapter, we take $n \ge 1$, drop $n$ from the notation, and
denote fields by $\varphi$. The $n = 0$ case is dealt with in \cite{BSW-saw-sa}.

%%%%%%%%%%%%%%%%%%%%%%%%%%%%%%%%%%%%%%%%%%%%%%%%%%%%%%%%%%%%%%%%%%%%%%%%%%%%%%%
%%%%%%%%%%%%%%%%%%%%%%%%%%%%%%%%%%%%%%%%%%%%%%%%%%%%%%%%%%%%%%%%%%%%%%%%%%%%%%%

\section{Initial coordinates for the renormalisation group}
\label{sec:K0bd}

We establish norm estimates on $K^{\pm}_0$ in Sections~\ref{sec:norms}--\ref{sec:KWcal}.
The initial coordinate $K^{\pm}_0$ depends on the coupling constants
$(g_0, \gamma_0, \nu_0, z_0)$ of \eqref{e:gg0} and regularity of $K_0$ as a
function of these variables is shown in Section~\ref{sec:Ksmooth}.

%%%%%%%%%%%%%%%%%%%%%%%%%%%%%%%%%%%%%%%%%%%%%%%%%%%%%%%%%%%%%%%%%%%%%%%%%%%%%%%

\subsection{Norms}
\label{sec:norms}

We need several properties of the $T_\phi$ semi-norm, whose proofs can be found in
\cite{BS-rg-norm}. We have already mentioned the product property in \eqref{e:prod}.
% First, there is the important \emph{product property}
% \cite[Proposition~\ref{norm-prop:prod}]{BS-rg-norm}
% \begin{equation}
% \label{e:prod}
% \|F G\|_{T_\phi} \leq \|F\|_{T_\phi} \|G\|_{T_\phi}.
% \end{equation}
An immediate consequence is that $\|e^{-F}\|_{T_\phi} \leq e^{\|F\|_{T_\phi}}$.
This is improved in \cite[Proposition~\ref{norm-prop:eK}]{BS-rg-norm},
which states that (recall that $F_\varnothing$ denotes the $0$-degree part of $F$)
\begin{equation}
\label{e:eK}
\|e^{-F}\|_{T_\phi} \leq e^{-2 {\rm Re} F_\varnothing(\phi) + \|F\|_{T_\phi}}.
\end{equation}

Each of the two choices $\varphi = \phi, \bar\phi$
can be viewed as a test function supported on sequences with
$|\vec x| = 1$ and $|\vec y| = 0$
and satisfying $\varphi_{\bar x} = \bar\varphi_x$.
In particular, $\|\phi\|_\Phi$ is defined as the norm of a test function.
We use \cite[Proposition~\ref{norm-prop:T0K}]{BS-rg-norm},
which states that if $F \in \Ncal$ is a polynomial in $\phi,\phib,\psi,\psib$ of
total degree $A \leq p_\Ncal$, then
\begin{equation}
\label{e:T0K}
\|F\|_{T_\phi} \leq \|F\|_{T_0} (1 + \|\phi\|_\Phi)^A.
\end{equation}

We write $x^\Box = \{y: |y-x|_\infty \le 2^d-1\}$,
where $|x|_\infty = \max\{|x_i| : 1 \le i \le d\}$
(this is the scale-0 version
of \cite[\eqref{IE-e:ssn}]{BS-rg-IE} for a single point).
The $\Phi_x \equiv \Phi(x^\square)$ norm of $\phi \in \C^\Lambda$ is defined by
\begin{equation}
\|\phi\|_{\Phi_x}
  =
\inf
\left\{
  \|\phi - f\|_\Phi : f \in \C^\Lambda \text{ such that } f_y = 0 \;\forall y \in x^\square
\right\}
.
\end{equation}
By taking the infimum in \eqref{e:T0K} over all possible
re-definitions of $\phi_y$ for $y \notin x^\square$, we get
\begin{equation}
\label{e:T0Kx}
\|F\|_{T_\phi}
  \leq
\|F\|_{T_0} (1 + \|\phi\|_{\Phi_x})^A
\end{equation}
when $F \in \Ncal(x^\square)$.

We need two choices of the parameter $\h_0$ (for both choices, $\h_0 \ge 1$):
either $\h_0 = \ell_0$, an $L$-dependent constant;
or $\h_0 = h_0 = k_0 \ggen_0^{-1/4}$, where $k_0$ is a small constant and
$\ggen_0$ is a constant which must be chosen small depending on $L$.
Some discussion of these constants occurs in the
proof of Proposition~\ref{prop:K0bd}.
In \cite{BS-rg-IE}, two \emph{regulators} are defined.
At scale $0$, these are given by
\begin{equation}
\lbeq{regdef}
G_0(x, \phi)
  = e^{\|\phi\|^2_{\Phi_x(\ell_0)}},
  \qquad
\tilde G_0(x, \phi)
  =
e^{\frac{1}{2} \|\phi\|^2_{\tilde\Phi_x(\ell_0)}}.
\end{equation}
The $\tilde \Phi_x$ norm in the definition of $\tilde G_0$,
is defined in \cite[\eqref{IE-e:Phitilnorm}]{BS-rg-IE};
it is a modification of the $\Phi_x$ norm that is invariant under shifts by
linear test functions.  Its specific properties do not play a direct role  in this paper.
Two regulator norms are defined for $F \in \Ncal(x^\square)$ by
\begin{equation}
\lbeq{reg0}
    \|F\|_{G_0} = \sup_{\phi\in\C^\Lambda} \frac{\|F\|_{T_\phi(\ell_0)}}{G_0(x,\phi)}
    , \quad
    \|F\|_{\tilde{G}^{\sf t}_0} = \sup_{\phi\in\C^\Lambda} \frac{\|F\|_{T_\phi(h_0)}}{\tilde{G}^{\sf t}_0(x,\phi)}
    ,
\end{equation}
where ${\sf t} \in (0, 1]$ is a constant power.

%%%%%%%%%%%%%%%%%%%%%%%%%%%%%%%%%%%%%%%%%%%%%%%%%%%%%%%%%%%%%%%%%%%%%%%%%%%%%%%%

\subsection{Bounds on \texorpdfstring{$K_0$}{K0}}
\label{sec:K0bds}

\todo{Do this with $n \ge 1$.}

The main estimate on $K^\pm_{0,x}$ is given by the following proposition.
Consistent with \cite[\eqref{IE-e:DV1-bis}]{BS-rg-IE}, we
fix a large constant $C_\DV$ and define
\begin{equation}
\label{e:DV0}
    \DV_0 = \DV_0(\ggen_0) = \{(g,\nu,z) \in \R^3 : C_{\DV}^{-1}\ggen_0 < g < C_{\DV}\ggen_0,
    \; |\nu|,|z| < C_{\DV}\ggen_0\}.
\end{equation}

\begin{prop}
\label{prop:K0bd}
Suppose that $V^\pm_0 \in \DV_0$, with $\ggen_0$ sufficiently small.
If $|\gamma_0| \leq  \ggen_0$, then
(with constants that may depend on $L$)
\begin{equation}
\lbeq{K0bds}
\|K^\pm_{0,x}\|_{G_0} = O(|\gamma_0|),
\quad
\|K^\pm_{0,x}\|_{\tilde G_0} = O(|\gamma_0|/g_0),
\end{equation}
where the bounds on $K^+$ and $K^-$ hold for $\gamma_0 \geq 0$
and $\gamma_0 < 0$, respectively.
\end{prop}


The form of the estimates \refeq{K0bds} can be anticipated from the definition of
$K_0^\pm$.
% in \refeq{Kpm}.
The upper bound arises from the small size of
$e^{-|\gamma_0|U_x^\pm}-1$.  For small fields, hence small $U_x^\pm$, this is of order $|\gamma_0|$,
as reflected by the $G_0$ norm estimate of \eqref{e:K0bds}.
For large fields, namely fields of size $|\phi| \approx \ggen_{0}^{-1/4}$, the difference
$e^{-|\gamma_0|U_x^\pm}-1$ is roughly of size $|\gamma_0|\,|\phi|^4 \approx |\gamma_0|/\ggen_0$.
This effect is measured by the $\tilde G_0$ norm.

Before proving the proposition, we
write
% \refeq{Kpm} for a singleton as
\begin{equation}
K^\pm_{0,x} = I^\pm_{0,x} J^\pm_x
  \label{e:KIJ},
\end{equation}
where, by the fundamental theorem of calculus,
\begin{align}
    I^\pm_{0,x} &= e^{-V^\pm_{0,x}} \\
    J^\pm_x
    &= e^{-|\gamma_0|U^\pm_x} - 1
    = - \int_0^{1} |\gamma_0| U^\pm_x e^{-t |\gamma_0| U^\pm_x} \; dt.
\label{e:J}
\end{align}
% As in \eqref{e:Kpm}, the $+$ versions of \eqref{e:KIJ}--\eqref{e:J} hold
% only for $\gamma_0 \geq 0$ and the $-$ versions only for $\gamma_0 < 0$.

Let $F \in \Ncal(x^\square)$ be a polynomial of degree at most $p_\Ncal$.
Then the stability estimates \cite[\eqref{IE-e:Iupper-a}--\eqref{IE-e:Iupper-b}]{BS-rg-IE}
imply that there exists $c_3 > 0$ and, for any $c_1 \geq 0$,
there exist positive constants $C, c_2$ such that
if $V_0^\pm \in \DV_0$ then
\begin{equation}
\label{e:Iupper}
\|I^\pm_{0,x} F\|_{T_\phi(\h_0)}
  \leq
C \|F\|_{T_0(\h_0)}
\begin{cases}
  e^{c_3 g_0 \left(1 + \|\phi\|^2_{\Phi_x(\ell_0)}\right)}
    & \h_0 = \ell_0 \\
  e^{-c_1 k_0^4 \|\phi\|^2_{\Phi_x(h_0)}} e^{c_2 k_0^4 \|\phi\|^2_{\tilde\Phi_x(\ell_0)}}
    & \h_0 = h_0.
\end{cases}
\end{equation}
This essentially reduces our task to estimating $J^\pm_x$.
The next lemma is an ingredient for this.

\begin{lemma}
\label{lem:FFnull-loc}
There is a universal constant $\tilde C$ such that
\begin{equation}
\label{e:FFnull}
\|U^\pm_x\|_{T_\phi(\h_0)}
  \leq
2 U^\pm_{\varnothing,x} + \tilde C \h_0^4 (1 + \|\phi\|^2_{\Phi_x(\h_0)}),
\end{equation}
where $U^\pm_\varnothing$ is the 0-degree part of $U^\pm$.
\end{lemma}

\begin{proof}
Let
\begin{equation}
M^+ = M^+_e = (\nabla^e \tau_x)^2,
\quad
M^- = M^-_e = 2 \tau_x \tau_{x+e},
\end{equation}
so that $U^\pm_x = \sum_{e\in\Ucal} M^\pm_e$.
It suffices to prove \eqref{e:FFnull} with $U^\pm_x$ replaced by $M^\pm$
(on both sides of the equation).
In addition, we can replace the $\Phi_x$ norm by the $\Phi$ norm;
the bound with the $\Phi_x$ norm then follows in the same way that \eqref{e:T0Kx} is a consequence of \eqref{e:T0K},
since $M^\pm \in \Ncal(x^\Box)$.

By definition of $\tau_x$,
\begin{equation}
M^\pm = M^\pm_{\varnothing} + R^\pm,
\end{equation}
where
\begin{alignat}{2}
&M^+_{\varnothing} = (\nabla^e |\phi_x|^2)^2,
  \quad
&&R^+ = 2 (\nabla^e |\phi_x|^2) \nabla^e (\psi_x\psib_x),
  \\
&M^-_\varnothing = 2 |\phi_x|^2 |\phi_{x+e}|^2,
  \quad
&&R^- = 2 (|\phi_x|^2 \psi_{x+e}\bar\psi_{x+e}
+ \psi_x\bar\psi_x |\phi_{x+e}|^2 + \psi_x\bar\psi_x\psi_{x+e}\bar\psi_{x+e}).
\end{alignat}
Thus, $\|M^\pm\|_{T_\phi} \leq \|M^\pm_{\varnothing}\|_{T_\phi} + \|R^\pm\|_{T_\phi}$.
A straightforward computation shows that
\begin{equation}
\label{e:Rpm-bound}
\|R^\pm\|_{T_\phi} = O(\h_0^4 (1 + \|\phi\|_\Phi)^2).
\end{equation}

By definition of the $T_\phi$ semi-norm,
\begin{equation}
\label{e:nabla-phi-sq-bd}
\|\nabla^e |\phi_x|^2\|_{T_\phi}
  \le
\nabla^e |\phi_x|^2 + 2 \h_0 (|\phi_x| + |\phi_{x+e}|) + 2 \h_0^2.
\end{equation}
Together with \eqref{e:Rpm-bound}, the product property,
and \eqref{e:testfcnbd}, this implies that
\begin{equation}
\|M^+\|_{T_\phi}
  \le
M^+_\varnothing
  + 2 |\nabla^e |\phi_x|^2| (2 \h_0 (|\phi_x| + |\phi_{x+e}|))
  + O(\h_0^4) (1 + \|\phi\|^2_\Phi).
\end{equation}
By the inequality
\begin{equation}
\label{e:young-ineq}
2|ab| \le |a|^2 + |b|^2
\end{equation}
and another application of \eqref{e:testfcnbd},
\begin{equation}
2 |\nabla^e |\phi_x|^2| (2 \h_0 (|\phi_x| + |\phi_{x+e}|))
  \le
M^+_\varnothing + O(\h_0^2 \|\phi\|^2_\Phi),
\end{equation}
and the bound on $M^+$ follows.

For the bound on $M^-$, we use the identity
\begin{equation}
\label{e:taunorm}
\|\tau_x\|_{T_\phi}
  =
(|\phi_x| + \h_0)^2 + \h_0^2
\end{equation}
from \cite[\eqref{norm-e:taunorm}]{BS-rg-norm}.
By the product property and \eqref{e:testfcnbd}, this implies that
\begin{equation}
\|M^-\|_{T_\phi}
  \le
2 |\phi_x|^2 |\phi_{x+e}|^2
  +
2 (|\phi_x| |\phi_{x+e}|) (2 \h_0 (|\phi_{x+e}| + |\phi_x|))
  +
O(\h_0^4) (1 + \|\phi\|^2_\Phi).
\end{equation}
Another application of \eqref{e:young-ineq} and \eqref{e:testfcnbd} gives
\begin{equation}
2 (|\phi_x| |\phi_{x+e}|) (2 \h_0 (|\phi_{x+e}| + |\phi_x|))
  \le
|\phi_x|^2 |\phi_{x+e}|^2 + O(\h_0^2 \|\phi\|^2_\Phi),
\end{equation}
and the proof is complete.
\end{proof}

An immediate consequence of Lemma~\ref{lem:FFnull-loc}, using \eqref{e:eK},
is that for any $s \ge 0$,
\begin{equation}
\label{e:Itilbd}
\|e^{-s U^\pm_x}\|_{T_\phi(\h_0)} \leq e^{\tilde C s \h_0^4 (1 + \|\phi\|^2_{\Phi_x(\h_0)})}.
\end{equation}

\begin{proof}[Proof of Proposition~\ref{prop:K0bd}]
According to the definition of the
regulator norms in \refeq{regdef}--\refeq{reg0},
it suffices to prove that, under the hypothesis on $\gamma_0$,
\begin{equation}
\label{e:K0bd}
  \|K^\pm_{0,x}\|_{T_\phi(\h_0)} = O(|\gamma_0| \h_0^4)
  \begin{cases}
  e^{\|\phi\|_{\Phi_x}^2} & (\h_0=\ell_0)
  \\
  e^{\frac{{\sf t}}{2} \|\phi\|_{\tilde\Phi}} & (\h_0=h_0).
  \end{cases}
\end{equation}
For $t \in [0,1]$, let $\tilde I^\pm_x(t) = e^{-t |\gamma_0| U^\pm_x}$.
By \eqref{e:KIJ}, \eqref{e:J}, and the product property,
\begin{align}
\label{e:K0x-est}
    \|K^\pm_{0,x}\|_{T_\phi(\h_0)}
    & \le |\gamma_0| \|I^\pm_{0,x} U^\pm_x\|_{T_\phi(\h_0)}
    \sup_{t\in [0, 1]} \|\tilde I^\pm_{x}(t)\|_{T_\phi(\h_0)}.
\end{align}
By \refeq{Iupper} and Lemma~\ref{lem:FFnull-loc},
there exists $c_3 > 0$, and, for any $c_1 \geq 0$ there exists $c_2 > 0$, such that
\begin{equation}
\label{e:Iupper-bis}
\|I^\pm_{0,x} U^\pm_x\|_{T_\phi(\h_0)}
  \leq
O(\h_0^4)
\begin{cases}
  e^{c_3 g_0  \|\phi\|^2_{\Phi_x(\ell_0)}}
    & \h_0 = \ell_0 \\
  e^{-c_1 k_0^4 \|\phi\|^2_{\Phi_x(h_0)}} e^{c_2 k_0^4 \|\phi\|^2_{\tilde\Phi_x(\ell_0)}}
    & \h_0 = h_0.
\end{cases}
\end{equation}
The constant in $O(|\gamma_0| \h_0^4)$ may depend on $c_1$,
but this is unimportant.
Also, by \eqref{e:Itilbd},
\begin{equation}
\sup_{t\in[0,1]} \|\tilde I_{x}^\pm(t) \|_{T_\phi(\h_0)}
  \le
e^{\tilde C |\gamma_0| \h_0^4 (1+\|\phi\|^2_{\Phi_x(\h_0)})}.
\end{equation}

Thus, for $\h_0=\ell_0$,
the total exponent in our estimate for the right-hand side of \refeq{K0x-est}
is
\begin{equation}
    \tilde C |\gamma_0| \ell_0^4
       +(c_3 g_0 + \tilde C |\gamma_0| \ell_0^4) \|\phi\|^2_{\Phi_x(\ell_0)}
     .
\end{equation}
This gives the $\h_0=\ell_0$ version of \refeq{K0bd} provided that
$g_0$ is small and $|\gamma_0|$ is small depending on $L$.

For $\h_0=h_0$, the total exponent in our estimate for the right-hand side of \refeq{K0x-est}
is
\begin{equation}
    \tilde C |\gamma_0| k_0^4 \ggen_0^{-1}
        + (\tilde C |\gamma_0| k_0^4 \ggen_0^{-1} - c_1 k_0^4) \|\phi\|^2_{\Phi_x(h_0)}
        + c_2 k_0^4 \|\phi\|^2_{\tilde\Phi_x(\ell_0)}.
\end{equation}
This gives the $\h_0=h_0$ version of \refeq{K0bd} provided that
$|\gamma_0| \le \ggen_0$, $c_1\ge \tilde C$, and $c_2 k_0^4 \le {\sf t}/2$.

All the provisos are satisfied
if we choose
$c_1 \ge \tilde C$,
$k_0$ small depending on $c_1$
and $\ggen_0$ small.
\end{proof}


\begin{rk}
By a small modification to the proof of Proposition~\ref{prop:K0bd},
it can be shown that if $M_x \in \Ncal(x^\square)$ is a monomial of
degree $r \le p_\Ncal -4$ (so that $M_xU_x^\pm$ has degree at most $p_\Ncal$), then
\begin{equation}
\label{e:K0bd-gen}
\|M_x K^\pm_{0,x}\|_{\Gcal_0} = O(|\gamma_0| \h_0^{4+r}).
\end{equation}
\end{rk}

%%%%%%%%%%%%%%%%%%%%%%%%%%%%%%%%%%%%%%%%%%%%%%%%%%%%%%%%%%%%%%%%%%%%%%%%%%%%%%%

\subsection{Unified bound on \texorpdfstring{$K_0$}{K0}}
\label{sec:KWcal}

The results of \cite{BS-rg-step,BBS-rg-flow} are formulated in a sequence of spaces $\Wcal_j$ that
enable the combination of small-field and large-field estimates into a single norm estimate.
In this section, we recast the result of Proposition~\ref{prop:K0bd} to see that $K_0^\pm$
fits into this formulation.

We restrict attention in this section to the $\Wcal_0$ norm,
whose definition is recalled below.
This requires several preliminaries.
Let $\Pcal_0 = \Pcal_0(\Lambda)$ denote the collection of subsets of vertices in $\Lambda$.
We refer to the elements of $\Pcal_0$ as \emph{polymers}.
We call a nonempty polymer $X\in \Pcal_0$ \emph{connected}
if for any $x, x' \in X$, there is a sequence
$x = x_0, \ldots, x_n = x' \in X$ such that
$|x_{i+1} - x_i|_\infty = 1$ for $i = 0, \ldots, n - 1$.
Let $\Ccal_0$ denote the set of connected polymers.
The \emph{small set neighbourhood} $X^\Box$ of $X\in\Pcal_0$ is defined by
\begin{equation}
    X^\Box =
    \{y \in \Lambda : \exists x \in \Lambda \; \text{such that}\; |y-x|_\infty \le 2^d\}.
\end{equation}
We extend the definitions of the regulators $\Gcal_0 = G_0, \tilde G_0^{\sf t}$,
defined in \refeq{regdef}, by setting
\begin{equation} \label{e:Gcalprod}
\Gcal_0(X, \phi) = \prod_{x\in X} \Gcal_0(x, \phi),
\end{equation}
and extend the definitions \refeq{reg0} to define norms, for $F \in \Ncal(X^\Box)$, by
\begin{equation}
\lbeq{reg0X}
    \|F\|_{G_0} = \sup_{\phi\in\C^\Lambda} \frac{\|F\|_{T_\phi(\ell_0)}}{G_0(X,\phi)}
    , \quad
    \|F\|_{\tilde{G}^{\sf t}_0} = \sup_{\phi\in\C^\Lambda} \frac{\|F\|_{T_\phi(h_0)}}{\tilde{G}^{\sf t}_0(X,\phi)}
    .
\end{equation}
It follows from the product property of the $T_\phi$ norm that these norms obey the product property
\begin{equation}
    \|F_1F_2\|_{\Gcal_0} \le   \|F_1\|_{\Gcal_0} \|F_2\|_{\Gcal_0}
    \quad \text{for $F_i\in \Ncal(X_i^\Box)$ with $X_1 \cap X_2=\varnothing$.}
\end{equation}

Given a map $K: \Pcal_0 \to \Ncal$ with the property that $K(X) \in \Ncal(X^\Box)$
for all $X \in \Pcal_0$,
we define the $\Fcal_0(\Gcal)$ norms (for $\Gcal = G, \tilde G$) by
\begin{align}
\|K\|_{\Fcal_0(G)}        &= \sup_{X\in\Ccal_0} \ggen_0^{-f_0(a, X)} \|K(X)\|_{G_0} \\
\|K\|_{\Fcal_0(\tilde G)} &= \sup_{X\in\Ccal_0}
\ggen_0^{-f_0(a, X)} \|K(X)\|_{\tilde G_0^{\sf t}},
\end{align}
with
\begin{equation}
    \label{e:f0def}
    f_0 (\amain, X)
    =
    \amain (|X|-2^d)_+
    =
    \begin{cases}
    a (|X| - 2^d)
    & \text{if } |X| > 2^d   \\
    0
    & \text{otherwise}.
    \end{cases}
\end{equation}
Here $a$ is a small constant;  its value is discussed below \cite[\eqref{step-e:T0dom}]{BS-rg-step}.
The $\Wcal_0$ norm is then defined by
\begin{align}
\label{e:9Kcalnorm}
\|K\|_{\Wcal_0}
  &=
  \max
  \Big\{
  \|K \|_{\Fcal_0(G)},\,
  \ggen_0^{9/4}
  \|K \|_{\Fcal_0(\tilde{G})}
  \Big\}.
\end{align}
Since this definition depends on $\ggen_0$ and the
volume $\Lambda$, we sometimes write $\Wcal_0 = \Wcal_0(\ggen_0, \Lambda)$.
The following proposition uses Proposition~\ref{prop:K0bd} to obtain a bound on the $\Wcal_0$ norm
of the map $K_0^\pm : \Pcal_0 \to \Ncal$ defined by
\begin{equation}
    K_0^\pm(X) = \prod_{x \in X} K_{0,x}^\pm \qquad (X \in \Pcal_0)
    .
\end{equation}

\begin{prop}
\label{prop:KWcal}
If $V_0^\pm \in \DV_0$ with $\ggen_0$ sufficiently small
(depending on $L$), and if $|\gamma_0| \le O(\ggen_0^{1+a'})$
for some $a' >a$,
then $\|K_0^\pm\|_{\Wcal_0} \le O(|\gamma_0|)$,
where all constants may depend on $L$.
\end{prop}

\begin{proof}
Let $X$ be a connected polymer in $\Pcal_0$.
By the product property and Proposition~\ref{prop:K0bd},
\begin{align}
\lbeq{K0prod}
    \|K_0^\pm(X)\|_{\Gcal_0} \le (c|\gamma_0|\h_0^4)^{|X|}
    &=
    (c|\gamma_0|\h_0^4)^{|X|\wedge 2^d} (c|\gamma_0|\h_0^4)^{(|X|-2^d)_+}.
\end{align}
For $\Gcal_0=G_0$, we use $\h_0=\ell_0$,
$(c|\gamma_0|\h_0^4)^{|X|\wedge 2^d}\le O(|\gamma_0|)$, and
\begin{equation}
    (c|\gamma_0|\h_0^4)^{(|X|-2^d)_+} \le (c' \ggen_0)^{(1+a')(|X|-2^d)_+} \le \ggen_0^{f_0(a,X)}.
\end{equation}
For $\Gcal_0=\tilde G_0$, we use $\h_0=h_0 = O(\ggen_0^{-1/4})$ and, since $a'>a$,
\begin{equation}
    (c|\gamma_0|\h_0^4)^{(|X|-2^d)_+} \le (c' \ggen_0)^{a'(|X|-2^d)_+} \le \ggen_0^{f_0(a,X)}.
\end{equation}
Since $|\gamma_0| \le \ggen_0$, it follows from \refeq{K0prod} that
\begin{equation}
    \ggen_0^{9/4}   \|K_0^\pm \|_{\Fcal_0(\tilde{G})}
    \le
    \ggen_0^{9/4}O(|\gamma_0| \ggen_0^{-1})
    \le |\gamma_0|,
\end{equation}
and the proof is complete.
\end{proof}

The above discussion is based on norms in the setting of the torus $\Lambda$.
As in \cite{BS-rg-step}, a version on the infinite lattice $\Zd$ is also required.
This can be done in exactly the same manner,
by defining
the polymers $\Pcal_0 = \Pcal_0(\Zd)$
to be the collection
of subsets of $\Zd$, with $K_0^\pm(X)$ defined for subsets of $\Zd$ by
$\prod_{x \in X} K_{0,x}^\pm$.
The $\Wcal_0 = \Wcal_0(\ggen_0, \Zd)$ norm (in infinite volume)
can be defined analogously to \eqref{e:9Kcalnorm}.
The hypotheses and conclusion of Proposition~\ref{prop:KWcal} remain the same
in the setting of $\Zd$.

%%%%%%%%%%%%%%%%%%%%%%%%%%%%%%%%%%%%%%%%%%%%%%%%%%%%%%%%%%%%%%%%%%%%%%%%%%%%%%%

\subsection{Smoothness of \texorpdfstring{$K_0$}{K0}}
\label{sec:Ksmooth}

Let $\Ccal_0(\Z^d) \subset \Pcal_0(\Z^d)$ be the set of connected polymers.
By definition, a connected polymer is nonempty.
Given $\ggen_0>0$, let
$\Wcal^*_0(\ggen_0, \Zd)$ denote the space of maps
$F :\Ccal_0(\Zd) \to \Ncal$,
with $F(X) \in \Ncal(X^\Box)$ and $\|F\|_{\Wcal_0(\ggen_0, \Zd)} < \infty$.
Addition in this space is defined by $(F_1+F_2)(X)=F_1(X)+F_2(X)$.
We extend any $F :\Ccal_0(\Zd) \to \Ncal$ to $F :\Pcal_0(\Zd) \to \Ncal$
by taking $F(X) = \prod_{Y} F(Y)$ where the product is over the connected components $Y$ of $X$.

Given any map $F : D \to \Wcal^*_0(\ggen_0, \Zd)$ for $D \subset \R$ an open interval,
write $F_X, F^\phi_X : D \to \Ncal(X^\square)$ for the
maps defined by partial evaluation of $F$ at $X$ and at
$(X, \phi)$, respectively. We say $F^\phi_X$ is $C^k$
if all of its coefficients in the decomposition \eqref{e:FinNcal}
are $C^k$ as functions $D \to \R$.

\begin{lemma}
\label{lem:smoothness}
Let $D \subset \R$ be open and $F : D \to \Wcal^*_0(\ggen_0, \Zd)$ be a map.
Suppose that $F^\phi_X$ is $C^2$ for all $X \in \Ccal_0$
and $\phi \in \C^\Lambda$, and define 
$F^{(i)} : D \to \Wcal^*_0(\ggen_0, \Zd)$ by $(F^{(i)}(t))^\phi_X = (F^\phi_X)^{(i)}(t)$ for $i = 1, 2$,
where the right-hand side denotes the (component-wise) $i^{\rm th}$
derivative of $F^\phi_X$.
If $\|F^{(i)}(t)\|_{\Wcal_0} < \infty$ for $i = 1, 2$ and $t \in D$, then 
$F^{(1)}$ is the derivative of $F$.
\end{lemma}

\begin{proof}
For $t, t + s \in D$, define $R(t, s) \in \Wcal_0$ by
\begin{equation}
R^\phi_X(t, s) = F^\phi_X(t + s) - F^\phi_X(t) - s (F^\phi_X)'(t).
\end{equation}
By Taylor's theorem, for any $\phi$ and $X$,
\begin{equation}
R^\phi_X(t, s) = s^2 \int_0^1 (F^\phi_X)''(t + u s) (1 - u) \; du,
\end{equation}
where the integral is taken component-wise.
It follows
that
\begin{equation}
\|R(t, s)\|_{\Wcal_0}
  \le |s|^2 \sup_{u\in[0,1]} \|F''(t+us)\|_{\Wcal_0}
  \le O(|s|^2),
\end{equation}
so $F$ is differentiable 
and its derivative satisfies $(F')^\phi_X = (F^\phi_X)'$.
Continuity of $F'$ follows similarly, since, by the
fundamental theorem of calculus,
\begin{equation}
\|F'(t+s) - F'(t)\|_{\Wcal_0}
  \le
|s| \sup_{u\in[t,t+s]} \|F''(u)\|_{\Wcal_0}
  \le
O(|s|),
\end{equation}
which suffices.
\end{proof}

Consider the map
\begin{equation}
(g_0, \gamma_0, \nu_0, z_0) \mapsto K_0 \in \Wcal^*_0(\ggen_0, \Zd)
\end{equation}
defined by
\begin{equation}
\label{e:K0def}
K_0(g_0, \gamma_0, \nu_0, z_0) =
\begin{cases}
K^+_0(g_0, \gamma_0, \nu_0, z_0)
  & (\gamma_0 \geq 0) \\
K^-_0(g_0, \gamma_0, \nu_0, z_0)
  & (\gamma_0 < 0),
\end{cases}
\end{equation}
for $(g_0, \gamma_0, \nu_0, z_0)$ satisfying the hypotheses
of Proposition~\ref{prop:KWcal}.
The map $K_0$ is in fact analytic away from $\gamma_0 = 0$.
However, we only prove the following, which is what we need later.

\begin{prop}
\label{prop:Ksmooth}
Suppose that $V_0^\pm \in \DV_0$, with $\ggen_0$ sufficiently small
(depending on $L$) and $|\gamma_0| \le O(\ggen_0^{1+a'})$
for some $a' >a$.
The map $K_0(g_0, \gamma_0, \nu_0, z_0)$ is jointly continuous
in its four variables, is
$C^1$ in $(g_0, \nu_0, z_0)$,
and (when $\gamma_0 \ne 0$) is $C^1$ in $(g_0, \gamma_0, \nu_0, z_0)$,
with partial derivatives with respect to $t = g_0$, $\nu_0$, and $z_0$ satisfying
\begin{equation}
\label{e:ddpK}
\|\partial K_0 / \partial t\|_{\Wcal_0} = O(|\gamma_0| \h_0^8).
\end{equation}
Moreover, $K_0$
is left- and right-differentiable in $\gamma_0$ at $\gamma_0 = 0$.
\end{prop}

\begin{proof}
Let $t$ denote any one of the coupling constants $g_0, \gamma_0, \nu_0$ or $z_0$.
We drop the subscript $0$, and let $K(t)$ denote $K_0$ viewed as a function of $t$,
with the remaining coupling constants fixed. Then $K^\phi_X$ is smooth for any $\phi, X$.
If $t$ is $g_0, \nu_0$ or $z_0$, then
\begin{align}
(K^\phi_x)'  &= -M_x(\phi) K^\phi_x, \quad
(K^\phi_x)'' = M_x^2(\phi) K^\phi_x,
\end{align}
where $M_x$ is $\tau_x^2, \tau_x$ or $\tau_{\Delta,x}$, respectively.
The maximal degree of $M_x$ is $4$, so
\eqref{e:K0bd-gen} implies that
\begin{equation}
\label{e:Kprime-bd1}
\|K'_x\|_{\Gcal_0} \le O(|\gamma_0| \h_0^{8}),
  \quad
\|K''_x\|_{\Gcal_0} \le O(|\gamma_0| \h_0^{12}).
\end{equation}

For $t$ denoting $\gamma_0$,
we restrict attention to $\gamma_0 > 0$, and write $U = U^+$
and $V_0 = V^+_0$ (the case $\gamma_0 < 0$ is similar). Then
\begin{equation}
\label{e:dKdgamma0}
(K^\phi_x)'  = -U_x(\phi) e^{-V_x(\phi) - \gamma_0 U_x(\phi)}, \quad
(K^\phi_x)'' = U_x^2(\phi) e^{-V_x(\phi) - \gamma_0 U_x(\phi)},
\end{equation}
and \eqref{e:Iupper} and \eqref{e:Itilbd} imply that
\begin{equation}
\label{e:Kprime-bd2}
\|K'_x\|_{\Gcal_0} \le O(\h_0^4),
  \quad
\|K''_x\|_{\Gcal_0} \le O(\h_0^8).
\end{equation}

By definition, $K_X = \prod_{x \in X} K_x$, so, for derivatives with respect to any one
of the four variables (with $\gamma_0 \neq 0$ when differentiating with respect to $\gamma_0$),
\begin{equation}
\label{e:KXprime}
(K^\phi_X)'  = \sum_{x \in X} (K^\phi_x)' K^\phi_{X \setminus x}, \quad
(K^\phi_X)'' = \sum_{x \in X} ((K^\phi_x)'' K^\phi_{X \setminus x} + (K^\phi_x)' (K^\phi_{X \setminus x})').
\end{equation}
Thus, by the product property, \eqref{e:Kprime-bd1}, and Proposition~\ref{prop:K0bd},
\begin{equation}
\|K'_X\|_{\Gcal_0}
  \le
O(|X|) |\gamma_0| \h_0^8 (|\gamma_0| \h_0^4)^{|X|-1}.
\end{equation}
when differentiating with respect to $g_0$, $\nu_0$, or $z_0$.
The bound \eqref{e:ddpK} then follows from the hypothesis on $\gamma_0$.
Similarly, using \eqref{e:Kprime-bd2},
\begin{equation}
\|K'_X\|_{\Gcal_0}
  \le
O(|X|) \h_0^4 (|\gamma_0| \h_0^4)^{|X|-1}
\end{equation}
when differentiating with respect to $\gamma_0$ away from $\gamma_0 = 0$.
In both cases, we have
\begin{equation}
\|K''_X\|_{\Gcal_0}
  \le
O(|X|^2) \h_0^8 (|\gamma_0| \h_0^4)^{(|X|-2) \wedge 0}.
\end{equation}
Thus, by Lemma~\ref{lem:smoothness}, $K$ is $C^1$ in any of its variables.
Therefore, $K$ is $C^1$ in $(g_0, \nu_0, z_0)$ on the whole domain and in all the variables when $\gamma_0 \ne 0$.

To show right-continuity in $\gamma_0$ at $\gamma_0 = 0$,
fix $(g_0, \nu_0, z_0)$ and define $F \in \Wcal^*_0$ by
\begin{equation}
F(X) =
\begin{cases}
  -U_x e^{-V_{0,x}}
    & X = \{ x \} \\
  0 & |X| > 1,
\end{cases}
\end{equation}
where $U_x, V_{0,x}$ are defined above.
Let $K'(\gamma_0)$ denote the $\gamma_0$ derivative of $K$ evaluated at $\gamma_0 > 0$.  Then
\eqref{e:dKdgamma0} and \eqref{e:KXprime} imply that
\begin{equation}
F(X) - K'_X(\gamma_0)
  =
\begin{cases}
  U_x K_x(\gamma_0)
    & X = \{ x \} \\
  \sum_{x \in X} K'_x(\gamma_0) K_{X \setminus x}(\gamma_0)
    & |X| > 1.
\end{cases}
\end{equation}
Thus, by \eqref{e:K0bd-gen}, \eqref{e:Kprime-bd2}, and Proposition~\ref{prop:K0bd},
\begin{equation}
\|F(X) - K'_X(\gamma_0)\|_{\Gcal_0}
  \le
\begin{cases}
  O(\gamma_0 \h_0^8)
    & X = \{ x \} \\
  O(|X|) \h_0^4 (\gamma_0 \h_0^4)^{|X|-1}
    & |X| > 1.
\end{cases}
\end{equation}
It follows that
\begin{equation}
\lim_{\gamma_0\downarrow 0} \|F - K'(\gamma_0)\|_{\Wcal_0} = 0,
\end{equation}
i.e., $F$ is the right-derivative of $K$ in $\gamma_0$ at $\gamma_0 = 0$.
Left-continuity is handled similarly.
\end{proof}

\begin{rk}
\label{rk:DK-base-case}
The bound \eqref{e:ddpK} verifies the condition
\begin{equation}
\|\partial K_0/\partial\nu_0\|_{\Wcal_0} \le O(g_0^3)
\end{equation}
required in the proof of \cite[Lemma~\ref{log-lem:gzmuprime}]{BBS-saw4-log}
(see \cite[\eqref{log-e:induct1}]{BBS-saw4-log}) and needed in Section~\ref{sec:suscept-conc}.
\end{rk}

\section{Renormalisation group flow}
\label{sec:flow}

The following theorem is an extension of \cite[Proposition~\ref{log-prop:flow-flow}]{BBS-saw4-log}
to non-trivial $K_0$. Such an extension is possible,
with only minor modifications to the proof of the $K_0 = \1_\varnothing$ case,
due to the generality allowed by the main result of \cite{BBS-rg-flow}.

The theorem provides, for any $N \ge 1$ and for initial error coordinate $K_0$
in a specified domain, a choice of initial condition $(\nu_0^c,z_0^c)$
for which there exists
a finite-volume renormalisation group flow $(V_j, K_j) \in \domRG_j$ for $0 \le j \le N$.
In order to ensure a degree of consistency amongst the sequences $(V_j, K_j)$, which depend on
the volume $\Lambda_N$, a notion of consistency must be imposed upon the collection of initial
error coordinates $K_{0,\Lambda} \in \Kcal_0(\Lambda)$ for varying $\Lambda$.
Specifically, the family $K_{0,\Lambda}$ is required to satisfy the property $(\Zd)$ of
\cite[Definition~\ref{step-defn:KZd}]{BS-rg-step}.
We refer to any such family as a $\Lambda$-family.
As discussed in \cite[Definition~\ref{step-defn:KZd}]{BS-rg-step},
any $\Lambda$-family
induces an infinite-volume error coordinate $K_{0,\Zd} \in \Kcal_0(\Zd)$ in a natural way.

\begin{theorem}
\label{thm:flow-flow}
Let $d = 4$.
There exists a constant $a_* > 0$ and continuous functions $\nu_0^c, z_0^c$
of $(m^2, g_0, K_0)$, defined for $(m^2, g_0) \in [0, \delta]^2$
(for some $\delta > 0$ sufficiently small) and for any $K_0 \in \Wcal_0(m^2, g_0, \Zd)$
with $\|K_0\|_{\Wcal_0(m^2, g_0, \Zd)} \leq a_* g_0^3$, such that
the following holds for $g_0 > 0$:
if $K_{0,\Lambda} \in \Kcal_0(\Lambda)$ is a $\Lambda$-family
that induces the infinite-volume coordinate $K_0$, and if
\begin{equation}
\label{e:flow-flow-ic}
V_0 = V_0^c(m^2, g_0, K_0) = (g_0, \nu_0^c(m^2,g_0,K_0), z_0^c(m^2,g_0,K_0)),
\end{equation}
then for any $N \in \N$ and $m^2 \in [\delta L^{-2 (N - 1)}, \delta]$,
there exists a sequence $(V_j, K_j) \in \domRG_j(m^2, g_0, \Lambda)$
such that
\begin{equation}
  \label{e:VjKjDj}
  (V_{j+1},K_{j+1}) = (V_{j+1}(V_j, K_j), K_{j+1}(V_j, K_j)) \text{ for all } j < N
\end{equation}
and \eqref{e:IcircKnew} is satisfied.
Moreover, $\nu_0^c,z_0^c$ are continuously differentiable in
$g_0 \in (0, \delta)$ and $K_0 \in B_{\Wcal_0(m^2, g_0, \Lambda)}(a_* g_0^3)$, and
\begin{align}
&\nu_0^c(m^2,0,0) = z_0^c(m^2,0,0) = 0,
\quad
\ddp{\nu_0^c}{g_0} = O(1),
\quad
\label{e:z0est}
\ddp{z_0^c}{g_0} = O(1),
\end{align}
where the estimates above hold uniformly in $m^2 \in [0, \delta]$.
\end{theorem}

\begin{proof}
The proof results from small modifications to the proofs of
\cite[Proposition~\ref{log-prop:flow-flow}]{BBS-saw4-log} and then to
\cite[Proposition~\ref{log-prop:KjNbd}]{BBS-saw4-log},
where (in both cases) we relax the requirement that $K_0 = \1_\varnothing$,
which was chosen in \cite{BBS-saw4-log} due to the fact that
$K_0 = \1_\varnothing$ when $\gamma=0$.
The more general condition that $K_0 \in B_{\Wcal_0(m^2, g_0, \Lambda)}(a_* g_0^3)$
comes from the hypothesis of \cite[Theorem~\ref{flow-thm:flow}]{BBS-rg-flow}
when $(m^2, g_0) = (\mgen^2, \ggen_0)$.
By \cite[Remark~\ref{flow-rk:Nrad}]{BBS-rg-flow}, no major changes to the proof
result from this choice of $K_0$.
The following paragraph outlines
in more detail the modifications to the proof of
\cite[Proposition~\ref{log-prop:flow-flow}]{BBS-saw4-log}.

By \cite[Theorem~\ref{flow-thm:flow}]{BBS-rg-flow} and
\cite[Corollary~\ref{flow-cor:masscont}]{BBS-rg-flow},
for any $(\mgen^2, \ggen_0) \in (0, \delta)^2$ and
$\tilde K_0 \in B_{\Wcal_0(\mgen^2, \ggen_0, \Zd)}(a_* \ggen_0^3)$,
there is a neighbourhood
${\sf N}(\ggen_0, \tilde K_0)$ of $(\ggen_0, \tilde K_0)$
such that for all
$(m^2, g_0, K_0) \in \Igen(\mgen^2) \times {\sf N}(\ggen_0, \tilde K_0)$,
there is an infinite-volume renormalisation group flow
\begin{equation}
(\Vch_j, K_j) = \xch^d_j(\mgen^2, \ggen_0, \tilde K_0; m^2, g_0, K_0)
\end{equation}
in \emph{transformed variables} $(\Vch_j, K_j)$.
The transformed variables are defined in
\cite[Section~\ref{log-sec:trans}]{BBS-saw4-log} and a flow
in the original variables can be recovered from the transformed flow.
The global solution is defined by
$\xch^c_j(m^2, g_0, K_0) = \xch^d_j(m^2, g_0, K_0; m^2, g_0, K_0)$
(or $\xch^c \equiv 0$ if $g_0 = 0$).
By \cite[Remark~\ref{flow-rk:Nrad}]{BBS-rg-flow},
the proof of regularity of $\xch^c$ can proceed as in \cite{BBS-saw4-log}.
The functions $(\nu_0^c, z_0^c)$ are given by the $(\nu_0, z_0)$ components
of $\xch^c_0 = (\Vch_0, K_0) = (V_0, K_0)$.
\end{proof}


\begin{rk}
The proof of \cite[Proposition~\ref{log-prop:flow-flow}]{BBS-saw4-log},
hence of Theorem~\ref{thm:flow-flow},
makes important use of the parameter $\ggen_0$ in order to prove regularity
of the renormalisation group flow in $g_0$. However, once the flow has been
constructed, we can and do set $\ggen_0 = g_0$.
\end{rk}

We wish to apply this theorem with $(\gcc_0, K_0) = (g_0, K^+_0)$.
It is straightforward to verify that $K^\pm_0 \in \Kcal_0$.
For instance, the fact that $K^\pm_0$ is supersymmetric
(which is required of all elements of $\Kcal_0$) follows
from the fact that $K^\pm_{0,x}$ is a function of $\tau_x$
(see \cite[Section~\ref{pt-sec:bulksym}]{BBS-rg-pt} for more on this topic).
It also follows from the definition that
the finite-volume coordinates $K^\pm_{0,\Lambda}$ form a $\Lambda$-family.

Moreover,
by Proposition~\ref{prop:KWcal}, if
$|\gamma_0|$ is sufficiently small (depending on $g_0$; we now take $\ggen_0=g_0$)
then $K_0 = K^\pm_0$ satisfies the bound required by Theorem~\ref{thm:flow-flow}.
However, we cannot apply the theorem immediately with this choice
of $K_0$,
due to the fact that $K^\pm_0$
depends on $(g_0, \nu_0, z_0)$.
We resolve this issue in the next section.

\section{Critical parameters}
\label{sec:nu0z0c}

% For convenience, let
% \begin{equation}
% \lbeq{g0hatdef}
% \hat g_0 = \hat g_0(g_0, \gamma_0) = g_0 + 4 d \gamma_0 \1_{\gamma_0 < 0}.
% \end{equation}
% Thus, $\hat g_0$ is the coefficient of $\tau_x^2$ in $V^+_{0,x}$
% when $\gamma_0 \ge 0$, and in $V^-_{0,x}$ when $\gamma_0 < 0$.
Recall the function $K_0(g_0, \gamma_0, \nu_0, z_0)$
defined in \eqref{e:K0def}.
We wish to initialise the renormalisation group with $(\nu_0, z_0)$ a solution
to the system of equations
\begin{align}
&\nu_0 = \nu_0^c(m^2, \gcc, K_0(g_0, \gamma_0, \nu_0, z_0)), \label{e:mu0c}
\\
&z_0 = z_0^c(m^2, \gcc, K_0(g_0, \gamma_0, \nu_0, z_0)) \label{e:z0c}
.
\end{align}
Such a choice of $(\nu_0, z_0)$ will be critical for $K_0$,
where $K_0$ is itself evaluated at this same choice of $(\nu_0, z_0)$.

When $\gamma_0 = 0$, we get $K_0 = \1_\varnothing$, so $K_0$ no longer depends on $(\nu_0, z_0)$
and this system is solved by $(\nu_0^c(m^2, g_0, 0), z_0^c(m^2, g_0, 0))$
for any (small) $m^2, g_0 \geq 0$.
Local solutions for $\gamma_0 \neq 0$ can then be
constructed using a version of the implicit function theorem from \cite{LS14}
that allows for the continuous but non-smooth behaviour of $K_0$ in $\gamma_0$.
In order to obtain global solutions with certain desired regularity properties
(needed in the next section), we make use of Proposition~\ref{prop:IFT},
which is based on a version of the implicit function theorem from \cite{LS14}.

Recall that $D(\delta, r)$ was defined in \eqref{e:Ddef}.

\begin{prop}
\label{prop:nuzhat}
There exists a continuous positive-definite function $\hat r : [0, \delta] \to [0, \infty)$
and continuous functions
$\hat\nu_0^c, \hat z_0^c \in C^{0,1,\pm}(D(\delta, \hat r))$ such that
the system \eqref{e:mu0c}--\eqref{e:z0c} is solved by $(\nu_0, z_0) = (\hat\nu_0^c, \hat z_0^c)$
whenever $(m^2, g_0, \gamma_0) \in D(\delta, \hat r)$.
Moreover, these functions satisfy the bounds
\begin{equation}
\label{e:hat-est-re}
\hat\nu_0^c = O(g_0),
\quad
\hat z_0^c = O(g_0)
\end{equation}
uniformly in $(m^2, \gamma_0)$.
\end{prop}

\begin{proof}
Let
\begin{equation}
F(m^2, g_0, \gamma_0, \nu_0, z_0)
= (\nu_0, z_0)
  -
  (\nu_0^c(m^2, \gcc_0, K_0),
  z_0^c(m^2, \gcc_0, K_0)
),
\end{equation}
where $K_0 = K_0(g_0, \gamma_0, \nu_0, z_0)$.
Then for $\delta > 0$ small and an appropriate constant $c > 0$ (depending on $a_*$),
$F$ is well-defined on
\begin{equation}
\{ (m^2, g_0, \gamma_0, \nu_0, z_0) : (m^2, \gcc_0, \gamma_0) \in D(\delta, c g_0^3),
|\nu_0|, |z_0| \leq C_\DV g_0 \}.
\end{equation}
Indeed, for $(m^2, g_0, \gamma_0, \nu_0, z_0)$ in this domain,
Proposition~\ref{prop:KWcal} (with $\ggen_0 = g_0$) implies that $(m^2, \gcc_0, K_0)$ is in the domain of
$(\nu_0^c, z_0^c)$.
By Theorem~\ref{thm:flow-flow} and Proposition~\ref{prop:Ksmooth},
$F$ is $C^1$ in $(g_0, \nu_0, z_0)$
and also in $\gamma_0$ away from $\gamma_0 = 0$,
continuous in $m^2$, and has one-sided derivatives in $\gamma_0$ at $\gamma_0 = 0$.

For fixed $(\bar m^2, {\bar g_0}) \in [0, \delta]^2$,
set $({\bar\nu_0}, \bar z_0) = (\nu_0^c(\bar m^2, \bar g_0, 0), z_0^c(\bar m^2, \bar g_0, 0))$
so that
\begin{equation}
F(\bar m^2, \bar g_0, 0, \bar\nu_0, \bar z_0) = (0, 0).
\end{equation}
By \eqref{e:ddpK}, at $(\bar g_0, 0, \bar\nu_0, \bar z_0)$,
\begin{equation}
\frac{\partial K_{0,x}}{\partial\nu_0}
= \frac{\partial K_{0,x}}{\partial z_0} = 0.
\end{equation}
It follows that $D_{\nu_0,z_0} F(\bar m^2, \bar g_0, 0, \bar\nu_0, \bar z_0)$
is the identity map on $\R^2$.
The existence of $\delta, \hat r$ and $\hat\nu_0^c, \hat z_0^c$
follows from Proposition~\ref{prop:IFT} with
$w = m^2, x = g_0, y = \gamma_0, z = (\nu_0, z_0)$,
and with $r_1(g_0) = c g_0^3$, $r_2(g_0) = C_\DV g_0$.

By the fundamental theorem of calculus, for any $0 < a < \gamma_0$,
\begin{equation}
\hat\nu_0^c(m^2, g_0, \gamma_0)
  =
\hat\nu_0^c(m^2, g_0, a)
  +
\int_a^{\gamma_0} \ddp{\hat\nu_0^c}{\gamma_0} (m^2, g_0, t) \; dt.
\end{equation}
Taking the limit $a\downarrow 0$ and using \eqref{e:z0est}, we obtain
\begin{equation}
|\hat\nu_0^c(m^2, g_0, \gamma_0)|
  \leq
O(g_0)
  +
\gamma_0
\sup_{t \in (0, \gamma_0]}
\left|\ddp{\hat\nu_0^c}{\gamma_0}(m^2, g_0, t)\right|.
\end{equation}
The supremum above is bounded by a constant and so
the first estimate of \eqref{e:hat-est-re} for $\gamma_0 \geq 0$
follows from the fact that $|\gamma_0| \leq \hat r(g_0)$
(since $\hat r(g_0)$ can be taken as small as desired).
The case $\gamma_0 < 0$ and the second estimate follow similarly.
\end{proof}

\begin{proof}[Proof of Theorem~\ref{thm:rhatflow}]
By Proposition~\ref{prop:KWcal},
and by taking $\hat r$ smaller if necessary,
$K_0 = K^\pm_0$ satisfies the estimate required by Theorem~\ref{thm:flow-flow}
whenever $(m^2, g_0, \gamma_0) \in D(\delta, \hat r)$. The
existence of the sequence \eqref{e:VjKjDj-hat} then follows from
Theorem~\ref{thm:flow-flow} and Proposition~\ref{prop:nuzhat}.
Although the presence of $\gamma_0$ causes a shift in initial
conditions, the second-order evolution of $V_j$ is still given by the map
$V_\pt$,
% (see \eqref{e:Vflow}),
which is independent of $\gamma_0$.
\end{proof}
