% Please see the ubcthesis documentation for details about the options.
\documentclass[phd,oneside]{ubcthesis}
% \usepackage{helvet}
% \renewcommand{\familydefault}{\sfdefault}

%%%%%%%%%%%%
% PACKAGES %
%%%%%%%%%%%%

%%%% Margins %%%%
% According to
% https://www.grad.ubc.ca/current-students/dissertation-thesis-preparation/page-layout-margins-numbering
% Left: 1.25 inches (32 mm) recommended for binding; 1 inch minimum.
% Right, top, and bottom: 1 inch recommended; 0.75 inches (19 mm) minimum
\usepackage[left=1.25in,right=1in,bottom=1in,top=1in]{geometry}

%%%% Drafthead %%%%
% \def\draft{}
\ifdefined\draft
	\usepackage{drafthead}
	\usepackage[notref]{showkeys}
	\renewcommand*\showkeyslabelformat[1]{\normalfont\tiny\ttfamily(#1)}
\fi

%%%% Line spacing %%%%
% According to
% 	grad.ubc.ca/current-students/dissertation-thesis-preparation/page-layout-margins-numbering
% -lines of text must be 1.5 or double spaced
% -quotations of more than one line can be single-spaced
% -acknowledgements, footnotes, table, figure and illustration captions and the bibliography can be single-spaced,
% 	provided that individual entries are separated by a full space
\usepackage{setspace}
\onehalfspacing

%%%% Other packages %%%%
\usepackage{mathrsfs,mathtools,wrapfig}
\usepackage{}

%%%%%%%%%%
% MACROS %
%%%%%%%%%%

\newcommand{\titlename}{{Renormalisation group analysis of self-interacting walks and spin systems}}
\ifdefined\draft
	\def\macrosH{}						% Use xr-hyper
	% include general packages
\usepackage{amsfonts}
\usepackage{amsmath,amssymb,amsthm}
\usepackage{appendix}
\usepackage{bbm} % used in \newcommand{\1}{\mathbbm{1}}
\usepackage{amsbsy}
\usepackage{enumerate}
\usepackage{cite}
\usepackage{enumerate}

% use hyperref by default (can be overridden by inserting
% \let\macrosH\undefined in macros_local.tex)
% \def\macrosH{}

\InputIfFileExists{./macros_local.tex}{}{}

% page setup
\ifdefined\macrosPa
  \usepackage[textwidth=465pt,textheight=650pt,centering]{geometry} % 11pt
\else\ifdefined\macrosPb
  \usepackage[textwidth=500pt,textheight=650pt,centering]{geometry} % 12pt
\fi\fi

% needed for Springer templates: restore bold math fonts
\ifdefined\macrosS
  \makeatletter
  %% make boldmath and boldsymbol an alias for \pmb{}
  \def\boldmath{\pmb}
  \def\boldsymbol{\pmb}
  \makeatother

  \usepackage{mathptmx}
  % standard mathcal fonts
  \DeclareMathAlphabet{\mathcal}{OMS}{cmsy}{m}{n}
\fi

% do not include graphics for Birkhaeuser
\ifdefined\macrosBirk
\else
\usepackage[dvips]{graphicx}
\fi



% use def99c macros
\ifdefined\macrosSB
\input ../def99c %for LaTeX2e, enables \eq  and \en (!!)
\else
\input def99c %for LaTeX2e, enables \eq  and \en (!!)
\fi

\UseSection   %Necessary to define Numbering scheme for Theorem, etc.
\setcounter{secnumdepth}{3} %Set the depth of sectioning.
\setcounter{tocdepth}{3}    %Set the depth of table of contents.

% define colors
\usepackage[usenames]{color}
\newcommand{\red}{\color{red}}
\newcommand{\blue}{\color{blue}}
\newcommand{\magenta}{\color{magenta}}
\newcommand{\green}{\color{green}}
\newcommand{\cyan}{\color{cyan}}
\definecolor{bw}{RGB}{240, 120, 0}
\definecolor{at}{rgb}{0.0, 0.5, 0.0} % green, darker than the standard green.
\newcommand{\colour}[1]{{\blue #1}}

% define chXX and commentXX
\newcommand{\chrb}[1]{{\magenta #1}}
\newcommand{\chdb}[1]{{\red #1}}
\newcommand{\chblue}[1]{{\blue #1}}
\newcommand{\chgs}[1]{{\blue #1}}
\newcommand{\chgat}[1]{{\color{at} #1}}
\newcommand{\chbw}[1]{{\color{bw} #1}}
\newcommand{\comment}[1] { \begin{quote}
                                { \bf Comment: #1 }
                           \end{quote} }
\newcommand{\commentgs}[1] { \begin{quote}
                                { \bf Comment from GS: #1 }
                           \end{quote} }
\newcommand{\commentdb}[1] { \begin{quote}
                                { \bf Comment from DB: #1 }
                           \end{quote} }
\newcommand{\commentrb}[1] { \begin{quote}
                                { \bf Comment from RB: #1 }
                           \end{quote} }
\newcommand{\commentat}[1] { \begin{quote}
                                { \bf Comment from AT: #1 }
                           \end{quote} }
\newcommand{\commentbw}[1] { \begin{quote}
                                { \bf Comment from BW: #1 }
                           \end{quote} }


% custom abbreviations

\newcommand{\wo}{\colon \!\!}
\newcommand{\lwo}{\colon \!\!}
\newcommand{\rwo}{\! \colon \!\!}
\newcommand{\shift}{\!\!\!\!}
\newcommand{\veee}[1]{|\!|\!|#1|\!|\!|}

\newcommand{\wick}[1]{\lwo#1\rwo}
\newcommand{\Tay}{{\rm Tay}}
\newcommand{\LTold}{\widetilde{\rm loc}  }
\newcommand{\LTbar}{{\rm loc}}
\newcommand{\LTsym}{{\rm loc}}
\newcommand{\LT}{{\rm Loc}  }
\newcommand{\Rel}{{\rm Proj}}
\newcommand{\Irr}{(1-{\rm Proj})}

\newcommand{\DV}{\Dcal}
\newcommand{\DVa}{\alpha}
\newcommand{\Dnull}{\Dcal_\varnothing}
\newcommand{\rhogen}{\tilde{\rho}}
\newcommand{\rD}{r}
\newcommand{\crho}{c_L}
\renewcommand{\to} {\rightarrow}
\renewcommand{\qed}{\hfill\rule{2mm}{2mm}\bigskip}


\newcommand{\sumtwo}[2]{\sum_{ \mbox{ \scriptsize
    $\begin{array}{c}
                        {#1} \\ {#2}
                        \end{array} $ }
    }
}



\newcommand{\R}{\Rbold}
\newcommand{\Z}{\Zbold}
\newcommand{\X}{\mathbb X}
\newcommand{\x}{\mathbb x}
\newcommand{\N}{\Nbold}
\newcommand{\C}{\mathbb{C}}
\newcommand{\volume}{\mathbb{V}}
\newcommand{\Lambdabold}{\boldsymbol{\Lambda}}
\newcommand{\Sigmabold}{\boldsymbol{\Sigma}}
\newcommand{\1}{\mathbbm{1}}
\newcommand{\Cbf}{\boldsymbol{C}}
\newcommand{\cbf}{\boldsymbol{c}}
\newcommand{\Sbf}{\boldsymbol{S}}
\newcommand{\Abf}{\boldsymbol{A}}
\newcommand{\wbf}{\boldsymbol{w}}

\newcommand{\la}{\langle}
\newcommand{\ra}{\rangle}
\newcommand{\supp}{\mathrm{supp}}

\newcommand{\cL}{{\cal L}}
\newcommand{\cP}{{\cal P}}

\newcommand{\nn}{\nonumber}

\newcommand{\smallsup}[1] {{\scriptscriptstyle{({#1}})}}


\newcommand{\FDel}{\lambda}


\newcommand{\alphab}{\bar\alpha}
\newcommand{\varphib}{\bar\varphi}
\newcommand{\phibar}{\bar\varphi}
\newcommand{\psib}{\bar\psi}
\newcommand{\ci}{\underline{i}}

\newcommand{\elll}{\chdb{l}}

\newcommand{\jm}{j_\Omega}
\newcommand{\jmass}{j_{\rm mass}}

\newcommand{\w}{{\sf w}}
\newcommand{\Mext}{M_\mathrm{ext}}



\newcommand{\card}[1]{\text{Card #1}}
\renewcommand{\d}{\,d}
\newcommand{\e}{\mathbf{e}}
\newcommand{\Xst}{x_{\ast}}
\newcommand{\Yst}{y_{\ast}}

\newcommand{\Yb}{\bar{Y}}
\newcommand{\cC}{\mathcal{C}}
\newcommand{\cF}{\mathcal{F}}

\newcommand{\zetab}{\bar{\zeta}}
\newcommand{\etab}{\bar{\eta}}
\newcommand{\Ex}{\mathbb{E}}

\newcommand{\Econst}{\alpha_{\Ebold}}
\newcommand{\Econstg}{\alpha_{G}}
\newcommand{\EGconst}{\alpha}
\newcommand{\Fconst}{f}
\newcommand{\IstabC}{\alpha_{I}}
\newcommand{\ItilstabC}{\alpha_{I}}

\newcommand{\cCov}{{\sf C}}
\newcommand{\chicCov}{{\chi}}
\newcommand{\ellconst}{\mathfrak{c}}


\newcommand{\NGrass}{N_{{\rm Grass}}}

\newcommand{\Qcalnabla}{\Qcal}

\newcommand{\lt}{\ell}

\newcommand{\bubble}{{\sf B}}

\newcommand{\Gtilp}{\gamma}



% SOME NEW COMMANDS.


\newcommand{\side}[1]{\mathrm{side}(#1)}
\newcommand{\degree}[1]{\mathrm{deg(#1)}}
\newcommand{\pair}[1]{\langle #1 \rangle}
\newcommand{\Phizero}{\Phi^{*0}}
\newcommand{\Phipol}{\Pi}
\newcommand{\Phipoltil}{\widetilde{\Pi}}
\newcommand{\diam}[1]{\textrm{diam}(#1)}

\newcommand{\units}{\Ucal}
\newcommand{\monomials}{\Mcal}

\newcommand{\cgam}{\gamma}
\newcommand{\ckap}{c_0}
\newcommand{\cQ}{c_Q}


\newcommand{\Gkappa}{\kappa}
\newcommand{\pT}{p_T}

\newcommand{\concat}{\circ}


\newcommand{\inversefiddle}{3}
\newcommand{\fiddle}{\frac{1}{\inversefiddle}}


\newcommand{\obs}{\mathrm{obs}}
\newcommand{\homog}{\mathrm{hom}}
\newcommand{\Ihom}{I_{\homog ,j+1}}
\newcommand{\Itildehom}{\tilde{I}_{\homog ,j+1}}
\newcommand{\deltaIhom}{\delta I_{\homog ,j}}
\newcommand{\Khom}{K_{\homog ,j}}
\newcommand{\khom}{k_{\homog}}
\newcommand{\Jhom}{J_{\homog ,j+1}}
\newcommand{\Jobs}{J_{\obs ,j+1}}
\newcommand{\Itilde}{\tilde{I}}
\newcommand{\pertconst}{c_{j+1}}

\newcommand{\Phitimes}{\Phi^{\times}}
\newcommand{\Phiprimetimes}{{\Phi'}^{\times}}
\newcommand{\Ttimes}{T}

\newcommand{\decay}{r}
\newcommand{\hldg}{h_{\rm lead}}
\newcommand{\cldg}{c_{\rm lead}}
\newcommand{\hpt}{h_{\rm pt}}
\newcommand{\htil}{{\tilde h}^{(0)}}
\newcommand{\hpttil}{{\tilde h}_{\rm pt}}
\newcommand{\pt}{{\rm pt}}
\newcommand{\Ipt}{I_{\rm pt}}
\newcommand{\Ipttil}{\tilde{I}_{\rm pt}}
\newcommand{\Upt}{U_{\rm pt}}
\newcommand{\Vpt}{V_{\rm pt}}
\newcommand{\gpt}{g_{\mathrm{pt}}}
\newcommand{\nupt}{\nu_{\mathrm{pt}}}
\newcommand{\mupt}{\mu_{\mathrm{pt}}}
\newcommand{\zpt}{z_{\mathrm{pt}}}
\newcommand{\ypt}{y_{\mathrm{pt}}}
\newcommand{\lambdapt}{\lambda_{\mathrm{pt}}}
\newcommand{\lambdaapt}{\lambda^{\pp}_{\mathrm{pt}}}
\newcommand{\lambdabpt}{\lambda^{\qq}_{\mathrm{pt}}}
\newcommand{\lambdaa}{\lambda^{\pp}}
\newcommand{\lambdab}{\lambda^{\qq}}
\newcommand{\qapt}{q^{\pp}_{\mathrm{pt}}}
\newcommand{\qbpt}{q^{\qq}_{\mathrm{pt}}}
\newcommand{\qa}{q^{\pp}}
\newcommand{\qb}{q^{\qq}}
\newcommand{\qpt}{q_{\mathrm{pt}}}
\newcommand{\rpt}{r^{\mathrm{pt}}}

\newcommand{\Vbulk}{U}

\newcommand{\xch}{\check{x}}
\newcommand{\Kch}{\check{K}}
\newcommand{\Vch}{\check{V}}
\newcommand{\Rch}{\check{R}}
\newcommand{\gch}{\check{g}}
\newcommand{\zch}{\check{z}}
\newcommand{\much}{\check{\mu}}
\newcommand{\nuch}{\check{\nu}}
\newcommand{\lambdach}{\check{\lambda}}
\newcommand{\rch}{\check{r}}
\newcommand{\vch}{\check{v}}
\newcommand{\uch}{\check{u}}
\newcommand{\alphach}{\check{\alpha}}
\newcommand{\Vptch}{\check{V}_\pt}

\newcommand{\dq}{\delta q}

\newtheorem{convention}       [theorem] {Convention}

\newcommand{\ccc}[1]{O (#1)}

\newcommand{\wt}{w}
\newcommand{\wta}{w_\alpha}

\newcommand{\two}{\ItilstabC}
\newcommand{\ball}{B}
\newcommand{\h}{\mathfrak{h}}
\ifdefined\macrosSB \else
  \newcommand{\D}{\mathrm{D}}
  \newcommand{\I}{\mathfrak{I}}
\fi
\newcommand{\Ifopt}{\mathfrak{I}_{\rm fopt}}
\newcommand{\Isopt}{\mathfrak{I}_{\rm sopt}}
\newcommand{\quadr}{Q_{2}}
\newcommand{\Q}{Q}
\newcommand{\q}{q}
\newcommand{\F}{F}
\newcommand{\Id}{\mathrm{Id}}
\newcommand{\ev}{\mathrm{ev}}
\newcommand{\rn}{\nabla_0}
\newcommand{\fun}{\Fcal}
\newcommand{\deriv}{\left[\frac{d}{d\alpha} \right]_{0}}
\newcommand{\remainder}{R}
\newcommand{\pert}{\mathrm{pt}}
\newcommand{\norm}[1]{\left\|#1\right\|_{T_{\phi}}}
\newcommand{\z}{z}
\newcommand{\redDroman}{\chdb{D}}
\newcommand{\redDcal}{\Dcal}

\newcommand{\Cdecomp}{C_{\mathrm{decomp}}}

\newcommand{\Lmain}{\Lcal_{\rm lin}}

\newcommand{\Lcallr}{\stackrel{\leftrightarrow}{\Lcal}}

\newcommand{\cpl}{\xi}
\newcommand{\cplbar}{\bar{\cpl}}
\newcommand{\cplhat}{\hat{\cpl}}
\newcommand{\gbar}{\bar{g}}
\renewcommand{\ghat}{\hat{g}}
\newcommand{\ggen}{\tilde{g}}
\newcommand{\sgen}{\tilde{s}}
\newcommand{\chigen}{\tilde{\chi}}
\newcommand{\mgen}{\tilde{m}}
\newcommand{\Iint}{\mathbb{I}}
\newcommand{\Igen}{\tilde{\mathbb{I}}}
\newcommand{\Sgen}{S}
\newcommand{\zbar}{\bar{z}}
\newcommand{\mubar}{\bar{\mu}}
\newcommand{\muhat}{\hat{\mu}}
\newcommand{\stab}{\mathrm{stab}}
\newcommand{\unstab}{\mathrm{unstable}}
\newcommand{\pline}{\underline{p}}
\newcommand{\qline}{\underline{q}}

\newcommand{\Zbar}{\bar{Z}}
\newcommand{\chibar}{\bar{\chi}}


\newcommand{\domRG}{\mathbb{D}}
\newcommand{\domRGch}{\check{\mathbb{D}}}
\newcommand{\domr}{R}

\newcommand{\Lip}{\mathrm{Lip}}

\newcommand{\betamax}{\beta_{\rm max}}


\newcommand{\pp}{a}
\newcommand{\qq}{b}
\newcommand{\sigmaa}{\sigma}
\newcommand{\sigmab}{\bar{\sigma}}
\newcommand{\s}{\mathfrak s}

\newcommand{\half}{\textstyle{\frac 12}}

\newcommand{\f}{f}
\newcommand{\poly}{P}
\newcommand{\ghB}{\eta_{B}}

\newcommand{\extpow}{\mbox{$\bigwedge$}}
\newcommand{\pr}{\Pi}
\newcommand{\ddp}[2]{\frac{\partial #1}{\partial #2}}
\newcommand{\dd}[2]{\frac{d #1}{d #2}}

\newcommand{\LL}{\tilde{L}}
\newcommand{\smp}{\epsilon}

\newcommand{\epP}{\epsilon_{P}}
\newcommand{\epW}{\epsilon_{W}}
\newcommand{\epWdot}{\epsilon_{\dot{W}}}
\newcommand{\epWnull}{\epsilon_{W_\varnothing}}
\newcommand{\epV}{\epsilon_{V}}
\newcommand{\epU}{\epsilon_{U}}
\newcommand{\epVnull}{\epsilon_{V_\varnothing}}
\newcommand{\epVbar}{\epsilon_{g\tau^{2}}}
\newcommand{\epVdot}{\epsilon_{\dot{V}}}
\newcommand{\epVpt}{\epsilon_{V_{\rm pt}}}
\newcommand{\epdV}{\bar{\epsilon}}
\newcommand{\epdVdot}{\epsilon_{\delta \dot{V}}}
\newcommand{\epEV}{\epsilon_{\Ex V}}
\newcommand{\epdVbar}{\bar{\epsilon}_{\delta V}}
\newcommand{\epEVbar}{\bar{\epsilon}_{\Ex V}}
\newcommand{\epK}{\epsilon_{K}}
\newcommand{\epdI}{\epsilon_{\delta I}}
\newcommand{\epI}{\epsilon_{I}}
\newcommand{\epdP}{\epsilon_{\delta P}}
\newcommand{\eplamdot}{\epsilon_{\dot{\lambda}}}

\newcommand{\gh}{\epV}
\newcommand{\ghbar}{\bar{\epsilon}_V}


\newcommand{\nobs}[1]{|#1|_{\text{obs}}}

\newcommand{\spec}{\mathrm{spec}}


\newcommand{\phib}{\bar\phi}



\newcommand{\zldg}{z_{\rm lead}}
\newcommand{\zh}{z_{h}}
\newcommand{\zK}{z_{K}}
\newcommand{\zDelta}{z_{\Delta}}


\newcommand{\McalnowM}{M}
\newcommand{\IcalnowN}{h_{\rm rem}}
\newcommand{\hred}{h_{\rm red}}
\newcommand{\hrem}{h_{\rm rem}}
\newcommand{\species}{\mathbf{s}}
\newcommand{\sgn}{\mathrm{sgn}}

\newcommand{\cc}{\mathbf{cc}}
\newcommand{\ob}{\mathbf{ob}}
\newcommand{\fn}{\mathbf{fn}}

\newcommand{\V}{\tilde{V}}
\newcommand{\Vspace}{\Vcal_{\mathrm{sym}}}
\newcommand{\Vdomain}{\Dcal_0}
\newcommand{\Vin}   {V}
\newcommand{\Vout}  {V_{\mathrm{out}}}
\newcommand{\Kin}   {K_{\mathrm{in}}}
\newcommand{\Uin}   {U_{\mathrm{in}}}
\newcommand{\Kout}  {K_{\mathrm{out}}}
\newcommand{\Kspace}{\Kcal}
\newcommand{\CKspace}{\Ccal\Kspace}
\newcommand{\BKspace}{\Bcal\Kspace}
\newcommand{\Ispace}{\text{$I$-space}}
\newcommand{\Iin}   {I_{\mathrm{in}}}
\newcommand{\Iout}  {I_{\mathrm{out}}}
\newcommand{\kprime}{k'}
\newcommand{\Vtilde}{\tilde{V}}
\newcommand{\Ktilde}{\tilde{K}}
\newcommand{\Kstar} {K^{*}}
\newcommand{\Jprime}{j}
\newcommand{\amain} {a}
\newcommand{\ain}   {a_{\mathrm{in}}}
\newcommand{\aout}  {a_{\mathrm{out}}}
\newcommand{\aahat} {\hat{a}}
\newcommand{\abig}  {\tilde{a}}
\newcommand{\astar} {a^*}
\newcommand{\epsin} {\epsilon_{\mathrm{in}}}
\newcommand{\rball} {r}

\newcommand{\qwt}{\mathfrak{q}}
\newcommand{\tail}{\mathfrak{t}}
\newcommand{\covc}{\mathfrak{c}}
\newcommand{\rgam}{\mathfrak{r}}

\newcommand{\RG}{\text{RG}}

\newcommand{\REF}{{\bf REF}\ }
\newcommand{\subnw}{\subset_{\text{nw}}}

% Some macros by AT

\newcommand{\prob}[1]{P\left\{ #1 \right\}}

\newcommand{\wm}{W}
%\DeclareMathOperator{\wm}{W}
\DeclareMathOperator{\wsaw}{wsaw}
\DeclareMathOperator{\srw}{srw}
\DeclareMathOperator{\Loc}{Loc} %\LT

\newcommand{\nl}{p}

\newcommand{\phifour}{$|\varphi|^4$}


%%%%%%%%%%%%%%%%%%%%%%%%%%%%%%%%%%%%%%%%%%%%%%%%%%%%%%%%%%%%%%%%%%%
%% do not delete: these are useful to make characters stand out during
%% proof-reading

%% how to hilight \Phipol
% \let\temp\Phipol
% \renewcommand{\Phipol}{\chdb{\temp}}

%% how to make the character $G$ red

% \mathchardef \tx \mathcode `G% storing current definition of G in \tx
% \ifnum \mathcode `\G = "8000% checking if we have already changed definition
% \else
% \begingroup
% \catcode `\G=13
% \global \mathchardef G \mathcode `\G
% \endgroup
% \mathcode `\G = "8000% 8000 declares G active in mathmode
% \fi

% \begingroup
% \catcode `\G=13
% \global \def G{\chdb{\tx}}%or make it any other command
% \endgroup
%%%%%%%%%%%%%%%%%%%%%%%%%%%%%%%%%%%%%%%%%%%%%%%%%%%%%%%%%%%%%%%%%%%%%


% hyperref setup -- needs to come at the end since section names
% can involve macros defined above

% use hyperref
\ifdefined\macrosH
  \usepackage{xr-hyper}
  \usepackage[colorlinks=true,
  	linkcolor=blue,
  	citecolor=blue,
  	linktoc=page,
  	pdfauthor={Benjamin C. Wallace},
  	pdftitle=\titlename]{hyperref}
  \hypersetup{hypertexnames=false}
  \hypersetup{colorlinks,citecolor=blue,linkcolor=blue}  % Comment out for printing!

  \externaldocument[norm-]{rg-norm}[rg-norm.pdf]
  \externaldocument[loc-]{rg-loc}[rg-loc.pdf]
  \externaldocument[pt-]{rg-pt}[rg-pt.pdf]
  \externaldocument[IE-]{rg-IE}[rg-IE.pdf]
  \externaldocument[step-]{rg-step}[rg-step.pdf]
  \externaldocument[saw4-]{saw4}[saw4.pdf]
  \externaldocument[log-]{saw4-log}[saw4-log.pdf]
  \externaldocument[phi4-log-]{phi4-log}[phi4-log.pdf]
  \externaldocument[flow-]{rg-flow}[rg-flow.pdf]
  \externaldocument[phi4-]{phi4}[phi4.pdf]
  \externaldocument[clp-]{clp}[clp.pdf]
  \externaldocument[sa-]{saw-sa}[saw-sa.pdf]
  % \externaldocument[strict5-]{saw-strict5}
% use hyperref with references to arxiv
\else\ifdefined\macrosHarxiv
  \usepackage{xr-hyper}
  \usepackage{hyperref}
  \hypersetup{hypertexnames=false}

  \externaldocument[norm-]{rg-norm}[http://arxiv.org/pdf/1403.7244v2.pdf]
  \externaldocument[loc-]{rg-loc}[http://arxiv.org/pdf/1403.7253v2.pdf]
  \externaldocument[pt-]{rg-pt}[http://arxiv.org/pdf/1403.7252v2.pdf]
  \externaldocument[IE-]{rg-IE}[http://arxiv.org/pdf/1403.7255v2.pdf]
  \externaldocument[step-]{rg-step}[http://arxiv.org/pdf/1403.7256v2.pdf]
  \externaldocument[flow-]{rg-flow}[http://arxiv.org/pdf/1211.2477.pdf]
  \externaldocument[saw4-]{saw4}[http://arxiv.org/pdf/1403.7268v2.pdf]
  \externaldocument[log-]{saw4-log}[http://arxiv.org/pdf/1403.7422v2.pdf]
  \externaldocument[phi4-log-]{phi4-log}[http://arxiv.org/pdf/1403.7424.pdf]
  \externaldocument[phi4-]{phi4}[http://arxiv.org/pdf/1412.2668.pdf]
% no hyperref
\else
  \newcommand{\texorpdfstring}[2]{#1}
  \usepackage{xr}
  \usepackage[colorlinks=true,
  	linkcolor=blue,
  	citecolor=blue,
  	linktoc=page,
  	pdfauthor={Benjamin C. Wallace},
  	pdftitle=\titlename]{hyperref}
  \externaldocument[norm-]{rg-norm}
  \externaldocument[loc-]{rg-loc}
  \externaldocument[pt-]{rg-pt}
  \externaldocument[IE-]{rg-IE}
  \externaldocument[step-]{rg-step}
  \externaldocument[flow-]{rg-flow}
  \externaldocument[saw4-]{saw4}
  \externaldocument[log-]{saw4-log}
  \externaldocument[phi4-log-]{phi4-log}
  \externaldocument[phi4-]{phi4}
  \externaldocument[clp-]{clp}
  % \externaldocument[sa-]{saw-sa}
  % \externaldocument[strict5-]{saw-strict5}
\fi\fi

\else
	% include general packages
\usepackage{amsfonts}
\usepackage{amsmath,amssymb,amsthm}
\usepackage{appendix}
\usepackage{bbm} % used in \newcommand{\1}{\mathbbm{1}}
\usepackage{amsbsy}
\usepackage{enumerate}
\usepackage{cite}
\usepackage{enumerate}

% use hyperref by default (can be overridden by inserting
% \let\macrosH\undefined in macros_local.tex)
% \def\macrosH{}

\InputIfFileExists{./macros_local.tex}{}{}

% page setup
\ifdefined\macrosPa
  \usepackage[textwidth=465pt,textheight=650pt,centering]{geometry} % 11pt
\else\ifdefined\macrosPb
  \usepackage[textwidth=500pt,textheight=650pt,centering]{geometry} % 12pt
\fi\fi

% needed for Springer templates: restore bold math fonts
\ifdefined\macrosS
  \makeatletter
  %% make boldmath and boldsymbol an alias for \pmb{}
  \def\boldmath{\pmb}
  \def\boldsymbol{\pmb}
  \makeatother

  \usepackage{mathptmx}
  % standard mathcal fonts
  \DeclareMathAlphabet{\mathcal}{OMS}{cmsy}{m}{n}
\fi

% do not include graphics for Birkhaeuser
\ifdefined\macrosBirk
\else
\usepackage[dvips]{graphicx}
\fi



% use def99c macros
\ifdefined\macrosSB
\input ../def99c %for LaTeX2e, enables \eq  and \en (!!)
\else
\input def99c %for LaTeX2e, enables \eq  and \en (!!)
\fi

\UseSection   %Necessary to define Numbering scheme for Theorem, etc.
\setcounter{secnumdepth}{3} %Set the depth of sectioning.
\setcounter{tocdepth}{3}    %Set the depth of table of contents.

% define colors
\usepackage[usenames]{color}
\newcommand{\red}{\color{red}}
\newcommand{\blue}{\color{blue}}
\newcommand{\magenta}{\color{magenta}}
\newcommand{\green}{\color{green}}
\newcommand{\cyan}{\color{cyan}}
\definecolor{bw}{RGB}{240, 120, 0}
\definecolor{at}{rgb}{0.0, 0.5, 0.0} % green, darker than the standard green.
\newcommand{\colour}[1]{{\blue #1}}

% define chXX and commentXX
\newcommand{\chrb}[1]{{\magenta #1}}
\newcommand{\chdb}[1]{{\red #1}}
\newcommand{\chblue}[1]{{\blue #1}}
\newcommand{\chgs}[1]{{\blue #1}}
\newcommand{\chgat}[1]{{\color{at} #1}}
\newcommand{\chbw}[1]{{\color{bw} #1}}
\newcommand{\comment}[1] { \begin{quote}
                                { \bf Comment: #1 }
                           \end{quote} }
\newcommand{\commentgs}[1] { \begin{quote}
                                { \bf Comment from GS: #1 }
                           \end{quote} }
\newcommand{\commentdb}[1] { \begin{quote}
                                { \bf Comment from DB: #1 }
                           \end{quote} }
\newcommand{\commentrb}[1] { \begin{quote}
                                { \bf Comment from RB: #1 }
                           \end{quote} }
\newcommand{\commentat}[1] { \begin{quote}
                                { \bf Comment from AT: #1 }
                           \end{quote} }
\newcommand{\commentbw}[1] { \begin{quote}
                                { \bf Comment from BW: #1 }
                           \end{quote} }


% custom abbreviations

\newcommand{\wo}{\colon \!\!}
\newcommand{\lwo}{\colon \!\!}
\newcommand{\rwo}{\! \colon \!\!}
\newcommand{\shift}{\!\!\!\!}
\newcommand{\veee}[1]{|\!|\!|#1|\!|\!|}

\newcommand{\wick}[1]{\lwo#1\rwo}
\newcommand{\Tay}{{\rm Tay}}
\newcommand{\LTold}{\widetilde{\rm loc}  }
\newcommand{\LTbar}{{\rm loc}}
\newcommand{\LTsym}{{\rm loc}}
\newcommand{\LT}{{\rm Loc}  }
\newcommand{\Rel}{{\rm Proj}}
\newcommand{\Irr}{(1-{\rm Proj})}

\newcommand{\DV}{\Dcal}
\newcommand{\DVa}{\alpha}
\newcommand{\Dnull}{\Dcal_\varnothing}
\newcommand{\rhogen}{\tilde{\rho}}
\newcommand{\rD}{r}
\newcommand{\crho}{c_L}
\renewcommand{\to} {\rightarrow}
\renewcommand{\qed}{\hfill\rule{2mm}{2mm}\bigskip}


\newcommand{\sumtwo}[2]{\sum_{ \mbox{ \scriptsize
    $\begin{array}{c}
                        {#1} \\ {#2}
                        \end{array} $ }
    }
}



\newcommand{\R}{\Rbold}
\newcommand{\Z}{\Zbold}
\newcommand{\X}{\mathbb X}
\newcommand{\x}{\mathbb x}
\newcommand{\N}{\Nbold}
\newcommand{\C}{\mathbb{C}}
\newcommand{\volume}{\mathbb{V}}
\newcommand{\Lambdabold}{\boldsymbol{\Lambda}}
\newcommand{\Sigmabold}{\boldsymbol{\Sigma}}
\newcommand{\1}{\mathbbm{1}}
\newcommand{\Cbf}{\boldsymbol{C}}
\newcommand{\cbf}{\boldsymbol{c}}
\newcommand{\Sbf}{\boldsymbol{S}}
\newcommand{\Abf}{\boldsymbol{A}}
\newcommand{\wbf}{\boldsymbol{w}}

\newcommand{\la}{\langle}
\newcommand{\ra}{\rangle}
\newcommand{\supp}{\mathrm{supp}}

\newcommand{\cL}{{\cal L}}
\newcommand{\cP}{{\cal P}}

\newcommand{\nn}{\nonumber}

\newcommand{\smallsup}[1] {{\scriptscriptstyle{({#1}})}}


\newcommand{\FDel}{\lambda}


\newcommand{\alphab}{\bar\alpha}
\newcommand{\varphib}{\bar\varphi}
\newcommand{\phibar}{\bar\varphi}
\newcommand{\psib}{\bar\psi}
\newcommand{\ci}{\underline{i}}

\newcommand{\elll}{\chdb{l}}

\newcommand{\jm}{j_\Omega}
\newcommand{\jmass}{j_{\rm mass}}

\newcommand{\w}{{\sf w}}
\newcommand{\Mext}{M_\mathrm{ext}}



\newcommand{\card}[1]{\text{Card #1}}
\renewcommand{\d}{\,d}
\newcommand{\e}{\mathbf{e}}
\newcommand{\Xst}{x_{\ast}}
\newcommand{\Yst}{y_{\ast}}

\newcommand{\Yb}{\bar{Y}}
\newcommand{\cC}{\mathcal{C}}
\newcommand{\cF}{\mathcal{F}}

\newcommand{\zetab}{\bar{\zeta}}
\newcommand{\etab}{\bar{\eta}}
\newcommand{\Ex}{\mathbb{E}}

\newcommand{\Econst}{\alpha_{\Ebold}}
\newcommand{\Econstg}{\alpha_{G}}
\newcommand{\EGconst}{\alpha}
\newcommand{\Fconst}{f}
\newcommand{\IstabC}{\alpha_{I}}
\newcommand{\ItilstabC}{\alpha_{I}}

\newcommand{\cCov}{{\sf C}}
\newcommand{\chicCov}{{\chi}}
\newcommand{\ellconst}{\mathfrak{c}}


\newcommand{\NGrass}{N_{{\rm Grass}}}

\newcommand{\Qcalnabla}{\Qcal}

\newcommand{\lt}{\ell}

\newcommand{\bubble}{{\sf B}}

\newcommand{\Gtilp}{\gamma}



% SOME NEW COMMANDS.


\newcommand{\side}[1]{\mathrm{side}(#1)}
\newcommand{\degree}[1]{\mathrm{deg(#1)}}
\newcommand{\pair}[1]{\langle #1 \rangle}
\newcommand{\Phizero}{\Phi^{*0}}
\newcommand{\Phipol}{\Pi}
\newcommand{\Phipoltil}{\widetilde{\Pi}}
\newcommand{\diam}[1]{\textrm{diam}(#1)}

\newcommand{\units}{\Ucal}
\newcommand{\monomials}{\Mcal}

\newcommand{\cgam}{\gamma}
\newcommand{\ckap}{c_0}
\newcommand{\cQ}{c_Q}


\newcommand{\Gkappa}{\kappa}
\newcommand{\pT}{p_T}

\newcommand{\concat}{\circ}


\newcommand{\inversefiddle}{3}
\newcommand{\fiddle}{\frac{1}{\inversefiddle}}


\newcommand{\obs}{\mathrm{obs}}
\newcommand{\homog}{\mathrm{hom}}
\newcommand{\Ihom}{I_{\homog ,j+1}}
\newcommand{\Itildehom}{\tilde{I}_{\homog ,j+1}}
\newcommand{\deltaIhom}{\delta I_{\homog ,j}}
\newcommand{\Khom}{K_{\homog ,j}}
\newcommand{\khom}{k_{\homog}}
\newcommand{\Jhom}{J_{\homog ,j+1}}
\newcommand{\Jobs}{J_{\obs ,j+1}}
\newcommand{\Itilde}{\tilde{I}}
\newcommand{\pertconst}{c_{j+1}}

\newcommand{\Phitimes}{\Phi^{\times}}
\newcommand{\Phiprimetimes}{{\Phi'}^{\times}}
\newcommand{\Ttimes}{T}

\newcommand{\decay}{r}
\newcommand{\hldg}{h_{\rm lead}}
\newcommand{\cldg}{c_{\rm lead}}
\newcommand{\hpt}{h_{\rm pt}}
\newcommand{\htil}{{\tilde h}^{(0)}}
\newcommand{\hpttil}{{\tilde h}_{\rm pt}}
\newcommand{\pt}{{\rm pt}}
\newcommand{\Ipt}{I_{\rm pt}}
\newcommand{\Ipttil}{\tilde{I}_{\rm pt}}
\newcommand{\Upt}{U_{\rm pt}}
\newcommand{\Vpt}{V_{\rm pt}}
\newcommand{\gpt}{g_{\mathrm{pt}}}
\newcommand{\nupt}{\nu_{\mathrm{pt}}}
\newcommand{\mupt}{\mu_{\mathrm{pt}}}
\newcommand{\zpt}{z_{\mathrm{pt}}}
\newcommand{\ypt}{y_{\mathrm{pt}}}
\newcommand{\lambdapt}{\lambda_{\mathrm{pt}}}
\newcommand{\lambdaapt}{\lambda^{\pp}_{\mathrm{pt}}}
\newcommand{\lambdabpt}{\lambda^{\qq}_{\mathrm{pt}}}
\newcommand{\lambdaa}{\lambda^{\pp}}
\newcommand{\lambdab}{\lambda^{\qq}}
\newcommand{\qapt}{q^{\pp}_{\mathrm{pt}}}
\newcommand{\qbpt}{q^{\qq}_{\mathrm{pt}}}
\newcommand{\qa}{q^{\pp}}
\newcommand{\qb}{q^{\qq}}
\newcommand{\qpt}{q_{\mathrm{pt}}}
\newcommand{\rpt}{r^{\mathrm{pt}}}

\newcommand{\Vbulk}{U}

\newcommand{\xch}{\check{x}}
\newcommand{\Kch}{\check{K}}
\newcommand{\Vch}{\check{V}}
\newcommand{\Rch}{\check{R}}
\newcommand{\gch}{\check{g}}
\newcommand{\zch}{\check{z}}
\newcommand{\much}{\check{\mu}}
\newcommand{\nuch}{\check{\nu}}
\newcommand{\lambdach}{\check{\lambda}}
\newcommand{\rch}{\check{r}}
\newcommand{\vch}{\check{v}}
\newcommand{\uch}{\check{u}}
\newcommand{\alphach}{\check{\alpha}}
\newcommand{\Vptch}{\check{V}_\pt}

\newcommand{\dq}{\delta q}

\newtheorem{convention}       [theorem] {Convention}

\newcommand{\ccc}[1]{O (#1)}

\newcommand{\wt}{w}
\newcommand{\wta}{w_\alpha}

\newcommand{\two}{\ItilstabC}
\newcommand{\ball}{B}
\newcommand{\h}{\mathfrak{h}}
\ifdefined\macrosSB \else
  \newcommand{\D}{\mathrm{D}}
  \newcommand{\I}{\mathfrak{I}}
\fi
\newcommand{\Ifopt}{\mathfrak{I}_{\rm fopt}}
\newcommand{\Isopt}{\mathfrak{I}_{\rm sopt}}
\newcommand{\quadr}{Q_{2}}
\newcommand{\Q}{Q}
\newcommand{\q}{q}
\newcommand{\F}{F}
\newcommand{\Id}{\mathrm{Id}}
\newcommand{\ev}{\mathrm{ev}}
\newcommand{\rn}{\nabla_0}
\newcommand{\fun}{\Fcal}
\newcommand{\deriv}{\left[\frac{d}{d\alpha} \right]_{0}}
\newcommand{\remainder}{R}
\newcommand{\pert}{\mathrm{pt}}
\newcommand{\norm}[1]{\left\|#1\right\|_{T_{\phi}}}
\newcommand{\z}{z}
\newcommand{\redDroman}{\chdb{D}}
\newcommand{\redDcal}{\Dcal}

\newcommand{\Cdecomp}{C_{\mathrm{decomp}}}

\newcommand{\Lmain}{\Lcal_{\rm lin}}

\newcommand{\Lcallr}{\stackrel{\leftrightarrow}{\Lcal}}

\newcommand{\cpl}{\xi}
\newcommand{\cplbar}{\bar{\cpl}}
\newcommand{\cplhat}{\hat{\cpl}}
\newcommand{\gbar}{\bar{g}}
\renewcommand{\ghat}{\hat{g}}
\newcommand{\ggen}{\tilde{g}}
\newcommand{\sgen}{\tilde{s}}
\newcommand{\chigen}{\tilde{\chi}}
\newcommand{\mgen}{\tilde{m}}
\newcommand{\Iint}{\mathbb{I}}
\newcommand{\Igen}{\tilde{\mathbb{I}}}
\newcommand{\Sgen}{S}
\newcommand{\zbar}{\bar{z}}
\newcommand{\mubar}{\bar{\mu}}
\newcommand{\muhat}{\hat{\mu}}
\newcommand{\stab}{\mathrm{stab}}
\newcommand{\unstab}{\mathrm{unstable}}
\newcommand{\pline}{\underline{p}}
\newcommand{\qline}{\underline{q}}

\newcommand{\Zbar}{\bar{Z}}
\newcommand{\chibar}{\bar{\chi}}


\newcommand{\domRG}{\mathbb{D}}
\newcommand{\domRGch}{\check{\mathbb{D}}}
\newcommand{\domr}{R}

\newcommand{\Lip}{\mathrm{Lip}}

\newcommand{\betamax}{\beta_{\rm max}}


\newcommand{\pp}{a}
\newcommand{\qq}{b}
\newcommand{\sigmaa}{\sigma}
\newcommand{\sigmab}{\bar{\sigma}}
\newcommand{\s}{\mathfrak s}

\newcommand{\half}{\textstyle{\frac 12}}

\newcommand{\f}{f}
\newcommand{\poly}{P}
\newcommand{\ghB}{\eta_{B}}

\newcommand{\extpow}{\mbox{$\bigwedge$}}
\newcommand{\pr}{\Pi}
\newcommand{\ddp}[2]{\frac{\partial #1}{\partial #2}}
\newcommand{\dd}[2]{\frac{d #1}{d #2}}

\newcommand{\LL}{\tilde{L}}
\newcommand{\smp}{\epsilon}

\newcommand{\epP}{\epsilon_{P}}
\newcommand{\epW}{\epsilon_{W}}
\newcommand{\epWdot}{\epsilon_{\dot{W}}}
\newcommand{\epWnull}{\epsilon_{W_\varnothing}}
\newcommand{\epV}{\epsilon_{V}}
\newcommand{\epU}{\epsilon_{U}}
\newcommand{\epVnull}{\epsilon_{V_\varnothing}}
\newcommand{\epVbar}{\epsilon_{g\tau^{2}}}
\newcommand{\epVdot}{\epsilon_{\dot{V}}}
\newcommand{\epVpt}{\epsilon_{V_{\rm pt}}}
\newcommand{\epdV}{\bar{\epsilon}}
\newcommand{\epdVdot}{\epsilon_{\delta \dot{V}}}
\newcommand{\epEV}{\epsilon_{\Ex V}}
\newcommand{\epdVbar}{\bar{\epsilon}_{\delta V}}
\newcommand{\epEVbar}{\bar{\epsilon}_{\Ex V}}
\newcommand{\epK}{\epsilon_{K}}
\newcommand{\epdI}{\epsilon_{\delta I}}
\newcommand{\epI}{\epsilon_{I}}
\newcommand{\epdP}{\epsilon_{\delta P}}
\newcommand{\eplamdot}{\epsilon_{\dot{\lambda}}}

\newcommand{\gh}{\epV}
\newcommand{\ghbar}{\bar{\epsilon}_V}


\newcommand{\nobs}[1]{|#1|_{\text{obs}}}

\newcommand{\spec}{\mathrm{spec}}


\newcommand{\phib}{\bar\phi}



\newcommand{\zldg}{z_{\rm lead}}
\newcommand{\zh}{z_{h}}
\newcommand{\zK}{z_{K}}
\newcommand{\zDelta}{z_{\Delta}}


\newcommand{\McalnowM}{M}
\newcommand{\IcalnowN}{h_{\rm rem}}
\newcommand{\hred}{h_{\rm red}}
\newcommand{\hrem}{h_{\rm rem}}
\newcommand{\species}{\mathbf{s}}
\newcommand{\sgn}{\mathrm{sgn}}

\newcommand{\cc}{\mathbf{cc}}
\newcommand{\ob}{\mathbf{ob}}
\newcommand{\fn}{\mathbf{fn}}

\newcommand{\V}{\tilde{V}}
\newcommand{\Vspace}{\Vcal_{\mathrm{sym}}}
\newcommand{\Vdomain}{\Dcal_0}
\newcommand{\Vin}   {V}
\newcommand{\Vout}  {V_{\mathrm{out}}}
\newcommand{\Kin}   {K_{\mathrm{in}}}
\newcommand{\Uin}   {U_{\mathrm{in}}}
\newcommand{\Kout}  {K_{\mathrm{out}}}
\newcommand{\Kspace}{\Kcal}
\newcommand{\CKspace}{\Ccal\Kspace}
\newcommand{\BKspace}{\Bcal\Kspace}
\newcommand{\Ispace}{\text{$I$-space}}
\newcommand{\Iin}   {I_{\mathrm{in}}}
\newcommand{\Iout}  {I_{\mathrm{out}}}
\newcommand{\kprime}{k'}
\newcommand{\Vtilde}{\tilde{V}}
\newcommand{\Ktilde}{\tilde{K}}
\newcommand{\Kstar} {K^{*}}
\newcommand{\Jprime}{j}
\newcommand{\amain} {a}
\newcommand{\ain}   {a_{\mathrm{in}}}
\newcommand{\aout}  {a_{\mathrm{out}}}
\newcommand{\aahat} {\hat{a}}
\newcommand{\abig}  {\tilde{a}}
\newcommand{\astar} {a^*}
\newcommand{\epsin} {\epsilon_{\mathrm{in}}}
\newcommand{\rball} {r}

\newcommand{\qwt}{\mathfrak{q}}
\newcommand{\tail}{\mathfrak{t}}
\newcommand{\covc}{\mathfrak{c}}
\newcommand{\rgam}{\mathfrak{r}}

\newcommand{\RG}{\text{RG}}

\newcommand{\REF}{{\bf REF}\ }
\newcommand{\subnw}{\subset_{\text{nw}}}

% Some macros by AT

\newcommand{\prob}[1]{P\left\{ #1 \right\}}

\newcommand{\wm}{W}
%\DeclareMathOperator{\wm}{W}
\DeclareMathOperator{\wsaw}{wsaw}
\DeclareMathOperator{\srw}{srw}
\DeclareMathOperator{\Loc}{Loc} %\LT

\newcommand{\nl}{p}

\newcommand{\phifour}{$|\varphi|^4$}


%%%%%%%%%%%%%%%%%%%%%%%%%%%%%%%%%%%%%%%%%%%%%%%%%%%%%%%%%%%%%%%%%%%
%% do not delete: these are useful to make characters stand out during
%% proof-reading

%% how to hilight \Phipol
% \let\temp\Phipol
% \renewcommand{\Phipol}{\chdb{\temp}}

%% how to make the character $G$ red

% \mathchardef \tx \mathcode `G% storing current definition of G in \tx
% \ifnum \mathcode `\G = "8000% checking if we have already changed definition
% \else
% \begingroup
% \catcode `\G=13
% \global \mathchardef G \mathcode `\G
% \endgroup
% \mathcode `\G = "8000% 8000 declares G active in mathmode
% \fi

% \begingroup
% \catcode `\G=13
% \global \def G{\chdb{\tx}}%or make it any other command
% \endgroup
%%%%%%%%%%%%%%%%%%%%%%%%%%%%%%%%%%%%%%%%%%%%%%%%%%%%%%%%%%%%%%%%%%%%%


% hyperref setup -- needs to come at the end since section names
% can involve macros defined above

% use hyperref
\ifdefined\macrosH
  \usepackage{xr-hyper}
  \usepackage[colorlinks=true,
  	linkcolor=blue,
  	citecolor=blue,
  	linktoc=page,
  	pdfauthor={Benjamin C. Wallace},
  	pdftitle=\titlename]{hyperref}
  \hypersetup{hypertexnames=false}
  \hypersetup{colorlinks,citecolor=blue,linkcolor=blue}  % Comment out for printing!

  \externaldocument[norm-]{rg-norm}[rg-norm.pdf]
  \externaldocument[loc-]{rg-loc}[rg-loc.pdf]
  \externaldocument[pt-]{rg-pt}[rg-pt.pdf]
  \externaldocument[IE-]{rg-IE}[rg-IE.pdf]
  \externaldocument[step-]{rg-step}[rg-step.pdf]
  \externaldocument[saw4-]{saw4}[saw4.pdf]
  \externaldocument[log-]{saw4-log}[saw4-log.pdf]
  \externaldocument[phi4-log-]{phi4-log}[phi4-log.pdf]
  \externaldocument[flow-]{rg-flow}[rg-flow.pdf]
  \externaldocument[phi4-]{phi4}[phi4.pdf]
  \externaldocument[clp-]{clp}[clp.pdf]
  \externaldocument[sa-]{saw-sa}[saw-sa.pdf]
  % \externaldocument[strict5-]{saw-strict5}
% use hyperref with references to arxiv
\else\ifdefined\macrosHarxiv
  \usepackage{xr-hyper}
  \usepackage{hyperref}
  \hypersetup{hypertexnames=false}

  \externaldocument[norm-]{rg-norm}[http://arxiv.org/pdf/1403.7244v2.pdf]
  \externaldocument[loc-]{rg-loc}[http://arxiv.org/pdf/1403.7253v2.pdf]
  \externaldocument[pt-]{rg-pt}[http://arxiv.org/pdf/1403.7252v2.pdf]
  \externaldocument[IE-]{rg-IE}[http://arxiv.org/pdf/1403.7255v2.pdf]
  \externaldocument[step-]{rg-step}[http://arxiv.org/pdf/1403.7256v2.pdf]
  \externaldocument[flow-]{rg-flow}[http://arxiv.org/pdf/1211.2477.pdf]
  \externaldocument[saw4-]{saw4}[http://arxiv.org/pdf/1403.7268v2.pdf]
  \externaldocument[log-]{saw4-log}[http://arxiv.org/pdf/1403.7422v2.pdf]
  \externaldocument[phi4-log-]{phi4-log}[http://arxiv.org/pdf/1403.7424.pdf]
  \externaldocument[phi4-]{phi4}[http://arxiv.org/pdf/1412.2668.pdf]
% no hyperref
\else
  \newcommand{\texorpdfstring}[2]{#1}
  \usepackage{xr}
  \usepackage[colorlinks=true,
  	linkcolor=blue,
  	citecolor=blue,
  	linktoc=page,
  	pdfauthor={Benjamin C. Wallace},
  	pdftitle=\titlename]{hyperref}
  \externaldocument[norm-]{rg-norm}
  \externaldocument[loc-]{rg-loc}
  \externaldocument[pt-]{rg-pt}
  \externaldocument[IE-]{rg-IE}
  \externaldocument[step-]{rg-step}
  \externaldocument[flow-]{rg-flow}
  \externaldocument[saw4-]{saw4}
  \externaldocument[log-]{saw4-log}
  \externaldocument[phi4-log-]{phi4-log}
  \externaldocument[phi4-]{phi4}
  \externaldocument[clp-]{clp}
  % \externaldocument[sa-]{saw-sa}
  % \externaldocument[strict5-]{saw-strict5}
\fi\fi
						% Just use xr
	\hypersetup{draft}					% Turns off hyperlinks
\fi
\newcommand{\generator}{Q}		  % Markov chain generator
\newcommand{\diag}{D}			  % diagonal matrix
\newcommand{\jay}{J}			  % adjacency matrix
\newcommand{\lap}{L}			  % graph Laplacian

\newcommand{\edges}{\Ecal}
\newcommand{\graph}{\Gcal}
\newcommand{\vertices}{\Vcal}

\DeclareMathOperator*{\argmax}{argmax}
\newcommand{\fugacity}{\zeta}
\newcommand{\gibbs}{\mathscr{G}}
\newcommand{\nubar}{\bar\nu}	  % scaling exponent
\renewcommand{\commentbw}[1]{}			% Uncomment to hide comments

% check ubcsample.tex for other potentially useful packages to include

%%%%%%%%%
% TITLE %
%%%%%%%%%

%%%% Institution %%%%
\institution{The University Of British Columbia}
\faculty{The Faculty of Graduate and Postdoctoral Studies}
\institutionaddress{Vancouver}

%%%% Degrees %%%%
\previousdegree{B.Sc.~Honours, Queen's University, 2012}
\previousdegree{M.Sc., Queen's University, 2013}

%%%% Program %%%%
\program{Mathematics}

%%%% Title and author %%%%o
\title{\titlename}
% \subtitle{Subtitle}
\author{Benjamin Wallace}					% Should be the name under which you are registered at UBC

%%%% Copyright %%%%
\copyrightyear{2017}

%%%% Date %%%%
% For the examination committee: month, year of submission to committe/examiner
% For final, post-defence submission: should be the month, year of final submissions of defended thesis
\submitdate{\monthname\ \number\year}
% \submitdate{June 2017}
% The "\ " is required after \monthname to
% prevent the command from eating the space.

%%%%%%%%%%%%%%%%%%%%%
% TABLE OF CONTENTS %
%%%%%%%%%%%%%%%%%%%%%

% See template regarding chapter numbers, etc.
\setcounter{tocdepth}{2}
\setcounter{secnumdepth}{2}

%%%%%%%%%%%%
% DOCUMENT %
%%%%%%%%%%%%

\begin{document}

%%%% Front matter %%%%
%% This starts numbering in Roman numerals as required for the thesis
%% style and is mandatory.
\frontmatter

%%% The order of the following components should be preserved.  The order
%%% listed here is the order currently required by FoGS:        \\
%%% Title (Mandatory)                                           \\
%%% Preface (Manditory if any collaborator contributions)       \\
%%% Abstract (Mandatory)                                        \\
%%% List of Contents, Tables, Figures, etc. (As appropriate)    \\
%%% Acknowledgements (Optional)                                 \\
%%% Dedication (Optional)                                       \\

\maketitle                      %% Mandatory
\begin{abstract}                %% Mandatory -  maximum 350 words
Abstract
\end{abstract}

\chapter{Preface} % Manditory if any of the conditions are met

You must include a preface if any part of your research was partly or
wholly published in articles, was part of a collaboration, or required
the approval of UBC Research Ethics Boards.

The Preface must include the following:

\begin{itemize}
\item A statement indicating the relative contributions of all
  collaborators and co-authors of publications (if any), emphasizing
  details of your contribution, and stating the proportion of research
  and writing conducted by you.
\item A list of any publications arising from work presented in the
  dissertation, and the chapter(s) in which the work is located.
\item The name of the particular UBC Research Ethics Board, and the
  Certificate Number(s) of the Ethics Certificate(s) obtained, if
  ethics approval was required for the research.
\end{itemize}

% %%% Sections and subsections etc. in the Preface should in general
% %%% not be listed in the table of contents, so use the starred form
% %%% of \section etc.
% \section*{Examples}
% Chapter~\ref{cha:apple_ref} is based on work conducted in UBC's Maple
% Syrup Laboratory by Dr. A.  Apple, Professor B. Boat, and Michael
% McNeil Forbes. I was responsible for tapping the trees in forests X
% and Z, conducted and supervised all boiling operations, and performed
% frequent quality control tests on the product.

% A version of chapter~\ref{cha:apple_ref} has been
% published~\cite{Apple:2010}. I conducted all the testing and wrote
% most of the manuscript. The section on ``Testing Implements'' was
% originally drafted by Boat, B.  Check the first pages of this
% chapter to see footnotes with similar information.

% Note that this preface must come before the table of contents.  Note
% also that this section ``Examples'' should not be listed in the table
% of contents, so we have used the starred form: \verb|\section*{Example}|.

\tableofcontents                %% Mandatory
% \listoftables                   %% Mandatory if thesis has tables
\listoffigures                  %% Mandatory if thesis has figures
% \listof{Program}{List of Programs} %% Optional
%% Any other lists should come here, i.e.
%% Abbreviation schemes, definitions, lists of formulae, list of
%% schemes, glossary, list of symbols etc.

\chapter{Acknowledgements}      %% Optional
Acknowledge: Gord, David, Roland, etc.

\chapter{Dedication} %% Optional
Dedication...

% Delete this
\chapter{To do}
\begin{itemize}
\item
Use $\Vcal$, etc. for polynomials, $\Ucal$ for unit vectors

\item
Make $U_{\gcc,\gamma}$ notation consistent with $U_{\gcc,\nu,N}$.
I have done this by getting rid of the latter (it is hardly used
explicitly in clp)

\item
Change $\beta, g$ to $\backslash gcc$

\item
Change $L$ to $\backslash lt$

\item
Make $G_x(g, \nu; n)$ consistent with $G_{\gcc,\gamma,\nu}(a, b)$.
Maybe use $G_x(g, \gamma, \nu)$ and $G_x(g, \nu; n)$

\item
Make it so that $U$ has observables but no constant and $V$ is $U$
plus the constant. Thus, we must change $V$ in saw-sa to $U$ (the $V$
in saw-sa has no observables, but these can be added in without harm).
We must also change $U^\pm$ in saw-sa to something else

\item
Change $\chi$ and $\tilde\chi$ to $\backslash chicCov$ and $\backslash chicCovgen$
\end{itemize}

% Any other unusual prefactory material should come here before the
% main body.
\input{todo}				% temporary


%%%% Main matter %%%%
\mainmatter

%% Chapter 1 %%
% Parts are the largest structural units, but are optional.
%\part{Thesis}

% Chapters are the next main unit.
\chapter{Introduction}

%%%%%%%%%%%%%%%%%%%%%%%%%%%%%%%%%%%%%%%%%%%%%%%%%%%%%%%%%%%%%%%%%%%%%%%%%%%%%%%
%%%%%%%%%%%%%%%%%%%%%%%%%%%%%%%%%%%%%%%%%%%%%%%%%%%%%%%%%%%%%%%%%%%%%%%%%%%%%%%

\section{Statistical mechanics}

\todo{Restructure this: just introduce Gibbs measure, don't make hard distinction
between canonical and grand canonical. Only mention GCE when discussing walks
later}

%%%%%%%%%%%%%%%%%%%%%%%%%%%%%%%%%%%%%%%%%%%%%%%%%%%%%%%%%%%%%%%%%%%%%%%%%%%%%%%

\subsection{Entropy}

Let $(\Omega, \lambda)$ be a measure space. The state of knowledge of a system on
$\Omega$ can be expressed by a probability measure $\mu$ on $\Omega$. Let
$\Mcal_\lambda(\Omega)$ denote the set of probability measures on $\Omega$ absolutely
continuous with respect to $\lambda$. For
$\mu \in \Mcal_\lambda(\Omega)$, we denote the Radon-Nikodym derivative of
$\mu$ with respect to $\lambda$ by $d\mu/d\lambda$ and define the
\emph{entropy} of $\mu$ with respect to $\lambda$ by
\begin{equation}
h(\mu) = h_\lambda(\mu) = -\int_\Omega \log\frac{d\mu}{d\lambda} \; d\mu.
\end{equation}
In many cases, specific information about the system under consideration is available,
so that we may restrict our attention to a subspace $M \subset \Mcal_\lambda(\Omega)$.
The \emph{principle of maximum entropy} \cite{Jaynes57} asserts that, in this
case, the measure best expressing the state of knowledge of the system is given by
\begin{equation}
\hat\mu = \argmax(h_\lambda(\mu) : \mu \in M),
\end{equation}
assuming such a measure exists and is unique.

%%%%%%%%%%%%%%%%%%%%%%%%%%%%%%%%%%%%%%%%%%%%%%%%%%%%%%%%%%%%%%%%%%%%%%%%%%%%%%%

\subsection{Microcanonical ensemble}

Consider an \emph{isolated} physical system on $\Omega$, that is, one that cannot
exchange energy with its surroundings. Such a system can be determined by a choice
of function $H : \Omega \to \R$, called the \emph{Hamiltonian}. The value $H(\omega)$
represents the total energy of the system in state
$\omega\in\Omega$.

\begin{example}
Let $\Omega = U^n \times \R^{3n}$, where $U \subset \R^3$, and denote a generic
element of $\Omega$ by $(q, p)$, where $q \in U^n$ and $p \in \R^{3n}$. Then
$\Omega$ is the state space of a system of $n$ point particles
$i = 1, \ldots, n$ with positions $q_i \in U$ and momenta $p_i \in \R^3$. Given
a $C^1$ Hamiltonian $H : \Omega \to \R$, the dynamics of such a system is determined
by \emph{Hamilton's equations}
\begin{align}
\dd{q}{t}   &= \nabla_q H(q(t), p(t)) \\
-\dd{p}{t}  &= \nabla_p H(q(t), p(t)).
\end{align}
An immediate consequence of these equations and the chain rule is the \emph{principle of conservation of energy}:
\begin{equation}
\dd{H}{t}(q(t), p(t)) = 0.
\end{equation}
Thus, a Hamiltonian system with initial configuration $(q(0), p(0))$ of energy
$E = H(q(0), p(0))$ will evolve on the constant energy shell $S_E = H^{-1}(E)$.
\end{example}

If $S_E = H^{-1}(E)$ is finite\footnote{We can view such a system as an approximation to a traditional continuous system with $S_E$ uncountable.}, then one can easily determine that the maximum entropy measure on $S_E$ is simply the uniform measure. For a system with finite state space $\Omega$ and Hamiltonian $H$, we define the \emph{microcanonical distribution} with energy $E$ to be the uniform measure on $S_E$.

%%%%%%%%%%%%%%%%%%%%%%%%%%%%%%%%%%%%%%%%%%%%%%%%%%%%%%%%%%%%%%%%%%%%%%%%%%%%%%%

\subsection{Canonical ensemble}

In practice, most systems of interest are not truly isolated: they may exchange
energy with their environments. Everyday experience, however, suggests that
a physical system that is left undisturbed for a sufficiently long time will achieve \emph{thermal equilibrium}, in which the system's temperature is constant and equal to that of its surroundings. We define the \emph{canonical ensemble} for a system with state space $\Omega$ and Hamiltonian $H$
(assumed to be integrable) on
$\Omega$ to be the maximum entropy distribution subject to the fixed average energy constraint
\begin{equation}
\int H \; d\mu = E.
\end{equation}
It can be shown by the method of Lagrange multipliers that the canonical ensemble is given by the \emph{Gibbs measure}
\begin{equation}
d\mu_\beta = \frac{1}{Z_\beta} e^{-\beta H} d\lambda,
\end{equation}
where
\begin{equation}
Z_\beta = \int e^{-\beta H} \; d\lambda
\end{equation}
is the normalizing constant, known as the \emph{canonical partition function}.
The quantity $\beta = \beta(E)$, which arises as a Lagrange multiplier, is known as the \emph{inverse temperature}.

The \emph{free energy} of this system is defined by
\begin{equation}
F_\beta = -\frac{1}{\beta} \log Z_\beta.
\end{equation}
This definition may seem obscure at first, but is elucidated by a computation of the entropy $h_\lambda(\mu_\beta)$ with $\beta = \beta(E)$, which implies that
\begin{equation}
F_\beta = E - \frac{1}{\beta} h_\lambda(\mu_\beta).
\end{equation}
This is the famous thermodynamic relation between free energy, internal energy $E$, temperature $1/\beta$, and entropy.


In the context of spin systems, there is a natural mathematical reason for studying measures of the above form. This is the Hammersley-Clifford theorem, which states that any Markov random field (a spatial generalization of a Markov chain) on a graph has a representation as a Gibbs measure whose Hamiltonian is a sum of ``local'' interactions.

%%%%%%%%%%%%%%%%%%%%%%%%%%%%%%%%%%%%%%%%%%%%%%%%%%%%%%%%%%%%%%%%%%%%%%%%%%%%%%%
%%%%%%%%%%%%%%%%%%%%%%%%%%%%%%%%%%%%%%%%%%%%%%%%%%%%%%%%%%%%%%%%%%%%%%%%%%%%%%%

\section{Graphs}

Most systems of interest do not have a finite (or even countable) state space.
Nevertheless, large finite systems may be used as approximations of real systems.
A natural approach to approximating spatially-extended systems is by studying
models on graphs.

An \emph{undirected graph} or simply a \emph{graph} is a pair $\graph = (\vertices, \edges)$,
where $\vertices$
is a set of \emph{vertices} and $\edges$ is a set of
\emph{edges} $\{ x, y \}$ with $x, y \in \vertices$; we will write $x \sim y$ if
$\{ x, y \} \in \edges$.
For simplicity, we will assume that $\vertices$ is countable annd that there are no
\emph{self-loops} $\{ x \} \in \edges$.
We will also assume that $\graph$ is \emph{(vertex-)transitive}: that is, for all pairs
of distinct
vertices $a, b \in \vertices$, there exists a mapping $f : \vertices \to \vertices$
such that $x \sim y$ if and only if $f(x) \sim f(y)$.
We fix a vertex $0\in\vertices$; the assumption of transitivity implies
that the particular choice of $0$ is immaterial.

\begin{example}\mbox{}\\
\smallskip\noindent
(i) We view any set $X \subset \Zd$ as a graph with $\vertices = X$ and
$x\sim y$ if $|x - y| = 1$. In particular, $X = \Zd$ is transitive.

\smallskip\noindent
(ii) Let $L > 1$ be an integer. For $N \ge 0$, let
\begin{equation}
\Lambda_N = \Zd/L^N\Zd.
\end{equation}
We call $\Lambda_N$ the \emph{discrete $d$-dimensional torus} of side $L^N$.
We view $\Lambda_N$ as a graph with $\vertices = \Lambda_N$ and $x \sim y$
if $|x - y| = 1$ modulo $L^N$.
\end{example}

%%%%%%%%%%%%%%%%%%%%%%%%%%%%%%%%%%%%%%%%%%%%%%%%%%%%%%%%%%%%%%%%%%%%%%%%%%%%%%%

\subsection{Functions on graphs}

Let us denote the components of a function $\varphi : \vertices \to \R^n$ by
$\varphi^i_x \in \R$ for $x \in \vertices$ and $i = 1, \ldots, n$.
The Euclidean inner product and norm on the space $(\R^n)^\vertices$ of such functions
are defined by
\begin{align}
\varphi\cdot\tilde\varphi
	&= \sum_{x\in\vertices} \varphi_x \cdot \tilde\varphi_y
  		= \sum_{i=1}^n \sum_{x\in\vertices} \varphi^i_x \tilde\varphi^i_x \\
	&=|\varphi|^2 = \varphi \cdot \varphi.
\end{align}
A $\vertices\times\vertices$ matrix $M = (M_{xy})_{x,y\in\vertices}$ acts on $\varphi$ via
\begin{equation}
(M \varphi)_x = \sum_{y\in\vertices} M_{xy} \varphi_y.
\end{equation}

%%%%%%%%%%%%%%%%%%%%%%%%%%%%%%%%%%%%%%%%%%%%%%%%%%%%%%%%%%%%%%%%%%%%%%%%%%%%%%%

\subsection{The graph Laplacian}

Let us say that a $\vertices\times\vertices$ matrix $M$ is \emph{indexed by} $\edges$
if $M_{xy} = 0$ if and only if $x \not\sim y$.
Let $\jay$ be a matrix indexed by $\edges$.
Throughout this thesis, we will assume that the $\jay$ has nonnegative entries;
thus, $\jay_{xy} \ge 0$ with equality if and only if $x\not\sim y$.
The pair $(\graph, \jay)$ is an example of a \emph{weighted} graph.
We will usually denote this weighted graph simply as $\graph$, with $\jay$
implicit.

Let $\diag$ be a diagonal $\vertices\times\vertices$ matrix with diagonal entries
\begin{equation}
d_x = \diag_{xx} = \sum_{y \sim x} J_{xy}.
\end{equation}
We say that $\graph$ is $d_0$-regular if $d_x = d_y$ for all $x, y$.

The \emph{(massless) graph Laplacian} on $\graph$ is defined by
\begin{equation}
-\lap = \diag - \jay.
\end{equation}
We also define the \emph{massive Laplacian} with squared \emph{mass} $m^2 > 0$
by
\begin{equation}
-\lap + m^2.
\end{equation}
Note that
\begin{equation}
\varphi \cdot (-\lap \varphi)
  =
\frac{1}{2} \sum_{x,y\in\vertices} J_{xy} |\varphi_x - \varphi_y|^2
  \ge
0,
\end{equation}
so $-\lap$ is positive-semidefinite.

\begin{example}
An important case is when $\jay$ has $\{0, 1 \}$-valued entries.
In this case, $d_x$ is the \emph{degree} of $x$ in $\graph$ and we denote $\lap$ by
$\Delta$, which has entries given by
\begin{equation}
-\Delta_{xy} = d_x \1_{x=y} - \1_{x \sim y}.
\end{equation}
\end{example}

%%%%%%%%%%%%%%%%%%%%%%%%%%%%%%%%%%%%%%%%%%%%%%%%%%%%%%%%%%%%%%%%%%%%%%%%%%%%%%%

\subsection{The Green function}

If $m^2 > 0$, then $-\lap + m^2$ is positive-definite, hence invertible with inverse
\begin{equation}
(-\lap + m^2)^{-1} = (m^2 + D)^{-1} \sum_{n=0}^\infty Z^n P^n,
\end{equation}
where
\begin{align}
Z = (m^2 + D)^{-1} D,
  \quad
P = D^{-1} J.
\end{align}
Let $z_x$ denote the diagonal elements of $Z$. The \emph{Green function} for
$-\lap + m^2$ is the kernel of $(-\lap + m^2)^{-1}$, given by
\begin{equation}
C(x, y)
  =
(m^2 + d_x)^{-1} \sum_{n=0}^\infty z_x^n P^n_{xy}.
\end{equation}

%%%%%%%%%%%%%%%%%%%%%%%%%%%%%%%%%%%%%%%%%%%%%%%%%%%%%%%%%%%%%%%%%%%%%%%%%%%%%%%
%%%%%%%%%%%%%%%%%%%%%%%%%%%%%%%%%%%%%%%%%%%%%%%%%%%%%%%%%%%%%%%%%%%%%%%%%%%%%%%

\section{Spin systems}

\subsection{The Ising model}

Suppose that $|\vertices| < \infty$.
The \emph{Ising model} on $\graph$ is defined by the Gibbs measure on $\Omega = \{ \pm 1 \}^\vertices$
with Hamiltonian
\begin{equation}
H_h(\sigma)
	=
-\frac{1}{2} \sum_{x \sim y} \sigma_x \sigma_y - h \sum_{x\in\vertices} \sigma_x,
\end{equation}
where $h \in \R$ is known as the \emph{external field}.
The first term encourages spins to \emph{align}: an edge $\{ x, y \}$ makes
a negative contribution to the total energy (so a positive contribution to
the probability of a configuration) when $\sigma_x = \sigma_y$.
For a similar reason, the second term encourages spins to adopt the same sign as $h$.

Let $\langle\cdot\rangle_{\beta,h}$ denote the expectation with respect to
this Gibbs measure and let $Z_{\beta,h}$ be the partition function.
The \emph{magnetization} is defined by
\begin{equation}
\langle \sigma_0 \rangle_{\beta,h} = \frac{1}{Z_\beta} \sum_{\sigma\in\Omega} \sigma_0 e^{-\beta H(\sigma)}.
\end{equation}
When $\beta > 0$, it is reasonable to expect that the magnetization has the
same sign as $h$.

To see this, define the \emph{magnetic susceptibility} by
\begin{equation}
\chi(\beta, h)
	=
\frac{1}{\beta} \dd{}{h} \langle \sigma_0 \rangle_{\beta,h}.
\end{equation}
A computation shows that
\begin{equation}
\chi(\beta, h) = \sum_{x\in\vertices} G_x(\beta, h),
\end{equation}
where
\begin{equation}
G_x(\beta, h)
	=
\Big(
	\langle \sigma_0 \sigma_x \rangle_{\beta,h}
		-
	\langle \sigma_0 \rangle_{\beta,h} \langle \sigma_x \rangle_{\beta,h}
\Big)
\end{equation}
is the \emph{two-point function}.
Thus, the susceptibility is positive and so the magnetization is increasing
in $h$. When $h = 0$, the Gibbs measure is invariant under the spin flip
$\sigma \mapsto -\sigma$, and so the magnetization is $0$. It follows that
$\langle \sigma_0 \rangle_{\beta,h} > 0$ if and only if $h > 0$.

%%%%%%%%%%%%%%%%%%%%%%%%%%%%%%%%%%%%%%%%%%%%%%%%%%%%%%%%%%%%%%%%%%%%%%%%%%%%%%%

\subsection{Spin systems on finite graphs}

We continue to assume that $|\vertices| < \infty$.
An $n$-component \emph{field} or \emph{spin configuration} on $\vertices$
with spins in $S \subset \R^n$ is an element of $\Omega = S^\vertices$.
% A \emph{spin system} is a probability measure $d\mu$ on $\Omega$.
Suppose that $S$ is equipped with a measure $d\lambda^0$.
Given a Hamiltonian $H : \Omega \to \R$, we wish to define the measure
\begin{equation}
d\mu_\beta(\varphi)
  =
\frac{1}{Z_\beta} e^{-\beta H(\varphi)} d\lambda(\varphi)
\end{equation}
on $\Omega$, where
\begin{equation}
d\lambda(\varphi) = \prod_{x\in\vertices} d\lambda^0(\varphi_x).
\end{equation}
We denote the expectation with respect to this measure by $\langle\cdot\rangle_\beta$.
% However, there are some problems with this definition when $\vertices$ is infinite.
% For one, $d\lambda$ may not be well-defined (for instance if
% $S = \R$ and $d\lambda^0$ is Lebesgue measure). Another issue is that it may be
% difficult to define a reasonable choice of $H$ on the infinite product space
% $\Omega$.
% 
% For this reason, we temporarily restrict our attention to finite graphs:
% \begin{equation}
% |\vertices| < \infty.
% \end{equation}
% Then the Gibbs measure $\mu = \mu_\beta$ is well-defined and we will denote the
% expectation with respect to this measure by $\langle \cdot \rangle_\mu$.

Following our discussion of the Ising model, we define the two-point function
and susceptibility by
\begin{equation}
G_x(\beta)
  =
\frac{1}{n}
\big(\langle \varphi_0 \cdot \varphi_x \rangle_\beta
  -
\langle \varphi_0 \rangle_\beta \cdot \langle \varphi_x \rangle_\beta\big)
\end{equation}
and
\begin{equation}
\chi(\beta) = \sum_x G_x(\beta).
\end{equation}

We will mainly be concerned with \emph{ferromagnetic} spin systems, for which the
Hamiltonian has the form
\begin{equation}
H(\varphi) = -\varphi \cdot M\varphi,
\end{equation}
where $M_{xy} \ge 0$.
% Thus, $H(\varphi)$ is smaller (and $d\mu_\beta(\varphi)$ larger) when the spins align
% (when $\varphi_x = \varphi_y$ for $x \sim y$).

%%%%%%%%%%%%%%%%%%%%%%%%%%%%%%%%%%%%%%%%%%%%%%%%%%%%%%%%%%%%%%%%%%%%%%%%%%%%%%%

% \subsection{The \texorpdfstring{$O(n)$}{O(n)} spin model}

\begin{example}[The $O(n)$ model]
Let $S = S^{n-1} \subset \R^n$ be the unit $(n-1)$-sphere equipped with the normalized sphere measure
$d\lambda^0$ (in particular, $S^0 = \{ \pm 1 \}$).
The \emph{$O(n)$ spin model} or \emph{$n$-vector model} is the
ferromagnetic spin system with Hamiltonian
\begin{equation}
H_J(\sigma) = -\frac{1}{2} \sigma \cdot J \sigma,
\end{equation}
which is clearly ferromagnetic.
When $n = 1$ we recover the Ising model. When $n = 2, 3$, we get the \emph{XY model} and the \emph{classical Heisenberg model}.
\end{example}

%%%%%%%%%%%%%%%%%%%%%%%%%%%%%%%%%%%%%%%%%%%%%%%%%%%%%%%%%%%%%%%%%%%%%%%%%%%%%%%

% \subsection{The \texorpdfstring{$|\varphi|^4$}{phi4} spin model}

\begin{example}[The $|\varphi|^4$ model]
Let $S = \R^n$. The $|\varphi|^4$ spin model on $\graph$ is the ferromagnetic spin system
defined by the quartic Hamiltonian
\begin{equation}
H_{\gcc,\nu}(\varphi)
  =
\sum_{x\in\vertices}
\left(
  \frac{1}{4} \gcc |\varphi_x|^4
    +
  \frac{1}{2} \nu |\varphi_x|^2
    +
  \frac{1}{2} \varphi_x \cdot (-\lap \varphi)_x
\right),
\end{equation}
where $\gcc > 0$ and $\nu\in\R$. Adjusting $\beta$ is equivalent to  rescaling
$\gcc$ and $\nu$, so we set $\beta = 1$ without loss of generality.
% This is a natural generalization of the $O(n)$ model
% in which spins are merely concentrated near a sphere.
When $\graph$ is $d_0$-regular, we can write
\begin{equation}
H_{\gcc,\nu}(\varphi)
  =
\sum_{x\in\vertices} U_{\gcc,\nu}(\varphi_x) - \frac{1}{2} \varphi \cdot \jay \varphi,
\end{equation}
where the \emph{single-spin potential} $U_{\gcc,\nu}$ is defined by
\begin{equation}
U_{\gcc,\nu}(t)
	=
\frac{1}{4} \gcc |t|^4
	+
\frac{1}{2} (\nu + d_0) |t|^2,
	\quad
t \in \R^n.
\end{equation}
\todo{Attach graph.}
When $\nu + d_0 < 0$, the potential has roots at $0$ and $\pm\sqrt{-2 (\nu + d_0) / \gcc}$.
It follows that the Gibbs measure for $|\varphi|^4$ model converges weakly to the
Gibbs measure for the $O(n)$ model in the limit $\gcc\to\infty$
with $\nu = -(d_0 + \gcc / 2)$.
\end{example}

%%%%%%%%%%%%%%%%%%%%%%%%%%%%%%%%%%%%%%%%%%%%%%%%%%%%%%%%%%%%%%%%%%%%%%%%%%%%%%%

% \subsection{Gaussian measures and the free field}

\begin{example}[The Gaussian free field]
Let $S = \R^n$ and let $C$ be a positive-definite symmetric $\vertices\times\vertices$
matrix. The $n$-component \emph{Gaussian measure} $d\mu_C$ on
$\Omega$ with mean $0$ and \emph{covariance} $C$ is defined by the Hamiltonian
\begin{equation}
H_C(\varphi) = \frac{1}{2} \varphi \cdot C^{-1} \varphi.
\end{equation}
Setting $\beta = 1$ again, the partition function $Z_C$ can be computed explicitly,
giving
\begin{equation}
Z_C
  =
\frac{1}{\sqrt{\det(2\pi C)}}.
\end{equation}
The correlations can also be computed explictly (this is sometimes known as
\emph{Wick's theorem}). In particular, the two-point function is the covariance:
\begin{equation}
\int \varphi_a \cdot \varphi_b \; d\mu(\varphi) = C_{ab}.
\end{equation}

An important case is the \emph{massive Gaussian free field} on $\graph$,
which is the $\gcc = 0$ case of the $|\varphi|^4$ model (with $\nu$ necessarily positive).
Thus, the the covariance is equal to the massive Green function $C = (-\lap + \nu)^{-1}$.
\end{example}

%%%%%%%%%%%%%%%%%%%%%%%%%%%%%%%%%%%%%%%%%%%%%%%%%%%%%%%%%%%%%%%%%%%%%%%%%%%%%%%

\subsection{The infinite-volume limit}

\todo{Briefly discuss the need for infinite-volume in terms of vanishing magnetization
and analytic free energy (stress the latter). Define infinite-volume quantities we will
need on $\Zd$. Mention relation to theory of Gibbs measure.}

The presence of a phase transition in a physical system is signalled by an abrupt
(i.e.\ non-analytic) change in an observable quantity (usually the free energy) as a
parameter (usually an external field) is varied. We make a broad distinction between
\emph{first-order} phase transitions in which the free energy has discontinuous first
derivative (with respect to an external field $h$) and \emph{continuous} phase transitions,
in which the free energy is differentiable but non-analytic.

The spin systems we have defined above all
have smooth free energy since the Hamiltonians are smooth functions. The reason we cannot
detect a phase transition in these systems is that they have been defined on finite volumes.
Thus, in order to study phase transitionns, we are forced to face the problem of defining
spin systems on an infinite graph.

A natural approach to defining such systems is via a procedure known as the
\emph{infinite-volume limit}, which we describe here. For any
finite subgraph $\Lambda \subset \graph$, let $H_\Lambda$ be the Hamiltonian
of one of the above spin systems on $\Lambda$ and let $\mu_\Lambda$ be the
corresponding Gibbs measure (which we can view as a measure on the full state
space $\Omega$).

Let $\Lambda_N \subset \graph$ be a sequence of subgraphs that exhaust
$\graph$, i.e.\ $\Lambda_N \uparrow \graph$. If the limits
\begin{equation}
\lim_{N\to\infty} \int f \; d\mu_{\Lambda_N}
\end{equation}
exist for a sufficiently rich class of functions $f$ (such as all bounded continuous functions), then they define a measure $\mu$ on $\Omega$ (with $\mu(f)$ the above limit), which we call a \emph{Gibbs state} or
\emph{infinite-volume Gibbs measure} on $\Omega$.

We remark that there is a more general approach to the study of spin systems in infinite volume developed by Dobrushin, Lanford, and Ruelle. We do not detail their approach here, but merely mention that, for the examples above, this approach involves defining a Gibbs measure for the collection
$H = (H_\Lambda)$ of Hamiltonians directly as a measure on $\Omega$ satisfying a system of constraints on its conditionals measures. This is somewhat in the spirit of Kolmogorov's consistency conditions with the importance difference that the resulting collection $\gibbs_\beta(H)$ of Gibbs states at inverse temperature
$\beta$ need not consist of only a single element. This is significant due to the interpretation of distinct elements of $\gibbs(H)$ as corresponding to different phases of the system under consideration.

For many models, including the $O(n)$ and $n$-component $|\varphi|^4$ models on $\Zd$ with $d > 2$, it is known that there exists a critical inverse temperature $\beta_c < \infty$ such that
$|\gibbs_\beta(H)| = 1$ if and only if $\beta \le \beta_c$ (\REF). In this thesis, our main concern is with the behaviour at $\beta_c$ and as
$\beta \uparrow \beta_c$. Thus, we need not concern ourselves with the precise nature of the infinite-volume limit.

\subsubsection{Translation-invariant systems}

Our interest will be in translation-invariant systems on $\Zd$, for which a particular approach to the infinite-volume limit will be convenient. Namely, for $L > 1$ we will let $\Lambda_N = \Zd/L^N\Zd$ be the discrete torus, which we view as a subset of $\Zd$. This allows us to preserve
translation-invariance of these models in finite volume. Strictly speaking, $\Lambda_N$ is not a subgraph of $\Zd$; nevertheless, it can be shown that the infinite-volume limit (if it exists) is a Gibbs measure in the usual sense; see \cite[Example 4.20]{Georgii11} for details.


We study a generalization of the $|\varphi|^4$ model defined by the Hamiltonian
\begin{equation}
\label{e:Vdef1}
V_{\gcc,\gamma,\nu,N}(\varphi)
	=
\sum_{x\in\Lambda_N}
\Big(
	\tfrac{1}{4} \gcc |\varphi_x|^4
		+
	\tfrac{1}{2} \nu |\varphi_x|^2
		+
	\tfrac{1}{2} \varphi_x \cdot (-\Delta \varphi)_x
		-
	\tfrac{1}{2 d} \gamma (\nabla |\phi_x|^2)^2
\Big),
\end{equation}
where
\begin{equation}
(\nabla |\phi_x|^2)^2
	=
\sum_{|e|=1} (\nabla^e |\phi_x|^2)^2.
\end{equation}
The expectation with respect to the corresponding Gibbs measure will
be denoted $\langle\cdot\rangle_{\gcc,\gamma,\nu,N}$.
We define the two-point function
\begin{equation}
\label{e:two-point-function-phi4}
G_{x, N}(g,\gamma,\nu; n)
	=
\frac{1}{n} \pair{\varphi_0 \cdot \varphi_x}_{g,\gamma,\nu,N},
	\quad
G_x(g,\gamma,\nu; n)
	=
\lim_{N \to \infty} G_{x, N}(g,\gamma,\nu; n).
\end{equation}
In the above limit, we identify a point $x \in \Zd$ with $x \in \Lambda_N$
for large $N$, by embedding the vertices of $\Lambda_N$ as an approximately
centred cube in $\Z^d$ (say as $[-\frac12 L^N+1,\frac12 L^N]^d \cap \Z^d$ if $L^N$ is even
and as $[-\frac12 (L^N-1), \frac12 (L^N-1)]^d \cap \Z^d$ if $L^N$ is odd).

The susceptibility and correlation length of order $p$ are defined by
\begin{equation}
\chi(\gcc, \gamma, \nu; n)
	=
\lim_{N\to\infty} \sum_{x\in\Lambda_N} G_{x,N}(\gcc, \gamma, \nu; n)
\end{equation}
and
\begin{equation}
\xi_p(\gcc, \gamma, \nu; n)
	=
\left(
\frac{\sum_{x\in\Zd} |x|^p G_x(\gcc, \gamma, \nu; n)}{\chi(\gcc, \gamma, \nu; n)}
\right)^{\tfrac{1}{p}}.
\end{equation}

%%%%%%%%%%%%%%%%%%%%%%%%%%%%%%%%%%%%%%%%%%%%%%%%%%%%%%%%%%%%%%%%%%%%%%%%%%%%%%%
%%%%%%%%%%%%%%%%%%%%%%%%%%%%%%%%%%%%%%%%%%%%%%%%%%%%%%%%%%%%%%%%%%%%%%%%%%%%%%%

\section{Critical behaviour and universality}

For this discussion, let us restrict our attention to the graph $\graph = \Zd$
with $d > 1$ and with $J_{xy} \in \{ 0, 1 \}$. Thus, $\lap = \Delta$.

\subsection{The critical point}

Many systems exhibit \emph{critical behaviour} at or near a particular parameter value.
For concreteness, let us discuss the $|\varphi|^4$ model. For convenience, we will
actually discuss a generalization of the $|\varphi|^4$ model. \todo{Define it and
the susceptibility, etc.}

Let
\begin{equation}
\nu_c = \nu_c(\gcc, \gamma, n) = \inf \{ \nu : \chi(g, \gamma, \nu; n) < \infty \}.
\end{equation}
Since the susceptibility can be written as the second derivative of the free energy
with respect to an external field, there is a second-order phase transition at $\nu_c$.
By the expression for the susceptibility in terms of the two-point function, it is
reasonable to expect rapid (i.e.\ summable) decay of $G_x(g, \gamma, \nu; n)$ in $|x|$ for
$\nu > \nu_c$ and much slower decay at $\nu = \nu_c$. In fact, the two-point function
is expected to decay exponentially above $\nu_c$ and sub-exponentially at $\nu_c$.
The \emph{correlation length} $\xi$ is defined to be the reciprocal of the exponential
rate of decay of the two-point function; concretely, we let
\begin{equation}
\xi(g, \gamma, \nu; n) = \limsup_{k\to\infty} \frac{-k}{\log G_{ke}(g, \gamma, \nu; n)},
\end{equation}
where $e \in \Zd$ is a unit vector. Roughly speaking, the correlation length acts as
a ``macroscopic length scale'' of the model; it is a measure of the largest scale at
which spins are strongly correlated.
Based on the above discussion, we expect $\xi$
to diverge as $\nu\downarrow\nu_c$. This divergence is usually seen as one of the
key features of critical behaviour and is indicative of strong correlations at all
scales. A related quantity is the \emph{correlation length of order $p$}, defined by
\begin{equation}
\xi_p(g, \gamma, \nu; n)
	=
\left(\frac{\sum_{x\in\Zd} |x|^p G_x(g, \gamma, \nu; n)}{\chi(g, \gamma, \nu; n)}\right)^{1/p}.
\end{equation}

\begin{rk}
There is a simple heuristic relationship between $\xi$ and $\xi_p$: Let us suppose that
the two-point function decays exponentially at rate $1/\xi$, possibly with some
sub-exponential multiplicative correction; for instance, suppose that
\begin{equation}
G_x(g, \gamma, \nu; n) \approx C |x|^{-\alpha} e^{-|x|/\xi(g, \gamma, \nu; n)},
\end{equation}
in some sense, where $\alpha$ and $C$ are positive constants independent of $\nu$.
Then the main contributions to the sum in the numerator in the definition of
$\xi_p$ should come from $|x| \le \xi = \xi(g, \gamma, \nu; n)$. For such $|x|$,
$G_x(g, \gamma, \nu; n) \approx C |x|^{-\alpha}$ and so
\begin{equation}
\sum_{x\in\Zd} |x|^p G_x(g, \gamma, \nu; n)
	\approx
C \sum_{|x| \le \xi} |x|^{-(\alpha-p)}
	\approx
C \xi(g, \gamma, \nu; n)^{-(\alpha-p)}.
\end{equation}
By definition, it follows that
\begin{equation}
\xi^p_p(g, \gamma, \nu; n) \approx \xi^p(g, \gamma, \nu; n).
\end{equation}
\end{rk}

\subsection{Critical exponents}

For simplicity, let us drop $\gcc$, $\gamma$, and $n$ from the notation.
It is predicted that there exist constants $\eta$, $\gammabar$, and $\nubar$,
known as \emph{critical exponents}, such that
\begin{align}
G_x(\nu_c)
	&\sim
C_1 |x|^{-(d - 2 + \eta)},
	\qquad
|x|\to\infty \\
\chi(\nu)
	&\sim
C_2 (\nu - \nu_c)^{-\gammabar},
	\qquad
\nu\downarrow\nu_c \\
\xi(\nu)
	&\sim
C_3 (\nu - \nu_c)^{-\nubar},
	\qquad
\nu\downarrow\nu_c \\
\xi_p(\nu)
	&\sim
C_4 (\nu - \nu_c)^{-\nubar},
	\qquad
\nu\downarrow\nu_c.
\end{align}
where $a \sim b$ means that $\lim (a/b) = 1$ and $C_i$ for $i = 1,2,3,4$
are constants that may depend on $g$ and $n$ (and $p$ when $i = 4$).
The critical exponents are expected to be \emph{universal} in the sense that they
only depend on ``large-scale properties'' of the model such as the global geometry
of the underlying graph and the symmetries of the Gibbs measure. In particular,
for the $n$-component $|\varphi|^4$ model on $\Zd$, these exponents should only
depend on $n$ and $d$ and independent of $g$ and $\gamma$ when $g > 0$ and $\gamma$
is sufficiently small (depending on $g$). In fact,
analogous relations are expected to hold for the $O(n)$ spin model, with the
\emph{same} critical exponents.

These and other relations are all believed to be manifestations of the existence of
a universal \emph{scaling limit} for the $|\varphi|^4$ model and other models in its
\emph{universality class}. That is, any spin system in this class, when appropriately
rescaled, is expected to converge in distribution to a unique continuum random field.
In this sense, the study of critical behaviour involves a set of far-reaching
generalizations of the central limit theorem.

\begin{example}[The Gaussian free field]
\todo{Needs work}

We have
\begin{equation}
-\Delta = 2 d (1 - P),
\end{equation}
where $P = (2 d)^{-1} J$ is the transition matrix for the simple random walk $X$ on $\Zd$.
For $d > 2$, the simple random walk is transient. Thus, letting $E_0$ denote the expectation
of $X$ conditioned so that $X_0 = 0$,
\begin{equation}
(1 - P)^{-1}_{0x} = \sum_{n=0}^\infty P^n_{0x} = E_0 \sum_{n=0}^\infty \1_{X_n=x} < \infty
\end{equation}
and we can define the \emph{massless} Gaussian free field on $\Zd$ to be the Gaussian
measure with covariance $(-\Delta)^{-1} = (2 d)^{-1} (1 - P)^{-1}$.

The two-point function is just the massive Green function
$(-\Delta + m^2)^{-1}$ which, on $\Zd$, has the well-known Ornstein-Uhlenbeck decay \todo{(show this; use random walks?)}. Moreover,
\begin{equation}
\chi
  =
\sum_{x\in\vertices} (-\Delta + m^2)^{-1}_{0x}
  =
\sum_{x\in\vertices} \sum_{n=0}^\infty z^n P^n_{0x}
  =
\sum_{n=0}^\infty z^n
  =
(1 - z)^{-1}.
\end{equation}
Thus, there is a critical point at $m^2 = 0$ ($z = 1$).

\todo{See candidacy report or preliminary version of it. For the Green function, see Theorem 1.5.4 in Lawler--Intersections of Random Walks}
\end{example}

%%%%%%%%%%%%%%%%%%%%%%%%%%%%%%%%%%%%%%%%%%%%%%%%%%%%%%%%%%%%%%%%%%%%%%%%%%%%%%%

\subsection{The upper-critical dimension}

The critical exponents are conjectured to take on the following values:
\begin{equation}
\eta =
	\begin{cases}
	\frac{5 + n}{24},		& d = 2, \, n = 1 \\
	\approx 0.03,			& d = 3, \\
	0,						& d \ge 4
	\end{cases}
\qquad
\gamma =
	\begin{cases}
	\frac{43 + 13 n}{32},	& d = 2, \, n = 1 \\
	\approx 1,				& d = 3 \\
	1,						& d \ge 4
	\end{cases}
\end{equation}
and
\begin{equation}
\nubar =
	\begin{cases}
	\frac{3 + n}{4},		& d = 2, \, n = 1 \\
	\approx 0.6,			& d = 3 \\
	\frac{1}{2},			& d \ge 4
	\end{cases}
\end{equation}
with logarithmic corrections in $d = 4$.
Thus, it is expected that, when $d > 4$, the critical exponents cease to depend
on the dimension and $n$. In fact, they are expected to equal the exponents of
the corresponding non-interacting model, the Gaussian free field.

This phenomenon is known as \emph{mean-field behaviour} and dimension $4$ is
called the \emph{upper-critical dimension} for this class of models. The behaviour
of models is generally better understood above than below their upper-critical
dimension.

%%%%%%%%%%%%%%%%%%%%%%%%%%%%%%%%%%%%%%%%%%%%%%%%%%%%%%%%%%%%%%%%%%%%%%%%%%%%%%%

\subsection{The renormalisation group}

\todo{General idea; Kadanoff decimation; fixed points and stable manifolds;
dependence on dimension}

%%%%%%%%%%%%%%%%%%%%%%%%%%%%%%%%%%%%%%%%%%%%%%%%%%%%%%%%%%%%%%%%%%%%%%%%%%%%%%%
%%%%%%%%%%%%%%%%%%%%%%%%%%%%%%%%%%%%%%%%%%%%%%%%%%%%%%%%%%%%%%%%%%%%%%%%%%%%%%%

\section{Walks}

\subsection{Discrete-time walks}

For simplicity, suppose that $J_{xy} \in \{ 0, 1 \}$ and that $\graph$ is
$d_0$-regular.

Let $[n] = \{ 0, \ldots, n \}$.
An $n$-step \emph{walk} is a function $\omega : [n] \to \vertices$ with
$\omega_i \sim \omega_{i+1}$ for all $i \in [n]$. Let $\dwalks_n$ denote the collection
of $n$-step walks $\omega$ with $\omega_0 = 0$ and let $\dwalks = \bigcup_n \dwalks_n$.
Given $\omega\in\dwalks_n$, we write $|\omega| = n$.
A model of discrete-time walks is defined by a weight function $w : \dwalks \to \R$.

The \emph{generating function} of a sequence $c_n$ is the function defined by
the power series
\begin{equation}
\sum_n c_n z^n.
\end{equation}
Let $\dwalks_n(x)$ denote the collection of walks $\omega\in\dwalks_n$ with $\omega_n = x$
and set $\dwalks(x) = \bigcup_n \dwalks_n(x)$.
The two-point function $G_x$ for walks with weight function $w$ is the generating
function of the sequence
\begin{equation}
c_n(x) = \sum_{\omega\in\dwalks_n(x)} w(\omega).
\end{equation}
That is,
\begin{equation}
G_x = \sum_n z^n c_n(x)
	= \sum_{\omega\in\dwalks(x)} w(\omega) z^{|\omega|}.
\end{equation}
The susceptibility is the generating function for $c_n = \sum_x c_n(x)$:
\begin{equation}
\chi = \sum_n c_n z^n = \sum_x G_x.
\end{equation}

We define a probability measure $\mu_n$ on $\dwalks_n$ by $\mu(\omega) = w(\omega) / c_n$.

%%%%%%%%%%%%%%%%%%%%%%%%%%%%%%%%%%%%%%%%%%%%%%%%%%%%%%%%%%%%%%%%%%%%%%%%%%%%%%%

\begin{example}[Simple random walk]
The (discrete-time) simple random walk $X_n$ is the Markov chain on $\graph$ with
transition matrix $P = d_0^{-1} J$. This is a model of walks in the sense above
with $w(\omega) = 1$ for all $\omega$.

There is a close relationship between the
simple random walk and the Gaussian free field. Since $P$ is a stochastic matrix,
the two-point function converges for $|z| < 1$. For $0 < z < 1$, we can write
$z = d_0 / (d_0 + m^2)$ with $m^2 > 0$ to get
\begin{equation}
G_x = \sum_n P^n_{0x} z^n = (2 d + m^2) (-\Delta + m^2)^{-1}_{0x}.
\end{equation}
In other words, the two-point function for simple random walk is the two-point
function for the Gaussian free field.

When $z = 1$, the power series above is the expected number of visits $X$ makes
to $x$ when started at $0$. Thus, $G_x$ converges at $z = 1$ when $X$ is transient.
% Since $\sum_x P^n_{0x} = 1$
\end{example}

%%%%%%%%%%%%%%%%%%%%%%%%%%%%%%%%%%%%%%%%%%%%%%%%%%%%%%%%%%%%%%%%%%%%%%%%%%%%%%%

% \subsection{Self-avoiding walk}

\begin{example}[Self-avoiding walk]
A walk $\omega$ is said to be \emph{self-avoiding} if $\omega_i \ne \omega_j$
for all $i \ne j$. Let $\Scal_n \subset \dwalks_n$ denote the set of $n$-step
self-avoiding walks starting at $0$. We define the weights $w(\omega) = \1_{\omega\in\Scal_n}$.
Then $c_n(x)$ is the number of $n$-step self-avoiding walks from $0$ to $x$
and $c_n = |\Scal_n|$.

Note that the sequence $c_n$ is sub-multiplicative: $c_{m+n} \le c_m c_n$.
Thus, Fekete's lemma implies that the existence of the \emph{connective constant}
$c(\graph)$ of $\graph$, defined by
\begin{equation}
c(\graph) = \lim_{n\to\infty} n^{-1} \log c_n.
\end{equation}
Note that, by the trivial bounds $1 \le c_n \le d_0^n$ for $n \ge 1$, $c(\graph) \in [0, d_0]$.
% Roughly speaking, this means that $c_n \approx c(\graph)^n$ for large $n$.
By definition, the susceptibility has radius of convergence $c(\graph)^{-1}$.
Since $c_n \ge 0$ for all $n$, the susceptibility diverges at $c(\graph)^{-1}$.

The measure $\mu_n$ on $\dwalks_n$ is the uniform measure.
These measures do not form a consistent family due to the possibility of ``traps''. That is, the equality
\begin{equation}
\mu_{|\omega|}(\omega) = \sum_{\tilde\omega \supset \omega} \mu_{|\tilde\omega|}(\tilde\omega)
\end{equation}
does not hold for all $\omega\in\dwalks$ (the sum here is over all self-avoiding walks extending $\omega$).
% \begin{wrapfigure}{R}{0.4\textwidth}
% \vspace{-0.5cm}
% \begin{center}
%   \includegraphics[width=0.3\textwidth]{trapped}
%   \caption{A trapped self-avoiding walk}
%   \label{fig:trap}
% \end{center}
% \vspace{-0.5cm}
% \end{wrapfigure}
For instance, consider the self-avoiding walk $\omega\in\dwalks_7$ on $\Zd$ in
Figure~\ref{fig:trap}. This walk has positive probability under $\mu_7$ but,
since there are no self-avoiding walks extending $\omega$, the sum on the 
right-hand side above is $0$.

As a result, the methods of stochastic processes cannot be directly used to
study the self-avoiding walk. The existence of traps also contributes to the
combinatorial difficulty of studying self-avoiding walk; for instance, it is
not clear how to express $c_{n+1}$ (the number of $(n+1)$-step self-avoiding walks)
in terms of $c_n$.

% \begin{figure}[!htb]
% \centering
% \caption{A trapped self-avoiding walk}
% \includegraphics{trapped}
% \end{figure}
\end{example}

%%%%%%%%%%%%%%%%%%%%%%%%%%%%%%%%%%%%%%%%%%%%%%%%%%%%%%%%%%%%%%%%%%%%%%%%%%%%%%%

\subsection{Gibbs measures on walks}

For any right-continuous function $\omega : [0, T] \to \vertices$, we define
$\tau_n = \tau_n(\omega)$ inductively by setting $\tau_0 = 0$ and
\begin{equation}
\tau_{n+1} = \inf(t > \tau_n : \omega_t \ne \omega_{\tau_n}).
\end{equation}
We call $\omega$ a \emph{walk} of length $T$ if
$\{ \tau_n : n \in \Z_+ \}$ does not have any cluster points and
$\omega_{\tau_n} \sim \omega_{\tau_{n+1}}$ for all $n$; thus, $\omega$
only jumps between neighbouring vertices.

A model of walks is determined by a choice of finite measure $d\mu_T$ on
$\cwalks_T$ for each $T$. Our focus will be on canonical Gibbs measures of the form
\begin{equation}
\int f \; d\mu_T = \frac{1}{c_T} E_0 (f e^{-H_T}),
\end{equation}
where $E_0$ is the expectation for a random walk on $\graph$ (in discrete or continuous time), $H_T : \cwalks_T \to \R$, and we have denoted the canonical partition function by $c_T$.
% \begin{equation}
% d\mu_T(\omega) = \frac{1}{c_T} e^{-H_T(\omega)} \; d\lambda_T(\omega),
% \end{equation}
% with respect to some measure $\lambda_T$ on $\cwalks_T$ (we have denoted the partition function by $c_T$).
% We will assume that models of discrete-time walks satisfy $\mu_T = \mu_{\lfloor T \rfloor}$ for all $T$
% (note that both are measures on $\cwalks_{\lfloor T \rfloor}$ in this case).
Given such a model, the corresponding grand canonical ensemble is given by
\begin{equation}
d\mu(\omega)
  =
\frac{1}{\chi(\nu)}
e^{-\nu |\omega| - H_{|\omega|}(\omega)}
d\lambda_{|\omega|}(\omega)
\end{equation}
and the grand canonical partition function, denoted $\chi(\nu)$, is known as the \emph{susceptibility}. In this setting, the fugacity $\nu$ is usually referred to as the \emph{killing rate}.
% The grand canonical partition function, denoted $\chi$, is known as the \emph{susceptibility}.
Note that
\begin{equation}
\mu(\cdot \mid \cwalks_T) = \mu_T.
\end{equation}

Let $\cwalks_T(x) \subset \cwalks_T$ denote the set of walks in $\cwalks_T$ from $0$ to $x$ and let $\cwalks(x) = \bigcup_{T \ge 0} \cwalks_T(x)$. We define the
\emph{two-point function}
\begin{equation}
G_x(\nu) = \mu(\cwalks(x)) \int_0^\infty e^{-\nu T} c_T(x) \; dT,
\end{equation}
where
\begin{equation}
c_T(x) = \int_{\cwalks_T(x)} e^{-H_T} \; d\lambda_T.
\end{equation}
Note that
\begin{equation}
\chi(\nu) = \sum_{x\in\vertices} G_x(\nu),
\end{equation}
which is consistent with the analogous relation for spin systems. Later we will discuss the relationship between the two-point function for walks and spin systems.

The \emph{critical point} $\nu_c$ for a model of walks is defined
\begin{equation}
\nu_c = \inf (\nu : \chi(\nu) < \infty).
\end{equation}

%%%%%%%%%%%%%%%%%%%%%%%%%%%%%%%%%%%%%%%%%%%%%%%%%%%%%%%%%%%%%%%%%%%%%%%%%%%%%%%

\subsection{Continuous-time simple random walk}

Let $\generator = -\lap$
Then $\generator$ is the generator of the $\vertices$-valued Markov process
$X = (X_t)_{t \ge 0}$ with transition probabilities
\begin{equation}
\Pr(X_t = y \mid X_0 = x) = (e^{-t \lap})_{xy},
\end{equation}
called the \emph{continuous-time simple random walk} on $\graph$.
We define the measure $\lambda_T$ on $\cwalks_T$ by
\begin{equation}
\lambda_T(d\omega) = \Pr(X_t = d\omega_t, \, t \le T \mid X_0 = 0).
\end{equation}
For the corresponding Gibbs measures (with $H_T \equiv 0$), we have
\begin{equation}
c_T(x) = (e^{-t \lap})_{0x},
  \quad
G_x = (\nu - \lap)^{-1}_{0x}
\end{equation}
for $\nu > 0$. Thus, the two-point function is the Green function for the
massive Laplacian.

\begin{example}
The discrete-time simple random walk is the Markov chain given by
$X_n = X_{\tau_n}$. Consider the discrete-time simple random walk on $\Zd$,
for which $X_n = X_0 + \sum_{i=1}^n Y_i$, where $Y_i$ are iid random variables
in $\{ e \in \Zd : |e| = 1 \}$ with $\Ex Y_i = 0$. Thus, $\Ex |X_n|^2 = n$ and
the central limit theorem implies that
$X_n / \sqrt n \Rightarrow \normal(0, 1)$. More generally, Donsker's invariance
principle states that a rescsaled version of $X$ converges to Brownian motion on $\Rd$.
\end{example}

%%%%%%%%%%%%%%%%%%%%%%%%%%%%%%%%%%%%%%%%%%%%%%%%%%%%%%%%%%%%%%%%%%%%%%%%%%%%%%%

\subsection{Weakly self-avoiding walk with self-attraction}

Define the \emph{local time} up to time $T$ of $\omega \in \cwalks$ at
$x \in \vertices$ by
\begin{equation}
\label{e:LTx-def}
\lt^x_T(\omega) = \int_0^T \1_{\omega(S)=x} \; dS.
\end{equation}
In the discrete-time case, $\lt^x_n$ is the number of times $\omega$ visits $x$
and is bounded by $n$. In the continuous-time case, $\lt^x_T$ is almost surely
finite for the continuous-time simple random walk.

We define the \emph{intersection local time}
\begin{equation}
\label{e:ITdef}
I_T(\omega) = \sum_{x\in\vertices} (\lt^x_T)^2
  =
\int_0^T \!\! \int_0^T \1_{\omega(S_1)=\omega(S_2)} \; dS_1 dS_2
\end{equation}
and the \emph{contact self-attraction}
\begin{equation}
\label{e:CTdef}
C_T(\omega)
	=
\sum_{x \in \vertices} \sum_{y \sim x} \lt_T^x(\omega) \lt_T^y(\omega)
	=
\int_0^T ds \int_0^T dt \; \1_{\omega_s \sim \omega_t}
\end{equation}
up to time $T$.
% Recall that we have set the inverse temperature equal to $1$.
Given a parameter $\gcc > 0$, and $\gamma \in \R$, let
\begin{equation}
\label{e:Udef-neg}
U_{\gcc,\gamma}(f)
=
\gcc \sum_{x\in\vertices} f_x^2
- \frac{\gamma}{2d}
\sum_{x\in\vertices} \sum_{y \sim x} f_x f_y
\end{equation}
for $f : \vertices \to \R$.
The \emph{weakly self-avoiding walk with self-attraction} (WSAW-SA) is defined via the Hamiltonian
\begin{equation}
U_{\gcc,\gamma,T} = U_{\gcc,\gamma} \circ L_T.
\end{equation}
We denote the canonical partition function by
\begin{equation}
c_T = c_T(\gcc, \gamma) = E_0 \left( e^{-\gcc I(T) + \gamma C(T)} \right),
\end{equation}
where $0 \in \vertices$ is fixed, and the susceptibility by
\begin{equation}
\chi(\gcc, \gamma, \nu) = \int_0^\infty c_T e^{-\nu T} \; dT.
\end{equation}

In the case $\gamma = 0$, the discrete-time version of this model is known as
the \emph{Domb-Joyce model}. In continuous-time, it is the
\emph{continuous-time weakly self-avoiding walk} (WSAW).

%%%%%%%%%%%%%%%%%%%%%%%%%%%%%%%%%%%%%%%%%%%%%%%%%%%%%%%%%%%%%%%%%%%%%%%%%%%%%%%

\subsection{Predicted behaviour}

\todo{State expected critical behaviour.}

\todo{
For walks, it is expected that
\begin{align}
c_T                       &\sim C_5 e^{-\nu_c T} T^{-\gammabar}, \\
\langle |X_T|^2 \rangle   &\sim C_6 T^{-\nubar}.
\end{align}
\todo{Motivate the above with an example somewhere.}}

%%%%%%%%%%%%%%%%%%%%%%%%%%%%%%%%%%%%%%%%%%%%%%%%%%%%%%%%%%%%%%%%%%%%%%%%%%%%%%%
%%%%%%%%%%%%%%%%%%%%%%%%%%%%%%%%%%%%%%%%%%%%%%%%%%%%%%%%%%%%%%%%%%%%%%%%%%%%%%%

\section{Main results}

The main result is stated below.
\todo{Discuss existence of $\nu_c$}

\begin{theorem} \label{thm:suscept}
  Let $d = 4$ and $n \ge 0$. For $L$ sufficiently large (depending on $n$),
  there exist $\gcc_* > 0$
  and a positive function $\gamma_* : (0, \gcc_*) \to \R$
  such that whenever $0 < \gcc < \gcc_*$ and $|\gamma| < \gamma_*(\gcc)$,
  there are constants $A_{\gcc,\gamma,n}$ and $B_{\gcc,\gamma,n}$ such that the following hold:

  \smallskip\noindent
  (i)
  The critical two-point function decays as
  \begin{equation}
    G_x(\gcc,\gamma,\nu_c; n)
        =
    A_{\gcc,\gamma} |x|^{-2} (1 + O((\log |x|)^{-1}))
        \quad
    \text{as $|x|\to\infty$},
  \end{equation}
  with $A_{\gcc,\gamma} = (4 \pi)^{-2} (1 + O(\gcc))$ as $\gcc \downarrow 0$.

  \smallskip\noindent
  (ii)
  The susceptibility diverges as
  \begin{equation} \label{e:chieps-asympt}
    \chi(\gcc, \gamma, \nu_c + \varepsilon; n)
      \sim B_{\gcc,\gamma,n} \varepsilon^{-1} (\log \varepsilon^{-1})^{(n+2)/(n+8)},
    \quad \varepsilon\downarrow 0
  \end{equation}
  with $B_{\gcc,\gamma,n} = ((n + 8) \gcc / 16\pi^2)^{(n+2)/(n+8)} (1 + O(\gcc))$
  as $\gcc \downarrow 0$.

  \smallskip\noindent
  (iii) For any $p >0$, if $L$ is chosen large and $\gcc_*$ small both depending on $p$,
  then the correlation length of order $p$ diverges as
  \begin{equation} \label{e:xieps-asympt}
    \xi_p(\gcc, \gamma, \nu_c + \varepsilon; n)
     \sim B_{\gcc,\gamma,n}^{1/2} {\sf c}_p \varepsilon^{-1/2} (\log \varepsilon^{-1})^{(n+2)/2(n+8)},
    \quad \varepsilon\downarrow 0
  \end{equation}
  with
  \begin{equation}
  {\sf c}_p^p = \int_{\R^4} |x|^p (-\Delta_{\R^4} + 1)^{-1}_{0x} \; dx.
  \end{equation}
\end{theorem}

The $\gamma = 0$ cases of (i) and (ii) were proved by Bauerschmidt, Brydges, and
Slade. The $n > 0$ case with $\gamma \ne 0$ is a new result in this thesis. We
will only discuss the proof of the $\gamma \ge 0$ case, which is of primary
interest. The proof of the $\gamma < 0$ case with $n = 0$ can be found in
\cite{BSW-saw-sa} and the extension to $n > 0$ is straightforward.

%%%%%%%%%%%%%%%%%%%%%%%%%%%%%%%%%%%%%%%%%%%%%%%%%%%%%%%%%%%%%%%%%%%%%%%%%%%%%%%
%%%%%%%%%%%%%%%%%%%%%%%%%%%%%%%%%%%%%%%%%%%%%%%%%%%%%%%%%%%%%%%%%%%%%%%%%%%%%%%

\section{Spin systems and models of walks}

We have already discussed the close relationship between the simple random walk
and the Gaussian free field, which ultimately stems from the representation of
matrix powers in terms of walks and which should be somewhat familiar to anyone
who has studied Markov chains. Namely, if $M$ is a $\vertices\times\vertices$
matrix, then
\begin{equation}
M^n_{ab} = \sum_{x_1,\ldots,x_n\in\vertices} M_{ax_1} M_{x_1x_2} \ldots M_{x_nb}.
\end{equation}
When $M$ is indexed by edges (i.e.\ when $M_{xy} = 0$ for $x \not\sim y$), the
above sum can be replaced by a sum over $n$-step walks from $a$ to $b$ on $\graph$.
When the entries of $M$ are non-negative, such a sum acquires a probabilistic
interpretation as an expectation with respect to the random walk whose steps
are weighted by the entries of $M$.

This idea can be extended to obtain relationships between certain models of
interacting walks and spin systems.

%%%%%%%%%%%%%%%%%%%%%%%%%%%%%%%%%%%%%%%%%%%%%%%%%%%%%%%%%%%%%%%%%%%%%%%%%%%%%%%

\subsection{High-temperature expansion of the \texorpdfstring{$O(n)$}{O(n)} model}

The high-temperature expansion of a spin system is based on the expansion of the
Boltzmann weight $e^{\beta H}$ around $\beta = 0$.
For the $O(n)$ spin model, an uncontrolled high-temperature expansion yields
\begin{align}
Z 	&= \int d\lambda(\sigma) \prod_{xy\in\edges} e^{\beta \sigma_x \cdot \sigma_y} \\
	&\approx \int d\lambda(\sigma) \prod_{xy\in\edges} (1 + \beta \sigma_x \cdot \sigma_y) \\
	&= \sum_{E \subset \edges} \beta^{|E|} \int d\lambda(\sigma) \prod_{xy \in E} \sigma_x \cdot \sigma_y
\end{align}
By reflection-invariance of the sphere measure, the last integral above is non-zero
if and only if every vertex in the product over $E$ appears an even number of times;
thus, the sum over subsets of $\edges$ can be replaced by a sum over collections of
loops (walks from a vertex to itself) in $\graph$.

A similar expansion can be performed for the numerator in the definition of the
two-point function, yielding:
\begin{equation}
\int d\lambda(\sigma) \sigma_a \cdot \sigma_b
	\approx
\sum_{E\subset\edges} \beta^{|E|}
\int d\lambda(\sigma) \sigma_a \cdot \sigma_b
\prod_{xy\in E} \sigma_x \cdot \sigma_y.
\end{equation}
Once again, every vertex must appear twice on the right-hand side. This time,
there are additional non-zero contributions from subsets $E$ of edges containg
a path from $a$ to $b$. For instance, if $a \sim b$, there is a non-zero
contribution from $E = \{ \{ a, b \} \}$.

%%%%%%%%%%%%%%%%%%%%%%%%%%%%%%%%%%%%%%%%%%%%%%%%%%%%%%%%%%%%%%%%%%%%%%%%%%%%%%%

\subsection{The \texorpdfstring{$n\to0$}{n approaches 0} limit}

A more careful analysis of the high-temperature expansion of the $O(n)$ model
above will reveal the $n$-dependence of the non-zero contributions to the
partition function and two-point function. When the spins are restricted to
the sphere of radius of $\sqrt n$, it can be shown that formally setting $n = 0$
results in $Z = 1$ and the two-point function receives contributions only from
self-avoiding walks from $a$ to $b$. See \cite[Section 2.3]{MS93} for details.

Based on this idea, De Gennes predicted the critical exponents of the self-avoiding
walk by setting $n = 0$ in the predicted exponents for the $O(n)$ spin model.
This is known as the $n \to 0$ ``limit''.

%%%%%%%%%%%%%%%%%%%%%%%%%%%%%%%%%%%%%%%%%%%%%%%%%%%%%%%%%%%%%%%%%%%%%%%%%%%%%%%

\subsection{Self-avoiding walk representation}
% This section based on saw-sa
\label{sec:intrep}

It is not clear how to make rigorous the $n\to0$ limit of De Gennes. Parisi and
Sourlas and, independently, McKane, discovered an alternative approach to the
predictions of De Gennes. They argued that the weakly
self-avoiding walk \todo{two-point function} could be represented as the two-point
function for a version of the $|\varphi|^4$ model involving both boson and fermion
fields. The formal appearance of $n = 0$ quantities was then explained as a
consequence of a symmetry between the bosons and fermions known as \emph{supersymmetry}.

In this section we describe an integral representation of the of WSAW-SA, which
is a special case of a result of Brydges, Imbrie, and Slade \cite{BIS09}.
We restrict out attention to the graph $\Lambda = \Lambda_N = \Zd/L^N\Zd$.
We begin by introducing the notions of bosons and fermion fields on $\Lambda$.

\subsubsection{Boson and fermion fields}
\label{sec:forms}

We fix $N$ and write $\Lambda = \Lambda_N$.
Given complex variables $\phi_x, \bar\phi_x$
(the boson field) for $x \in \Lambda$,
we define the differentials (the fermion field)
\begin{equation}
\psi_x = \frac{1}{\sqrt{2\pi i}} d\phi_x,
\quad
\bar\psi_x = \frac{1}{\sqrt{2\pi i}} d\bar\phi_x,
\end{equation}
where we fix a choice of complex square root.
The fermion fields are multiplied with each other
via the anti-commutative wedge product,
though we suppress this in our notation.

A differential form that is the
product of a function of $(\phi, \bar\phi)$
with $p$ differentials is said to have degree $p$.
A sum of forms of even degree is said to be \emph{even}.
We introduce a copy $\bar\Lambda$ of $\Lambda$
and we denote the copy of $X \subset \Lambda$ by $\bar X \subset \bar\Lambda$.
We also denote the copy of $x \in \Lambda$
by $\bar x \in \bar\Lambda$ and define $\phi_{\bar x} = \bar\phi_x$ and $\psi_{\bar x} = \bar\psi_x$.
Then any differential form $F$ can be written
\begin{equation}
\lbeq{FinNcal}
F
=
\sum_{\vec y}
F_{\vec y} (\phi, \bar\phi)
\psi^{\vec y}
\end{equation}
where the sum is over finite sequences $\vec y$ over $\Lambda\sqcup\bar\Lambda$,
and $\psi^{\vec y} = \psi_{y_1} \ldots \psi_{y_p}$ when $\vec y = (y_1, \ldots, y_p)$.
When $\vec y = \varnothing$ is the empty sequence,
$F_\varnothing$ denotes the $0$-degree (bosonic) part of $F$.

In order to apply the results of \cite{BBS-saw4-log,BBS-saw4,BSTW-clp}, we require
smoothness of the coefficients $F_{\vec y}$ of $F$.  For Theorem~\ref{thm:suscept}(i,ii),
we need these coefficients to be $C^{10}$, and for Theorem~\ref{thm:suscept}(iii) we require
a $p$-dependent number of derivatives for the analysis of $\xi_p$, as discussed in \cite{BSTW-clp}.
We let $\Ncal^\varnothing$ be the algebra of even forms with sufficiently smooth coefficients
and we let $\Ncal^\varnothing(X) \subset \Ncal^\varnothing$ be the sub-algebra of even forms only depending on fields
in $X$. Thus, for $F \in \Ncal^\varnothing(X)$, the sum in \eqref{e:FinNcal} runs over sequences $\vec y$
over $X \sqcup \bar X$.
Note that $\Ncal^\varnothing = \Ncal^\varnothing(\Lambda)$.


Now let $F = (F_j)_{j \in J}$ be a finite collection of even forms
indexed by a set $J$
and write $F_\varnothing = (F_{\varnothing,j})_{j \in J}$.
Given a $C^\infty$ function $f : \R^J \to \C$, we define
$f(F)$ by its Taylor expansion about $F_\varnothing$:
\begin{equation}
f(F) = \sum_\alpha \frac{1}{\alpha!} f^{(\alpha)}(F_\varnothing) (F - F_\varnothing)^\alpha.
\end{equation}
The summation terminates as a finite sum,
since $\psi_x^2 = \bar\psi_x^2 = 0$ due to the anti-commut\-ative product.

We define the integral
$\int F$
of a differential form $F$ in the usual way
as the Riemann integral of its top-degree part
(which may be regarded as a function
of the boson field).
In particular, given a positive-definite
$\Lambda \times \Lambda$ symmetric matrix $C$
with inverse $A = C^{-1}$,
we define the \emph{Gaussian expectation}
(or \emph{super-expectation}) of $F$ by
\begin{equation}
\lbeq{ExCF}
\Ex_C F = \int e^{-S_A} F,
\end{equation}
where
\begin{equation}
\label{e:action}
S_A = \sum_{x\in\Lambda} \Big(\phi_x (A\bar\phi)_x + \psi_x (A \bar\psi)_x\Big).
\end{equation}

Finally, for $F = f(\phi, \bar\phi) \psi^{\vec y}$,
we let
\begin{equation}
\theta F = f(\phi + \xi, \bar\phi + \bar\xi) (\psi + \eta)^{\vec y},
\end{equation}
where $\xi$ is a new boson field, $\eta = (2\pi i)^{-1/2} d\xi$ a new fermion field,
and $\bar\xi, \bar\eta$ are the corresponding conjugate fields.
We extend $\theta$ to all $F \in \Ncal^\varnothing$ by linearity
and define the convolution operator $\Ex_C\theta$ by letting
$\Ex_C\theta F \in \Ncal^\varnothing$ denote the Gaussian expectation of $\theta F$ with respect
to $(\xi, \bar\xi, \eta, \bar\eta)$, with $\phi,\phib,\psi,\psib$ held fixed.

\subsubsection{Integral representation of the two-point function}
\label{sec:Gintrep}

An integral representation formula applying to general local time functionals
is given in \cite{BEI92,BIS09}; see also \cite[Appendix~A]{ST-phi4}.
We state the result we need in the proposition below.

Let $\Delta$ denote the Laplacian on $\Lambda$,
i.e.\ $\Delta_{xy}$ is given by the right-hand side of
\eqref{e:Deltaxy} for $x, y \in \Lambda$.
We define the differential forms:
\begin{align}
\label{e:taudef}
\tau_x
	&=
\phi_x \bar\phi_x + \psi_x \bar\psi_x
	\\
\label{e:addDelta}
\tau_{\Delta,x}
	&=
\frac 12
\Big(
	\phi_{x} (- \Delta \bar{\phi})_{x} + (- \Delta \phi)_{x} \bar{\phi}_{x}
		+
	\psi_{x}  (- \Delta \bar{\psi})_{x} + (- \Delta \psi)_{x}  \bar{\psi}_{x}
\Big)
	\\
\label{e:nablatau}
|\nabla \tau_x|^2
	&=
\sum_{|e|=1} (\nabla^e \tau)_x^2.
\end{align}
Let
\begin{equation}
\label{e:Vdef2}
V_{\gcc,\gamma,\nu,N}
	=
\sum_{x\in\Lambda_N}
\Big(
	\gcc \tau_x^2 + \nu \tau_x + \tau_{\Delta,x} - \tfrac{1}{2 d} \gamma |\nabla \tau_x|^2
\Big)
\end{equation}

\begin{prop}
Let $d > 0$ and $\gcc > 0$. For $\gamma < \gcc$ and $\nu \in \R$,
\begin{align}
\label{e:Grep-pos-bis}
G_{N,\gcc,\gamma,\nu}(a, b)
	&=
\int e^{-U_{\gcc,\gamma,\nu,N}} \bar\phi_a \phi_b.
\end{align}
\end{prop}

\subsubsection{Finite-volume approximation}

In order to make use of the integral representation above, we must approximate the
WSAW-SA on $\Zd$ by a model on $\Lambda_N$.

Let $X^{L^N}$ denote the simple random walk on $\Lambda_N$.
For $F_T = F_T(X)$ any one of the functions $L_T^x,I_T,C_T$
of $X$ defined in \eqref{e:LTx-def}--\eqref{e:CTdef},
we write $F_{N,T} = F_T(X^{L^N})$. For instance, with $n=L^N$,
\begin{equation}
    L^x_{N,T} = \int_0^T \1_{X^{n}_t=\;x} \; dt,
    \quad I_{N,T} = \sum_{x \in \Lambda_N}(L_{N,T}^x)^2 .
\end{equation}

As before, we identify the vertices of $\Lambda_N$ with nested subsets of $\Zd$,
centred at the origin (approximately if $L$ is even),
with $\Lambda_{N+1}$ paved by $L^d$ translates of $\Lambda_N$.
% We can thus define $\partial \Lambda_N$ to be the inner vertex boundary of $\Lambda_N$.
We denote the expectation of $X^{L^N}$ started from $a \in \Lambda_N$ by $E^{\Lambda_N}_a$
and define
\begin{align}
\label{e:cN}
c_{N,T}(a, b)
    &= E^{\Lambda_N}_a \left( e^{-U_{\beta,\gamma,T}} \1_{X(T)=b} \right)
    \quad (a, b \in \Lambda_N), \\
c_{N,T}
    &= E^{\Lambda_N}_0 \left( e^{-U_{\beta,\gamma,T}} \right).
\end{align}
The finite-volume two-point function and susceptibility
are defined by
\begin{align}
G_{N,\beta,\gamma,\nu}(a,b)
    &= \int_0^\infty c_{N,T}(a, b) e^{-\nu T} \; dT, \\
\chi_N(\beta, \gamma, \nu)
    &= \int_0^\infty c_{N,T} e^{-\nu T} \; dT
    .
    \label{e:chiNdef}
\end{align}

\begin{prop}
\label{prop:finvol}
Let $d >0$, $\beta >0$ and $\gamma < \beta$. For all $\nu \in \R$,
\begin{equation}
\label{e:Givlc}
\lim_{N \to \infty}
G_{N,\beta,\gamma,\nu}(a,b)
=
G_{\beta,\gamma,\nu}(a,b)
\end{equation}
and
\begin{equation}
\label{e:chilim}
\lim_{N\to\infty}\chi_N(\beta,\gamma,\nu)=   \chi(\beta,\gamma,\nu).
\end{equation}
\end{prop}

The proof is in the appendix.
\section{Main results}
\label{sec:mr}

For any integer $n \ge 1$, let $G_x(g, \gamma, \nu; n)$
denote the two-point point function for the version of the $|\varphi|^4$ model
defined by \eqref{e:two-point-function-phi4}.
We let $G_x(g, \gamma, \nu; 0)$ denote the two-point function of the WSAW-SA,
defined in \eqref{e:Gsa}; this notation will be explained in Section~\ref{sec:spin-walk}.
We employ similar conventions for the susceptibility, correlation length of order $p$,
and critical point, which we denote by
$\chi(g, \gamma, \nu; n)$, $\xi_p(g, \gamma, \nu; n)$, $\nu_c(g, \gamma, \nu; n)$, respectively,
with $n \ge 0$ an integer.
When $n = 0$, these correspond to the WSAW-SA, whereas for $n \ge 1$ they correspond to the
Ising model. The following theorem is the main result of this thesis.

\begin{theorem}
\label{thm:mr}
Let $d = 4$ and $n \ge 0$. For $L$ sufficiently large (depending on $n$),
there exists $\gcc_* > 0$ and a positive function $\gamma_* : (0, \gcc_*) \to \R$
such that whenever $0 < \gcc < \gcc_*$ and $|\gamma| < \gamma_*(\gcc)$,
there are constants $A_{\gcc,\gamma,n}$ and $B_{\gcc,\gamma,n}$ such that the following hold:

\smallskip\noindent
(i)
The critical two-point function decays as
\begin{equation}
G_x(\gcc,\gamma,\nu_c; n)
    =
A_{\gcc,\gamma,n} |x|^{-2} \big(1 + O((\log |x|)^{-1})\big)
    \quad
\text{as $|x|\to\infty$},
\end{equation}
with $A_{\gcc,\gamma,n} = (4 \pi)^{-2} (1 + O(\gcc))$ as $\gcc \downarrow 0$.

\smallskip\noindent
(ii)
The susceptibility diverges as
\begin{equation} \label{e:chieps-asympt}
\chi(\gcc, \gamma, \nu_c + \varepsilon; n)
	\sim
B_{\gcc,\gamma,n} \varepsilon^{-1} (\log \varepsilon^{-1})^{(n+2)/(n+8)},
	\quad
\varepsilon\downarrow 0
\end{equation}
with $B_{\gcc,\gamma,n} = ((n + 8) \gcc / 16\pi^2)^{(n+2)/(n+8)} (1 + O(\gcc))$
as $\gcc \downarrow 0$.

\smallskip\noindent
(iii)
For any $p >0$, if $L$ is chosen large and $\gcc_*$ small (both depending on $p$),
then the correlation length of order $p$ diverges as
\begin{equation} \label{e:xieps-asympt}
\xi_p(\gcc, \gamma, \nu_c + \varepsilon; n)
	\sim
B_{\gcc,\gamma,n}^{1/2} {\sf c}_p \varepsilon^{-1/2} (\log \varepsilon^{-1})^{(n+2)/2(n+8)},
	\quad
\varepsilon\downarrow 0
\end{equation}
with
\begin{equation}
\label{e:cpdef}
{\sf c}_p^p
	=
\int_{\R^4} |x|^p (-\Delta_{\R^4} + 1)^{-1}_{0x} \; dx.
\end{equation}
\end{theorem}

The $\gamma = 0$ cases of (i) and (ii) were proved by Bauerschmidt, Brydges,
and Slade in \cite{BBS-saw4,BBS-saw4-log};
in fact, the $n = 1$ case of their results were first obtained in
\cite{Hara87,HT87,GK85,FMRS87}.
The $n > 0$ case with $\gamma \ne 0$ is a new result in this thesis. We
will only discuss the proof of the $\gamma \ge 0$ case, which is of primary
interest. The proof of the $\gamma < 0$ case with $n = 0$ can be found in
\cite{BSW-saw-sa} and the extension to $n \ge 1$ is straightforward.				% main results
\section{Relations between models}
\label{sec:spin-walk}

One way to understand universality is via representation theorems that relate
different models. For instance, the \emph{Kac-Siegert transformation}
can be used to write the partition function of the $O(n)$ model as a partition function
for a perturbation of the $|\varphi|^4$ model (we will discuss this further in
Section~\ref{sec:hard-core}). In the other direction, the
Griffiths-Simon construction \cite{SG73} can be used to approximate the $1$-component
$|\varphi|^4$ model by a sum of Ising models. Such theorems do not necessarily
imply universality
(in the sense that models related in this way have the same critical exponents or
scaling limit), but tend to be suggestive of it and may in some cases
be used as the basis for the proof of a universality-type result.

We have already noted in Example~\ref{ex:gff-asymp} the close relationship between
the simple random walk
and the Gaussian free field, which ultimately stems from the representation of
matrix powers in terms of walks and which is familiar to anyone
who has studied Markov chains. Namely, if $M$ is a $\vertices\times\vertices$
matrix, then
\begin{equation}
M^n_{ab} = \sum_{x_1,\ldots,x_n\in\vertices} M_{ax_1} M_{x_1x_2} \ldots M_{x_nb}.
\end{equation}
When $M$ is indexed by $\edges$, the
sum above can be replaced by a sum over $n$-step walks from $a$ to $b$ on $\graph$.
When the entries of $M$ are non-negative, such a sum acquires a probabilistic
interpretation as an expectation with respect to the random walk whose steps
are weighted by the entries of $M$.

It was discovered by Symanzik \cite{Syma69} that certain spin systems could be
represented as models of interacting walks in a background of interacting loops.
Symanzik used this insight to study quantum field theories in terms of walks.
Such representations were also studied, e.g.\ in \cite{BFS82,Dynk83}. A comprehensive
reference is \cite{FFS92}.

In the opposite direction, one can consider studying walks by looking for corresponding
spin systems. In \cite{Genn72}, De Gennes argued that the self-avoiding walk
corresponds to an $n \to 0$ ``limit'' of the $O(n)$ spin model and used this
to predict the values of its critical exponents. Since $n$ is
the number of components of the spins, the $O(n)$ model is only well-defined
for $n$ a positive integer and it is not clear how to make sense of such a limit.

Parisi and Sourlas \cite{PS80} and McKane \cite{McKa80} discovered
an alternative approach to the predictions of De Gennes. They argued that the weakly
self-avoiding walk two-point function could be represented as the two-point function
for a \emph{supersymmetric} version of the $|\varphi|^4$ model, involving boson and
fermion fields (we discuss these below). The formal appearance of $n = 0$ quantities
was then explained as a consequence of a symmetry between the bosons and fermions known
as \emph{supersymmetry}.

In Section~\ref{sec:ntozero}, we provide a brief description of the heuristic
relation between spin systems and self-avoiding walk. Then in Section~\ref{sec:intrep},
we describe the rigorous representation of WSAW-SA as a supersymmetric field
theory.

%%%%%%%%%%%%%%%%%%%%%%%%%%%%%%%%%%%%%%%%%%%%%%%%%%%%%%%%%%%%%%%%%%%%%%%%%%%%%%%

\subsection{The \texorpdfstring{$n\to0$}{n approaches 0} limit}
\label{sec:ntozero}

\commentbw{Note that we have an unnormalized measure on the sphere}

The heuristic relation between self-avoiding and spin systems is most easily treated
on graphs of degree $3$, so we restrict our attention to this case.

The \emph{high-temperature expansion} of a spin system is based on the expansion of the
Boltzmann weight $e^{\beta H}$ around $\beta = 0$.
For the $O(n)$ spin model, an uncontrolled approximation of this expansion yields
\begin{align}
Z 
	&=
\int d\lambda(\sigma) \prod_{xy\in\edges} e^{\beta \sigma_x \cdot \sigma_y} \nonumber \\
	&\approx
\int d\lambda(\sigma) \prod_{xy\in\edges} (1 + \beta \sigma_x \cdot \sigma_y) \nonumber \\
	&=
\sum_{E \subset \edges} \beta^{|E|} \int d\lambda(\sigma) \prod_{xy \in E} \sigma_x \cdot \sigma_y
\end{align}
By reflection-invariance of the sphere measure, the last integral above is non-zero
if and only if every vertex in the product over $E$ appears an even number of times.
On a graph of degree $3$, this is only possible if $E$ consists of mutually avoiding
(i.e.\ disjoint) self-avoiding
loops (i.e.\ walks from a vertex to itself that are self-avoiding everywhere except this
vertex). Thus, the sum over subsets of $\edges$ can be replaced by a sum over collections of
such loops.

A similar expansion can be performed for the numerator in the definition of the
two-point function, yielding:
\begin{equation}
\int d\lambda(\sigma) \sigma_a \cdot \sigma_b
	\approx
\sum_{E\subset\edges} \beta^{|E|}
\int d\lambda(\sigma) \sigma_a \cdot \sigma_b
\prod_{xy\in E} \sigma_x \cdot \sigma_y.
\end{equation}
Once again, every vertex must appear twice on the right-hand side. This time,
non-zero contributions come from subsets $E$ containing mutually avoiding self-avoiding
loops together with a self-avoiding walk from $a$ to $b$ that is also disjoint from these
loops. As a very simple example, if $a \sim b$, then there is a non-zero contribution from
$E = \{ \{ a, b \} \}$.

\commentbw{Non-zero contributions vanish for loops when $n = 0$. I think spins need
to be scaled by $\sqrt n$}

\commentbw{Refer to Madras-Slade \cite[Section 2.3]{MS93} for more general}

%%%%%%%%%%%%%%%%%%%%%%%%%%%%%%%%%%%%%%%%%%%%%%%%%%%%%%%%%%%%%%%%%%%%%%%%%%%%%%%

\subsection{Self-avoiding walk representation}
% This section based on saw-sa
\label{sec:intrep}

In this section we describe an integral representation of the of WSAW-SA
% which is a special case of a result of Brydges, Imbrie, and Slade \cite{BIS09}.
% We restrict out attention to
on the discrete torus $\Lambda$. We begin with the necessary
background on Grassmann integration, introduced in \cite{Bere66}. However, we
follow the treatment of \cite{BIS09}.

\subsubsection{Boson and fermion fields}
\label{sec:forms}

Let $\phi_x$, $\bar\phi_x$ denote complex variables indexed by $x\in\Lambda$.
We refer to $(\phi, \phib)$ as a \emph{boson} field. Let $u_x, v_x$ denote
the real and imaginary parts of $\phi_x$ and define the differentials
$d\phi_x = du_x + i dv_x$ and likewise for $d\phib_x$.
We multiply differential forms
in the usual way via the wedge product $\wedge$ but drop this in our notation;
in particular,
\begin{equation}
d\phib_x d\phi_x = 2 i du_x dv_x.
\end{equation}

\begin{example}
Let $C$ be a positive-definite symmetric $\Lambda\times\Lambda$ matrix. The
\emph{complex Gaussian measure} with covariance $C$ is the probability measure
on $\R^{2\Lambda}$ given by
\begin{equation}
d\mu_C(\phi, \phib)
	=
\frac{d\phib d\phi}{\det(2\pi i C)} e^{-\phi \cdot A \phib}
\end{equation}
where $A = C^{-1}$ and
\begin{equation}
d\phib d\phi \coloneqq \prod_{x\in\Lambda} d\phib_x d\phi_x
\end{equation}
The order in which the product over $x\in\Lambda$ is taken does not matter
since the $d\phib_x d\phi_x$ commute. The complex Gaussian satisfies a
version of Wick's theorem. In particular,
\begin{equation}
\label{e:wick-complex2}
\int \phib_x \phi_y \; d\mu_C(\phi, \phib) = C_{xy}.
\end{equation}
\end{example}

Let
\begin{equation}
\psi_x = \frac{1}{\sqrt{2\pi i}} d\phi_x,
\quad
\bar\psi_x = \frac{1}{\sqrt{2\pi i}} d\bar\phi_x,
\end{equation}
where we fix a choice of complex square root. We refer to $(\psi_x, \psib_x)_{x\in\Lambda}$
as a \emph{fermion} field.
A differential form that is the
product of a function of $(\phi, \bar\phi)$
with $p$ differentials is said to have \emph{degree} $p$.
A sum of forms of even degree is said to be \emph{even}.

We introduce a copy $\bar\Lambda$ of $\Lambda$
and we denote the copy of $X \subset \Lambda$ by $\bar X \subset \bar\Lambda$.
We also denote the copy of $x \in \Lambda$
by $\bar x \in \bar\Lambda$ and define $\phi_{\bar x} = \bar\phi_x$ and $\psi_{\bar x} = \bar\psi_x$.
Then any differential form $F$ can be written
\begin{equation}
\lbeq{FinNcal}
F
=
\sum_{\vec y}
F_{\vec y} (\phi, \bar\phi)
\psi^{\vec y}
\end{equation}
where the sum is over finite sequences $\vec y$ over $\Lambda\sqcup\bar\Lambda$,
and $\psi^{\vec y} = \psi_{y_1} \ldots \psi_{y_p}$
% (in some canonical order)
when $\vec y = (y_1, \ldots, y_p)$. Here, we take the sequences to be ordered in
some canonical fashion. When $\vec y = \varnothing$ is the empty sequence,
$F_\varnothing$ denotes the $0$-degree (bosonic) part of $F$.

In order to apply the results of \cite{BBS-saw4-log,BBS-saw4,BSTW-clp}, we require
smoothness of the coefficients $F_{\vec y}$ of $F$.  For Theorem~\ref{thm:mr}(i,ii),
we need these coefficients to be $C^{10}$, and for Theorem~\ref{thm:mr}(iii) we require
a $p$-dependent number of derivatives for the analysis of $\xi_p$, as discussed in \cite{BSTW-clp}.
In either case, we let $p_\Ncal$ denote the desired degree of smoothness.
We will discuss this further in Section~\ref{sec:newnorm}.

We let $\Ncal^\varnothing$ be the algebra of even forms with sufficiently smooth coefficients
and we let $\Ncal^\varnothing(X) \subset \Ncal^\varnothing$ be the sub-algebra of even forms only depending on fields
in $X$. Thus, for $F \in \Ncal^\varnothing(X)$, the sum in \eqref{e:FinNcal} runs over sequences $\vec y$
over $X \sqcup \bar X$.

Now let $F = (F_j)_{j \in J}$ be a finite collection of even forms
indexed by a set $J$
and write $F_\varnothing = (F_{\varnothing,j})_{j \in J}$.
Given a $C^\infty$ function $f : \R^J \to \C$, we define
$f(F)$ by its Taylor expansion about $F_\varnothing$:
\begin{equation}
f(F) = \sum_\alpha \frac{1}{\alpha!} f^{(\alpha)}(F_\varnothing) (F - F_\varnothing)^\alpha.
\end{equation}
The summation terminates as a finite sum,
since $\psi_x^2 = \bar\psi_x^2 = 0$
by anti-commutativity.
% due to the anti-commut\-ative product.

We define the integral $\int F$ of a differential form $F$ in the usual way
as the Riemann integral of its top-degree part (which may be regarded as a function
of the boson field).
In particular, given a positive-definite symmetric
$\Lambda \times \Lambda$ matrix $C$ with inverse $A = C^{-1}$,
we define the \emph{Gaussian expectation} (or \emph{super-expectation}) of $F$ by
\begin{equation}
\lbeq{ExCF}
\Ex_C F = \int e^{-S_A} F,
\end{equation}
where
\begin{equation}
\label{e:action}
S_A = \sum_{x\in\Lambda} \Big(\phi_x (A\bar\phi)_x + \psi_x (A \bar\psi)_x\Big).
\end{equation}
% Note that the computation of \eqref{e:ExCF} always reduces to an integral of the form
% \begin{equation}
% \int f(\phi, \phib) \psi^{\Lambda\sqcup\bar\Lambda},
% \end{equation}
% where $\psi^{\Lambda\sqcup\bar\Lambda}$ denotes the product of all the fermionic fields
% in their canonical ordering. Thus, in order to make the change of variables
% $\psi_x \mapsto a \psi_x$ for some $x$ we must simply scale the integral by the factor
% $a^{-1}$. This is the opposite effect of a bosonic change of variables $\phi_x \mapsto a \phi_x$
The super-expectation possesses the remarkable property that
\begin{equation}
\label{e:tau-iso}
\int e^{-S_A} F(\tau) = F(0).
\end{equation}
In particular,
\begin{equation}
\label{e:self-normal}
\int e^{-S_A} = 1.
\end{equation}
Moreover, if $F$ is a degree-$0$ form, then
\begin{equation}
\Ex_C F = \int F \; d\mu_C.
\end{equation}
There is also a version of Wick's theorem for fermions. In particular,
\begin{equation}
\label{e:wick-fermion2}
\int e^{-S_A} \psi_x \psib_x = C_{xx}.
\end{equation}

For $F = f(\phi, \bar\phi) \psi^{\vec y}$, we let
\begin{equation}
\theta F = f(\phi + \xi, \bar\phi + \bar\xi) (\psi + \eta)^{\vec y},
\end{equation}
where $\xi$ is a new boson field, $\eta = (2\pi i)^{-1/2} d\xi$ a new fermion field,
and $\bar\xi, \bar\eta$ are the corresponding conjugate fields.
We extend $\theta$ to all $F \in \Ncal^\varnothing$ by linearity
and define the convolution operator $\Ex_C\theta$ by letting
$\Ex_C\theta F \in \Ncal^\varnothing$ denote the Gaussian expectation of $\theta F$ with respect
to $(\xi, \bar\xi, \eta, \bar\eta)$, with $\phi,\phib,\psi,\psib$ held fixed.

\subsubsection{Integral representation of the two-point function}
\label{sec:Gintrep}

An integral representation formula applying to general local time functionals
is given in \cite{BEI92,BIS09}. We state the result we need in the proposition below.
A direct proof can be obtained by a small modification to the proof in
\cite[Appendix~A]{ST-phi4}.

We define the differential forms:
\begin{align}
\label{e:taudef}
\tau_x
	&=
\phi_x \bar\phi_x + \psi_x \bar\psi_x
	\\
\label{e:addDelta}
\tau_{\Delta,x}
	&=
\frac 12
\Big(
	\phi_{x} (- \Delta \bar{\phi})_{x} + (- \Delta \phi)_{x} \bar{\phi}_{x}
		+
	\psi_{x}  (- \Delta \bar{\psi})_{x} + (- \Delta \psi)_{x}  \bar{\psi}_{x}
\Big)
	\\
\label{e:nablatau}
|\nabla \tau_x|^2
	&=
\sum_{|e|=1} (\nabla^e \tau)_x^2.
\end{align}
Recall \eqref{e:Udef-pos} and define
\begin{equation}
\label{e:Vdef2}
V_{\gcc,\gamma,\nu,N}
	=
U_{\gcc,\gamma}(\tau)
	+
\sum_{x\in\Lambda_N}
\Big(
	\nu \tau_x + \tau_{\Delta,x}
\Big)
% \sum_{x\in\Lambda_N}
% \Big(
% 	\gcc \tau_x^2 + \nu \tau_x + \tau_{\Delta,x} - \tfrac{1}{2 d} \gamma |\nabla \tau_x|^2
% \Big)
\end{equation}

\begin{prop}
Let $d > 0$ and $\gcc > 0$. For $\gamma < \gcc$ and $\nu \in \R$,
\begin{equation}
\label{e:Grep-pos-bis}
G_{x,N}(\gcc, \gamma, \nu; 0)
	=
\int e^{-V_{\gcc,\gamma,\nu,N}} \phib_0 \phi_x.
% \int e^{-U_{\gcc,\gamma,\nu,N}} \bar\phi_a \phi_b.
\end{equation}
\end{prop}

\subsubsection{Finite-volume approximation}

In order to make use of the integral representation above, we must approximate the
WSAW-SA on $\Zd$ by a model on $\Lambda_N$.

Let $X^{L^N}$ denote the simple random walk on $\Lambda_N$.
For $F_T = F_T(X)$ any one of the functions $L_T^x,I_T,C_T$
of $X$ defined in \eqref{e:LTx-def}--\eqref{e:CTdef},
we write $F_{N,T} = F_T(X^{L^N})$. For instance, with $n=L^N$,
\begin{equation}
    L^x_{N,T} = \int_0^T \1_{X^{n}_t=\;x} \; dt,
    \quad I_{N,T} = \sum_{x \in \Lambda_N}(L_{N,T}^x)^2 .
\end{equation}

As before, we identify the vertices of $\Lambda_N$ with nested subsets of $\Zd$,
centred at the origin (approximately if $L$ is even),
with $\Lambda_{N+1}$ paved by $L^d$ translates of $\Lambda_N$.
% We can thus define $\partial \Lambda_N$ to be the inner vertex boundary of $\Lambda_N$.
We denote the expectation of $X^{L^N}$ started from $0 \in \Lambda_N$ by $E^{\Lambda_N}_0$
and define
\begin{align}
\label{e:cN}
c_{N,T}(x)
    &= E^{\Lambda_N}_0 \left( e^{-U_{\gcc,\gamma,T}} \1_{X(T)=b} \right),
    \quad x \in \Lambda_N \\
c_{N,T}
    &= E^{\Lambda_N}_0 \left( e^{-U_{\gcc,\gamma,T}} \right).
\end{align}
The finite-volume two-point function and susceptibility
are defined by
\begin{align}
G_{x,N}(\gcc,\gamma,\nu; 0)
    &=
\int_0^\infty c_{N,T}(x) e^{-\nu T} \; dT, \\
\label{e:chiNdef-pre}
\chi_N(\gcc, \gamma, \nu; 0)
    &=
\int_0^\infty c_{N,T} e^{-\nu T} \; dT.
\end{align}
The proof of the following proposition is given in Appendix~\ref{sec:finvol}.

\begin{prop}
\label{prop:finvol}
Let $d >0$, $\gcc >0$ and $\gamma < \gcc$. For all $\nu \in \R$,
\begin{equation}
\label{e:Givlc}
\lim_{N \to \infty}
G_{x,N}(\gcc,\gamma,\nu; 0)
=
G_x(\gcc,\gamma,\nu; 0)
\end{equation}
and
\begin{equation}
\label{e:chilim-pre}
\lim_{N\to\infty}\chi_N(\gcc,\gamma,\nu; 0) =   \chi(\gcc,\gamma,\nu; 0).
\end{equation}
In fact, $\chi_N$ and $\chi$ are analytic in ${\rm Re} \nu > \nu_c$ and
$\chi_N \to \chi$ uniformly on compact subsets of this domain.
\end{prop}
					% integral representations/finite-volume approximation

%% Chapter 2 %%
\chapter{Renormalisation group method}
\label{sec:rg}

\renewcommand{\pm}{+}					% only consider \gamma \ge 0

In this chapter, we outline the renormalisation group method of Bauerschmidt,
Brydges, and Slade developed in the series of papers
\cite{BS-rg-norm,BS-rg-loc,BBS-rg-pt,BS-rg-IE,BS-rg-step} and
applied in \cite{BBS-phi4-log,BBS-saw4-log,BBS-saw4,ST-phi4}. We will often
state results from these papers without proof.

The main contribution of this
thesis, which is based on the work in \cite{BSTW-clp,BSW-saw-sa}, is the
improvement of the estimates in Theorem~\ref{thm:step-mr-fv} and the extension
to $\gamma_0 \ne 0$ in Theorem~\ref{thm:rhatflow}.

%%%%%%%%%%%%%%%%%%%%%%%%%%%%%%%%%%%%%%%%%%%%%%%%%%%%%%%%%%%%%%%%%%%%%%%%%%%%%%%
%%%%%%%%%%%%%%%%%%%%%%%%%%%%%%%%%%%%%%%%%%%%%%%%%%%%%%%%%%%%%%%%%%%%%%%%%%%%%%%

\section{Notation}

To unify our treatment of the two models,
we define the forms $\tau_x, \tau_{\Delta,x}, |\nabla\tau_x|^2$ according
to \eqref{e:taudef}--\eqref{e:nablatau} if $n = 0$ and
\begin{equation}
\label{e:tauphi}
\tau_x = \tfrac{1}{2} |\varphi_x|^2,
	\quad
\tau_{\Delta,x} = \tfrac{1}{2} \varphi_x \cdot (-\Delta \varphi)_x,
	\quad
|\nabla\tau_x|^2 = \sum_{|e|=1} (\nabla^e |\phi_x|^2)^2
\end{equation}
if $n \ge 1$.

Then by \eqref{e:Vdef1}, \eqref{e:Udef-pos}, and \eqref{e:Vdef2},
% $V_{g,\gamma,\nu,N} \in \Ncal^\varnothing$ is given by
\begin{equation}
V_{g,\gamma,\nu,N}
	=
\sum_{x\in\Lambda_N}
\Big(
	(g - \gamma) \tau_x^2 + \nu \tau_x + \tau_{\Delta,x} + \tfrac{1}{4 d} \gamma |\nabla\tau_x|^2
\Big)
\end{equation}
for any choice of $n$.
We write
\begin{equation}
\langle F \rangle_{g,\gamma,\nu,N}
	=
\begin{cases}
\displaystyle \int F e^{-U_{g,\gamma,\nu,N}},				& n = 0 \\
\displaystyle \frac{1}{Z_{g,\gamma,\nu,N}}
	\int F(\varphi) e^{-U_{g,\gamma,\nu,N}} \; d\varphi,	& n \ge 1.
\end{cases}
\end{equation}
The action $S_A$ is defined by \eqref{e:action} if $n = 0$ and
\begin{equation}
S_A = \frac12 \sum_{x\in\Lambda} \varphi_x \cdot (A \varphi)_x
\end{equation}
if $n \ge 1$. In either case, if $A = -\Delta + m^2$, then
\begin{equation}
\label{e:SAtauDelta}
S_A = \sum_{x\in\Lambda} (\tau_{\Delta,x} + m^2 \tau_x).
\end{equation}
Thus, if $\Ex_C \theta$ is the super-expectation \eqref{e:ExCF} for $n = 0$
and Gaussian integration over $(\R^n)^\Lambda$ if $n \ge 1$ (recall \eqref{e:gauss-density}),
then for $\nu > 0$,
\begin{equation}
\label{e:phi4-gauss}
\langle F \rangle_{0,0,m^2,N}
	=
\Ex_C F,
	\qquad
C = (-\Delta + m^2)^{-1}.
\end{equation}

By \eqref{e:two-point-function-phi4}, \eqref{e:Givlc}, and \eqref{e:Grep-pos-bis},
\begin{equation}
G_x(g, \gamma, \nu; n) = \lim_{N\to\infty} G_{x,N}(g, \gamma, \nu; n),
\end{equation}
where
\begin{equation}
G_{x,N}(g, \gamma, \nu; n)
	=
\begin{cases}
\langle \phib_0 \phi_x \rangle_{g,\gamma,\nu,N},      & n = 0 \\
\langle \varphi_0 \cdot \varphi_x \rangle_{g,\gamma,\nu,N}  & n \ge 1.
\end{cases}
\end{equation}

%%%%%%%%%%%%%%%%%%%%%%%%%%%%%%%%%%%%%%%%%%%%%%%%%%%%%%%%%%%%%%%%%%%%%%%%%%%%%%%
%%%%%%%%%%%%%%%%%%%%%%%%%%%%%%%%%%%%%%%%%%%%%%%%%%%%%%%%%%%%%%%%%%%%%%%%%%%%%%%

\section{Reformulation of the problem}

Given $m^2>0$ and $z_0 >-1$, let
\begin{equation}
\label{e:gg0}
g_0 = (\gcc - \gamma) (1 + z_0)^2,
	\quad
\nu_0 = \nu (1 + z_0) - m^2,
	\quad
\gamma_0 = \frac{1}{4d} \gamma (1 + z_0)^2.
\end{equation}
We fix the two points $0,x\in \Lambda$
and introduce \emph{observable fields} $\sigmaa, \sigmab \in \R$.
In summary, we distinguish between the following kinds of fields:
real ($\varphi$) and complex ($\phi$, $\phib$) bosonic fields,
observable fields ($\sigma$), and fermionic fields ($\psi$, $\psib$).

For any $y\in\Lambda$, we define the polynomials
\begin{equation}
\label{e:V0def}
\Vp^+_{0,y}
	=
g_0\tau_y^2 + \nu_0 \tau_y + z_0 \tau_{\Delta,y}
- f_0 \sigma_0 \1_{y=x}
- f_x \sigma_x \1_{y=x},
	\quad
U^+_y
	=
|\nabla \tau_y|^2
\end{equation}
where
\begin{equation}
\label{e:obs-couple}
f_u =
\begin{cases}
\phib_0,		& n = 0, u = 0 \\
\phi_x,			& n = 0, u = x \\
\varphi^1_u,	& n \ge 1.
\end{cases}
\end{equation}
These are examples of local polynomials, which are polynomials
in the fields and their derivatives at a point $y\in\Lambda$. For any such local
polynomial $V_y$, we will usually write
\begin{equation}
\label{e:VX}
V(X) = \sum_{y\in X} V_y.
\end{equation}

Let
\begin{equation}
\label{e:Z0def}
Z_0 = \prod_{x\in \Lambda} e^{-(\Vp^+_{0,x} + \gamma_0 U^+_x)}
\end{equation}
and
\begin{equation}
\label{e:ZNdef}
Z_N = \Ex_C \theta Z_0
\end{equation}
where the covariance is given by $C = (-\Delta + m^2)^{-1}$.
In particular,
\begin{equation}
\label{e:exp-conv}
\Ex_C Z_0 = Z_{N,\varnothing}(0).
\end{equation}
Recall here that $Z_{N,\varnothing}$ denotes the $0$-degree part of $Z_N$;
this is a function of the bosonic fields, which we have set to $0$ on the
right-hand side.

Recall that the Gaussian convolution operator $\Ex_C\theta$ was defined in
Section~\ref{sec:forms}; recall also from this section that $Z_{N,\varnothing}$
denotes the $0$-degree part of $Z_N$ (when $n \ge 1$, $Z_{N,\varnothing} = Z_N$).
When $n = 0$, this is a function of the fields $(\phi, \bar\phi)$ and when $n \ge 1$
it is a function of $\varphi$. We define a test function $\1: \Lambda_N \to \R$ by
$\1_y=1$ for all $y$. If $F$ is a sufficiently smooth form, let
% and write $D^2 Z_{N,\varnothing}(0; \1, \1)$ for the directional derivative of
% $Z_{N,\varnothing}$ evaluated with all fields equal to $0$ and
% % at $(\phi, \bar\phi) = (0, 0)$,
% with both directions equal to $\1$. That is,
\begin{equation}
D^2 F(0; \1, \1)
  =
\ddp{^2}{s\partial t}\Big|_0
\begin{cases}
F(s \1, t\1), & n = 0 \\
F(s \1 + t\1), & n \ge 1
\end{cases}
\end{equation}
where the derivative is evaluated with all fields (bulk and observable) and $s, t$ set to $0$.
We will also denote by $D^2_{\sigma_0\sigma_x} F(0)$ denote the second partial derivative
with respect to $\sigma_0$ and $\sigma_x$ evaluated with all fields $0$.

\begin{prop}
\label{prop:intrep}
% from saw-sa
Let $d > 0$, $\gamma, \nu \in \R$, $\gcc >0$ and $\gamma <\gcc$.
If the relations \eqref{e:gg0} hold, then
\begin{align}
% from clp
\label{e:generating-fn}
G_{x,N}(g, \gamma, \nu; n)
	&=
(1+z_0)
D^2_{\sigma_0\sigma_x}
\log \Ex_C Z_0
\end{align}
% \begin{equation}
% \label{e:GG2}
% G_{x,N}(\gcc,\gamma,\nu)
%     =
% (1+z_0)
% \Ex_C (Z_0 \bar\phi_a \phi_b),
% \end{equation}
and
\begin{equation}
\label{e:chichibar}
\chi_N\left(\gcc,\gamma,\nu; n\right)
	=
(1+z_0)\hat\chi_N(m^2, g_0, \gamma_0, \nu_0, z_0; n),
\end{equation}
with
\begin{equation}
\label{e:chibarm}
\hat\chi_N(m^2, g_0, \gamma_0, \nu_0, z_0; n)
	=
\frac{1}{m^2}
	+
\frac{1}{m^4} \frac{1}{|\Lambda|} \frac{D^2 Z_{N,\varnothing}(0; \1, \1)}{Z_N(0)}.
\end{equation}
\end{prop}

\begin{proof}
We prove the case $n = 0$ and drop the parameter $n$ from the notation; note that
$Z_N(0)\big|_{\sigma_0=\sigma_x=0} = 1$ in this case. The proof for $n \ge 1$ is similar
and involves only ordinary integration with to real boson fields.

We make the change of variables $\phi_x \mapsto (1 + z_0)^{1/2} \phi_x$
and likewise for $\bar\phi_x, \psi_x, \bar\psi_x$ in \eqref{e:Grep-pos-bis}, and obtain
\begin{equation}
\label{e:Grep-pos}
G_{x,N}(\gcc,\gamma,\nu)
	=
(1+z_0) \int e^{-\sum_{x\in\Lambda}
\left(
	g_0 \tau_x^2 + \gamma_0 |\nabla \tau_x|^2 + \nu (1+z_0) \tau_x + (1+z_0)\tau_{\Delta,x}
\right)}
\bar\phi_a \phi_b.
\end{equation}
Note here that there is no Jacobian factor due to the change of variables in $\psi, \psib$.
For any $m^2 \in\R$, it follows that
\begin{equation}
\lbeq{GNint}
G_{x,N}(\gcc,\gamma,\nu)
    =
(1 + z_0) \int
e^{-\sum_{x\in\Lambda} (\tau_{\Delta,x} + m^2 \tau_x)}
Z_0 \phib_0 \phi_x
\end{equation}
($m^2$ simply cancels with $\nu_0$ on the right-hand side).
We use this with $m^2>0$, so that the inverse matrix $C=(-\Delta+m^2)^{-1}$ exists and
\begin{equation}
\label{e:G-gauss}
G_{x,N}(g, \gamma, \nu)
	=
(1 + z_0) \Ex_C (Z_0 \phib_0 \phi_x)
\end{equation}
% and \eqref{e:GNint} is a Gaussian expectation
by \eqref{e:phi4-gauss}
% By symmetry of the matrix $\Delta$, \refeq{action} gives
% \begin{equation}
% \label{e:SAtauDelta}
% S_{(-\Delta+m^2)}
% =
% \sum_{x\in\Lambda} \left( \tau_{\Delta,x}
% + m^2  \tau_x \right).
% \end{equation}
and \eqref{e:generating-fn} follows.

Summation of \eqref{e:G-gauss} over $x\in \Lambda_N$ gives the formula
$\chi_N(\gcc,\gamma,\nu) = (1+z_0)\sum_{x\in \Lambda} \Ex_C (Z_0\phib_0\phi_x)$.
Call the right-hand side $\hat\chi_N(\gcc,\gamma,\nu)$. To show that this is
consistent with \eqref{e:chibarm}, begin by noting that
\begin{equation}
\hat\chi_N(\gcc,\gamma,\nu)
	=
|\Lambda|^{-1} \frac{D^2 \Sigma(0; \1, \1)}{Z_N(0)},
\end{equation}
where
\begin{equation}
\Sigma(J, \bar J) = \Ex_C (Z_0 e^{J \cdot \phib + \phi \cdot \bar J}).
\end{equation}
Completing the square yields
\begin{equation}
\Sigma(J, \bar J)
	=
e^{J \cdot C \bar J} Z_{N,\varnothing}(C J, C \bar J)
\end{equation}
and differentiating this expression gives
\begin{equation}
D^2 \Sigma(0; \1, \1)
	=
(\1, C \1) + D^2 Z_{N,\varnothing}(0; C\1, C\1)
\end{equation}
The result then follows from the fact that
\begin{equation}
C \1 = A^{-1} \1 = m^{-2} \1.
\end{equation}
\end{proof}

%%%%%%%%%%%%%%%%%%%%%%%%%%%%%%%%%%%%%%%%%%%%%%%%%%%%%%%%%%%%%%%%%%%%%%%%%%%%%%%
%%%%%%%%%%%%%%%%%%%%%%%%%%%%%%%%%%%%%%%%%%%%%%%%%%%%%%%%%%%%%%%%%%%%%%%%%%%%%%%

\section{Progressive integration}
% Based on both
\label{sec:prog}

By Proposition~\ref{prop:intrep}, our task is to understand the Gaussian expectation
$Z_N = \Ex_C Z_0$ and its derivatives to leading order, uniformly in the volume
$\Lambda_N$ and the mass $m^2$ near $0$.

We proceed using the covariance decomposition
\begin{equation}
\label{e:NCj}
C = C_1 + \cdots + C_{N-1} + C_{N,N}
\end{equation}
constructed in \cite{Baue13a}; a similar decomposition was also constructed in \cite{BGM04}.
The covariances $C_1, \ldots, C_{N-1}$ are independent of the volume $\Lambda_N$. The final
covariance $C_{N,N}$ \emph{does} depend on the volume; so, for instance, $C_{N,N} \ne C_{N,N+1}$.
Nevertheless, we will often write $C_N \coloneqq C_{N,N}$ for simplicity when the volume is implicit.

The covariances $C_j$ have the following important \emph{finite-range property}:
\begin{equation}
C_{j;xy} = 0 \text{ if } |x - y| \ge \tfrac12 L^j.
\end{equation}
Thus, $\zeta$ is a Gaussian field with covariance $C_j$, then $\zeta_x$ is independent
of $\zeta_y$ whenever $|x - y| \ge \tfrac12 L^j$. In particular,
if $F_x, F_y$ are functions of the fields in $x, y$, respectively, then
\begin{equation}
\label{e:uncorr}
\Ex_{C_{j+1}} (F_x F_y) = (\Ex_{C_{j+1}} F_x) (\Ex_{C_{j+1}} F_y).
\end{equation}
In addition, we have the following covariance bounds (this is a restatement of
\cite[Proposition~\ref{pt-prop:Cdecomp}(a)]{BBS-rg-pt}).

\begin{prop}
\label{prop:Cdecomp}
  Let $d >2$, $L\geq 2$, $j \ge 1$, $\bar m^2 >0$.
  For multi-indices $\alpha,\beta$ with
  $\ell^1$ norms $|\alpha|_1,|\beta|_1$ at most
  some fixed value $p$,
  and for any $k$, and for $m^2 \in [0,\bar m^2]$,
  \begin{equation}
    \label{e:scaling-estimate}
    |\nabla_x^\alpha \nabla_y^\beta C_{j;x,y}|
    \leq c(1+m^2L^{2(j-1)})^{-k}
    L^{-(j-1)(d-2 +|\alpha|_1+|\beta|_1)},
  \end{equation}
  where $c=c(p,k,\bar m^2)$ is independent of $m^2,j,L$.
  The same bound holds for $C_{N,N}$ if
  $m^2L^{2(N-1)} \ge \varepsilon$ for some $\varepsilon >0$,
  with $c$ depending on $\varepsilon$ but independent of $N$.
\end{prop}

% By Proposition~\ref{prop:Cdecomp}, the decomposition \eqref{e:NCj} is a multiscale decomposition,
% in the sense that $C_{j+1} = O(L^{-1} C_j)$. Moreover, the covariances are approximately
% constant on blocks in the sense that $\nabla C_j = O(L^{-1} C_j)$.

It is a basic property  of the Gaussian distribution that a sum of independent Gaussian random
variables with covariances
$C'$ and $C''$ is itself Gaussian with covariance $C' + C''$. It follows that for any
boson field $F$,
\begin{equation}
\Ex_{C'+C''}\theta F = \Ex_{C'}\theta \circ \Ex_{C''}\theta F.
\end{equation}
% When $C' = s C_0$ and $C'' = t C_0$ for some covariance $C_0$ and $s, t > 0$,
% this is essentially a restatement of the semigroup property of the heat kernel.
This extends to any sufficiently smooth form $F$ (see \cite{BS-rg-norm}).
It follows that
\begin{equation}
\label{e:prog-int}
Z_N =
\Ex_{C_N}\theta \circ \Ex_{C_{N-1}}\theta \circ \ldots \circ \Ex_{C_1}\theta Z_0.
\end{equation}
We define the \emph{renormalisation group map} $Z_j \mapsto Z_{j+1}$ by
\begin{equation}
\label{e:rgmapZ}
Z_{j+1} = \Ex_{C_{j+1}} \theta Z_j, \quad j < N.
\end{equation}

\begin{rk}
This is truly a ($j$-dependent) map and not just a fixed sequence of field
functionals.  Indeed, recall that the initial condition $Z_0$ is not fixed:
it depends on the two free parameters $m^2$ and $z_0$. We think of the family
of possible sequences $Z_j$ as a (non-autonomous) dynamical system.

The key to understanding
$Z_N$ for large $N$ is the careful choice of \emph{critical} initial conditions
$(m^2, z_0)$. In a sense, these initial conditions, which are functions of
$(g, \gamma, \nu)$, define the stable manifold for the renormalisation group
and the fixed point for this stable manifold is the Gaussian measure with covariance
$(1 + z_0) (-\Delta + m^2)^{-1}$.

We have scaled out the factor $1 + z_0$
in the change of variables performed in the proof of Proposition~\ref{prop:intrep}.
This factor plays a similar role to the standard deviation $\sigma$ in the central
limit theorem: rather than a unique fixed point, there is a one-parameter family
of fixed points.
\end{rk}

%%%%%%%%%%%%%%%%%%%%%%%%%%%%%%%%%%%%%%%%%%%%%%%%%%%%%%%%%%%%%%%%%%%%%%%%%%%%%%%
%%%%%%%%%%%%%%%%%%%%%%%%%%%%%%%%%%%%%%%%%%%%%%%%%%%%%%%%%%%%%%%%%%%%%%%%%%%%%%%

\section{The space of field functionals}
\label{sec:Ncal}

For the analysis of the dynamical system \eqref{e:rgmapZ}, we require a suitable
space on which this system evolves.

Let $\Ncal^\varnothing$ be defined as in Section~\ref{sec:intrep} if $n = 0$ and
\begin{equation}
\label{e:Ncaldef}
\Ncal^\varnothing
	= \Ncal^\varnothing(\Lambda)
	= C^{p_\Ncal}((\R^n)^\Lambda,\R)
\end{equation}
if $n \ge 1$. Recall that $p_\Ncal$ is the smoothness parameter discussed in
Section~\ref{sec:intrep}.
% In Section~\ref{sec:newnorm},
% we explain that $p_\Ncal$ must be chosen in a way that depends on the
% parameter $p$ in Theorem~\ref{thm:mr}(iii).

We extend $\Ncal^\varnothing$ to a space $\Ncal$
that includes functions of the observable fields $\sigma_0$ and $\sigma_x$.
Recalling \eqref{e:generating-fn}, this space is defined in such a way that
functions of the observable fields are identified to second order. A precise
definition is given in
in \cite[Section~\ref{phi4-sec:phi4observables_representation}]{ST-phi4}.
The upshot is that every $F \in \Ncal$ has the form
\begin{equation}
\label{e:obs-decomp}
F = F_\varnothing + F_a + F_b + F_{ab},
	\quad
F_\alpha \in \Ncal^\varnothing.
\end{equation}
There are natural projections $\pi_\alpha : \Ncal \to \Ncal_\alpha$ with
$\alpha = \varnothing, a, b, ab$ such that $\pi_\alpha F = F_\alpha$.
For $X \subset \Lambda$, we let $\Ncal(X)$ denote the subset of $\Ncal$ consisting
of field functionals that only depend on fields in $X$.

In order to control the evolution of $Z_j$ on $\Ncal$, we make use of a family
$\|\cdot\|_{T_{\phi,j}(\h_j)}$ of scale-dependent dependent seminorms depending on a
sequence of weights $\h_j > 0$; the field $\phi$ lies in $\C^\Lambda$ if $n = 0$ and
$(\R^n)^\Lambda$ if $n \ge 1$. For convenience,
we will simply write $\|\cdot\|_{T_\phi(\h_j)}$ with the scale $j$ implied by the
choice of parameter $\h_j$. The precise definitions are given below.

%%%%%%%%%%%%%%%%%%%%%%%%%%%%%%%%%%%%%%%%%%%%%%%%%%%%%%%%%%%%%%%%%%%%%%%%%%%%%%%

\subsection{Test functions}

We define the $T_\phi$ seminorm for $n = 0$. The case $n \ge 1$ involves only
minor changes, which we describe in Remark~\ref{rk:Tphi-n}.

Recall the notation introduced in Section~\ref{sec:forms}.
A \emph{test function} $g$ is defined to be a function $(\vec x, \vec y) \mapsto g_{\vec x,\vec y}$,
where $\vec x$ and $\vec y$ are finite sequences of elements in $\Lambda \sqcup \bar\Lambda$.
When $\vec x$ or $\vec y$ is the empty sequence $\varnothing$,
we drop it from the notation as long as this causes no confusion;
e.g., we may write $g_{\vec x} = g_{\vec x,\varnothing}$.
The length of a sequence $\vec x$ is denoted $|\vec x|$.
Gradients of test functions are defined component-wise.
Thus, if $\vec x = (x_1, \ldots, x_m)$
and $\alpha = (\alpha_1, \ldots, \alpha_m)$
with each $\alpha_i \in \N_0^\Ucal$, and similarly for $\vec y=(y_1,\ldots,y_n)$ and
$\beta=(\beta_1,\ldots,\beta_n)$,
then
\begin{equation}
\nabla^{\alpha,\beta}_{\vec x,\vec y} g_{\vec x,\vec y}
  =
\nabla^{\alpha_1}_{x_1} \ldots \nabla^{\alpha_m}_{x_m}
\nabla^{\beta_1}_{y_1} \ldots \nabla^{\beta_n}_{y_n}  g_{x_1,\ldots,x_m,y_1,\ldots,y_n}.
\end{equation}

We fix a positive constant $p_\Phi\ge 4$ and restrict our attention to test functions
that vanish when $|\vec x|  +|\vec y| > p_\Ncal$.
% For Theorem~\ref{thm:suscept}(i-ii), any choice of $p_\Ncal \ge 10$ is sufficient,
% whereas for Theorem~\ref{thm:suscept}(iii) it is necessary to choose $p_\Ncal$ large
% depending on $p$ \cite{BSTW-clp}.
The $\Phi_j = \Phi(\h_j)$ norm on such test functions is defined by
\begin{equation}
\|g\|_{\Phi_j}
	=
\sup_{\vec x, \vec y} \h_j^{-(|\vec x| +|\vec y|)}
	\shift\shift
\sup_{\alpha,\beta: |\alpha|_1+|\beta|_1 \le p_\Phi}
L^{j (|\alpha|_1 + |\beta|_1)}
|\nabla^{\alpha,\beta} g_{\vec x, \vec y}|,
\end{equation}
where $|\alpha|_1$ denotes the total order of the differential operator $\nabla^\alpha$.
Thus, for any test function $g$ and for sequences
$\vec x, \vec y$ with $|\vec x| +|\vec y| \leq p_\Ncal$ and
corresponding $\alpha, \beta$ with $|\alpha|_1 + |\beta|_1 \leq p_\Phi$,
\begin{equation}
\label{e:testfcnbd}
|\nabla^{\alpha,\beta} g_{\vec x,\vec y}|
	\leq
\h_j^{|\vec x| + |\vec y|} L^{-j (|\alpha|_1 + |\beta|_1)} \|g\|_{\Phi_j}.
\end{equation}

%%%%%%%%%%%%%%%%%%%%%%%%%%%%%%%%%%%%%%%%%%%%%%%%%%%%%%%%%%%%%%%%%%%%%%%%%%%%%%%

\subsection{The \texorpdfstring{$T_\phi$}{Tphi} seminorm}
\label{sec:Tphi}

If $n = 0$, then for any $F \in \Ncal^\varnothing$, there are
\emph{unique} functions $F_{\vec y}$ of $(\phi, \bar\phi)$
that are anti-symmetric under permutations of $\vec y$, such that
\begin{equation}
F = \sum_{\vec y} \frac{1}{|\vec y|!} F_{\vec y}(\phi, \bar\phi) \psi^{\vec y}.
\end{equation}
Given a sequence $\vec{x}$ with $|\vec{x}| = m$, we define
\begin{equation}
F_{\vec x, \vec y} = \ddp{^m F_{\vec y}}{\phi_{x_1} \ldots \partial\phi_{x_m}}.
\end{equation}
We define a $\phi$-dependent pairing of elements of $\Ncal$ with test functions, by
\begin{equation}
\label{e:pairing}
\langle F, g \rangle_\phi
  =
\sum_{\vec x, \vec y}
\frac{1}{|\vec x|! |\vec y|!}
F_{\vec x,\vec y}(\phi, \bar\phi)
g_{\vec x,\vec y}.
\end{equation}

Let $B(\Phi)$ denote the unit $\Phi$-ball in the space of test functions. Then the
$T_\phi = T_\phi(\h_j)$ semi-norm on $\Ncal^\varnothing$ is defined by
\begin{equation}
\label{e:Tphi-def}
\|F\|_{T_\phi} = \sup_{g\in B(\Phi_j)} |\langle F, g \rangle_\phi|.
\end{equation}

\begin{rk}
\label{rk:Tphi-n}
If $n \ge 1$, a test function is a function $g$ on sequences over $\Lambda\times\{1,\ldots,n\}$.
For any such sequence $\vec x = ((x_1, i_1), \ldots, (x_m, i_m))$, we write $|\vec x| = m$
and set
\begin{equation}
F_{\vec x}
	=
\ddp{^m F}{\varphi^{i_1}_{x_1} \ldots \partial\varphi^{i_m}_{x_m}}
\end{equation}
and
\begin{equation}
\langle F, g \rangle_\varphi
	=
\sum_{\vec x} \frac{1}{|\vec x|!} F_{\vec x}(\varphi) g_{\vec x}.
\end{equation}
Then the $T_\varphi$ seminorm can be defined as in \eqref{e:Tphi-def}.
\end{rk}

To extend the $T_\phi$ seminorm to $\Ncal$, we make use of an additional sequence
of parameters $\h_{\sigma,j}$. For any $F \in \Ncal$ of the form \eqref{e:obs-decomp},
we let
\begin{equation}
\|F\|_{T_\phi}
	=
\|F_\varnothing\|_{T_\phi}
	+ (\|F_a\|_{T_\phi} + \|F_b\|_{T_\phi}) \h_\sigma
	+ \|F_{ab}\|_{T_\phi} \h_\sigma^2.
\end{equation}

By its definition, the $T_\phi$ seminorm controls the values of $F$ and its derivatives
(up to order $p_\Ncal$) at $\phi$. For instance, we will make use of the following facts.

\begin{lemma}
\label{lem:deriv-norm-bds}
For $F \in \Ncal$, we have $|F_\varnothing(0)| \le \|F\|_{T_0}$ and
\begin{equation}
\label{e:deriv-norm-bd}
|D^2 F_\varnothing(0; \1, \1)|
	\le
2 \|F\|_{T_0(\h_j)} \|\1\|^2_{\Phi_N(\h_j)}
	=
2 \|F\|_{T_0(\h_j)} \h_j^{-1}
\end{equation}
and
\begin{equation}
|D^2_{\sigma_0\sigma_x} F_\varnothing|
	\le
\h_{\sigma,j}^{-2} \|F\|_{T_0}.
\end{equation}
\end{lemma}

An essential property of the $T_\phi$ seminorm is the following \emph{product property},
which is essential to fully take advantage of the factorization property \eqref{e:uncorr}.

\begin{prop}
\label{prop:prod}
If $F, G \in \Ncal$, then $\|F G\|_{T_\phi} \le \|F\|_{T_\phi} \|G\|_{T_\phi}$.
\end{prop}

\begin{rk}
This follows essentially from the fact that
the series expansion of the product of two functions is the product of their
respective series expansions (see \cite{BS-rg-norm}). This is why the $T_\phi$
seminorm was defined in terms of the pairing \eqref{e:pairing}.
\end{rk}

%%%%%%%%%%%%%%%%%%%%%%%%%%%%%%%%%%%%%%%%%%%%%%%%%%%%%%%%%%%%%%%%%%%%%%%%%%%%%%%

\subsection{Norm weights}
\label{sec:weights}

Control of the $T_\phi$ seminorm is needed for all values of
$\phi$ in order to control the Gaussian expectation in \eqref{e:rgmapZ}.
This will be discussed further in Section~\ref{sec:newnorm}.

For now, we turn our attention to the special case of the $T_0$ seminorm. Recalling
\eqref{e:exp-conv}, it is natural to choose the weights $\h_j$ so that
$\Ex_{C_{j+1}} F$ is of order $\|F\|_{T_0(\h_j)}$.
By Wick's theorem \eqref{e:wick}, for a $1$-component field $\varphi$,
\begin{equation}
\Ex_{C_{j+1}} \varphi_x^{2p} = (2p - 1)!! C_{j+1;00}^p
\end{equation}
and similar statements hold for complex and fermionic fields by the analogues of
Wick's theorem for such fields.
On the other hand, by definition of the $T_0$ seminorm,
\begin{equation}
\label{e:gauss-moments}
\|\varphi_x^{2p}\|_{T_0(\h_j)} \asymp \h_j^{2p}.
\end{equation}
This suggests defining $\h_j$ so that $|C_{j+1;00}| \le O(\h^2_j)$.
% Bounds on the covariance were stated in \eqref{e:scaling-estimate}.
% For instance, with $k = 0$, these become
% \begin{equation}
% \label{e:massless-cov-bd}
% |C_{j;xy}| \le O(L^{-j (d - 2)}).
% \end{equation}

The key to our analysis of the correlation length is that we make a choice of norm
weights that takes full advantage of the
$k$-dependence in the covariance bounds \eqref{e:scaling-estimate}.
With $k = s + 1$, this estimate together with the elementary bound
\begin{equation}
\label{e:mass-decay}
(1 + m^2 L^{2j})^{-k} \le c_L L^{-2(s+1)(j - j_m)_+}
\end{equation}
imply that
\begin{equation}
|C_{j;xy}| \le O(L^{-j (d - 2) - s (j - j_m)_+}),
\end{equation}
where $j_m$ is the \emph{mass scale}, defined by
\begin{equation}
\label{e:jmdef}
j_m	= \lfloor\log_{L} m^{-1}\rfloor.
\end{equation}
Based on this, when $d = 4$, we define the following weights:
\begin{align}
\label{e:elldef-zz}
\ell_j &= \ell_0 L^{-j - s (j - j_m)_+}, \quad
\ell_{\sigma,j}
=
\ell_{j \wedge j_{x}}^{-1} 2^{(j - j_{x})_+} \ggen_j,
\end{align}
where
\begin{equation}
\label{e:jxdef}
j_x = \max\{0,\lfloor \log_{L} (2 |x|)\rfloor\}
\end{equation}
is the \emph{coalescence scale}
and the sequence $\ggen_j = \ggen_j(m^2,g_0)$ will be discussed
in Section~\ref{sec:pt}.
% For now, we remark only that it is bounded above and below by constant multiples of
% the sequence $\gbar$ defined in
% \eqref{e:gbar}, by \cite[Lemma~\ref{log-lem:gbarmcomp}]{BBS-saw4-log}.
The origin of the definition of $\ell_{\sigma,j}$ is discussed in
\cite[Remark~\ref{IE-rk:hsigmot}]{BS-rg-IE}.

We will set $\h_j = \ell_j$ to estimate ``small'' fields. These are fields which
are assumed not to deviate too much from their expected value. A different norm
parameter $\h_j = h_j$ will be used to control ``large'' fields.
This will be discussed in Section~\ref{sec:newnorm}.

%%%%%%%%%%%%%%%%%%%%%%%%%%%%%%%%%%%%%%%%%%%%%%%%%%%%%%%%%%%%%%%%%%%%%%%%%%%%%%%

\subsection{Symmetries}

It is useful to restrict our attention to field functionals $F \in \Ncal$ that
obey certain symmetry conditions preserved by Gaussian expectation (and which
are obeyed by $V_0$).

We let any automorphism $E$ of $\Lambda$ act on $\Ncal$ by $EF(\varphi) = F(E\varphi)$.
We say that $F\in\Ncal$ is \emph{Euclidean invariant} if $EF = F$ for all such automorphisms.

Let $n = 0$.
Define the \emph{supersymmetry generator}
\begin{equation}
Q = (2\pi i)^{1/2} \sum_{x\in\Lambda}
\left(
	\psi_x \ddp{}{\phi_x} + \psib_x \ddp{}{\phib_x}
		-
	\phi_x \ddp{}{\psi_x} + \phib_x \ddp{}{\psib_x}.
\right)
\end{equation}
A form $F \in \Ncal$ is said to be \emph{supersymmetric} if $Q F = 0$.
Such a form is said to be \emph{gauge invariant} if it is invariant under the
\emph{gauge flow} $(q, \bar q) \mapsto (e^{-2\pi it} q, 2^{2\pi it} \bar q)$
for $q = \phi_x, \psi_x$ and all $x\in\Lambda$.

Let $n \ge 1$.
An $n \times n$ matrix $T$ acts on $\Ncal$ via $T F(\varphi) = F(T \varphi)$.
We say that $F\in\Ncal$ is \emph{$O(n)$ invariant} if $TF = F$ for all
orthogonal matrices $T$.

%%%%%%%%%%%%%%%%%%%%%%%%%%%%%%%%%%%%%%%%%%%%%%%%%%%%%%%%%%%%%%%%%%%%%%%%%%%%%%%
%%%%%%%%%%%%%%%%%%%%%%%%%%%%%%%%%%%%%%%%%%%%%%%%%%%%%%%%%%%%%%%%%%%%%%%%%%%%%%%

\section{Perturbative coordinate}

As mentioned in Section~\ref{sec:rg-intro}, one of Wilson's key insights was that the renormalisation
group could be well-approximated by a finite-dimensional dynamical system. In this
section, we reformulate Wilson's insights in terms of the covariance decomposition
and define a subspace on which this finite-dimensional system will evolve.

% The dynamical system is analysed via a perturbative part which is tracked accurately
% to second order in $g$, together with a third-order non-perturbative part whose study
% forms the main part of our effort.  For the perturbative part, we first introduce
% an appropriate space of local field polynomials.

%%%%%%%%%%%%%%%%%%%%%%%%%%%%%%%%%%%%%%%%%%%%%%%%%%%%%%%%%%%%%%%%%%%%%%%%%%%%%%%

\subsection{Dimensional analysis}

% We define the \emph{scaling dimension}
% \begin{equation}
% [\varphi] = \frac{d - 2}{2}.
% \end{equation}
We call $M_x \in \Ncal$ a local monomial if it is a monomial in $\varphi_x$ and
its (discrete) gradients, i.e.\ if $M_x$ has the form
\begin{equation}
\label{e:field-mon}
M_x = (\nabla^{\alpha_1} \varphi_x) \ldots (\nabla^{\alpha_p} \varphi_x).
\end{equation}
The $T_0$ seminorm of a local monomial $M_x$ essentially
just counts the number of fields and derivatives in $M_x$. For instance, let
$\varphi$ be a $1$-component Gaussian field with covariance $C_j$. Then for $M_x$
as above,
\begin{equation}
\|M_x\|_{T_0(\ell_j)}
	=
O(L^{-j (|\alpha| + p [\varphi])})
\end{equation}
where $|\alpha| = |\alpha_1| + \cdots + |\alpha_p|$ and
\begin{equation}
[\varphi] = \frac{d - 2}{2}
\end{equation}
is the \emph{scaling dimension}. Based on this observation, we define the
\emph{dimension} of $M_x$ by
\begin{equation}
\label{e:mon-dim}
[M_x] = |\alpha| + p [\varphi].
\end{equation}
Note here that we have neglected the rapid decay of fields above the mass scale.

By \eqref{e:scaling-estimate}, $\varphi$ is approximately constant on blocks of side $L^j$. In a sense,
the fields on a block act as a unit and this contributes to a volume factor $L^{jd}$.
This leads us to compare the dimension of a monomial with the dimension $d$ of the
lattice. We say that $M_x$ is \emph{relevant} if $[M_x] < d$, \emph{marginal} if
$[M_x] = d$, and \emph{irrelevant} if $[M_x] > d$.

%%%%%%%%%%%%%%%%%%%%%%%%%%%%%%%%%%%%%%%%%%%%%%%%%%%%%%%%%%%%%%%%%%%%%%%%%%%%%%%

\subsection{Local field polynomials}
% Based on clp

For $y \in \Lambda$, we supplement \eqref{e:taudef}--\eqref{e:nablatau} and \refeq{tauphi}
by defining
\begin{equation}
\label{e:tauphi2}
\quad \tau_{\nabla\nabla,y}
	=
\begin{cases}
\frac 12 \sum_{e \in \units}
\left(
	(\nabla^e \phi)_y (\nabla^e \bar\phi)_y +
	(\nabla^e \psi)_y (\nabla^e \bar\psi)_y
\right),
	& n = 0 \\
\frac{1}{4} \sum_{|e| = 1} \nabla^e \varphi_y \cdot \nabla^e \varphi_y,
	& n \ge 1.
\end{cases}
\end{equation}

When $n = 0$, it can be shown that the only marginal and relevant local monomials
that are Euclidean invariant and supersymmetric are constant multiples of
\begin{equation}
1, \quad \tau_x, \quad \tau_x^2, \quad \tau_{\Delta,x}, \quad \tau_{\nabla\nabla,x}.
\end{equation}
When $n \ge 1$, these are the only marginal and relevant monomials that are Euclidean
invariant and $O(n)$-invariant (see \cite{BBS-rg-pt}).

The marginal and relevant contributions to the evolution of the renormalisation group
will be tracked by a \emph{local polynomial} (a sum of local monomials) of the form
$\sum_{x\in\Lambda} \Vc_y$, where (recall \eqref{e:obs-couple})
\begin{align}
\lbeq{Vy}
\Vc_y
	&=
g \tau_y^2 + \nu \tau_y + z \tau_{\Delta,y}
	% + y \tau_{\nabla\nabla,y}
	+ u
		\nnb&\quad
	- \1_{y=\pp}\lambda_{\pp} f_0 \sigma_0
	- \1_{y=\qq}\lambda_{\qq} f_x \sigma_x
		\nnb&\quad
	- \textstyle{\frac 12} (\1_{y=\pp} q_\pp + \1_{y=\qq}q_\qq )\sigma_\pp\sigma_\qq.
\end{align}
We have omitted $\tau_{\nabla\nabla}$ as summation by parts gives
\begin{equation}
\label{e:nabla-delta}
\sum_{x\in\Lambda} \tau_{\nabla\nabla,x} = \sum_{x\in\Lambda} \tau_{\Delta,x}.
\end{equation}

\begin{rk}
\label{rk:noconst}
When $n = 0$, we can also omit $u$ since constant terms are not produce by the 
Gaussian super-expectation. For example, $\Ex_C\theta \tau_x$ has constant part
$0$ by \eqref{e:wick-complex2} and \eqref{e:wick-fermion2}. More generally,
this is a consequence of supersymmetry (see \cite{BBS-rg-pt}).
\end{rk}

We define $\Vcalc$ to be the space of all polynomials of the form $\Vc_y$.
Given $X \subset \Lambda$, we let
\begin{equation}
\label{e:Vcalesig}
\Vcalc(X) = \{\Vc(X) : \Vc \in \Vcalc \},
\end{equation}
where $\Vc(X)$ is defined as in \eqref{e:VX}.
We also make use of the % subspaces $\Vcalc^{(1)} \subseteq \Vcalc$ consisting
% of polynomials with $y = 0$, as well as the
subspace $\Vcalp$ of polynomials with
$u = y = q_\pp=q_\qq = 0$.
We will usually denote an element of $\Vcalp$ as $\Vp$.
For $\Vc \in \Vcalc$, we define % maps $\Vc \mapsto \Vc^{(1)} \in \Vcalc^{(1)}$ and
the map $\Vc \mapsto \Vc^{(0)} \in \Vcalp$, which sets
% Both maps replace $z\tau_{\Delta}+y\tau_{\nabla\nabla}$
% by $(z+y)\tau_{\Delta}$, and the latter additionally
$u = q_\pp = q_\qq = 0$.

We define the $\Vcalc = \Vcalc_j$ norm by
\begin{equation}
\label{e:Vnormdef}
\begin{aligned}
\|\Vc\|_{\Vcalc} &=
\max\Big\{
|g|, L^{2j}|\nu|, |z|, |y|,  L^{4j}|u|,
\ell_j\ell_{\sigma,j}(|\lambda_\pp|\vee|\lambda_\qq|),\;
%\\
%& \qquad\qquad\qquad
 \ell_{\sigma,j}^{2} (|q_\pp|\vee|q_\qq|)
\Big\}
\end{aligned}
\end{equation}
on $\Vc \in \Vcalc$, which depends on the parameters $\ell_j$ and $\ell_{\sigma,j}$.
The $\Vcalc = \Vcalc_j$ norm is equivalent to the $T_0(\ell_j)$ seminorm on $\Vcalc(B)$
when $|B| = L^{jd}$:
\begin{equation}
\|\Vc\|_\Vcalc \asymp \|\Vc(B)\|_{T_0(\ell_j)} = L^{jd} \|\Vc_y\|_{T_0(\ell_j)}.
\end{equation}

%%%%%%%%%%%%%%%%%%%%%%%%%%%%%%%%%%%%%%%%%%%%%%%%%%%%%%%%%%%%%%%%%%%%%%%%%%%%%%%

\subsection{Perturbative flow}
\label{sec:pt}

Here we discuss how to maintain the form $Z_j \approx e^{-\Vp_j(\Lambda)}$ to
second order with $\Vp_j\in\Vcalp$. The basic idea begins with the \emph{cumulant
expansion}
\begin{equation}
\Ex_C \theta e^{-\Vp(\Lambda)}
	\approx
e^{-\Ex_C \theta \Vp(\Lambda) + \scriptstyle{\frac12} \Ex_C (\theta \Vp(\Lambda); \theta \Vp(\Lambda))}.
\end{equation}
In \cite{BS-rg-loc} an operator $\Loc_x$ is defined so that $\Loc_x F$ is an
approximation of $F$ by a local polynomial at $x$. We make the split
\begin{equation}
\frac12 \Ex_C (\theta \Vp_x; \theta \Vp(\Lambda))
	=
\frac12 \Loc_x \Ex_C (\theta \Vp_x; \theta \Vp(\Lambda))
	+
\frac12 (1 - \Loc_x) \Ex_C (\theta \Vp_x; \theta \Vp(\Lambda))
\end{equation}
With $\Vp(\Lambda) = \sum_{x\in\Lambda}$ and $e^F \approx 1 + F$, we get
\begin{equation}
\Ex_C \theta e^{-\Vp(\Lambda)}
	\approx
e^{-\Ex_C \theta \Vp(\Lambda)
	+
\scriptstyle{\frac12} \Loc_x \Ex_C (\theta \Vp_x; \theta \Vp(\Lambda))}
\Big(1 + \tfrac12 (1 - \Loc_x) \Ex_C (\theta \Vp_x; \theta \Vp(\Lambda))\Big).
\end{equation}

Based on this idea, in \cite{BBS-rg-pt}
a map\footnote{In \cite{BBS-rg-pt}, $\Vpt$ is defined on a
larger space including $\tau_{\nabla\nabla}$. Here, following \eqref{e:nabla-delta},
we define $\Vpt$ by composing that map with the map that replaces
$z \tau_\Delta + y \tau_{\nabla\nabla}$ by $(z + y) \tau_\Delta$.}
$\Vpt : \Vcalp\to\Vcalc$ of the form
\begin{equation}
\label{e:Vpt-def}
\Vpt(\Vp) = \Ex_C\theta\Vp - P(\Vp)
\end{equation}
is defined so as to maintain the approximation
\begin{equation}
\label{e:Zapprox}
Z_j \approx e^{-\Vc_j(\Lambda)} (1 + W_j),
\end{equation}
where $P(\Vp)$ is a local polynomial and $W_j = W_j(\Vp)$ is non-local.
Both are explicitly defined and second-order in $V_j$;
in fact, by \cite[\eqref{IE-e:W-logwish}]{BS-rg-IE},
\begin{equation}
\label{e:Wbilinbd}
\|W_j\|_{T_0(\ell_j)}
	\le
O(\chicCov_j) \|V\|_\Vcal^2,
\end{equation}
where $\chicCov_j$ is a parameter that decays exponentially above the mass scale
and will be discussed in Section~\ref{sec:step}. We will elaborate on the  meaning
of \eqref{e:Zapprox} in Section~\ref{sec:rgcoord}.

% Precisely, $\Vpt$ depends on a covariance $C$ and satisfies
% \begin{equation}
% \Ex_C\theta e^{-V(\Lambda)} (1 + W(\Vp))
% 	=
% e^{-\Vpt(\Lambda)} (1 + W(\Vpt^{(0)})) + O(\Vp^3),
% \end{equation}
% where $W = W(V)$ is an explicit polynomial that is quadratic in $V\in\Vcalp$; see
% \cite[\eqref{pt-e:WLTF}]{BBS-rg-pt} for its definitin, which also depends on $C$.
% \todo{Explain $O(V^3)$.}

The map $\Vpt$ depends on the covariance $C$ and
in practice we set $C = C_{j+1}$ and obtain a sequence $\Vpt = \Vc_{\mathrm{pt},j+1}$.
By successively iterating these maps, we generate a sequence of coupling constants that
we refer to as the \emph{perturbative flow}. The equations defining this flow can be
computed exactly by way of Feynman diagrams or with a computer program \cite{BBS-rg-ptsoft}.
In \cite{BBS-rg-pt}, these flow equations are summarized and it is shown that a change of
variables can be used to triangularize the resulting system of equations up to third-order
errors. Below, we summarize these transformed flow equations for $g$, $\lambda$, and $q$.

\subsubsection{The flow of \texorpdfstring{$g$}{g}}

The (transformed) perturbative flow of $g$ takes the form
\begin{equation}
\label{e:gbar}
\gbar_{j+1}
	=
\gbar_j - \beta_j  \gbar_j^{2}, \qquad \gbar_0
	=
g_0
\end{equation}
where
\begin{equation}
\beta_j = (8 + n) \sum_{x\in\Zd} (w_{j+1;0x}^2 - w_{j;0x}^2),
	\quad
w_j = \sum_{i=1}^j C_i.
\end{equation}
The sequence $\beta_j$ is closely related to the free bubble diagram,
which is the $\ell_2(\Zd)$ norm of the Green function $C$ (this is
the expected number of intersections of two independent, killed simple
random walks).

The recursion \eqref{e:gbar} was analyzed in \cite{BBS-rg-pt}. It was shown in
\cite[Proposition~\ref{log-prop:approximate-flow}]{BBS-saw4-log}
that
\begin{equation}
\label{e:gjxgjmbd}
\gbar_{j}
	=
O((\log m^{-1})^{-1}) \;\; \text{for $j \geq j_m$},
	\quad
\gbar_{j_x}
	=
O((\log |x|)^{-1}) \;\; \text{for $j_x \leq j_m$.}
\end{equation}

\begin{rk}
A heuristic argument is as follows: Using Proposition~\ref{prop:Cdecomp}, it is
straightforward to show that
\begin{equation}
\beta_j = O(L^{-j (d - 4)}).
\end{equation}
Thus, a crude approximation to the flow of $\gbar$ is the recursion
\begin{equation}
x_{j+1} = x_j - c \1_{j \le j_m} x_j^2,
	\quad
c > 0.
\end{equation}
Comparing this to the differential equation $\dot x = - c x^2$, which has solutions
of the form $x(t) = (C + c t)^{-1}$, it is reasonable to expect that $x_j \approx (c j)^{-1}$
for $j \le j_m$ and $x_j \approx x_{j_m}$ for $j > j_m$. The relations \eqref{e:gjxgjmbd}
follow easily if $g_j$ behaves in a similar way.
\end{rk}

Following \cite[\eqref{log-e:ggendef}]{BBS-saw4-log}, we define the parameter
$\ggen_j$ in \eqref{e:elldef-zz} as a function of two variables $(\mgen^2, \ggen_0)$ by
\begin{equation}
\label{e:ggendef}
\ggen_j(\mgen^2,\ggen_0)
	=
\gbar_j(0,\ggen_0) \1_{j \le j_{\mgen}} + \gbar_{j_{\mgen}}(0,\ggen_0) \1_{j > j_{\mgen}}.
\end{equation}
These parameters play an important role in Section~\ref{sec:step} and in the proof of
Theorem~\ref{thm:rhatflow} (see Section~\ref{sec:flow}.

\subsubsection{The flow of \texorpdfstring{$\lambda$ and $q$}{lambda and q}}

It was shown in \cite[\eqref{pt-e:lambdapt2}--\eqref{pt-e:qpt2}]{BBS-rg-pt} (for $n = 0$)
and \cite[Proposition~\ref{phi4-prop:pt}]{ST-phi4} (for $n \ge 1$) that,
with $C = C_{j+1}$ and $u = 0, x$,
\begin{align}
\label{e:lampt}
\lambda_{u,\pt}
	&=
\begin{cases}
(1 - \delta[\nu w^{(1)}]) \lambda_u,
	& j + 1 < j_x \\
\lambda_u,
	& j + 1 \ge j_x
\end{cases}
	\\
\label{e:qpt}
q_\pt
	&=
q + \lambda_0 \lambda_x C_{0x},
\end{align}
where
\begin{equation}
\label{e:deltanuw1}
\delta[\nu w^{(1)}] = (\nu + 2 g C_{00}) w^{(1)}_{j+1} - \nu w^{(1)}_j,
	\qquad
w^{(1)}_j = \sum_{x\in\Lambda} \sum_{i=1}^j C_{i;0x}.
\end{equation}
Note that $q_\pt = q$ for $j + 1 < j_x$.

%%%%%%%%%%%%%%%%%%%%%%%%%%%%%%%%%%%%%%%%%%%%%%%%%%%%%%%%%%%%%%%%%%%%%%%%%%%%%%%
%%%%%%%%%%%%%%%%%%%%%%%%%%%%%%%%%%%%%%%%%%%%%%%%%%%%%%%%%%%%%%%%%%%%%%%%%%%%%%%

\section{Non-perturbative coordinate}
% Based on clp
\label{sec:rgcoord}

Let $\volume$ denote either $\Lambda_N$ or $\Zd$. We allow $\Ncal$ to depend on
$\volume$. If $\volume = \Lambda$, then $\Ncal = \Ncal(\Lambda)$ was defined in
Section~\ref{sec:Ncal}. Otherwise, we set
\begin{equation}
\Ncal(\Zd) = \bigcup_{\text{finite } X \subset \volume} \Ncal(X).
\end{equation}

We set $N(\volume) = N$ if
$\volume = \Lambda_N$ and $N(\volume) = \infty$ if $\volume = \Zd$.
For $j \le N(\volume)$ (meaning $j < \infty$ if $N(\volume) = \infty$), we partition
$\volume$ into disjoint
\emph{scale-$j$ blocks} of side length $L^j$, each of which is a translate of
the block $\{ x \in \Lambda : 0 \le x_i < L^j, i = 1, \ldots, d\}$.
A scale-$j$ \emph{polymer} is a union of scale-$j$ blocks.
Given a scale-$j$ polymer $X$ and $k \le j$, we let $\Bcal_k(X)$
(respectively, $\Pcal_k(X)$)
denote the set of all scale-$k$ blocks (respectively, scale-$k$ polymers) in $X$.
We sometimes write $\Bcal_j = \Bcal_j(\volume)$ and $\Pcal_j = \Pcal_j(\volume)$
when the volume $\volume$ is implicit.

Given maps $F, G : \Pcal_j(\Lambda) \to \Ncal$ (sometimes called \emph{polymer activities}),
we define the \emph{circle product} $F \circ G : \Pcal_j(\Lambda) \to \Ncal$ by
\begin{equation}
\label{e:circ}
(F \circ G)(X) = \sum_{Y\in\Pcal_j(X)} F(X \setminus Y) G(Y).
\end{equation}
The circle product is commutative, associative, and has the identity element
\begin{equation}
\1_\varnothing(X) =
\begin{cases}
1,	& X = \varnothing \\
0,	& X \ne \varnothing
\end{cases}.
\end{equation}
We track $Z_j$ using \emph{renormalisation group coordinates}
$u_j, q_{0,j}, q_{x,j} \in \R$,
$I_j, K_j : \Pcal_j \to \Ncal$, defined such that
\begin{equation}
\label{e:IcircKnew}
	Z_j = e^{\zeta_j}(I_j\circ K_j)(\Lambda),
	\qquad
	\zeta_j= - u_j|\Lambda|
	+ \textstyle{\frac 12} (q_{\pp,j} + q_{\qq,j}) \sigma_\pp\sigma_\qq
	.
\end{equation}
The coordinate $I_j = I_j(\Vp, \cdot)$ is defined by
\begin{equation}
\label{e:Idef}
I_j(\Vp, X)
	=
\prod_{B \in \Bcal_j(X)} e^{-\Vp(B)} (1 + W_j(B, \Vp)), \quad X \in \Pcal_j,
	\quad
\Vp \in \Vcalp
\end{equation}
and is said to satisfy the \emph{block-factorization} property.
This gives a rigorous implementation of \eqref{e:Zapprox}.

Before defining the space in which $K_j$ lies, we need the following notions:
\begin{itemize}
\item
We call a nonempty polymer $X\in \Pcal_j$ \emph{connected}
if for any $x, x' \in X$, there is a sequence
$x = x_0, \ldots, x_n = x' \in X$ such that
$|x_{i+1} - x_i|_\infty = 1$ for $i = 0, \ldots, n - 1$.
Let $\Ccal_0 = \Ccal_0(\volume)$ denote the set of connected polymers.

\item
For $X \in \Pcal_j$, let $|X|_j$ denote the number of scale-$j$ blocks in $X$.
We call a connected polymer $X\in\Ccal_j$ a \emph{small set} if $|X|_j \le 2^d$.
Let $\Scal_j = \Scal_j(\volume)$ denote the collection of small sets.
The \emph{small set neighbourhood} $X^\square$ of a polymer $X$ is defined by
\begin{equation}
\label{e:ssn}
X^\Box = \bigcup_{Y\in\Scal_j : Y \cap X \ne \varnothing} Y.
\end{equation}

\item
Two polymers $X, Y$ \emph{do not touch} if $\min(|x - y|_\infty : x \in X, y \in Y) > 1$.
We let ${\rm Comp}(X)$ denote the set of maximal connected components that do not touch
in $X$.
\end{itemize}

\begin{defn}
For $j \le N(\volume)$, let $\CKspace_j = \CKspace_j(\volume)$ denote the complex
vector space of maps $K : \C_j(\volume) \to \Ncal(\volume)$ satisfying the following
properties:
\begin{itemize}
\item
Field Locality: If $X \in \Ccal_j$, then $K(X) \in \Ncal(X^\square)$.
Also: (i) $\pi_\alpha K(X) = 0$ unless $\alpha \in X$ for $\alpha = a, b$;
(ii) $\pi_{ab} K(X) = 0$ unless $a\in X$ and $b \in X^\square$ or vice-versa;
and (iii) $\pi_{ab} K(X) = 0$ if $X \in \Scal_j$ and $j < j_x$.

\item
Symmetry: (i) $K$ is Euclidean invariant and, if $n = 0$, $K$ is gauge-invariant;
(ii) $\pi_\varnothing K$ is supersymmetric and has no constant part if $n = 0$
or $O(n)$ invariant if $n \ge 1$;
and (iii) $\pi_\varnothing K$ is Euclidean covariant.
\end{itemize}
We let $\Kspace_j = \Kspace_j(\volume)$ denote the complex vector space of functions
$K \in \CKspace_j$ with the following additional property:
\begin{itemize}
\item
Component factorization: If $X \in \Pcal_j$, then $K(X) = \prod_{Y\in{\rm Comp}(X)} K(Y)$.
\end{itemize}
\end{defn}

Addition in $\CKspace_j$ is defined by $(F_1 + F_2)(X) = F_1(X) + F_2(X)$.
We extend any $F \in \CKspace_j$ to $\Kspace_j$ by defining
$F(X) = \prod_{Y\in{\rm Comp}(X)} F(Y)$.

\subsection{Initial coordinates}

At scale $j = 0$, we are given $\Vp^+_0$ as defined in \eqref{e:V0def}
and we set $\zeta_0 = 0$. In particular,
the initial values of $u$, $q_0$, $q_x$ are zero, and the initial values of $\lambda_0$, $\lambda_x$
are $1$. By definition, $W_0 = 0$.
For $X \subset \Lambda$, we define
\begin{equation}
\label{e:IK0def}
I_0^+(X) = I_0(\Vp^+_0, X) = \prod_{x\in X} e^{-\Vp^+_{0,x}},
	\qquad
K_0^+(X) = \prod_{x \in X} I_{0,x}^+ (e^{-\gamma_0 U^{+}_{x}} - 1).
\end{equation}
It is straightforward to verify that $K_0 \in \Kcal_0$.
With these choices, $Z_0$ (recall \refeq{Z0def})
takes the form \eqref{e:IcircKnew}, and we seek
$(u_j, q_{0,j}, q_{x,j}, \Vp_j, K_j)$ such that this continues to hold as the scale advances.

Equivalently, given $(\Vp_j, K_j)$, we must define
$(\delta u_{j+1}, \delta q_{0,j+1}, \delta q_{x,j+1}, V_{j+1}, K_{j+1})$ so that
\begin{equation} \label{e:IcircKdu}
	\Ex_{j+1}\theta(I_j \circ K_j)(\Lambda)
	=
	e^{-\delta \zeta_{j+1}}(I_{j+1} \circ K_{j+1})(\Lambda).
\end{equation}
Moreover, we need $K_j$ to contract as the scale advances, under an appropriate norm.
The construction of (scale-dependent) maps $\Vc_+$ and $K_+$ such that
\eqref{e:IcircKdu} holds with
\begin{equation}
(\delta u_{j+1}, \delta q_{0,j+1}, \delta q_{x,j+1}, \Vp_{j+1})
	=
\Vc_+(\Vp_j, K_j),
	\quad
K_{j+1} =  K_+(\Vp_j, K_j)
\end{equation}
is the main accomplishment of \cite{BS-rg-step}.

%%%%%%%%%%%%%%%%%%%%%%%%%%%%%%%%%%%%%%%%%%%%%%%%%%%%%%%%%%%%%%%%%%%%%%%%%%%%%%%
%%%%%%%%%%%%%%%%%%%%%%%%%%%%%%%%%%%%%%%%%%%%%%%%%%%%%%%%%%%%%%%%%%%%%%%%%%%%%%%

\section{Renormalisation group step}
% from saw-sa and clp
\label{sec:step}

For fixed $(\mgen^2, \ggen_0) \in [0, \delta) \times (0, \delta)$,
we define a sequence $\ggen_j = \ggen_j(\mgen^2, \ggen_0)$ by
\begin{equation} \label{e:ggendef}
  \ggen_j(m^2,g_0) =
  \gbar_j(0,g_0) \1_{j \le j_m} + \gbar_{j_m}(0,g_0) \1_{j > j_m},
\end{equation}
where $\gbar_j = \gbar_j(m^2, g_0)$ is the sequence discussed in Section~\ref{sec:pt};
in particular, $\ggen_0(\mgen^2, \ggen_0) = \ggen_0$.
In \cite[Section~\ref{step-sec:Knorms}]{BS-rg-step},
a sequence of norms $\|\cdot\|_{\Wcal_j} = \|\cdot\|_{\Wcal_j(\mgen^2, \ggen_j, \Lambda)}$
parameterised by $(\mgen^2, \ggen_j)$ is defined on $\Kcal_j$.
These are defined in terms of the $T_\phi(\h_j)$ norms with parameters $\h_j = \ell_j, h_j$.
In order to make use of the improved norm parameters with $s > 1$,
we must modify the definition of $\Wcal_j$ when $j$ is above the mass scale.
This will be discussed in Chapter~\ref{sec:RGstep}.
We note here only the fact that (for any $s \ge 0$) the $\Wcal_j(\Lambda)$
norm dominates the $T_0(\ell_j)$ norm in the following sense:
\begin{equation}
\label{e:T0dom}
\|F(\Lambda)\|_{T_0(\ell_j)} \le \|F\|_{\Wcal_j}.
\end{equation}
We let $\Wcal_j = \Wcal_j(\volume)$ denote the space of $K\in\Kcal_j(\volume)$ with
finite $\Wcal_j$ norm and
denote the ball of radius $r$ in the normed space $\Wcal_j$ by $B_{\Wcal_j}(r)$.

In \cite[\eqref{log-e:mass-scale}--\eqref{log-e:chidef}]{BBS-saw4-log},
a function $\chicCov_j = \chicCov_j(m^2)$ (denoted $\chi_j$ in \cite{BBS-saw4-log})
is defined in such a way that $\chicCov_j$ decays exponentially
when $j$ is sufficiently large depending on $m$. We write $\chicCovgen_j = \chicCov_j(\mgen^2)$.
Given constants $\alpha > 0$ and $C_\DV > 0$, we define the (finite-volume)
renormalisation group domains
\begin{align}
\label{e:DVdef}
\DV_j
	&=
\{ \Vp\in \Vcalp :
	g > C_{\DV}^{-1} \ggen_j, \; \|\Vp\|_{\Vcalc} < C_{\DV} \ggen_j \}, \\
\label{e:domRG}
\domRG_j
	&= \domRG_j(\volume)
	= \DV_j \times B_{\Wcal_j}(\alpha \chicCovgen_j \ggen_j^3).
\end{align}
The domain $\DV_j$ is independent of the volume $\volume$ while $\domRG_j$
depends on $\volume$ through $\Wcal_j$.

In the statement of the following theorem, we fix the scale $j$ and
consider maps $\Vc_+ = \Vc_{j+1}$ and $K_+ = K_{j+1}$ that act on the domain
$\domRG_j$ and map into $\Vcalc_{j+1}$, $\Kcal_{j+1}$, respectively.
We will drop the scale $j$ from the notation for objects at scale $j$
and replace $j + 1$ with $+$.
The deviation of the map $\Vc_+$ from the perturbative map $\Vpt$
is denoted by $R_+$:
\begin{equation}
\label{e:Rplusdef}
    R_+(\Vp,K) = \Vc_+(\Vp,K) -\Vpt(\Vp).
\end{equation}

The renormalisation group map depends also on the mass $m^2$ through its
dependence on the covariance $C_{j+1}$.
We let $\Igen_j(\mgen^2)$ be the neighbourhood of $\mgen^2$ defined by
\begin{equation}
\lbeq{Itilint}
    \Igen_j = \Igen_j(\mgen^2) =
    \begin{cases}
    [\frac 12 \mgen^2, 2 \mgen^2] \cap \Iint_j & (\mgen^2 \neq 0)
    \\
    [0,L^{-2(j-1)}] \cap \Iint_j & (\mgen^2 =0)
    \end{cases},
\end{equation}
where $\Iint_j = [0, \delta]$ if $j < N$ and $\Iint_N = [\delta L^{-2 (N - 1)}, \delta]$.

\begin{theorem}
\label{thm:step-mr-fv}
Let $d = 4$ and let $n \ge 0$. Fix $s > 1$. Let $C_\DV$ and $L$
be sufficiently large. There exist $M>0$, $\delta >0$,
and $\kappa = O(L^{-1})$											% added
such that for $\ggen \in (0,\delta)$
and $\mgen^2 \in \Iint_+$,											% added
and with the domain $\domRG$ defined using any $\DVa> M$, the maps
\begin{equation}
\label{e:RKplusmaps}
R_+:\domRG \times \Igen_+ % (\mgen^2)
	\to \Vcal,
		\quad
K_+:\domRG \times \Igen_+ % (\mgen^2)
	\to \Wcal_{+}%(\sgen_+)
\end{equation}
are analytic in $(V, K)$											% added
and satisfy the estimates
\begin{equation}
\label{e:RKplus}
\|R_+\|_{\Vcal}
\le
M\chigen_+\ggen_+^{3}
, \qquad
\|K_+\|_{\Wcal_+}
\le
M\chigen_+ \ggen_+^{3}
\end{equation}
and
\begin{equation}
\label{e:DKkappa}
\|D_K K_+\|_{L(\Wcal,\Wcal_+)} \le \kappa.
\end{equation}
When $\volume = \Lambda$, these maps define $(\Vp,K)\mapsto (\Vc_+,K_+)$
obeying \eqref{e:IcircKdu}.
\end{theorem}

With $s = 0$ in the choice of weights $\ell_j$ and $\ell_{\sigma,j}$,
this theorem was the main achievement of \cite{BS-rg-step}. With $s > 1$
arbitrary, the bounds \eqref{e:RKplus} are greatly improved beyond the
mass scale. The statement in \cite{BS-rg-step} with $s = 0$ contains
bounds on the derivatives of the maps $R_+$ and $K_+$. Our improvements
apply to these bounds as well, but we do not state them here as we will
not make direct use of these bounds\footnote{These derivative bounds are
indirectly used to prove Theorem~\ref{thm:rhatflow}.}.

Note that the maps $R_+$ and $K_+$ are \emph{independent} of $s$.
The proof of Theorem~\ref{thm:step-mr-fv} involves showing that the
inductive estimates \eqref{e:RKplus} hold for any $s$. In some cases,
we will make use of these estimates both with $s > 1$ and $s = 0$.

\commentbw{I don't like the next paragraph.}

The proof of Theorem~\ref{thm:step-mr-fv} is an adaptation of the proof
of the $s = 0$ case contained
in\footnote{For $n \ge 1$, there is an additional step to deal with observables.
This is dealt with in the proof of \cite[Theorem~\ref{phi4-thm:step-mr-fv}]{ST-phi4}
and is unchanged in the present context.}
\cite{BS-rg-IE,BS-rg-step}.
Some steps in this proof continue to hold unchanged whereas others require
some modification. As mentioned above, a major change that is required is
a new definition of $\Wcal_j$ above the mass scale. A detailed verification
that the proof holds for $s > 1$ is carried out in Chapter~\ref{sec:RGstep}.

Theorem~\ref{thm:step-mr-fv} expresses a contractive property of the map $K_+$
through \eqref{e:DKkappa}.
% as it takes $K$ in a ball whose radius involves $\alpha=4M$ at scale $j$ to
% an image which lies in a ball whose radius involves the smaller number $M$ at scale $j+1$.
The fact that $K_+$ is a contraction is used, in Theorem~\ref{thm:rhatflow},
to prove that, for $m^2$ and $g_0$ sufficiently small, there exist
\emph{critical} initial conditions
$\nu_0 = \nu_0^c(m^2, g_0, \gamma_0; n)$ and $z_0 = z_0^c(m^2, g_0,\gamma_0; n)$
such that for any $N$,
iteration of the maps $(\Vc_+,K_+)$ defines a sequence $(\Vp_j, K_j)$
which lies in the domain $\domRG_j$ and obeys the estimates \refeq{RKplus}
\emph{for all} $j = 1, \ldots, N$.
This construction of critical initial conditions uses the $s=0$ version
of \refeq{elldef-zz}.

%%%%%%%%%%%%%%%%%%%%%%%%%%%%%%%%%%%%%%%%%%%%%%%%%%%%%%%%%%%%%%%%%%%%%%%%%%%%%%%
%%%%%%%%%%%%%%%%%%%%%%%%%%%%%%%%%%%%%%%%%%%%%%%%%%%%%%%%%%%%%%%%%%%%%%%%%%%%%%%

\section{Renormalisation group flow}
% mainly from saw-sa

Theorem~\ref{thm:step-mr-fv} allows us to perform a single renormalisation group
step. In \cite[Proposition~\ref{log-prop:flow-flow}]{BBS-saw4-log} it was shown
that there exist \emph{critical} initial conditions
$\nu_0^c, z_0^c$ depending on $(m^2, g_0)$ such that this map can be iterated
indefinitely, i.e.\ for $j = 1, \ldots, N$ and any $N$. This results in a
sequence $(\Vc_j, K_j)$ generated by the renormalisation group map, hence
whose elements lie in the domains $\domRG_j$. This was proved with $s = 0$,
but the sequence itself is independent of $s$ and continues to exist in our
setting. However, Theorem~\ref{thm:step-mr-fv} shows that this sequence satisfies
improved estimates. Thus, there is no difficulty in extending
\cite[Proposition~\ref{log-prop:flow-flow}]{BBS-saw4-log} to the $s$-dependent
domains used here.

However, \cite{BBS-saw4-log} studied the WSAW ($\gamma = 0$). In order to study
the WSAW-SA, we extend \cite[Proposition~\ref{log-prop:flow-flow}]{BBS-saw4-log}
by taking advantage of the additional generality in \cite{BBS-rg-flow}, whose
main result is the basis for the proof of this proposition. The result, which we
state as Theorem~\ref{thm:rhatflow}, is the
construction of critical initial conditions $\hat\nu_0^c, \hat z_0^c$ that now
depend on $\gamma_0$ as well as $(m^2, g_0)$. Thus, this theorem is an extension
to $s > 1$ and $\gamma_0 \ne 0$ of \cite[Proposition~\ref{log-prop:flow-flow}]{BBS-saw4-log}.

Let $\delta > 0$ and suppose $r : [0, \delta] \to [0, \infty)$
is a continuous \emph{positive-definite}\footnote{Note that our usage of this term is
different from that in the theory of quadratic forms.} function; by this we
mean that $r(x) > 0$ if $x > 0$ and $r(0) = 0$.
We define
\begin{equation}
\lbeq{Ddef}
D(\delta, r)
	=
\{ (w, x, y) \in [0, \delta]^2 \times (-\delta, \delta) : |y| \leq r(x) \}
\end{equation}
and we let $C^{0,1,\pm}(D(\delta, r))$ denote the space of continuous functions
$f = f(w, x, y)$ on $D(\delta, r)$
that are $C^1$ in $(x, y)$ away from $y = 0$, $C^1$ in $x$ everywhere,
and have left- and right-derivatives in $y$ at $y = 0$.
In our applications, we take $w = m^2$, $x = g_0$ or $\gcc$,
and $y = \gamma_0$ or $\gamma$.

\begin{theorem}
\label{thm:rhatflow}
There exists a domain $D(\delta, \hat r)$ (with $\delta > 0$ and $\hat r$
positive-definite) and functions $\hat\nu_0^c, \hat z_0^c \in C^{0,1,\pm}(D(\delta, \hat r))$
such that for any $(m^2, g_0, \gamma_0) \in D(\delta, \hat r)$
with $g_0 > 0$ and $m^2 \in [\delta L^{-2 (N - 1)}, \delta)$, the following holds:
if $(\Vc_0, K_0) = (\Vp^+_0, K^+_0)$ with $(\nu_0, z_0) = (\hat\nu_0^c, \hat z_0^c)$,
then for any $N \in \N$, there exists a sequence $(\Vc_j, K_j) \in \domRG_j$ such that
\begin{equation}
\label{e:VjKjDj-hat}
(\Vc_{j+1},K_{j+1})
	=
(\Vc_{j+1}(\Vp_j, K_j), K_{j+1}(\Vp_j, K_j)) \text{ for all } j < N
\end{equation}
and \eqref{e:IcircKdu} is satisfied.
Moreover, the sequence $\Vc_j, j = 1, \ldots, N$ is independent of the volume $\Lambda$ and
\begin{equation}
\label{e:hat-est}
\hat\nu_0^c = O(g_0),
\quad
\hat z_0^c = O(g_0)
\end{equation}
uniformly in $(m^2, \gamma_0)$.
% Moreover, the second-order evolution equation for $V_j$ is independent of $\gamma_0$.
\end{theorem}

%%%%%%%%%%%%%%%%%%%%%%%%%%%%%%%%%%%%%%%%%%%%%%%%%%%%%%%%%%%%%%%%%%%%%%%%%%%%%%%

\subsection{Historical remarks}

Expansions such as \eqref{e:circ} go back to the work of Gruber and Kunz \cite{GK71}.
The use of such expansions together with carefully weighted norms to achieve
rigorous control of the renormalisation group map goes back to \cite{BY90}.
Although the notion of a covariance decomposition was implicit in Wilson's
work, its utility in the context of the renormalisation group was first clearly
articulated in
\cite{BCGNOPS78}. Renormalisation group analysis based on a \emph{finite-range}
decomposition was first used in \cite{MS00} based on a suggestion of Brydges.
In \cite{BGM04}, covariance decompositions on the lattice were constructed.
The notes \cite{Bryd09} describe an implementation of the renormalisation group
based on the decomposition of \cite{BGM04}; these notes discuss models on the
Euclidean lattice as well as on a \emph{hierarchical lattice}, which was introduced
by Dyson \cite{Dyso69}. A version of the weakly self-avoiding walk on the hierarchical
lattice was defined and studied by a renormalisation group method in \cite{BEI92,BI03c,BI03d}.
					% setup and statement of main flow result

%% Chapter 3 %%
\chapter{Analysis of critical behaviour}
\label{sec:chi-G-xi}

\setcounter{footnote}{0}

\commentbw{Generally I am having a hard time to learn the structure of the thesis and
of the proof.  Somewhere as early as possible I would like to see a description
of the structure of the thesis:
- the main ingredients of the proof of Thm 1.7.1
- where these ingredients are proved
- what are you including about the proof and what results are you citing
  from the literature.
This relates to one of my comments about Chapter 2 concerning the roles
of Thms 2.7.1 and 2.8.1.}

In this chapter, we prove Theorem~\ref{thm:mr} using Theorem~\ref{thm:rhatflow}.
For simplicity, we drop the parameter $n$ from the notation. In order to employ
Theorem~\ref{thm:rhatflow}, we fix
\begin{equation}
\nu_0 = \hat\nu_0^c(m^2, g_0, \gamma_0),
	\quad
z_0 = \hat z_0^c(m^2, g_0, \gamma_0).
\end{equation}
Then Theorem~\ref{thm:rhatflow} defines a sequence
\begin{equation}
(U_j, K_j) \in \domRG_j,
	\quad
0 \le j \le N
\end{equation}
for any $N$. Moreover, $U_j$ is independent of the volume, so we actually have
an \emph{infinite} sequence
\begin{equation}
U_j \in \DV_j
	\quad
j \ge 0.
\end{equation}
Throughout this section we write $U_j$ as
\begin{equation}
U_{j;x}
	=
g_j \tau_y^2 + \nu_j \tau_y + z_j \tau_{\Delta,y} + u_j
- \lambda_{0,j} f_0 \1_{y=0}
- \lambda_{x,j} f_x \1_{y=x}
- \tfrac12 (\1_{y=0} q_{0,j} + \1_{y=x} q_{x,j}) \sigma_0 \sigma_x.
\end{equation}


%%%%%%%%%%%%%%%%%%%%%%%%%%%%%%%%%%%%%%%%%%%%%%%%%%%%%%%%%%%%%%%%%%%%%%%%%%%%%%%
%%%%%%%%%%%%%%%%%%%%%%%%%%%%%%%%%%%%%%%%%%%%%%%%%%%%%%%%%%%%%%%%%%%%%%%%%%%%%%%

\section{Susceptibility}
\label{sec:suscept}

% By Theorem~\ref{thm:rhatflow}, for any $N$ there exist
% $(V_N, K_N) \in \domRG_N$ such that \eqref{e:IcircKnew} becomes
The proof of Theorem~\ref{thm:mr}(ii) involves some small changes to the proof
of the $\gamma_0 = 0$ case in \cite{BBS-saw4-log}. Rather than specifying the
individual changes that need to be made, here we sketch the complete argument.

Since the only scale-$N$ blocks are the empty set and $\Lambda$, at scale $j = N$
the representation \eqref{e:IcircKnew} becomes
\begin{equation}
\label{e:ZNINKN}
Z_N = e^{\zeta_N} (I_N(\Lambda) + K_N(\Lambda)).
\end{equation}
In particular, \eqref{e:DVdef}--\eqref{e:domRG},
\eqref{e:Vnormdef}, \eqref{e:Wbilinbd}, and \eqref{e:T0dom} imply that
\begin{align}
\label{e:unu-bd}
u_N |\Lambda_N| = O(1),
	&\qquad
\nu_N = O(L^{-2 N} g_N),
	\\
\label{e:WKbd}
\|W_N\|_{T_0(\ell_N)} \le O(\chicCov_N g_N^2),
	&\qquad
\|K_N\|_{T_0(\ell_N)} \le O(\chicCov_N g_N^3).
\end{align}
Now by \eqref{e:ZNINKN} and \eqref{e:chibarm} together with the definitions of
$I_N$ and $V_N$,
\begin{equation}
\label{e:chiNhat-approx}
\hat\chi_N(m^2, \gcc_0, \gamma_0, \nu_0^c, z_0^c)
	=
\frac{1}{m^2}
	+
\frac{1}{m^4 |\Lambda|}
\frac
{-\nu_N |\Lambda| + D^2 W_N(0; \1, \1) + D^2 K_N(0; \1, \1)}
{e^{u_N |\Lambda|} (1 + W_N(0) + K_N(0))}.
\end{equation}
\commentbw{(3.1.4):  dropped label $\bulk$ for bulk on RHS.  Also we now know
that $D^2 W = 0$.  There should be no $e^{u_N |\Lambda|}$ in denominator.}
Using Lemma~\ref{lem:deriv-norm-bds} with $s = 0$ (recall \eqref{e:elldef-zz}),
we see that the last term vanishes as $N\to\infty$ leaving
% By \eqref{e:Wbilinbd}, \eqref{e:VKbds}, and \eqref{e:VjKjDj-hat} with $s = 0$
% in the choice of $\ell_j$, the second term vanishes as $N\to\infty$ leaving
\begin{equation}
\label{e:chi-m-hat}
\hat\chi(m^2, \gcc_0, \gamma_0, \nu_0, z_0)
	=
\lim_{N\to\infty} \hat\chi_N(m^2, \gcc_0, \gamma_0, \nu_0, z_0)
	=
\frac{1}{m^2}.
\end{equation}
In order to identify the asymptotics of $m^2$ as $\nu$ approaches the
critical point, we will need information about the derivative of $\hat\chi$ with
respect to $\nu_0$. Let us denote by $F'$ the derivative of a function $F$ with
respect to $\nu_0$. By \eqref{e:chiNhat-approx}, the derivative $\hat\chi_N'$
will contain a term $-\nu_N'/m^4$. An argument using Lemma~\ref{lem:deriv-norm-bds}
shows that the remaining terms are of strictly higher order. Together with a careful
analysis of the derivatives of the renormalisation group flow with respect to the initial
condition $\nu_0$ (as in \cite[Section~\ref{log-sec:pfmr}]{BBS-saw4-log} for $\gamma = 0$),
we get
\begin{equation}
\label{e:chiprime-m-hat}
\hat\chi'(m^2, g_0, \gamma_0, \nu_0^c, z_0^c)
	\sim
-\frac{1}{m^4} \frac{c(\hat\gcc_0, \gamma_0)}{(\hat\gcc_0\bubble_{m^2})^{(n+2)/(n+8)}}
	\quad
\text{as $(m^2,\gcc_0,\gamma_0) \to (0,\hat\gcc_0,\hat\gamma_0)$},
\end{equation}
where $c$ is a continuous function. The bubble diagram $\bubble_{m^2}$ was defined
in \eqref{e:bubble} and its logarithmic divergence as $m^2\downarrow0$ is ultimately
the source of the logarithmic corrections in Theorem~\ref{thm:mr}.
% The proof for $\gamma_0 \ne 0$ is almost entirely a direct adaptation of the proof
% found there using the critical parameters $\hat\nu_0^c, \hat z_0^c$ constructed
% in Theorem~\ref{thm:rhatflow} rather than their $\gamma_0 = 0$ analogues.

\begin{rk}
There is one aspect of the proof of \eqref{e:chiprime-m-hat} that must be modified when
$\gamma_0 = 0$: This is the verification of the third bound
in the base case ($j = 0$) of the inductive hypothesis \cite[\eqref{log-e:induct1}]{BBS-saw4-log}.
This will be done in Section~\ref{sec:Ksmooth} (see Remark~\ref{rk:DK-base-case}).
\end{rk}

%%%%%%%%%%%%%%%%%%%%%%%%%%%%%%%%%%%%%%%%%%%%%%%%%%%%%%%%%%%%%%%%%%%%%%%%%%%%%%%

\subsection{Change of parameters}
\label{sec:nuztilde}

We wish to recover the asymptotics of $\chi$ from \eqref{e:chi-m-hat} and
\eqref{e:chiprime-m-hat}. By \eqref{e:chichibar},
\begin{equation}
\label{e:chichihat}
\chi_N(\gcc, \gamma, \nu)
	=
(1 + z_0) \hat\chi_N(m^2, \gcc_0, \gamma_0, \nu_0, z_0),
\end{equation}
whenever the variables on the left- and right-hand sides satisfy
\begin{equation}
\label{e:gg0-re}
\gcc_0 = (\gcc_0 - \gamma) (1 + z_0)^2,
\quad
\nu_0 = \nu (1 + z_0) - m^2,
\quad
\gamma_0 = \frac{1}{4d} \gamma (1 + z_0)^2.
\end{equation}
On the other hand, \eqref{e:chi-m-hat} is contingent on the initialization of
the renormalisation group with the critical parameters
\begin{equation}
\label{e:crit-constraint}
\nu_0 = \hat \nu_0^c(m^2, \gcc_0, \gamma_0),
	\quad
z_0   = \hat z_0^c(m^2, \gcc_0, \gamma_0).
\end{equation}

Given $\gcc,\gamma,\nu$,
the relations \eqref{e:gg0-re} leave free two of the variables
$(m^2, \gcc_0, \gamma_0, \nu_0, z_0)$.
More generally, if any three of the variables
$(\gcc, \gamma, \nu, m^2, \gcc_0, \gamma_0, \nu_0, z_0)$
are fixed, then two of the remaining variables are free.
In the following two propositions, which together form an extension of
\cite[Proposition~\ref{log-prop:changevariables}]{BBS-saw4-log},
we fix three variables and show that the addition of the constraints
\eqref{e:crit-constraint}
% \begin{equation}
% \label{e:crit-constraint}
% \nu_0 = \hat \nu_0^c(m^2, \gcc_0, \gamma_0),
% 	\quad
% z_0   = \hat z_0^c(m^2, \gcc_0, \gamma_0)
% \end{equation}
allows us to uniquely specify the two remaining variables.
For this, we make use of the following version of the
implicit function theorem, which we prove in Appendix~\ref{sec:IFT}.

\begin{prop}
\label{prop:IFT}
Let $\delta > 0$, and let $r_1, r_2$ be continuous positive-definite functions on $[0, \delta]$.
Recalling \eqref{e:Ddef}, set
\begin{equation}
D(\delta, r_1, r_2)
	=
\{ (w, x, y, z) \in D(\delta, r_1) \times \R^n : |z| \leq r_2(x) \},
\end{equation}
and let $F$ be a continuous function on $D(\delta, r_1, r_2)$ that is $C^1$ in $(x, z)$.
Suppose that for all $(\bar w, \bar x) \in [0, \delta]^2$ there exists $\bar z$
such that both $F(\bar w, \bar x, 0, \bar z) = 0$
and $D_Y F(\bar w, \bar x, 0, \bar z)$ is invertible.
Then there is a continuous positive-definite function $r$ on $[0, \delta]$ and
a continuous map $f : D(\delta, r) \to \R^n$
that is $C^1$ in $x$
and such that $F(w, x, y, f(w, x, y)) = 0$
for all $(w, x, y) \in D(\delta, r)$.
Moreover, if $F$ is left-differentiable
(respectively, right-differentiable) in $y$ at some point $(w, x, y, z)$,
then $f$ is left-differentiable (respectively, right-differentiable) at $(w, x, y)$.
\end{prop}
Our first application of this result is Proposition~\ref{prop:changevariables1},
in which the three fixed variables are $(m^2, \gcc_0, \gamma)$.

\begin{prop}
\label{prop:changevariables1}
There exist $\delta_* > 0$,
a continuous positive-definite function $r_* : [0, \delta_*] \to [0, \infty)$,
and continuous functions $(\nu^*, \gcc_0^*, \gamma_0^*, \nu_0^*, z_0^*)$
defined for $(m^2, \gcc, \gamma) \in D(\delta_*, r_*)$, such that
\eqref{e:gg0-re} and \eqref{e:crit-constraint} hold with $\nu = \nu^*$ and
$(\gcc_0, \gamma_0, \nu_0, z_0) = (\gcc^*_0, \gamma^*_0, \nu^*_0, z^*_0)$.
Moreover,
\begin{gather}
\label{e:gznustarbd}
\gcc_0^* = \gcc_0 + O(\gcc_0^2),
\quad
\nu_0^* = O(\gcc_0),
\quad
z_0^* = O(\gcc_0).
\end{gather}
\end{prop}

\begin{proof}
Suppose we have found the desired continuous functions $(\gcc_0^*, \gamma_0^*)$
and that $\gcc_0^*$ satisfies the first bound in \eqref{e:gznustarbd}.
Then the functions defined by
\begin{equation}
\nu_0^* = \hat\nu_0^c(m^2, \gcc_0^*, \gamma_0^*),
	\quad
z_0^* = \hat z_0^c(m^2, \gcc_0^*, \gamma_0^*),
	\quad
\nu^* = \frac{\nu_0^* + m^2}{1 + z_0^*}
\end{equation}
are continuous, satisfy \eqref{e:gg0-re}, and satisfy the remaining bounds in
\eqref{e:gznustarbd} by \eqref{e:hat-est}.

In order to construct $(g_0^*, \gamma_0^*)$, we first solve the third equation
of \eqref{e:gg0-re}, and then solve the first equation of \eqref{e:gg0-re}.
To this end, we begin by defining
\begin{equation}
f_1(m^2, \gcc_0, \gamma, \gamma_0)
	=
\gamma_0 - (4d)^{-1} \gamma (1 + \hat z_0^c(m^2, \gcc_0, \gamma_0))^2
\end{equation}
for $(m^2, \gcc_0, \gamma_0) \in D(\delta, \hat r)$
and $|\gamma| \le \hat r(\gcc_0)$.
% (recall that $\hat r$ is defined in Proposition~\ref{prop:nuzhat});
Although $f_1$ is well-defined
for any $\gamma \in \R$, we restrict the domain in preparation
for our application of Proposition~\ref{prop:IFT}.
Note that $f_1$ is $C^1$ in $\gamma$ and
$f_1(\cdot, \cdot, \gamma, \cdot) \in C^{0,1,\pm}(D(\delta, \hat r))$ for any $\gamma$.
The equation $f_1(m^2, \gcc_0, \gamma, \gamma_0) = 0$
has the solution $\gamma_0 = 0$ when $\gamma = 0$
and, for any $\gamma_0 \neq 0$,
\begin{equation}
\ddp{f_1}{\gamma_0}
	=
1 - (2d)^{-1} \gamma (1 + \hat z_0^c(m^2, \gcc_0, \gamma_0)) \ddp{\hat z_0^c}{\gamma_0}.
\end{equation}
By Theorem~\ref{thm:rhatflow}, the one-sided $\gamma_0$ derivatives of $\hat z_0^c$ exist
at $\gamma_0 = 0$. Thus, the $\gamma_0$ derivative of $f_1$ is well-defined
and equal to $1$ when $\gamma = 0$ for any small $\gamma_0$ (including $\gamma_0 = 0$).
It follows by Proposition~\ref{prop:IFT}
(with $w = m^2$, $x = \gcc_0$, $y = \gamma$, $z = \gamma_0$ and $r_1 = r_2 = \hat r$)
that there exists a continuous function $\gamma^{(1)}_0(m^2, \gcc_0, \gamma)$
on $D(\delta, r^{(1)})$ (for some continuous positive-definite function $r^{(1)}$ on $[0, \delta]$)
such that $f_1(m^2, \gcc_0, \gamma, \gamma^{(1)}_0) = 0$.
Moreover, $\gamma^{(1)}_0$ is $C^1$ in $(\gcc_0, \gamma)$.

Next, we define
\begin{equation}
f_2(m^2, \gcc, \gamma, \gcc_0)
	=
\gcc_0 - (\gcc - \gamma) (1 + \hat z_0^c(m^2, \gcc_0, \gamma^{(1)}_0(m^2, \gcc, \gamma)))^2
\end{equation}
for $(m^2, \gcc_0, \gamma) \in D(\delta, r^{(1)})$ and $\gcc \in [0, \delta_*]$,
where $\delta_* > 0$ will be made sufficiently small below.
Then $f_2(m^2, \gcc, \gamma, \gcc_0) = 0$ is solved by
$(\gamma, \gcc_0) = (0, \gcc_0^*(m^2, \gcc, 0))$,
where $\gcc_0^*(m^2, \gcc, 0)$ was constructed in \cite[\eqref{log-e:ccstar2}]{BBS-saw4-log}.
By \cite[\eqref{log-e:gznustarbd}]{BBS-saw4-log}, $\gcc_0^* = \gcc + O(\gcc^2)$,
so we may restrict the domain of $f_2$ so that $|\gcc_0| \le 2 \gcc$.
Moreover,
\begin{equation}
\ddp{f_2}{\gcc_0}
	=
1 - 2 (\gcc - \gamma) (1 + \hat z_0^c(m^2, \gcc_0, \gamma^{(1)}_0))
\left( \ddp{\hat z_0^c}{\gcc_0} + \ddp{\hat z_0^c}{\gamma_0} \ddp{\gamma^{(1)}_0}{\gcc_0} \right).
\end{equation}
Differentiating both sides of
\begin{equation}
\gamma^{(1)}_0
	=
\frac{1}{4d} \gamma (1 + \hat z_0^c(m^2, \gcc_0, \gamma^{(1)}_0))^2,
\end{equation}
and solving for $\ddp{\gamma^{(1)}_0}{\gcc_0}$, gives
\begin{equation}
\ddp{\gamma^{(1)}_0}{\gcc_0}
	=
\frac{\gamma (1 + \hat z_0^c)
	\ddp{\hat z_0^c}{\gcc_0}}{2 d - \gamma (1 + \hat z_0^c) \ddp{\hat z_0^c}{\gamma_0}},
\end{equation}
where $\hat z_0^c$ and its derivatives are evaluated at $(m^2, \gcc_0, \gamma^{(1)}_0)$.
Thus, $\ddp{\gamma^{(1)}_0}{\gcc_0} = 0$ when $\gamma = 0$.
It follows that $\partial f_2/\partial \gcc_0$
is well-defined when $(\gamma, \gcc_0) = (0, \gcc_0^*(m^2, \gcc, 0))$ and equals
\begin{equation}
1 - 2 \gcc (1 + \hat z_0^c(m^2, \gcc_0^*, 0)) \ddp{\hat z_0^c}{\gcc_0}(m^2, \gcc_0^*, 0),
\end{equation}
% THIS WAS A TYPO IN SAW-SA
which is positive when $\delta_*$ is small, by \eqref{e:hat-est}.
Thus, by Proposition~\ref{prop:IFT}
(with $w = m^2$, $x = \gcc$, $y = \gamma$, $z = \gcc_0$ and $r_1 = r^{(1)}$, $r_2(\gcc) = 2\gcc$),
there exists a function $\gcc_0^*(m^2, \gcc, \gamma) \in C^{0,1,\pm}(D(\delta_*, r^{(2)}))$
(for some continuous positive-definite function $r^{(2)}$ on $[0, \delta_*]$)
such that $f_2(m^2, \gcc, \gamma, \gcc_0^*) = 0$.

By the fact that $\gcc_0^*$ solves $f_2 = 0$,
\begin{equation}
\gcc_0^* = (\gcc - \gamma) + O((\gcc - \gamma)^2).
\end{equation}
Since $|\gamma| \le r^{(2)}(\gcc_0)$ and $r^{(2)}(\gcc_0)$ can be taken
as small as desired, this implies the first estimate in \eqref{e:gznustarbd}.
Thus, by taking $r_*$ sufficiently small, if $|\gamma| \le r_*(\gcc_0)$, then
$|\gamma| \le r^{(2)}(\gcc_0^*(m^2, \gcc, \gamma))$.
Thus, for $\gcc < \delta_*$ and $|\gamma| \leq r_*(\gcc)$,
we can define
\begin{equation}
\gamma_0^*(m^2, \gcc, \gamma)
	=
\gamma^{(1)}_0(m^2, \gcc_0^*(m^2, \gcc, \gamma), \gamma),
\end{equation}
which completes the proof.
\end{proof}

Using Proposition~\ref{prop:changevariables1}, it is possible to
identify the critical point $\nu_c$, as follows.
By \eqref{e:chi-m-hat}, \eqref{e:chichihat}, Proposition~\ref{prop:finvol}, and Proposition~\ref{prop:changevariables1},
\begin{equation}
\label{e:chistar}
\chi(\gcc, \gamma, \nu^*) = \frac{1 + z_0^*}{m^2} = \frac{1 + O(\gcc)}{m^2}.
\end{equation}
Thus, with $\nu = \nu^*$, we see that $\chi < \infty$ when $m^2 > 0$, and
$\chi = \infty$ when $m^2 = 0$.
By \eqref{e:nuc-def}, this implies that
\begin{equation}
\label{e:nustarbd}
\nu_c(\gcc, \gamma) = \nu^*(0, \gcc, \gamma) = O(\gcc),
	\quad
\nu_c(\gcc, \gamma) < \nu^*(m^2, \gcc, \gamma)
	\quad
(m^2 > 0).
\end{equation}
It follows that
\begin{equation}
\chi(\gcc, \gamma, \nu_c) = \infty,
\end{equation}
which is a fact that cannot be concluded immediately from the definition \refeq{nuc-def}.

In \eqref{e:chistar}, $\chi$ is evaluated at $\nu^* = \nu^*(m^2, \gcc, \gamma)$.
However, in the setting of Theorem~\ref{thm:mr},
we need to evaluate $\chi$ at a \emph{given} value of $\nu$
and then take $\nu \downarrow \nu_c$.
To do so, we must determine a choice of $m^2$ in terms of $\nu$
such that \eqref{e:gg0-re} is satisfied and this choice
must approach $0$ (as it should by \eqref{e:nustarbd})
right-continuously as $\nu\downarrow\nu_c$.
The following proposition carries out this construction.
In the following, the functions $\tilde m^2, \tilde \gcc_0$ should not be
confused with the parameter $\mgen^2, \ggen_0$ that appeared previously
in the $\Wcal_j$ norms (these norms are not used in this chapter).

\begin{prop}
\label{prop:changevariables2}
Write $\nu = \nu_c + \varepsilon$.
There exist functions $\tilde m^2, \tilde \gcc_0, \tilde\gamma_0, \tilde\nu_0, \tilde z_0$
of $(\varepsilon, \gcc, \gamma) \in D(\delta_*, r_*)$
(all right-continuous as $\varepsilon\downarrow 0$)
such that \eqref{e:gg0-re} and \eqref{e:crit-constraint} hold with
\begin{equation}
(m^2, \gcc_0, \gamma_0, \nu_0, z_0) = (\tilde m^2, \tilde \gcc_0, \tilde\gamma_0, \tilde\nu_0, \tilde z_0).
\end{equation}
Moreover,
\begin{gather}
\label{e:mtildebd}
\tilde m^2(0, \gcc, \gamma) = 0,
		\qquad
\tilde m^2(\varepsilon, \gcc, \gamma) > 0
		\quad
(\varepsilon > 0) \\
\label{e:gznutildebd}
\tilde \gcc_0 = \gcc + O(\gcc^2),
		\quad
\tilde \nu_0 = O(\gcc),
		\quad
\tilde z_0 = O(\gcc).
\end{gather}
\end{prop}

\begin{proof}
The proof is a minor modification of the proof in \cite{BBS-saw4-log},
using Proposition~\ref{prop:changevariables1}.
Define
\begin{equation}
\label{e:mtildef}
\tilde m^2
		=
\tilde m^2 (\varepsilon,\gcc,\gamma)
		=
\inf \{m^2 > 0 : \nu^*(m^2, \gcc, \gamma) = \nu_c(\gcc, \gamma) + \varepsilon \},
\end{equation}
on $D(\delta_*, r_*)$. By continuity of $\nu^*$, the infimum is attained and
\begin{equation}
\nu_c(\gcc, \gamma) + \varepsilon
	=
\nu^*(\tilde m^2(\varepsilon, \gcc, \gamma), \gcc, \gamma).
\end{equation}
From the above expression, continuity of $\nu^*$, and \eqref{e:nustarbd},
it follows that $\tilde m^2$ is right-continuous as $\varepsilon\downarrow 0$.
It is immediate that \eqref{e:mtildebd} holds.
Also, the functions of $(\varepsilon,\gcc,\gamma)$ defined by
\begin{align}
&\tilde\nu_0 = \nu_0^*(\tilde m^2, \gcc, \gamma),
	\quad
\tilde z_0 = z_0^*(\tilde m^2, \gcc, \gamma),
	\\
&\tilde \gcc_0 = (\gcc - \gamma) (1 + \tilde z_0)^2,
	\quad
\tilde\gamma_0 = \frac{1}{4d} \gamma (1 + \tilde z_0)^2
\end{align}
are right-continuous as $\varepsilon \downarrow 0$ and satisfy \eqref{e:gg0-re}.
The bounds \eqref{e:gznutildebd} follow from the definitions
and \eqref{e:gznustarbd}, and the proof is complete.
\end{proof}

%%%%%%%%%%%%%%%%%%%%%%%%%%%%%%%%%%%%%%%%%%%%%%%%%%%%%%%%%%%%%%%%%%%%%%%%%%%%%%%

\subsection{Conclusion of the argument}
\label{sec:suscept-conc}

We sketch the remainder of the argument, which follows as in
\cite[Section~\ref{log-sec:chvar}]{BBS-saw4-log}.
By Proposition~\ref{prop:finvol}, \eqref{e:chichihat}, \eqref{e:chi-m-hat},
and Proposition~\ref{prop:changevariables2},
\begin{equation}
\label{e:chi-renorm}
\chi(\gcc, \gamma, \nu)
	=
\frac{1 + \tilde z_0}{\tilde m^2}.
\end{equation}
Similarly, from \eqref{e:chiprime-m-hat} (using \eqref{e:chi-renorm}), we get
\begin{equation}
\label{e:diff-reln}
\new{\chi'(g, \gamma, \nu)}
	\sim
-\chi^2(g, \gamma, \nu)
\frac{c_0(g, \gamma)}{(\tilde g_0 {\sf B}_{\tilde m^2})^{\frac{n+2}{n+8}}}
\end{equation}
% \commentbw{(3.1.33): LHS should be derivative}
with $c_0(g, \gamma) = \lim_{\varepsilon\downarrow0} c(\tilde g_0, \tilde\gamma_0)$.
% By a careful analysis of the derivatives of the renormalisation group flow with
% respect to the initial condition $\nu_0$, it can be shown that
% \begin{equation}
% \label{e:chiprime-m-hat}
% \ddp{\hat\chi}{\nu_0} \left(m^2,\gcc_0, \gamma_0,\hat\nu_0^c, \hat z_0^c \right)
% 	\sim
% -\frac{1}{m^4} \frac{c(\gcc_0^*, \gamma_0)}{(\gcc_0^*\bubble_{m^2})^{1/4}}
% 	\quad
% \text{as $(m^2,\gcc_0,\gamma_0) \to (0,\gcc_0^*,\gamma_0^*)$},
% \end{equation}
% where $c$ is a continuous function and the \emph{bubble diagram} $\bubble_{m^2}$ is
% is asymptotic to $(2\pi^2)^{-1} \log m^{-2}$, as $m^2 \downarrow 0$, when $d = 4$.
% This was shown for $\gamma_0 = 0$ in \cite[Section~\ref{log-sec:pfmr}]{BBS-saw4-log}.
% The proof for $\gamma_0 \ne 0$ is almost entirely a direct adaptation of the proof
% found there using the critical parameters $\hat\nu_0^c, \hat z_0^c$ constructed
% in Theorem~\ref{thm:rhatflow} rather than their $\gamma_0 = 0$ analogues.
By exactly the same argument as in \cite[Section~\ref{log-sec:pfsuscept}]{BBS-saw4-log},
the differential relation \eqref{e:diff-reln}
% \eqref{e:chi-renorm} and \eqref{e:chiprime-m-hat} allow us to obtain
% a differential relation between $\ddp{\chi}{\nu}$ and $\chi$,
can be solved, which gives the result of Theorem~\ref{thm:mr}(ii).

\begin{rk}
It is a consequence of \eqref{e:chieps-asympt} and \eqref{e:chi-renorm} that
\begin{equation}
\label{e:mass-epsilon-asympt}
\tilde m^2
	\sim
\tilde A_{g,n}^{-1} \varepsilon (\log \varepsilon^{-1})^{-\frac{n + 2}{n + 8}}
	\quad
\text{as $\varepsilon \downarrow 0$}.
\end{equation}
\end{rk}

%%%%%%%%%%%%%%%%%%%%%%%%%%%%%%%%%%%%%%%%%%%%%%%%%%%%%%%%%%%%%%%%%%%%%%%%%%%%%%%
%%%%%%%%%%%%%%%%%%%%%%%%%%%%%%%%%%%%%%%%%%%%%%%%%%%%%%%%%%%%%%%%%%%%%%%%%%%%%%%

\section{Two-point function}

Our analysis of the two-point function and finite-order correlation length is
based on the following proposition.

\begin{prop}
\label{prop:R}
Let $d=4$, $n \ge 0$, $\varepsilon \in (0,\delta)$ with $\delta$ sufficiently small,
and $\nu = \nu_c + \varepsilon$.
Let $x \in \Z^4$ with $x \neq 0$.
Fix $s = 0$ or $s > 1$.
For $L$ sufficiently large and for $g > 0$ sufficiently small (depending on $s$),
\begin{equation}
\label{e:Gab-to-sum-Rqj}
\frac{1}{1+\tilde z_0} G_x(g, \gamma, \nu)
	=
(1 + O(\gbar_{j_x})) G_x(0, 0, \tilde m^2) + R_x(\tilde m^2)
\end{equation}
and the remainder $R_x$ satisfies the bound
\begin{equation}
\label{e:Rab-bound}
|R_{x}(m^2)|
	\le
\frac{O(\gbar_{j_{x}}) }{|x|^2}
	\times
\begin{cases}
1,				& (m|x|\le 1) \\
(m|x|)^{-2s},	& (m|x|\ge 1)
\end{cases}
\end{equation}
with the  constant depending on $L$ and $s$.
\end{prop}

\begin{proof}
% [Proof of Proposition~\ref{prop:R} (assuming Theorem~\ref{thm:step-mr-fv})]
Let $D_{\sigma_0}$ and $D_{\sigma_x}$ denote differentiation with respect to
$\sigma_0$ and $\sigma_x$, respectively, evaluated with fields set to $0$.
By \eqref{e:generating-fn}, \eqref{e:ZNINKN}, and \eqref{e:IcircKnew},
\begin{equation}
\lbeq{GK}
\frac{1}{1+\tilde z_0} G_{x,N}(g,\gamma,\nu)
	=
\frac{1}{2} (q_{0,N} + q_{x,N})
	+
\frac{D^2_{\sigma_0\sigma_x}K_{N}}{1 + K_{N}}
	-
\frac{\left(D_{\sigma_0}K_{N}\right) \left(D_{\sigma_x}K_{N}\right)}{(1 + K_{N})^2},
\end{equation}
\commentbw{(3.2.3): missing explanation of why no $W$ on RHS}
where the quantities on the right-hand side are evaluated at
$(\tilde m^2, \tilde g_0, \tilde\gamma_0, \tilde\nu_0, \tilde z_0)$.
By \eqref{e:WKbd} and Lemma~\ref{lem:deriv-norm-bds}, the last two terms
vanish as $N \to \infty$ leaving
\begin{equation}
\frac{1}{1+\tilde z_0} G_x(g, \gamma, \nu) = \frac{1}{2} (q_{0,\infty} + q_{x,\infty}).
\end{equation}

Now it is a straightforward computation using \eqref{e:lampt}--\eqref{e:deltanuw1}
and \eqref{e:Rplusdef} to show that
\begin{equation}
\label{e:q}
q_{u,\infty}
	=
\lambda_{\pp, j_\qq} \lambda_{\qq, j_\qq}  G_x(0, 0, \tilde m^2)
	+
\sum_{i = j_\qq}^\infty R^{q_u}_i,
\quad u = 0, x
\end{equation}
where $R^{q_u}_i$ is the coefficient of $\1_{y=u}\sigma_0\sigma_x$
(recall \refeq{Vy}) in $R_{+,i}$.
Moreover, as in \cite[\eqref{phi4-e:lam-star}]{ST-phi4} and \cite[Corollary~\ref{phi4-cor:vx}]{ST-phi4},
\begin{equation}
\lambda_{u,j_\qq} = 1 + O(\chicCov_{j_\qq} \gbar_{j_\qq}).
\end{equation}
It follows that
\begin{equation}
\frac{1}{1+\tilde z_0} G_x(g, \gamma, \nu)
	=
(1 + O(\gbar_{j_x})) G_x(0, \tilde m^2) + R_x
\end{equation}
with
\begin{equation}
\label{e:Rabdef}
R_x = \frac{1}{2} \sum_{i=j_\qq}^\infty (R^{q_0}_i + R^{q_x}_i).
\end{equation}

By the first bound of \eqref{e:RKplus} and the definition \eqref{e:Vnormdef}
of the $\Vcal$ norm,
\begin{align}
\label{e:vq-new}
    |R^{q_u}_{+,i}|
&
\le O(\ell_{\sigma,i}^{-2}\chicCov_i\gbar_i^{3}).
\end{align}
We insert the definition of $\ell_{\sigma,j}$ from \refeq{elldef-zz} into \refeq{vq-new}.
We also use $\ggen_j^{-2} = O(\gbar_j^{-2})$, $\chicCov_i \le 1$, $\ell_0^2 \le O(1)$,
as well as $\gbar_{j} \leq O(\gbar_{j_{x}})$ for $j \geq j_x$.
The definitions of
the coalescence scale $j_x$ and the mass scale $j_m$ imply that  $L^{-2j_x} \le O(|x|^{-2})$
and $L^{ -  (j_{x} - j_m)_+} \le O((m|x|)^{-1})$.
All this leads to
\begin{align}
\sum_{j = j_{x}}^\infty |R^{q_u}_j|
	&\leq
L^{-2j_{x} - 2s (j_{x} - j_m)_+}
\sum_{j = j_{x}}^\infty O(\gbar_{j}) 4^{-(j - j_{x})}
%\nnb
%&\leq
%L^{-2j_{x} - 2s (j_{x} - j_m)_+} O(\gbar_{j_{x}})
	\nnb
	&\leq
|x|^{-2} (m|x|)^{-2s} O(\gbar_{j_{x}})
.
\label{e:vq-sum}
\end{align}
This gives the desired estimate \eqref{e:Rab-bound}.
\end{proof}

A version of this result with $s = 0$ and $\gamma = 0$ was obtained in \cite{BBS-saw4,ST-phi4}.
This version is sufficient for studying the \emph{critical} two-point function with $\gamma = 0$.
With the extension to $\gamma \ne 0$, we can complete the proof of the first part of
Theorem~\ref{thm:mr}.

\begin{proof}[Proof of Theorem~\ref{thm:mr}(i)]
% With $\nu_c = 0$ (hence $m^2 = 0$),  			% ???
% With $s = 0$, we have $R_x = O(\gbar_{j_x}) G_x(0, 0, 0)$, so
We apply Proposition~\ref{prop:R} with $s = 0$ to get
\begin{equation}
\frac{1}{1 + \tilde z_0} G_x(g, \gamma, \nu)
	=
(1 + O(\gbar_{j_x})) G_x(0, 0, \tilde m^2) + R_x(\tilde m^2).
\end{equation}
By Proposition~\ref{prop:changevariables2}, $\tilde m^2 = 0$
when $\nu = \nu_c$. Since $R_x(0) = O(\gbar_{j_x}) G_x(0, 0, 0)$,
\begin{equation}
\frac{1}{1 + \tilde z_0} G_x(g, \gamma, \nu_c)
	=
(1 + O(\gbar_{j_x})) G_x(0, 0, 0)
\end{equation}
and the result follows from \eqref{e:gjxgjmbd}.
\end{proof}

%%%%%%%%%%%%%%%%%%%%%%%%%%%%%%%%%%%%%%%%%%%%%%%%%%%%%%%%%%%%%%%%%%%%%%%%%%%%%%%
%%%%%%%%%%%%%%%%%%%%%%%%%%%%%%%%%%%%%%%%%%%%%%%%%%%%%%%%%%%%%%%%%%%%%%%%%%%%%%%

\section{Finite-order correlation length}

An elementary ingredient in the proof of Theorem~\ref{thm:mr}(iii) is the following result
for the $g=0$ case, which is independent of $n\ge 0$.
For simplicity, we restrict attention to dimensions $d>2$, as only $d=4$ is used here.
A proof is provided in Appendix~\ref{app:free-moments}.

\begin{prop}\label{prop:Gab-free-moment-estimate}
Let ${\sf c}_p$ be the constant defined by \eqref{e:cpdef}.
For all dimensions $d>2$ and all $p>0$,
as $m^2 \downarrow 0$,
\begin{equation}
\label{e:Gab-free-moment-estimate}
\sum_{x\in\Zd} |x|^p G_{x}(0, 0, m^2)
=
{\sf c}_p^p m^{-(p + 2)} (1 + O(m)).
\end{equation}
In particular, $\xi_p(0,0,\varepsilon) = {\sf c}_p \varepsilon^{-1/2}
(1+O(\varepsilon^{1/2}))$ as $\varepsilon \downarrow 0$.
\end{prop}



\begin{proof}[Proof of Theorem~\ref{thm:mr}]
We multiply \eqref{e:Gab-to-sum-Rqj} by $|x|^p$, sum over $x \in \Z^4$,
and use \eqref{e:chistar}
% \eqref{e:susceptibility-mass-identity},
to obtain
\begin{equation}
\xi_p^p(g,\gamma,\nu)
	=
\sum_{x \in \Z^4} |x|^p \frac{G_{x}(g, \gamma, \nu)}{\chi(g, \gamma, \nu)}
	=
\tilde m^2 \sum_{x \in \Z^4} |x|^p \Big(G_{x}(0, 0, m^2) + r_{x}(\tilde m^2) \Big),
\end{equation}
with
\begin{equation}
\lbeq{rx}
    r_x(m^2) = O(\gbar_{j_x})  G_x(0, 0, m^2) + R_{x}(m^2).
\end{equation}
By
Proposition~\ref{prop:Gab-free-moment-estimate},
this gives (as $\tilde m^2 \downarrow 0$)
\begin{equation}
\lbeq{ximasy}
\xi_p^p(g,\gamma,\nu)
	\sim
{\sf c}_p^p \tilde m^{-p} +
\tilde m^2 \sum_{x \in \Z^4} |x|^p r_{x}(\tilde m^2).
\end{equation}
By \eqref{e:mass-epsilon-asympt}, it suffices to prove that
the first term on the right-hand side of \refeq{ximasy} is dominant.

For the term $O(\gbar_{j_x}) G_x(0, 0, m^2)$ in \refeq{rx},
we apply \eqref{e:gjxgjmbd} to obtain
\begin{align}
\lbeq{easyerror}
&\sum_{x \in \Z^4} \gbar_{j_x} |x|^p G_x(0, 0, \tilde m^2)
	\nnb & \quad \le
\sum_{x : 0 < j_x \le j_{\tilde m}} \frac{c |x|^p}{\log |x|} G_x(0, 0, \tilde m^2)
	+
\frac{c}{\log \tilde m^{-1}} \sum_{x : j_x >j_{\tilde m}}  |x|^p G_x(0, 0, \tilde m^2).
\end{align}
In the first term,
we use $G_x(0, 0, m^2) \le G_x(0, 0, 0) \le O(|x|^{-2})$.
The restriction $j_x \le j_{\tilde m}$ ensures that $|x| \le O(\tilde m^{-1})$.
Therefore the first term is bounded above by a multiple of
$(\tilde m^{-1})^{d+p-2}(\log \tilde m^{-1})^{-1}$, which suffices.
For the term with $j_x > j_{\tilde m}$, we extend the sum to $x \in \Z^4$
and apply Proposition~\ref{prop:Gab-free-moment-estimate}
to obtain a bound of the same form as for the first term.

For the term $R_x$ of \eqref{e:rx}, we use Proposition~\ref{prop:R}
to see that
\begin{equation}\label{e:rab-bound-scales-bis}
|R_x(\tilde m^2)|
	=
O(\gbar_{j_x})
L^{-2j_x - 2s (j_x - j_{\tilde m})_+}.
\end{equation}
We divide $\Z^4$ into shells $S_1 = \{x : |x| < \frac 12 L\}$ and, for $j \ge 2$,
$S_j = \{x : \frac 12 L^{j-1} \le |x| < \frac 12 L^{j}\}$.
The number of points in $S_j$ is bounded by $O(L^{4j})$.
Note that $j_x$ is the unique scale so that
\begin{equation}
   \label{e:Phi-def-jc}
    x \in S_{j_x +1}
   .
\end{equation}
By \eqref{e:rab-bound-scales-bis} with $s>\frac 12 (p+2)$ and \refeq{Phi-def-jc},
\begin{equation}
\label{e:xpr-shells}
\sum_{x\in\Z^4} |x|^p |R_{x}(\tilde m^2)|
	=
\sum_{j = 1}^\infty\sum_{x\in S_j}   |x|^p |R_{x}(\tilde m^2)| \\
    =
\sum_{j = 1}^\infty L^{4j + pj - 2j - 2s (j - j_{\tilde m})_+} O(\gbar_{j}),
\end{equation}
with an $L$-dependent constant.
By Lemma~\ref{lem:mass-scale-sum} below (with $a=p+2$ and $b=1$),
we obtain
\begin{equation}\label{e:xpr-mbound}
\tilde m^2 \sum_{x\in\Z^4} |x|^p |R_{x}(\tilde m^2)|
	=
O\big(\tilde m^{-p}(\log \tilde m^{-1})^{-1}\big).
\end{equation}
The first term on the right-hand side of \refeq{ximasy} therefore dominates,
and the proof is complete.
\end{proof}

The estimate used to obtain \eqref{e:xpr-mbound} is given by the following lemma,
which is stated more generally for use in the proof of Proposition~\ref{prop:Gab-free-moment-estimate}.

\begin{lemma} \label{lem:mass-scale-sum}
Let $L>1$, $2s> a > 0$, $b \geq 0$, and let $\gbar_0>0$ be sufficiently small.
Then
\begin{equation} \label{e:gmsumbd}
\sum_{j=1}^\infty L^{aj - 2s (j - j_m)_+}
\gbar_j^b = O(m^{-a} \gbar_{j_m}^b) = O(m^{-a} (\log m^{-1})^{-b}).
\end{equation}
\end{lemma}

\begin{proof}
We divide the sum at the mass scale as
\begin{equation} \label{e:xpr-2sums}
\sum_{j=1}^\infty L^{aj - 2s (j - j_m)_+} \gbar_j^{b}
= \sum_{j=1}^{j_m} L^{aj} \gbar_j^{b} +  \sum_{j=j_m+1}^\infty L^{aj - 2s (j - j_m)} \gbar_{j}^{b}.
\end{equation}
For the second sum on the right-hand side, we use $\gbar_j = O(\gbar_{j_m})$ for $j > j_m$
by \eqref{e:gjxgjmbd},
and obtain
a bound consistent with the first equality of \refeq{gmsumbd}.
For the first term, we use the crude bound
$\gbar_i/\gbar_{i+1} = 1+O(g_0)$ (by
\cite[Lemma~\ref{flow-lem:elementary-recursion}]{BBS-rg-flow}), and find
\begin{equation}
  \sum_{j=1}^{j_m} L^{aj} \gbar_j^b
  \leq
  L^{aj_m} \gbar_{j_m}^b
  \sum_{j=1}^{j_m} ((1+O(\gbar_0))L^{-a})^{j_m-j}
  =
  O(L^{aj_m} \gbar_{j_m}^b),
\end{equation}
for sufficiently small $\gbar_0>0$.
This proves the first equality in \eqref{e:gmsumbd}.
The second equality then follows since
$\gbar_{j_m} = O(\log m^{-1})$ by \eqref{e:gjxgjmbd}.
\end{proof}				% proof of main result

%% Chapter 4 %%
\chapter{Renormalisation group step}

%%%%%%%%%%%%%%%%%%%%%%%%%%%%%%%%%%%%%%%%%%%%%%%%%%%%%%%%%%%%%%%%%%%%%%%%%%%%%%%
%%%%%%%%%%%%%%%%%%%%%%%%%%%%%%%%%%%%%%%%%%%%%%%%%%%%%%%%%%%%%%%%%%%%%%%%%%%%%%%

\section{The basic idea}

\todo{The main problem is to get the inductive estimates on $(V, K)$.
We cannot prove bounds that are ``too good''. Thus, the weights must
accurately reflect the size of the covariance and of large fields.
The following shows why this is the case.}

\section{Improved norm}
\label{sec:Rpf1}

The proof of Theorem~\ref{thm:step-mr-fv} is based on the observation that
it is possible to use the parameters \refeq{elldef-zz}
in the norm used in \cite{BS-rg-IE}, instead of the $s=0$ version used
previously.  In this section, we first
state improved covariance estimates, thereby indicating why it is possible
to improve the norm.
This leads to a discussion of simplified norm pairs beyond the mass
scale.  A lemma concerning the fluctuation-field regulator indicates why the
simplification is possible.
In the following, we use the notation appropriate for the spin field
$\varphi \in (\R^n)^\Lambda$ for $n \ge 1$; only notational modifications are needed for
$n=0$.

%%%%%%%%%%%%%%%%%%%%%%%%%%%%%%%%%%%%%%%%%%%%%%%%%%%%%%%%%%%%%%%%%%%%%%%%%%%%%%%

\subsection{Covariance bounds}
\label{sec:Cbds}

The estimate in \cite{ST-phi4} which yields the $s = 0$ case of \refeq{Rab-bound}
uses the norms defined in \cite{BS-rg-IE}.
One of these norms is the $\Phi_j(\ell_j)$ norm defined by
\begin{equation}
\lbeq{phinorm}
\|\varphi\|_{\Phi_j(\ell_j)}
=
\ell_j^{-1}
\sup_{x\in \Lambda}
\sup_{|\alpha|_1  \le p_\Phi}
L^{j|\alpha|_1}
|\nabla^{\alpha} \varphi_x|,
\end{equation}
which depends on the parameter $\ell_j$,
and on the maximal number of discrete derivatives $p_\Phi$
(fixed to be at least $4$ in \cite{BS-rg-IE}).
As in \refeq{elldef-zz}, we now define
\begin{align}
\label{e:elldef}
\ell_j &= \ell_0 L^{-j - s (j - j_m)_+}, \quad
\ell_{\sigma,j}
=
\ell_{j \wedge j_{x}}^{-1} 2^{(j - j_{x})_+} \ggen_j.
\end{align}
The analysis of \cite{BS-rg-IE,BS-rg-step} uses the norm parameters $\ell_j$ and $\ell_{\sigma,j}$ with $s = 0$.
To distinguish these from our
new choice \refeq{elldef} of $\ell_j$ and $\ell_{\sigma,j}$, we write
\begin{equation}
\label{e:ell-old}
    \ell_j^\oldrm = \ell_0 L^{-j},
    \quad
    \ell_{\sigma,j}^\oldrm  =
    (\ell_{j \wedge j_{x}}^{\rm old})^{-1}2^{(j - j_{x})_+}\ggen_j.
\end{equation}

In the more general terminology and notation of \cite{BS-rg-norm,BS-rg-IE},
we may regard a covariance $C_j$
in the decomposition \eqref{e:NCj}
as a test function depending on
two arguments $x,y$, and with this identification its $\Phi_j(\ell_j)$
norm is
\begin{equation}
    \label{e:Phinorm}
    \|C_j\|_{\Phi_{j}(\ell_j)}  =
    \ell_j^{-2}
    \sup_{x,y\in \Lambda}
    \;
    \sup_{|\alpha|_1 + |\beta|_1 \le p_\Phi}
    L^{(|\alpha|_1+  |\beta|_1)j}
    |\nabla_x^{\alpha} \nabla_y^{\beta} C_{j;x,y}|.
\end{equation}
The purpose of the $\Phi_j(\ell_j)$ norm is to measure the size of typical
fluctuation fields $\varphi$ with covariance $C_j$.
The parameter $\ell_j$ is chosen so that the norm of a typical field should
be $O(1)$, independent of $j$.

The following lemma justifies our choice of $\ell_j$
in \refeq{elldef}, by showing that the
bound \cite[\eqref{IE-e:CLbd}]{BS-rg-IE}, proved there only for the $s=0$ version
$\ell_j^\oldrm$ of \refeq{ell-old},
remains true with the stronger
choice of norm parameter $\ell_j$ that permits arbitrary $s \ge 0$.
%The sequence $\chicCov_j$ in the lemma is called $\chi_j$ in \cite{BS-rg-IE}, but here we use
%a different symbol to avoid confusion with the susceptibility.
In its statement, the bounded sequence $\chicCov_j$ decays exponentially after the
mass scale and may be taken to be equal to
$2^{-(j-j_m)_+}$; its details are given
in \cite[Section~\ref{IE-sec:frp}]{BS-rg-IE} (where it is called $\chi_j$ rather
than $\chicCov_j$).

\begin{lemma}
[{Extension of \cite[\eqref{IE-e:CLbd}]{BS-rg-IE}}]
\label{lem:Cbd}
Given $\ellconst \in (0, 1]$, $\ell_0$ can be chosen large (depending on $L,\ellconst,s$)
so that
\begin{equation}
\lbeq{Cbd}
\|C_j\|_{\Phi_{j}(\ell_j)} \leq \min(\ellconst, \chicCov_j).
\end{equation}
\end{lemma}



The proof of Lemma~\ref{lem:Cbd} uses an estimate from
\cite[Proposition~\ref{pt-prop:Cdecomp}]{BBS-rg-pt}, which we repeat here as
the following proposition.

\begin{prop}[{Restatement of \cite[Proposition~\ref{pt-prop:Cdecomp}(a)]{BBS-rg-pt}}]
\label{prop:Cdecomp}
  Let $d >2$, $L\geq 2$, $j \ge 1$, $\bar m^2 >0$.
  For multi-indices $\alpha,\beta$ with
  $\ell^1$ norms $|\alpha|_1,|\beta|_1$ at most
  some fixed value $p$,
  and for any $k$, and for $m^2 \in [0,\bar m^2]$,
  \begin{equation}
    \label{e:scaling-estimate}
    |\nabla_x^\alpha \nabla_y^\beta C_{j;x,y}|
    \leq c(1+m^2L^{2(j-1)})^{-k}
    L^{-(j-1)(d-2 +|\alpha|_1+|\beta|_1)},
  \end{equation}
  where $c=c(p,k,\bar m^2)$ is independent of $m^2,j,L$.
  The same bound holds for $C_{N,N}$ if
  $m^2L^{2(N-1)} \ge \varepsilon$ for some $\varepsilon >0$,
  with $c$ depending on $\varepsilon$ but independent of $N$.
\end{prop}

\begin{proof}[Proof of Lemma~\ref{lem:Cbd}]
For $d=4$, insertion of \refeq{scaling-estimate} into \refeq{Phinorm} gives
\begin{equation}
    \label{e:Phinorm2}
    \|C_j\|_{\Phi_{j}(\ell_j)}
    \le
    c
    L^{p_\Phi}
    \ell_j^{-2}(1+m^2L^{2(j-1)})^{-k}
    L^{-2(j-1)}.
\end{equation}
With $s=0$ in \eqref{e:elldef}, \refeq{Phinorm2} gives
$\|C_j\|_{\Phi_{j}(\ell_j)} \le c_L \ell_0^{-2} (1+m^2L^{2(j-1)})^{-k}$
for an $L$-dependent constant $c_L$ (whose value may now change from line to line).
The estimate \cite[\eqref{IE-e:CLbd}]{BS-rg-IE}
is wasteful in that it does not make any use of the factor
$(1+m^2L^{2(j-1)})^{-k}$ in \refeq{Phinorm2} beyond extraction of the factor $\chicCov_j$.
To improve this, we now allow arbitrary $s$, and fix the arbitrary parameter $k$ to be $k=s+1$
in \refeq{Phinorm2} so that
\begin{equation} \label{e:mass-decay}
(1 + m^2 L^{2j})^{-k} \le c_L L^{-2(s+1)(j - j_m)_+}.
\end{equation}
We insert \refeq{mass-decay} and the definition $\ell_j=\ell_0 L^{-j-s(j-j_m)_+}$ from
\refeq{elldef} into
\eqref{e:Phinorm2}, to conclude that there exists $c_0 = c_0(s, L)$ such that
\begin{equation}
    \|C_j\|_{\Phi_{j}(\ell_j)} \leq c_0 \ell_0^{-2} L^{-2(j - j_m)_+}
    .
\end{equation}
By definition of $\chicCov_j$ (see \cite[Section~\ref{IE-sec:frp}]{BS-rg-IE}),
$L^{-2(j - j_m)_+}$ is bounded by a multiple of $\chicCov_j$.
It thus suffices to choose $\ell_0$ large enough that
$\ell_0^2 \ge c_0 \ellconst^{-1}$.
\end{proof}

%%%%%%%%%%%%%%%%%%%%%%%%%%%%%%%%%%%%%%%%%%%%%%%%%%%%%%%%%%%%%%%%%%%%%%%%%%%%%%%

\subsection{New choice of norm beyond the mass scale}
\label{sec:newnorm}


As in \cite[(\ref{IE-e:PhiXdef})]{BS-rg-IE}, we use the localised version
of \eqref{e:phinorm}, defined for subsets
$X \subset \Lambda$  by
\begin{align}
\label{e:PhiXdef}
    \|\varphi\|_{\Phi_j(X)}
    &=
    \inf \{ \|\varphi -f\|_{\Phi_j} :
    \text{$f \in \C^\Lambda$ such that $f_{x} = 0$
    $\forall x\in X$}\}.
\end{align}
A \emph{small set} is defined to be a
connected polymer $X \in \Pcal_j$ consisting of at most $2^d$ blocks
(the specific number
$2^d$ plays no direct role here),
and $\Scal_j \subset \Pcal_j$ denotes the set of small sets.
The \emph{small set neighbourhood} of $X \subset \Lambda $ is
the enlargement of $X$ defined by
$
    X^{\Box}
=
    \bigcup_{Y\in \Scal_{j}:X\cap Y \not =\varnothing } Y$.

Given $X \subset \Lambda$ and $\varphi \in (\R^n)^{\Lambda}$,
we recall from \cite[\eqref{IE-e:GPhidef}]{BS-rg-IE}
that the
\emph{fluctuation-field regulator} $G_j$
is defined by
\begin{align}
\label{e:GPhidef}
    G_j(X,\varphi)
    =
    \prod_{x \in X} \exp
    \left(|B_{x}|^{-1}\|\varphi\|_{\Phi_j (B_{x}^\Box,\ell_j )}^2 \right)
    ,
\end{align}
where $B_{x}\in \Bcal_j$ is the unique block that contains $x$,
and hence $|B_x| = L^{dj}$.
The \emph{large-field regulator} is defined in \cite[\eqref{IE-e:9Gdef}]{BS-rg-IE} by
\begin{align}
\label{e:9Gdef}
    \tilde G_j  (X,\varphi)
    =
    \prod_{x \in X}
    \exp \left(
    \frac 12 |B_{x}|^{-1}\|\varphi\|_{\tilde\Phi_j (B_{x}^\Box,\ell_j)}^2
    \right)
    .
\end{align}
The $\tilde\Phi_j$ norm appearing on the right-hand side of \refeq{9Gdef} is
similar to the $\Phi_j$ norm, with the important difference that it is insensitive to
shifts by linear test functions; see \cite[\eqref{IE-e:Phitilnorm}]{BS-rg-IE} for the
precise definition.
The two regulators serve as weights in the \emph{regulator norms} of
\cite[Definition~\ref{IE-def:Gnorms}]{BS-rg-IE}.
The regulator norms are defined,  with $\gamma \in (0,1]$ and
for $F$ in the space $\Ncal(X^\Box)$ of functionals
of the field (see \cite[\eqref{norm-e:NXdef}]{BS-rg-norm}), by
\begin{align}
\label{e:Gnormdef1}
    \| F\|_{G_j(\ell_j)}
    &=
    \sup_{\varphi \in (\R^n)^\Lambda}
    \frac{\|F\|_{T_{\varphi,j}(\ell_j)}}{G_{j}(X,\varphi)}
    ,
\\
\label{e:Gnormdef2}
    \|F\|_{\tilde G_j^{\Gtilp}(h_j)}
    &=
    \sup_{\varphi \in (\R^n)^\Lambda}
    \frac{\|F \|_{T_{\varphi,j}(h_j)}}{\tilde{G}^{\Gtilp}_{j}(X,\varphi)}
    .
\end{align}
The parameter $\ell_j$ that appears in the regulators \refeq{GPhidef}--\refeq{9Gdef} and
in the numerator of \refeq{Gnormdef1} was taken to be $\ell_j^\oldrm$ in \cite{BS-rg-IE},
but now we use $\ell_j$ instead. As in \cite{BS-rg-IE},
the parameter $h_j$ and its observable counterpart $h_{\sigma,j}$ are given by
\begin{align}
\label{e:h}
    h_{j} &= k_0 \ggen_j^{-1/4}L^{-j},
    \quad
    h_{\sigma,j}  = (\ell_{j \wedge j_{x}}^{\rm old})^{-1}
    2^{(j - j_{x})_+}\ggen_j^{1/4}.
\end{align}

In \cite{BS-rg-IE}, estimates on $\|\cdot\|_{j+1}$ are given in terms of
$\|\cdot\|_j$, where the pair $(\|\cdot\|_j, \|\cdot\|_{j+1})$ refers to
either of the norm pairs
\begin{equation}
\label{e:np1}
    \|F\|_j = \|F\|_{G_j(\ell_j^\oldrm)}
    \quad \text{and} \quad
    \|F\|_{j+1} = \|F\|_{T_{0,j+1}(\ell_{j+1}^\oldrm)},
\end{equation}
or
\begin{equation}
\label{e:np2}
    \|F\|_j = \|F\|_{\tilde{G}_j(h_j)}
    \quad \text{and} \quad
    \|F\|_{j+1} = \|F\|_{\tilde{G}_{j+1}^{\Gtilp}(h_{j+1})}.
\end{equation}
We will show that, \emph{above the mass scale} $j_m$ (see \eqref{e:jmdef}), the results of \cite{BS-rg-IE} hold  with
both norm pairs in \eqref{e:np1} and \eqref{e:np2} replaced by the single new norm pair
\begin{equation}
\label{e:npmass}
    \|F\|_j = \|F\|_{G_j(\ell_j)}
    \quad \text{and} \quad
    \|F\|_{j+1} = \|F\|_{G_{j+1}(\ell_{j+1})},
\end{equation}
with the improved $\ell_j$ of \eqref{e:elldef} with $s>0$ fixed as large as desired.

The space $\Ncal$ containing the functionals $F$ appearing above requires control on
up to $p_\Ncal$ derivatives of $F$ with respect to the field $\varphi$,
where $p_\Ncal$ is a parameter of the $T_\varphi$-norm.
In the proof of Proposition~\REF % \ref{prop:cl}
below, we must choose $p_\Ncal$ to be large depending on $p$,
in order to analyse the correlation length
of order $p$.  The renormalisation group analysis is predicated on fixed (but arbitrary)
$p_\Ncal$, so it can proceed with this modification.  However,
we do not prove that constants are uniform in $p_\Ncal$,
and in particular we do not prove that the required smallness of $g$ in
Theorem~\ref{thm:mr} is uniform in the choice of $p_\Ncal$.
Thus we do not have a result for \emph{all} $p>0$ for any fixed $g$.


The use of two norm pairs adds intricacy to \cite{BS-rg-IE,BS-rg-step}.
The pair \refeq{np1} is insufficient, on its own, because the scale-$(j+1)$ norm
is the $T_0$ semi-norm which controls only small fields, and an estimate in this norm
does not imply an estimate for the $G_{j+1}$ norm.  The norm pair \refeq{np2} is
used to supplement the norm pair \refeq{np1}, and estimates in both of the scale-$(j+1)$
norms can be combined to provide an estimate for the $G_{j+1}$ norm.  This then
sets the stage for the next renormalisation group step.  Above the mass scale,
the use of \refeq{npmass} now bypasses many issues.  For example, for $j>j_m$
 the $\Wcal_j$ norm of \cite[\eqref{step-e:9Kcalnorm}]{BS-rg-step} is replaced
 simply by the $\Fcal_j(G)$ norm, and there is no need for the $\Ycal_j$ norm of
\cite[\eqref{step-e:Ycaldef}]{BS-rg-step} nor for \cite[Lemma~\ref{step-lem:KKK}]{BS-rg-step}.

The need for both norm pairs \eqref{e:np1}--\eqref{e:np2} is discussed in
\cite[Section~\ref{IE-sec:lfp}]{BS-rg-IE} and is related to the
so-called \emph{large-field problem}. Roughly speaking, the
norm pair \refeq{np2} is used to take advantage of the quartic term in the interaction to
suppress the effects of large values of the fields. This approach
relies on the fact that the interaction polynomial is dominated by the
quartic term in the $h$-norm, as expressed by
\cite[\eqref{IE-e:tau2dom}]{BS-rg-IE}, together with the lower bound
\cite[\eqref{IE-e:epVbark0}]{BS-rg-IE} on the quartic term.
However, above the mass scale, large fields are naturally suppressed
by the rapid decay of the covariance.
This idea is captured in Lemma~\ref{lem:mart} below, which replaces
\cite[Lemma~\ref{IE-lem:mart}]{BS-rg-IE} above the mass scale.
The regulators in its statement are defined by \refeq{GPhidef} with the $s$-dependent
$\ell_j$ of \refeq{elldef}.


\begin{lemma}[{Replacement for \cite[Lemma~\ref{IE-lem:mart}]{BS-rg-IE}}]
\label{lem:mart}
Let $X \subset \Lambda$ and assume that $s > 1$.
For any $q >0$, if $L$ is sufficiently large depending on $q$, then for $j_m \leq j < N$,
\begin{equation}
\label{e:mart}
    G_{j}(X, \varphi)^{q}
    \le
    G_{j+1}(X, \varphi).
\end{equation}
\end{lemma}
\begin{proof}
By \eqref{e:GPhidef}, it suffices to show that, for any scale-$j$ block $B_j$ and any scale-$(j+1)$ block $B_{j+1}$ containing $B_j$,
\begin{equation}
q \|\varphi\|^2_{\Phi_j (B_j^\Box,\ell_j )}
\leq
L^{-4} \|\varphi\|^2_{\Phi_{j+1} (B_{j+1}^\Box,\ell_{j+1})}.
\end{equation}
In fact, since $\|\varphi\|_{\Phi_j (B_j^\Box,\ell_j )}
\leq \|\varphi\|_{\Phi_j (B_{j+1}^\Box,\ell_j )}$ by definition,
it suffices to prove the above bound with $B_j$ replaced by $B_{j+1}$ on the left-hand side.
According to the definition of the norm in \eqref{e:PhiXdef},
to show this it suffices to prove that
\begin{equation}
\lbeq{martwant}
q \|\varphi\|_{\Phi_j(\ell_j)}^2 \leq L^{-4} \|\varphi\|_{\Phi_{j+1}(\ell_{j+1})}^2
\end{equation}
(then we replace $\varphi$ by $\varphi -f$ in the above and take the infimum).

By definition,
\begin{equation}
\|\varphi\|_{\Phi_j(\ell_j)}
\le
\ell_j^{-1} \ell_{j+1}
\sup_{x\in \Lambda} \sup_{|\alpha| \leq p_\Phi}
\ell_{j+1}^{-1}
L^{(j+1) |\alpha|}
|\nabla^\alpha \varphi_x|,
\end{equation}
with the inequality due to replacement of $L^{j |\alpha|}$ on the left-hand
side by $L^{(j+1) |\alpha|}$ on the right-hand side.
Since $\ell_j^{-1} \ell_{j+1} = L^{-1 - s \1_{j \geq j_m}}$,
\begin{equation}
\|\varphi\|_{\Phi_j(\ell_j)} \leq L^{-1 - s \1_{j \geq j_m}} \|\varphi\|_{\Phi_{j+1}(\ell_{j+1})}.
\end{equation}
Thus,
\begin{equation}
q \|\varphi\|_{\Phi_j(\ell_j)}^2
\leq q L^{-4} L^{2 - 2s \1_{j \geq j_m}} \|\varphi\|^2_{\Phi_{j+1}(\ell_{j+1})},
\end{equation}
and then \refeq{martwant}
follows once $L$ is large enough that $q L^{2 - 2s} \leq 1$.
\end{proof}

\begin{rk}
The elimination of the $h$-norm after the mass scale is more than a convenience.
It becomes a necessity when we improve the $\ell$-norm.
Briefly, the reason is as follows. In the proof of
\cite[Lemma~\ref{step-lem:KKK}]{BS-rg-step}, the ratio
$\ell_{\sigma}/h_{\sigma}$
must be bounded. For this, we would need
to increase $h_{\sigma}$
beyond the mass scale  (since $\ell_{\sigma}$ has been increased).
This forces a compensating decrease in $h$
beyond $j_m$, to keep the product $hh_{\sigma}$ bounded for stability
(as in Section~\REF % \ref{sec:stability1}
below). But if we do this, we lose the lower bound required on $\epsilon_{g\tau^2}$
required for stability in the $h$-norm (see \cite[\eqref{IE-e:epVbardefz-app}]{BS-rg-IE}).
\end{rk}
				% improved bounds on rg-step from clp

%% Chapter 5 %%
\chapter{Critical initial conditions}

%%%%%%%%%%%%%%%%%%%%%%%%%%%%%%%%%%%%%%%%%%%%%%%%%%%%%%%%%%%%%%%%%%%%%%%%%%%%%%%
%%%%%%%%%%%%%%%%%%%%%%%%%%%%%%%%%%%%%%%%%%%%%%%%%%%%%%%%%%%%%%%%%%%%%%%%%%%%%%%

\section{Initial coordinates for the renormalisation group}
\label{sec:K0bd}

Following the approach of \cite{BBS-saw4-log},
we represent $Z_j$
by a pair of coordinates $I_j$ and $K_j$ that capture the \emph{relevant} (expanding),
\emph{marginal},
and \emph{irrelevant} (contracting) parts of $Z_j$.
We begin in Section~\ref{sec:IK} by defining coordinates $(I_0, K_0)$ for $Z_0$.
Norms used to control the evolution of these coordinates are introduced
in Section~\ref{sec:norms}, and it is shown in
Sections~\ref{sec:K0bds}--\ref{sec:KWcal} that $K_0$ satisfies norm estimates
that permit the results of \cite{BS-rg-step,BBS-rg-flow} to be applied.
The initial coordinate $K_0$
depends on the coupling constants $(g_0, \gamma_0, \nu_0, z_0)$
of \eqref{e:gg0}, and regularity of $K_0$ as a function of these variables
is established in Section~\ref{sec:Ksmooth}.

%%%%%%%%%%%%%%%%%%%%%%%%%%%%%%%%%%%%%%%%%%%%%%%%%%%%%%%%%%%%%%%%%%%%%%%%%%%%%%%

\subsection{Norms}
\label{sec:norms}

In this section, we recall some definitions and basic facts concerning norms,
from \cite{BS-rg-norm}.
For now, we only consider the case of scale $j = 0$.

Recall the notation introduced in Section~\ref{sec:forms}.
A \emph{test function} $g$ is defined to be a function $(\vec x, \vec y) \mapsto g_{\vec x,\vec y}$,
where $\vec x$ and $\vec y$ are finite sequences of elements in $\Lambda \sqcup \bar\Lambda$.
When $\vec x$ or $\vec y$ is the empty sequence $\varnothing$,
we drop it from the notation as long as this causes no confusion;
e.g., we may write $g_{\vec x} = g_{\vec x,\varnothing}$.
The length of a sequence $\vec x$ is denoted $|\vec x|$.
Gradients of test functions are defined component-wise.
Thus, if $\vec x = (x_1, \ldots, x_m)$
and $\alpha = (\alpha_1, \ldots, \alpha_m)$
with each $\alpha_i \in \N_0^\Ucal$, and similarly for $\vec y=(y_1,\ldots,y_n)$ and
$\beta=(\beta_1,\ldots,\beta_n)$,
then
\begin{equation}
\nabla^{\alpha,\beta}_{\vec x,\vec y} g_{\vec x,\vec y}
  =
\nabla^{\alpha_1}_{x_1} \ldots \nabla^{\alpha_m}_{x_m}
\nabla^{\beta_1}_{y_1} \ldots \nabla^{\beta_n}_{y_n}  g_{x_1,\ldots,x_m,y_1,\ldots,y_n}.
\end{equation}

Let $\h_0 > 0$
be a parameter, which we set below.
We fix positive constants $p_\Phi\ge 4$ and
$p_\Ncal$
and assume that all test functions
vanish when $|\vec x|  +|\vec y| > p_\Ncal$.
For Theorem~\ref{thm:suscept}(i-ii), any choice of $p_\Ncal \ge 10$ is sufficient,
whereas for Theorem~\ref{thm:suscept}(iii) it is necessary to choose $p_\Ncal$ large
depending on $p$ \cite{BSTW-clp}.
The $\Phi = \Phi(\h_0)$ norm on such test functions is defined by
\begin{equation}
\|g\|_\Phi
  =
    \sup_{\vec x, \vec y} \h_0^{-(|\vec x| +|\vec y|)}
    \shift\shift \sup_{\alpha,\beta: |\alpha|_1+|\beta|_1 \le p_\Phi}
    |\nabla^{\alpha,\beta} g_{\vec x, \vec y}|,
\end{equation}
where $|\alpha|_1$ denotes the total order of the differential operator $\nabla^\alpha$.
Thus, for any test function $g$ and for sequences
$\vec x, \vec y$ with $|\vec x| +|\vec y| \leq p_\Ncal$ and
corresponding $\alpha, \beta$ with $|\alpha|_1 + |\beta|_1 \leq p_\Phi$,
\begin{equation}
\label{e:testfcnbd}
|\nabla^{\alpha,\beta} g_{\vec x,\vec y}| \leq \h_0^{|\vec x| + |\vec y|} \|g\|_\Phi.
\end{equation}

For any $F \in \Ncal$,
there exist \emph{unique} functions $F_{\vec y}$ of $(\phi, \bar\phi)$
that are anti-symmetric under permutations of $\vec y$, such that
\begin{equation}
F = \sum_{\vec y} \frac{1}{|\vec y|!} F_{\vec y}(\phi, \bar\phi) \psi^{\vec y}.
\end{equation}
Given a sequence $\vec{x}$ with $|\vec{x}| = m$, we define
\begin{equation}
F_{\vec x, \vec y} = \ddp{^m F_{\vec y}}{\phi_{x_1} \ldots \partial\phi_{x_m}}.
\end{equation}
We define a $\phi$-dependent pairing of elements of $\Ncal$ with test functions, by
\begin{equation}
\langle F, g \rangle_\phi
  =
\sum_{\vec x, \vec y} \frac{1}{|\vec x|! |\vec y|!} F_{\vec x,\vec y}(\phi, \bar\phi) g_{\vec x,\vec y}.
\end{equation}
Let $B(\Phi)$ denote the unit $\Phi$-ball
in the space of test functions. Then the
$T_\phi = T_\phi(\h_0)$ semi-norm on $\Ncal$
is defined by
\begin{equation}
\|F\|_{T_\phi} = \sup_{g\in B(\Phi)} |\langle F, g \rangle_\phi|.
\end{equation}

We need several properties of the $T_\phi$ semi-norm,
whose proofs can be found in \cite{BS-rg-norm}.
First, there is the important \emph{product property}
\cite[Proposition~\ref{norm-prop:prod}]{BS-rg-norm}
\begin{equation}
\label{e:prod}
\|F G\|_{T_\phi} \leq \|F\|_{T_\phi} \|G\|_{T_\phi}.
\end{equation}
An immediate consequence is that $\|e^{-F}\|_{T_\phi} \leq e^{\|F\|_{T_\phi}}$.
This is improved in \cite[Proposition~\ref{norm-prop:eK}]{BS-rg-norm},
which states that (recall that $F_\varnothing$ denotes the $0$-degree part of $F$)
\begin{equation}
\label{e:eK}
\|e^{-F}\|_{T_\phi} \leq e^{-2 {\rm Re} F_\varnothing(\phi) + \|F\|_{T_\phi}}.
\end{equation}

Each of the two choices $\varphi = \phi, \bar\phi$
can be viewed as a test function supported on sequences with
$|\vec x| = 1$ and $|\vec y| = 0$
and satisfying $\varphi_{\bar x} = \bar\varphi_x$.
In particular, $\|\phi\|_\Phi$ is defined as the norm of a test function.
We use \cite[Proposition~\ref{norm-prop:T0K}]{BS-rg-norm},
which states that if $F \in \Ncal$ is a polynomial in $\phi,\phib,\psi,\psib$ of
total degree $A \leq p_\Ncal$, then
\begin{equation}
\label{e:T0K}
\|F\|_{T_\phi} \leq \|F\|_{T_0} (1 + \|\phi\|_\Phi)^A.
\end{equation}

We write $x^\Box = \{y: |y-x|_\infty \le 2^d-1\}$,
where $|x|_\infty = \max\{|x_i| : 1 \le i \le d\}$
(this is the scale-0 version
of \cite[\eqref{IE-e:ssn}]{BS-rg-IE} for a single point).
The $\Phi_x \equiv \Phi(x^\square)$ norm of $\phi \in \C^\Lambda$ is defined by
\begin{equation}
\|\phi\|_{\Phi_x}
  =
\inf
\left\{
  \|\phi - f\|_\Phi : f \in \C^\Lambda \text{ such that } f_y = 0 \;\forall y \in x^\square
\right\}
.
\end{equation}
By taking the infimum in \eqref{e:T0K} over all possible
re-definitions of $\phi_y$ for $y \notin x^\square$, we get
\begin{equation}
\label{e:T0Kx}
\|F\|_{T_\phi}
  \leq
\|F\|_{T_0} (1 + \|\phi\|_{\Phi_x})^A
\end{equation}
when $F \in \Ncal(x^\square)$.

We need two choices of the parameter $\h_0$ (for both choices, $\h_0 \ge 1$):
either $\h_0 = \ell_0$, an $L$-dependent constant;
or $\h_0 = h_0 = k_0 \ggen_0^{-1/4}$, where $k_0$ is a small constant and
$\ggen_0$ is a constant which must be chosen small depending on $L$.
Some discussion of these constants occurs in the
proof of Proposition~\ref{prop:K0bd}.
In \cite{BS-rg-IE}, two \emph{regulators} are defined.
At scale $0$, these are given by
\begin{equation}
\lbeq{regdef}
G_0(x, \phi)
  = e^{\|\phi\|^2_{\Phi_x(\ell_0)}},
  \qquad
\tilde G_0(x, \phi)
  =
e^{\frac{1}{2} \|\phi\|^2_{\tilde\Phi_x(\ell_0)}}.
\end{equation}
The $\tilde \Phi_x$ norm in the definition of $\tilde G_0$,
is defined in \cite[\eqref{IE-e:Phitilnorm}]{BS-rg-IE};
it is a modification of the $\Phi_x$ norm that is invariant under shifts by
linear test functions.  Its specific properties do not play a direct role  in this paper.
Two regulator norms are defined for $F \in \Ncal(x^\square)$ by
\begin{equation}
\lbeq{reg0}
    \|F\|_{G_0} = \sup_{\phi\in\C^\Lambda} \frac{\|F\|_{T_\phi(\ell_0)}}{G_0(x,\phi)}
    , \quad
    \|F\|_{\tilde{G}^{\sf t}_0} = \sup_{\phi\in\C^\Lambda} \frac{\|F\|_{T_\phi(h_0)}}{\tilde{G}^{\sf t}_0(x,\phi)}
    ,
\end{equation}
where ${\sf t} \in (0, 1]$ is a constant power.

%%%%%%%%%%%%%%%%%%%%%%%%%%%%%%%%%%%%%%%%%%%%%%%%%%%%%%%%%%%%%%%%%%%%%%%%%%%%%%%%

\subsection{Bounds on \texorpdfstring{$K_0$}{K0}}
\label{sec:K0bds}

\todo{Do this with $n \ge 1$.}

The main estimate on $K^\pm_{0,x}$ is given by the following proposition.
Consistent with \cite[\eqref{IE-e:DV1-bis}]{BS-rg-IE}, we
fix a large constant $C_\DV$ and define
\begin{equation}
\label{e:DV0}
    \DV_0 = \DV_0(\ggen_0) = \{(g,\nu,z) \in \R^3 : C_{\DV}^{-1}\ggen_0 < g < C_{\DV}\ggen_0,
    \; |\nu|,|z| < C_{\DV}\ggen_0\}.
\end{equation}

\begin{prop}
\label{prop:K0bd}
Suppose that $V^\pm_0 \in \DV_0$, with $\ggen_0$ sufficiently small.
If $|\gamma_0| \leq  \ggen_0$, then
(with constants that may depend on $L$)
\begin{equation}
\lbeq{K0bds}
\|K^\pm_{0,x}\|_{G_0} = O(|\gamma_0|),
\quad
\|K^\pm_{0,x}\|_{\tilde G_0} = O(|\gamma_0|/g_0),
\end{equation}
where the bounds on $K^+$ and $K^-$ hold for $\gamma_0 \geq 0$
and $\gamma_0 < 0$, respectively.
\end{prop}


The form of the estimates \refeq{K0bds} can be anticipated from the definition of
$K_0^\pm$ in \refeq{Kpm}.  The upper bound arises from the small size of
$e^{-|\gamma_0|U_x^\pm}-1$.  For small fields, hence small $U_x^\pm$, this is of order $|\gamma_0|$,
as reflected by the $G_0$ norm estimate of \eqref{e:K0bds}.
For large fields, namely fields of size $|\phi| \approx \ggen_{0}^{-1/4}$, the difference
$e^{-|\gamma_0|U_x^\pm}-1$ is roughly of size $|\gamma_0|\,|\phi|^4 \approx |\gamma_0|/\ggen_0$.
This effect is measured by the $\tilde G_0$ norm.

Before proving the proposition, we
write \refeq{Kpm} for a singleton as
\begin{equation}
K^\pm_{0,x} = I^\pm_{0,x} J^\pm_x
  \label{e:KIJ},
\end{equation}
where, by the fundamental theorem of calculus,
\begin{align}
    I^\pm_{0,x} &= e^{-V^\pm_{0,x}} \\
    J^\pm_x
    &= e^{-|\gamma_0|U^\pm_x} - 1
    = - \int_0^{1} |\gamma_0| U^\pm_x e^{-t |\gamma_0| U^\pm_x} \; dt.
\label{e:J}
\end{align}
As in \eqref{e:Kpm}, the $+$ versions of \eqref{e:KIJ}--\eqref{e:J} hold
only for $\gamma_0 \geq 0$ and the $-$ versions only for $\gamma_0 < 0$.

Let $F \in \Ncal(x^\square)$ be a polynomial of degree at most $p_\Ncal$.
Then the stability estimates \cite[\eqref{IE-e:Iupper-a}--\eqref{IE-e:Iupper-b}]{BS-rg-IE}
imply that there exists $c_3 > 0$ and, for any $c_1 \geq 0$,
there exist positive constants $C, c_2$ such that
if $V_0^\pm \in \DV_0$ then
\begin{equation}
\label{e:Iupper}
\|I^\pm_{0,x} F\|_{T_\phi(\h_0)}
  \leq
C \|F\|_{T_0(\h_0)}
\begin{cases}
  e^{c_3 g_0 \left(1 + \|\phi\|^2_{\Phi_x(\ell_0)}\right)}
    & \h_0 = \ell_0 \\
  e^{-c_1 k_0^4 \|\phi\|^2_{\Phi_x(h_0)}} e^{c_2 k_0^4 \|\phi\|^2_{\tilde\Phi_x(\ell_0)}}
    & \h_0 = h_0.
\end{cases}
\end{equation}
This essentially reduces our task to estimating $J^\pm_x$.
The next lemma is an ingredient for this.

\begin{lemma}
\label{lem:FFnull-loc}
There is a universal constant $\tilde C$ such that
\begin{equation}
\label{e:FFnull}
\|U^\pm_x\|_{T_\phi(\h_0)}
  \leq
2 U^\pm_{\varnothing,x} + \tilde C \h_0^4 (1 + \|\phi\|^2_{\Phi_x(\h_0)}),
\end{equation}
where $U^\pm_\varnothing$ is the 0-degree part of $U^\pm$.
\end{lemma}

\begin{proof}
Let
\begin{equation}
M^+ = M^+_e = (\nabla^e \tau_x)^2,
\quad
M^- = M^-_e = 2 \tau_x \tau_{x+e},
\end{equation}
so that $U^\pm_x = \sum_{e\in\Ucal} M^\pm_e$.
It suffices to prove \eqref{e:FFnull} with $U^\pm_x$ replaced by $M^\pm$
(on both sides of the equation).
In addition, we can replace the $\Phi_x$ norm by the $\Phi$ norm;
the bound with the $\Phi_x$ norm then follows in the same way that \eqref{e:T0Kx} is a consequence of \eqref{e:T0K},
since $M^\pm \in \Ncal(x^\Box)$.

By definition of $\tau_x$,
\begin{equation}
M^\pm = M^\pm_{\varnothing} + R^\pm,
\end{equation}
where
\begin{alignat}{2}
&M^+_{\varnothing} = (\nabla^e |\phi_x|^2)^2,
  \quad
&&R^+ = 2 (\nabla^e |\phi_x|^2) \nabla^e (\psi_x\psib_x),
  \\
&M^-_\varnothing = 2 |\phi_x|^2 |\phi_{x+e}|^2,
  \quad
&&R^- = 2 (|\phi_x|^2 \psi_{x+e}\bar\psi_{x+e}
+ \psi_x\bar\psi_x |\phi_{x+e}|^2 + \psi_x\bar\psi_x\psi_{x+e}\bar\psi_{x+e}).
\end{alignat}
Thus, $\|M^\pm\|_{T_\phi} \leq \|M^\pm_{\varnothing}\|_{T_\phi} + \|R^\pm\|_{T_\phi}$.
A straightforward computation shows that
\begin{equation}
\label{e:Rpm-bound}
\|R^\pm\|_{T_\phi} = O(\h_0^4 (1 + \|\phi\|_\Phi)^2).
\end{equation}

By definition of the $T_\phi$ semi-norm,
\begin{equation}
\label{e:nabla-phi-sq-bd}
\|\nabla^e |\phi_x|^2\|_{T_\phi}
  \le
\nabla^e |\phi_x|^2 + 2 \h_0 (|\phi_x| + |\phi_{x+e}|) + 2 \h_0^2.
\end{equation}
Together with \eqref{e:Rpm-bound}, the product property,
and \eqref{e:testfcnbd}, this implies that
\begin{equation}
\|M^+\|_{T_\phi}
  \le
M^+_\varnothing
  + 2 |\nabla^e |\phi_x|^2| (2 \h_0 (|\phi_x| + |\phi_{x+e}|))
  + O(\h_0^4) (1 + \|\phi\|^2_\Phi).
\end{equation}
By the inequality
\begin{equation}
\label{e:young-ineq}
2|ab| \le |a|^2 + |b|^2
\end{equation}
and another application of \eqref{e:testfcnbd},
\begin{equation}
2 |\nabla^e |\phi_x|^2| (2 \h_0 (|\phi_x| + |\phi_{x+e}|))
  \le
M^+_\varnothing + O(\h_0^2 \|\phi\|^2_\Phi),
\end{equation}
and the bound on $M^+$ follows.

For the bound on $M^-$, we use the identity
\begin{equation}
\label{e:taunorm}
\|\tau_x\|_{T_\phi}
  =
(|\phi_x| + \h_0)^2 + \h_0^2
\end{equation}
from \cite[\eqref{norm-e:taunorm}]{BS-rg-norm}.
By the product property and \eqref{e:testfcnbd}, this implies that
\begin{equation}
\|M^-\|_{T_\phi}
  \le
2 |\phi_x|^2 |\phi_{x+e}|^2
  +
2 (|\phi_x| |\phi_{x+e}|) (2 \h_0 (|\phi_{x+e}| + |\phi_x|))
  +
O(\h_0^4) (1 + \|\phi\|^2_\Phi).
\end{equation}
Another application of \eqref{e:young-ineq} and \eqref{e:testfcnbd} gives
\begin{equation}
2 (|\phi_x| |\phi_{x+e}|) (2 \h_0 (|\phi_{x+e}| + |\phi_x|))
  \le
|\phi_x|^2 |\phi_{x+e}|^2 + O(\h_0^2 \|\phi\|^2_\Phi),
\end{equation}
and the proof is complete.
\end{proof}

An immediate consequence of Lemma~\ref{lem:FFnull-loc}, using \eqref{e:eK},
is that for any $s \ge 0$,
\begin{equation}
\label{e:Itilbd}
\|e^{-s U^\pm_x}\|_{T_\phi(\h_0)} \leq e^{\tilde C s \h_0^4 (1 + \|\phi\|^2_{\Phi_x(\h_0)})}.
\end{equation}

\begin{proof}[Proof of Proposition~\ref{prop:K0bd}]
According to the definition of the
regulator norms in \refeq{regdef}--\refeq{reg0},
it suffices to prove that, under the hypothesis on $\gamma_0$,
\begin{equation}
\label{e:K0bd}
  \|K^\pm_{0,x}\|_{T_\phi(\h_0)} = O(|\gamma_0| \h_0^4)
  \begin{cases}
  e^{\|\phi\|_{\Phi_x}^2} & (\h_0=\ell_0)
  \\
  e^{\frac{{\sf t}}{2} \|\phi\|_{\tilde\Phi}} & (\h_0=h_0).
  \end{cases}
\end{equation}
For $t \in [0,1]$, let $\tilde I^\pm_x(t) = e^{-t |\gamma_0| U^\pm_x}$.
By \eqref{e:KIJ}, \eqref{e:J}, and the product property,
\begin{align}
\label{e:K0x-est}
    \|K^\pm_{0,x}\|_{T_\phi(\h_0)}
    & \le |\gamma_0| \|I^\pm_{0,x} U^\pm_x\|_{T_\phi(\h_0)}
    \sup_{t\in [0, 1]} \|\tilde I^\pm_{x}(t)\|_{T_\phi(\h_0)}.
\end{align}
By \refeq{Iupper} and Lemma~\ref{lem:FFnull-loc},
there exists $c_3 > 0$, and, for any $c_1 \geq 0$ there exists $c_2 > 0$, such that
\begin{equation}
\label{e:Iupper-bis}
\|I^\pm_{0,x} U^\pm_x\|_{T_\phi(\h_0)}
  \leq
O(\h_0^4)
\begin{cases}
  e^{c_3 g_0  \|\phi\|^2_{\Phi_x(\ell_0)}}
    & \h_0 = \ell_0 \\
  e^{-c_1 k_0^4 \|\phi\|^2_{\Phi_x(h_0)}} e^{c_2 k_0^4 \|\phi\|^2_{\tilde\Phi_x(\ell_0)}}
    & \h_0 = h_0.
\end{cases}
\end{equation}
The constant in $O(|\gamma_0| \h_0^4)$ may depend on $c_1$,
but this is unimportant.
Also, by \eqref{e:Itilbd},
\begin{equation}
\sup_{t\in[0,1]} \|\tilde I_{x}^\pm(t) \|_{T_\phi(\h_0)}
  \le
e^{\tilde C |\gamma_0| \h_0^4 (1+\|\phi\|^2_{\Phi_x(\h_0)})}.
\end{equation}

Thus, for $\h_0=\ell_0$,
the total exponent in our estimate for the right-hand side of \refeq{K0x-est}
is
\begin{equation}
    \tilde C |\gamma_0| \ell_0^4
       +(c_3 g_0 + \tilde C |\gamma_0| \ell_0^4) \|\phi\|^2_{\Phi_x(\ell_0)}
     .
\end{equation}
This gives the $\h_0=\ell_0$ version of \refeq{K0bd} provided that
$g_0$ is small and $|\gamma_0|$ is small depending on $L$.

For $\h_0=h_0$, the total exponent in our estimate for the right-hand side of \refeq{K0x-est}
is
\begin{equation}
    \tilde C |\gamma_0| k_0^4 \ggen_0^{-1}
        + (\tilde C |\gamma_0| k_0^4 \ggen_0^{-1} - c_1 k_0^4) \|\phi\|^2_{\Phi_x(h_0)}
        + c_2 k_0^4 \|\phi\|^2_{\tilde\Phi_x(\ell_0)}.
\end{equation}
This gives the $\h_0=h_0$ version of \refeq{K0bd} provided that
$|\gamma_0| \le \ggen_0$, $c_1\ge \tilde C$, and $c_2 k_0^4 \le {\sf t}/2$.

All the provisos are satisfied
if we choose
$c_1 \ge \tilde C$,
$k_0$ small depending on $c_1$
and $\ggen_0$ small.
\end{proof}


\begin{rk}
By a small modification to the proof of Proposition~\ref{prop:K0bd},
it can be shown that if $M_x \in \Ncal(x^\square)$ is a monomial of
degree $r \le p_\Ncal -4$ (so that $M_xU_x^\pm$ has degree at most $p_\Ncal$), then
\begin{equation}
\label{e:K0bd-gen}
\|M_x K^\pm_{0,x}\|_{\Gcal_0} = O(|\gamma_0| \h_0^{4+r}).
\end{equation}
\end{rk}

%%%%%%%%%%%%%%%%%%%%%%%%%%%%%%%%%%%%%%%%%%%%%%%%%%%%%%%%%%%%%%%%%%%%%%%%%%%%%%%

\subsection{Unified bound on \texorpdfstring{$K_0$}{K0}}
\label{sec:KWcal}

The results of \cite{BS-rg-step,BBS-rg-flow} are formulated in a sequence of spaces $\Wcal_j$ that
enable the combination of small-field and large-field estimates into a single norm estimate.
In this section, we recast the result of Proposition~\ref{prop:K0bd} to see that $K_0^\pm$
fits into this formulation.

We restrict attention in this section to the $\Wcal_0$ norm,
whose definition is recalled below.
This requires several preliminaries.
Let $\Pcal_0 = \Pcal_0(\Lambda)$ denote the collection of subsets of vertices in $\Lambda$.
We refer to the elements of $\Pcal_0$ as \emph{polymers}.
We call a nonempty polymer $X\in \Pcal_0$ \emph{connected}
if for any $x, x' \in X$, there is a sequence
$x = x_0, \ldots, x_n = x' \in X$ such that
$|x_{i+1} - x_i|_\infty = 1$ for $i = 0, \ldots, n - 1$.
Let $\Ccal_0$ denote the set of connected polymers.
The \emph{small set neighbourhood} $X^\Box$ of $X\in\Pcal_0$ is defined by
\begin{equation}
    X^\Box =
    \{y \in \Lambda : \exists x \in \Lambda \; \text{such that}\; |y-x|_\infty \le 2^d\}.
\end{equation}
We extend the definitions of the regulators $\Gcal_0 = G_0, \tilde G_0^{\sf t}$,
defined in \refeq{regdef}, by setting
\begin{equation} \label{e:Gcalprod}
\Gcal_0(X, \phi) = \prod_{x\in X} \Gcal_0(x, \phi),
\end{equation}
and extend the definitions \refeq{reg0} to define norms, for $F \in \Ncal(X^\Box)$, by
\begin{equation}
\lbeq{reg0X}
    \|F\|_{G_0} = \sup_{\phi\in\C^\Lambda} \frac{\|F\|_{T_\phi(\ell_0)}}{G_0(X,\phi)}
    , \quad
    \|F\|_{\tilde{G}^{\sf t}_0} = \sup_{\phi\in\C^\Lambda} \frac{\|F\|_{T_\phi(h_0)}}{\tilde{G}^{\sf t}_0(X,\phi)}
    .
\end{equation}
It follows from the product property of the $T_\phi$ norm that these norms obey the product property
\begin{equation}
    \|F_1F_2\|_{\Gcal_0} \le   \|F_1\|_{\Gcal_0} \|F_2\|_{\Gcal_0}
    \quad \text{for $F_i\in \Ncal(X_i^\Box)$ with $X_1 \cap X_2=\varnothing$.}
\end{equation}

Given a map $K: \Pcal_0 \to \Ncal$ with the property that $K(X) \in \Ncal(X^\Box)$
for all $X \in \Pcal_0$,
we define the $\Fcal_0(\Gcal)$ norms (for $\Gcal = G, \tilde G$) by
\begin{align}
\|K\|_{\Fcal_0(G)}        &= \sup_{X\in\Ccal_0} \ggen_0^{-f_0(a, X)} \|K(X)\|_{G_0} \\
\|K\|_{\Fcal_0(\tilde G)} &= \sup_{X\in\Ccal_0}
\ggen_0^{-f_0(a, X)} \|K(X)\|_{\tilde G_0^{\sf t}},
\end{align}
with
\begin{equation}
    \label{e:f0def}
    f_0 (\amain, X)
    =
    \amain (|X|-2^d)_+
    =
    \begin{cases}
    a (|X| - 2^d)
    & \text{if } |X| > 2^d   \\
    0
    & \text{otherwise}.
    \end{cases}
\end{equation}
Here $a$ is a small constant;  its value is discussed below \cite[\eqref{step-e:T0dom}]{BS-rg-step}.
The $\Wcal_0$ norm is then defined by
\begin{align}
\label{e:9Kcalnorm}
\|K\|_{\Wcal_0}
  &=
  \max
  \Big\{
  \|K \|_{\Fcal_0(G)},\,
  \ggen_0^{9/4}
  \|K \|_{\Fcal_0(\tilde{G})}
  \Big\}.
\end{align}
Since this definition depends on $\ggen_0$ and the
volume $\Lambda$, we sometimes write $\Wcal_0 = \Wcal_0(\ggen_0, \Lambda)$.
The following proposition uses Proposition~\ref{prop:K0bd} to obtain a bound on the $\Wcal_0$ norm
of the map $K_0^\pm : \Pcal_0 \to \Ncal$ defined by
\begin{equation}
    K_0^\pm(X) = \prod_{x \in X} K_{0,x}^\pm \qquad (X \in \Pcal_0)
    .
\end{equation}

\begin{prop}
\label{prop:KWcal}
If $V_0^\pm \in \DV_0$ with $\ggen_0$ sufficiently small
(depending on $L$), and if $|\gamma_0| \le O(\ggen_0^{1+a'})$
for some $a' >a$,
then $\|K_0^\pm\|_{\Wcal_0} \le O(|\gamma_0|)$,
where all constants may depend on $L$.
\end{prop}

\begin{proof}
Let $X$ be a connected polymer in $\Pcal_0$.
By the product property and Proposition~\ref{prop:K0bd},
\begin{align}
\lbeq{K0prod}
    \|K_0^\pm(X)\|_{\Gcal_0} \le (c|\gamma_0|\h_0^4)^{|X|}
    &=
    (c|\gamma_0|\h_0^4)^{|X|\wedge 2^d} (c|\gamma_0|\h_0^4)^{(|X|-2^d)_+}.
\end{align}
For $\Gcal_0=G_0$, we use $\h_0=\ell_0$,
$(c|\gamma_0|\h_0^4)^{|X|\wedge 2^d}\le O(|\gamma_0|)$, and
\begin{equation}
    (c|\gamma_0|\h_0^4)^{(|X|-2^d)_+} \le (c' \ggen_0)^{(1+a')(|X|-2^d)_+} \le \ggen_0^{f_0(a,X)}.
\end{equation}
For $\Gcal_0=\tilde G_0$, we use $\h_0=h_0 = O(\ggen_0^{-1/4})$ and, since $a'>a$,
\begin{equation}
    (c|\gamma_0|\h_0^4)^{(|X|-2^d)_+} \le (c' \ggen_0)^{a'(|X|-2^d)_+} \le \ggen_0^{f_0(a,X)}.
\end{equation}
Since $|\gamma_0| \le \ggen_0$, it follows from \refeq{K0prod} that
\begin{equation}
    \ggen_0^{9/4}   \|K_0^\pm \|_{\Fcal_0(\tilde{G})}
    \le
    \ggen_0^{9/4}O(|\gamma_0| \ggen_0^{-1})
    \le |\gamma_0|,
\end{equation}
and the proof is complete.
\end{proof}

The above discussion is based on norms in the setting of the torus $\Lambda$.
As in \cite{BS-rg-step}, a version on the infinite lattice $\Zd$ is also required.
This can be done in exactly the same manner,
by defining
the polymers $\Pcal_0 = \Pcal_0(\Zd)$
to be the collection
of subsets of $\Zd$, with $K_0^\pm(X)$ defined for subsets of $\Zd$ by
$\prod_{x \in X} K_{0,x}^\pm$.
The $\Wcal_0 = \Wcal_0(\ggen_0, \Zd)$ norm (in infinite volume)
can be defined analogously to \eqref{e:9Kcalnorm}.
The hypotheses and conclusion of Proposition~\ref{prop:KWcal} remain the same
in the setting of $\Zd$.

%%%%%%%%%%%%%%%%%%%%%%%%%%%%%%%%%%%%%%%%%%%%%%%%%%%%%%%%%%%%%%%%%%%%%%%%%%%%%%%

\subsection{Smoothness of \texorpdfstring{$K_0$}{K0}}
\label{sec:Ksmooth}

Let $\Ccal_0(\Z^d) \subset \Pcal_0(\Z^d)$ be the set of connected polymers.
By definition, a connected polymer is nonempty.
Given $\ggen_0>0$, let
$\Wcal^*_0(\ggen_0, \Zd)$ denote the space of maps
$F :\Ccal_0(\Zd) \to \Ncal$,
with $F(X) \in \Ncal(X^\Box)$ and $\|F\|_{\Wcal_0(\ggen_0, \Zd)} < \infty$.
Addition in this space is defined by $(F_1+F_2)(X)=F_1(X)+F_2(X)$.
We extend any $F :\Ccal_0(\Zd) \to \Ncal$ to $F :\Pcal_0(\Zd) \to \Ncal$
by taking $F(X) = \prod_{Y} F(Y)$ where the product is over the connected components $Y$ of $X$.

Given any map $F : D \to \Wcal^*_0(\ggen_0, \Zd)$ for $D \subset \R$ an open interval,
write $F_X, F^\phi_X : D \to \Ncal(X^\square)$ for the
maps defined by partial evaluation of $F$ at $X$ and at
$(X, \phi)$, respectively. We say $F^\phi_X$ is $C^k$
if all of its coefficients in the decomposition \eqref{e:FinNcal}
are $C^k$ as functions $D \to \R$.

\begin{lemma}
\label{lem:smoothness}
Let $D \subset \R$ be open and $F : D \to \Wcal^*_0(\ggen_0, \Zd)$ be a map.
Suppose that $F^\phi_X$ is $C^2$ for all $X \in \Ccal_0$
and $\phi \in \C^\Lambda$, and define 
$F^{(i)} : D \to \Wcal^*_0(\ggen_0, \Zd)$ by $(F^{(i)}(t))^\phi_X = (F^\phi_X)^{(i)}(t)$ for $i = 1, 2$,
where the right-hand side denotes the (component-wise) $i^{\rm th}$
derivative of $F^\phi_X$.
If $\|F^{(i)}(t)\|_{\Wcal_0} < \infty$ for $i = 1, 2$ and $t \in D$, then 
$F^{(1)}$ is the derivative of $F$.
\end{lemma}

\begin{proof}
For $t, t + s \in D$, define $R(t, s) \in \Wcal_0$ by
\begin{equation}
R^\phi_X(t, s) = F^\phi_X(t + s) - F^\phi_X(t) - s (F^\phi_X)'(t).
\end{equation}
By Taylor's theorem, for any $\phi$ and $X$,
\begin{equation}
R^\phi_X(t, s) = s^2 \int_0^1 (F^\phi_X)''(t + u s) (1 - u) \; du,
\end{equation}
where the integral is taken component-wise.
It follows
that
\begin{equation}
\|R(t, s)\|_{\Wcal_0}
  \le |s|^2 \sup_{u\in[0,1]} \|F''(t+us)\|_{\Wcal_0}
  \le O(|s|^2),
\end{equation}
so $F$ is differentiable 
and its derivative satisfies $(F')^\phi_X = (F^\phi_X)'$.
Continuity of $F'$ follows similarly, since, by the
fundamental theorem of calculus,
\begin{equation}
\|F'(t+s) - F'(t)\|_{\Wcal_0}
  \le
|s| \sup_{u\in[t,t+s]} \|F''(u)\|_{\Wcal_0}
  \le
O(|s|),
\end{equation}
which suffices.
\end{proof}

Consider the map
\begin{equation}
(g_0, \gamma_0, \nu_0, z_0) \mapsto K_0 \in \Wcal^*_0(\ggen_0, \Zd)
\end{equation}
defined by
\begin{equation}
\label{e:K0def}
K_0(g_0, \gamma_0, \nu_0, z_0) =
\begin{cases}
K^+_0(g_0, \gamma_0, \nu_0, z_0)
  & (\gamma_0 \geq 0) \\
K^-_0(g_0, \gamma_0, \nu_0, z_0)
  & (\gamma_0 < 0),
\end{cases}
\end{equation}
for $(g_0, \gamma_0, \nu_0, z_0)$ satisfying the hypotheses
of Proposition~\ref{prop:KWcal}.
The map $K_0$ is in fact analytic away from $\gamma_0 = 0$.
However, we only prove the following, which is what we need later.

\begin{prop}
\label{prop:Ksmooth}
Suppose that $V_0^\pm \in \DV_0$, with $\ggen_0$ sufficiently small
(depending on $L$) and $|\gamma_0| \le O(\ggen_0^{1+a'})$
for some $a' >a$.
The map $K_0(g_0, \gamma_0, \nu_0, z_0)$ is jointly continuous
in its four variables, is
$C^1$ in $(g_0, \nu_0, z_0)$,
and (when $\gamma_0 \ne 0$) is $C^1$ in $(g_0, \gamma_0, \nu_0, z_0)$,
with partial derivatives with respect to $t = g_0$, $\nu_0$, and $z_0$ satisfying
\begin{equation}
\label{e:ddpK}
\|\partial K_0 / \partial t\|_{\Wcal_0} = O(|\gamma_0| \h_0^8).
\end{equation}
Moreover, $K_0$
is left- and right-differentiable in $\gamma_0$ at $\gamma_0 = 0$.
\end{prop}

\begin{proof}
Let $t$ denote any one of the coupling constants $g_0, \gamma_0, \nu_0$ or $z_0$.
We drop the subscript $0$, and let $K(t)$ denote $K_0$ viewed as a function of $t$,
with the remaining coupling constants fixed. Then $K^\phi_X$ is smooth for any $\phi, X$.
If $t$ is $g_0, \nu_0$ or $z_0$, then
\begin{align}
(K^\phi_x)'  &= -M_x(\phi) K^\phi_x, \quad
(K^\phi_x)'' = M_x^2(\phi) K^\phi_x,
\end{align}
where $M_x$ is $\tau_x^2, \tau_x$ or $\tau_{\Delta,x}$, respectively.
The maximal degree of $M_x$ is $4$, so
\eqref{e:K0bd-gen} implies that
\begin{equation}
\label{e:Kprime-bd1}
\|K'_x\|_{\Gcal_0} \le O(|\gamma_0| \h_0^{8}),
  \quad
\|K''_x\|_{\Gcal_0} \le O(|\gamma_0| \h_0^{12}).
\end{equation}

For $t$ denoting $\gamma_0$,
we restrict attention to $\gamma_0 > 0$, and write $U = U^+$
and $V_0 = V^+_0$ (the case $\gamma_0 < 0$ is similar). Then
\begin{equation}
\label{e:dKdgamma0}
(K^\phi_x)'  = -U_x(\phi) e^{-V_x(\phi) - \gamma_0 U_x(\phi)}, \quad
(K^\phi_x)'' = U_x^2(\phi) e^{-V_x(\phi) - \gamma_0 U_x(\phi)},
\end{equation}
and \eqref{e:Iupper} and \eqref{e:Itilbd} imply that
\begin{equation}
\label{e:Kprime-bd2}
\|K'_x\|_{\Gcal_0} \le O(\h_0^4),
  \quad
\|K''_x\|_{\Gcal_0} \le O(\h_0^8).
\end{equation}

By definition, $K_X = \prod_{x \in X} K_x$, so, for derivatives with respect to any one
of the four variables (with $\gamma_0 \neq 0$ when differentiating with respect to $\gamma_0$),
\begin{equation}
\label{e:KXprime}
(K^\phi_X)'  = \sum_{x \in X} (K^\phi_x)' K^\phi_{X \setminus x}, \quad
(K^\phi_X)'' = \sum_{x \in X} ((K^\phi_x)'' K^\phi_{X \setminus x} + (K^\phi_x)' (K^\phi_{X \setminus x})').
\end{equation}
Thus, by the product property, \eqref{e:Kprime-bd1}, and Proposition~\ref{prop:K0bd},
\begin{equation}
\|K'_X\|_{\Gcal_0}
  \le
O(|X|) |\gamma_0| \h_0^8 (|\gamma_0| \h_0^4)^{|X|-1}.
\end{equation}
when differentiating with respect to $g_0$, $\nu_0$, or $z_0$.
The bound \eqref{e:ddpK} then follows from the hypothesis on $\gamma_0$.
Similarly, using \eqref{e:Kprime-bd2},
\begin{equation}
\|K'_X\|_{\Gcal_0}
  \le
O(|X|) \h_0^4 (|\gamma_0| \h_0^4)^{|X|-1}
\end{equation}
when differentiating with respect to $\gamma_0$ away from $\gamma_0 = 0$.
In both cases, we have
\begin{equation}
\|K''_X\|_{\Gcal_0}
  \le
O(|X|^2) \h_0^8 (|\gamma_0| \h_0^4)^{(|X|-2) \wedge 0}.
\end{equation}
Thus, by Lemma~\ref{lem:smoothness}, $K$ is $C^1$ in any of its variables.
Therefore, $K$ is $C^1$ in $(g_0, \nu_0, z_0)$ on the whole domain and in all the variables when $\gamma_0 \ne 0$.

To show right-continuity in $\gamma_0$ at $\gamma_0 = 0$,
fix $(g_0, \nu_0, z_0)$ and define $F \in \Wcal^*_0$ by
\begin{equation}
F(X) =
\begin{cases}
  -U_x e^{-V_{0,x}}
    & X = \{ x \} \\
  0 & |X| > 1,
\end{cases}
\end{equation}
where $U_x, V_{0,x}$ are defined above.
Let $K'(\gamma_0)$ denote the $\gamma_0$ derivative of $K$ evaluated at $\gamma_0 > 0$.  Then
\eqref{e:dKdgamma0} and \eqref{e:KXprime} imply that
\begin{equation}
F(X) - K'_X(\gamma_0)
  =
\begin{cases}
  U_x K_x(\gamma_0)
    & X = \{ x \} \\
  \sum_{x \in X} K'_x(\gamma_0) K_{X \setminus x}(\gamma_0)
    & |X| > 1.
\end{cases}
\end{equation}
Thus, by \eqref{e:K0bd-gen}, \eqref{e:Kprime-bd2}, and Proposition~\ref{prop:K0bd},
\begin{equation}
\|F(X) - K'_X(\gamma_0)\|_{\Gcal_0}
  \le
\begin{cases}
  O(\gamma_0 \h_0^8)
    & X = \{ x \} \\
  O(|X|) \h_0^4 (\gamma_0 \h_0^4)^{|X|-1}
    & |X| > 1.
\end{cases}
\end{equation}
It follows that
\begin{equation}
\lim_{\gamma_0\downarrow 0} \|F - K'(\gamma_0)\|_{\Wcal_0} = 0,
\end{equation}
i.e., $F$ is the right-derivative of $K$ in $\gamma_0$ at $\gamma_0 = 0$.
Left-continuity is handled similarly.
\end{proof}

\section{Renormalisation group flow}
\label{sec:flow}

The following theorem is an extension of \cite[Proposition~\ref{log-prop:flow-flow}]{BBS-saw4-log}
to non-trivial $K_0$. Such an extension is possible,
with only minor modifications to the proof of the $K_0 = \1_\varnothing$ case,
due to the generality allowed by the main result of \cite{BBS-rg-flow}.

The theorem provides, for any $N \ge 1$ and for initial error coordinate $K_0$
in a specified domain, a choice of initial condition $(\nu_0^c,z_0^c)$
for which there exists
a finite-volume renormalisation group flow $(V_j, K_j) \in \domRG_j$ for $0 \le j \le N$.
In order to ensure a degree of consistency amongst the sequences $(V_j, K_j)$, which depend on
the volume $\Lambda_N$, a notion of consistency must be imposed upon the collection of initial
error coordinates $K_{0,\Lambda} \in \Kcal_0(\Lambda)$ for varying $\Lambda$.
Specifically, the family $K_{0,\Lambda}$ is required to satisfy the property $(\Zd)$ of
\cite[Definition~\ref{step-defn:KZd}]{BS-rg-step}.
We refer to any such family as a $\Lambda$-family.
As discussed in \cite[Definition~\ref{step-defn:KZd}]{BS-rg-step},
any $\Lambda$-family
induces an infinite-volume error coordinate $K_{0,\Zd} \in \Kcal_0(\Zd)$ in a natural way.

\begin{theorem}
\label{thm:flow-flow}
Let $d = 4$.
There exists a constant $a_* > 0$ and continuous functions $\nu_0^c, z_0^c$
of $(m^2, g_0, K_0)$, defined for $(m^2, g_0) \in [0, \delta]^2$
(for some $\delta > 0$ sufficiently small) and for any $K_0 \in \Wcal_0(m^2, g_0, \Zd)$
with $\|K_0\|_{\Wcal_0(m^2, g_0, \Zd)} \leq a_* g_0^3$, such that
the following holds for $g_0 > 0$:
if $K_{0,\Lambda} \in \Kcal_0(\Lambda)$ is a $\Lambda$-family
that induces the infinite-volume coordinate $K_0$, and if
\begin{equation}
\label{e:flow-flow-ic}
V_0 = V_0^c(m^2, g_0, K_0) = (g_0, \nu_0^c(m^2,g_0,K_0), z_0^c(m^2,g_0,K_0)),
\end{equation}
then for any $N \in \N$ and $m^2 \in [\delta L^{-2 (N - 1)}, \delta]$,
there exists a sequence $(V_j, K_j) \in \domRG_j(m^2, g_0, \Lambda)$
such that
\begin{equation}
  \label{e:VjKjDj}
  (V_{j+1},K_{j+1}) = (V_{j+1}(V_j, K_j), K_{j+1}(V_j, K_j)) \text{ for all } j < N
\end{equation}
and \eqref{e:ZjIjKj} is satisfied.
Moreover, $\nu_0^c,z_0^c$ are continuously differentiable in
$g_0 \in (0, \delta)$ and $K_0 \in B_{\Wcal_0(m^2, g_0, \Lambda)}(a_* g_0^3)$, and
\begin{align}
&\nu_0^c(m^2,0,0) = z_0^c(m^2,0,0) = 0,
\quad
\ddp{\nu_0^c}{g_0} = O(1),
\quad
\label{e:z0est}
\ddp{z_0^c}{g_0} = O(1),
\end{align}
where the estimates above hold uniformly in $m^2 \in [0, \delta]$.
\end{theorem}

\begin{proof}
The proof results from small modifications to the proofs of
\cite[Proposition~\ref{log-prop:flow-flow}]{BBS-saw4-log} and then to
\cite[Proposition~\ref{log-prop:KjNbd}]{BBS-saw4-log},
where (in both cases) we relax the requirement that $K_0 = \1_\varnothing$,
which was chosen in \cite{BBS-saw4-log} due to the fact that
$K_0 = \1_\varnothing$ when $\gamma=0$.
The more general condition that $K_0 \in B_{\Wcal_0(m^2, g_0, \Lambda)}(a_* g_0^3)$
comes from the hypothesis of \cite[Theorem~\ref{flow-thm:flow}]{BBS-rg-flow}
when $(m^2, g_0) = (\mgen^2, \ggen_0)$.
By \cite[Remark~\ref{flow-rk:Nrad}]{BBS-rg-flow}, no major changes to the proof
result from this choice of $K_0$.
The following paragraph outlines
in more detail the modifications to the proof of
\cite[Proposition~\ref{log-prop:flow-flow}]{BBS-saw4-log}.

By \cite[Theorem~\ref{flow-thm:flow}]{BBS-rg-flow} and
\cite[Corollary~\ref{flow-cor:masscont}]{BBS-rg-flow},
for any $(\mgen^2, \ggen_0) \in (0, \delta)^2$ and
$\tilde K_0 \in B_{\Wcal_0(\mgen^2, \ggen_0, \Zd)}(a_* \ggen_0^3)$,
there is a neighbourhood
${\sf N}(\ggen_0, \tilde K_0)$ of $(\ggen_0, \tilde K_0)$
such that for all
$(m^2, g_0, K_0) \in \Igen(\mgen^2) \times {\sf N}(\ggen_0, \tilde K_0)$,
there is an infinite-volume renormalisation group flow
\begin{equation}
(\Vch_j, K_j) = \xch^d_j(\mgen^2, \ggen_0, \tilde K_0; m^2, g_0, K_0)
\end{equation}
in \emph{transformed variables} $(\Vch_j, K_j)$.
The transformed variables are defined in
\cite[Section~\ref{log-sec:trans}]{BBS-saw4-log} and a flow
in the original variables can be recovered from the transformed flow.
The global solution is defined by
$\xch^c_j(m^2, g_0, K_0) = \xch^d_j(m^2, g_0, K_0; m^2, g_0, K_0)$
(or $\xch^c \equiv 0$ if $g_0 = 0$).
By \cite[Remark~\ref{flow-rk:Nrad}]{BBS-rg-flow},
the proof of regularity of $\xch^c$ can proceed as in \cite{BBS-saw4-log}.
The functions $(\nu_0^c, z_0^c)$ are given by the $(\nu_0, z_0)$ components
of $\xch^c_0 = (\Vch_0, K_0) = (V_0, K_0)$.
\end{proof}


\begin{rk}
The proof of \cite[Proposition~\ref{log-prop:flow-flow}]{BBS-saw4-log},
hence of Theorem~\ref{thm:flow-flow},
makes important use of the parameter $\ggen_0$ in order to prove regularity
of the renormalisation group flow in $g_0$. However, once the flow has been
constructed, we can and do set $\ggen_0 = g_0$.
\end{rk}

We wish to obtain a version of \eqref{e:chi-m}--\eqref{e:chiprime-m}
with the initial conditions of Section~\ref{sec:IK}, i.e.\ with
$(\hat g_0, K_0) = (g_0, K^+_0)$.
It is straightforward to verify that $K^\pm_0 \in \Kcal_0$.
For instance, the fact that $K^\pm_0$ is supersymmetric
(which is required of all elements of $\Kcal_0$) follows
from the fact that $K^\pm_{0,x}$ is a function of $\tau_x$
(see \cite[Section~\ref{pt-sec:bulksym}]{BBS-rg-pt} for more on this topic).
It also follows from the definition that
the finite-volume coordinates $K^\pm_{0,\Lambda}$ form a $\Lambda$-family.

Moreover,
by Proposition~\ref{prop:KWcal}, if
$|\gamma_0|$ is sufficiently small (depending on $g_0$; we now take $\ggen_0=g_0$)
then $K_0 = K^\pm_0$ satisfies the bound required by Theorem~\ref{thm:flow-flow}.
However, we cannot apply the theorem immediately with this choice
of $K_0$,
due to the fact that $K^\pm_0$
depends on $(g_0, \nu_0, z_0)$.
We resolve this issue in the next section.

\section{Critical parameters}
\label{sec:nu0z0c}

% For convenience, let
% \begin{equation}
% \lbeq{g0hatdef}
% \hat g_0 = \hat g_0(g_0, \gamma_0) = g_0 + 4 d \gamma_0 \1_{\gamma_0 < 0}.
% \end{equation}
% Thus, $\hat g_0$ is the coefficient of $\tau_x^2$ in $V^+_{0,x}$
% when $\gamma_0 \ge 0$, and in $V^-_{0,x}$ when $\gamma_0 < 0$.
Recall the function $K_0(g_0, \gamma_0, \nu_0, z_0)$
defined in \eqref{e:K0def}.
We wish to initialise the renormalisation group with $(\nu_0, z_0)$ a solution
to the system of equations
\begin{align}
&\nu_0 = \nu_0^c(m^2, \hat g_0(g_0, \gamma_0), K_0(g_0, \gamma_0, \nu_0, z_0)), \label{e:mu0c}
\\
&z_0 = z_0^c(m^2, \hat g_0(g_0, \gamma_0), K_0(g_0, \gamma_0, \nu_0, z_0)) \label{e:z0c}
.
\end{align}
Such a choice of $(\nu_0, z_0)$ will be critical for $K_0$,
where $K_0$ is itself evaluated at this same choice of $(\nu_0, z_0)$.

When $\gamma_0 = 0$, we get $K_0 = \1_\varnothing$, so $K_0$ no longer depends on $(\nu_0, z_0)$
and this system is solved by $(\nu_0^c(m^2, g_0, 0), z_0^c(m^2, g_0, 0))$
for any (small) $m^2, g_0 \geq 0$.
Local solutions for $\gamma_0 \neq 0$ can then be
constructed using a version of the implicit function theorem from \cite{LS14}
that allows for the continuous but non-smooth behaviour of $K_0$ in $\gamma_0$.
In order to obtain global solutions with certain desired regularity properties
(needed in the next section), we make use of Proposition~\ref{prop:IFT},
which is based on a version of the implicit function theorem from \cite{LS14}.

Recall that $D(\delta, r)$ was defined in \eqref{e:Ddef}.

\begin{prop}
\label{prop:nuzhat}
There exists a continuous positive-definite function $\hat r : [0, \delta] \to [0, \infty)$
and continuous functions
$\hat\nu_0^c, \hat z_0^c \in C^{0,1,\pm}(D(\delta, \hat r))$ such that
the system \eqref{e:mu0c}--\eqref{e:z0c} is solved by $(\nu_0, z_0) = (\hat\nu_0^c, \hat z_0^c)$
whenever $(m^2, g_0, \gamma_0) \in D(\delta, \hat r)$.
Moreover, these functions satisfy the bounds
\begin{equation}
\label{e:hat-est-re}
\hat\nu_0^c = O(g_0),
\quad
\hat z_0^c = O(g_0)
\end{equation}
uniformly in $(m^2, \gamma_0)$.
\end{prop}

\begin{proof}
Recall the definition of $\ghat_0$ in \refeq{g0hatdef}, and
let
\begin{equation}
F(m^2, g_0, \gamma_0, \nu_0, z_0)
= (\nu_0, z_0)
  -
  (\nu_0^c(m^2, \hat g_0, K_0),
  z_0^c(m^2, \hat g_0, K_0)
),
\end{equation}
where $K_0 = K_0(g_0, \gamma_0, \nu_0, z_0)$.
Then for $\delta > 0$ small and an appropriate constant $c > 0$ (depending on $a_*$),
$F$ is well-defined on
\begin{equation}
\{ (m^2, g_0, \gamma_0, \nu_0, z_0) : (m^2, \hat g_0, \gamma_0) \in D(\delta, c g_0^3),
|\nu_0|, |z_0| \leq C_\DV g_0 \}.
\end{equation}
Indeed, for $(m^2, g_0, \gamma_0, \nu_0, z_0)$ in this domain,
Proposition~\ref{prop:KWcal} (with $\ggen_0 = g_0$) implies that $(m^2, \hat g_0, K_0)$ is in the domain of
$(\nu_0^c, z_0^c)$.
By Theorem~\ref{thm:flow-flow} and Proposition~\ref{prop:Ksmooth},
$F$ is $C^1$ in $(g_0, \nu_0, z_0)$
and also in $\gamma_0$ away from $\gamma_0 = 0$,
continuous in $m^2$, and has one-sided derivatives in $\gamma_0$ at $\gamma_0 = 0$.

For fixed $(\bar m^2, {\bar g_0}) \in [0, \delta]^2$,
set $({\bar\nu_0}, \bar z_0) = (\nu_0^c(\bar m^2, \bar g_0, 0), z_0^c(\bar m^2, \bar g_0, 0))$
so that
\begin{equation}
F(\bar m^2, \bar g_0, 0, \bar\nu_0, \bar z_0) = (0, 0).
\end{equation}
By \eqref{e:ddpK}, at $(\bar g_0, 0, \bar\nu_0, \bar z_0)$,
\begin{equation}
\frac{\partial K_{0,x}}{\partial\nu_0}
= \frac{\partial K_{0,x}}{\partial z_0} = 0.
\end{equation}
It follows that $D_{\nu_0,z_0} F(\bar m^2, \bar g_0, 0, \bar\nu_0, \bar z_0)$
is the identity map on $\R^2$.
The existence of $\delta, \hat r$ and $\hat\nu_0^c, \hat z_0^c$
follows from Proposition~\ref{prop:IFT} with
$w = m^2, x = g_0, y = \gamma_0, z = (\nu_0, z_0)$,
and with $r_1(g_0) = c g_0^3$, $r_2(g_0) = C_\DV g_0$.

By the fundamental theorem of calculus, for any $0 < a < \gamma_0$,
\begin{equation}
\hat\nu_0^c(m^2, g_0, \gamma_0)
  =
\hat\nu_0^c(m^2, g_0, a)
  +
\int_a^{\gamma_0} \ddp{\hat\nu_0^c}{\gamma_0} (m^2, g_0, t) \; dt.
\end{equation}
Taking the limit $a\downarrow 0$ and using \eqref{e:z0est}, we obtain
\begin{equation}
|\hat\nu_0^c(m^2, g_0, \gamma_0)|
  \leq
O(g_0)
  +
\gamma_0
\sup_{t \in (0, \gamma_0]}
\left|\ddp{\hat\nu_0^c}{\gamma_0}(m^2, g_0, t)\right|.
\end{equation}
The supremum above is bounded by a constant and so
the first estimate of \eqref{e:hat-est-re} for $\gamma_0 \geq 0$
follows from the fact that $|\gamma_0| \leq \hat r(g_0)$
(since $\hat r(g_0)$ can be taken as small as desired).
The case $\gamma_0 < 0$ and the second estimate follow similarly.
\end{proof}

\begin{proof}[Proof of Theorem~\ref{thm:rhatflow}]
By Proposition~\ref{prop:KWcal},
and by taking $\hat r$ smaller if necessary,
$K_0 = K^\pm_0$ satisfies the estimate required by Theorem~\ref{thm:flow-flow}
whenever $(m^2, g_0, \gamma_0) \in D(\delta, \hat r)$. The
existence of the sequence \eqref{e:VjKjDj-hat} then follows from
Theorem~\ref{thm:flow-flow} and Proposition~\ref{prop:nuzhat}.
Although the presence of $\gamma_0$ causes a shift in initial
conditions, the second-order evolution of $V_j$ is still given by the map
$V_\pt$ (see \eqref{e:Vflow}),
which is independent of $\gamma_0$.
\end{proof}

\begin{rk}
We have invoked \eqref{e:ddpK} above in order to satisfy the condition
\begin{equation}
\|\partial K_0/\partial\nu_0\|_{\Wcal_0} \le O(g_0^3)
\end{equation}
required in the proof of \cite[Lemma~\ref{log-lem:gzmuprime}]{BBS-saw4-log}
(see \cite[\eqref{log-e:induct1}]{BBS-saw4-log}). This condition holds trivially
when $K_0$ does not depend on $\nu_0$, as in \eqref{e:chi-m}--\eqref{e:chiprime-m}.
\end{rk}				% critical parameters from saw-sa

%% Chapter 6 %%
\chapter{Conclusion}

\setcounter{footnote}{0}

We end with a discussion of some open problems that may be accessible by
extensions to the renormalisation group method discussed in this thesis.
We will try to point out some of the main obstacles that must be overcome.

%%%%%%%%%%%%%%%%%%%%%%%%%%%%%%%%%%%%%%%%%%%%%%%%%%%%%%%%%%%%%%%%%%%%%%%%%%%%%%%
%%%%%%%%%%%%%%%%%%%%%%%%%%%%%%%%%%%%%%%%%%%%%%%%%%%%%%%%%%%%%%%%%%%%%%%%%%%%%%%

\section{Other models}

In order to apply the renormalisation group to the models we have considered,
we had to express them as perturbations of a Gaussian measure whose covariance
admits an appropriate finite-range decomposition. Here we discuss other models
that can be written in this way.

%%%%%%%%%%%%%%%%%%%%%%%%%%%%%%%%%%%%%%%%%%%%%%%%%%%%%%%%%%%%%%%%%%%%%%%%%%%%%%%

\subsection{Long-range models}

In \cite{WF72}, Wilson and Fisher suggested studying models in $d_c - \epsilon$
dimensions with $\epsilon > 0$ small and $d_c = 4$ the upper-critical dimension.
By setting $\epsilon = 1$, they obtained approximate values for critical exponents
in $3$ dimensions. One approach to the rigorous implementation of this idea
involves studying models in dimension $d$ (an integer) whose upper-critical
dimension is $d_c + \epsilon$. This is not as problematic as considering
models in fractional dimensions, as the upper-critical dimension $d_c$ need not
be the actual dimension of some ambient space.

Given a massless covariance $C'$, the upper-critical
dimension is simply a number $d_c$ such that some class of models scales like a
Gaussian model with covariance $C'$ if and only if $d > d_c$. Recalling Remark~\ref{rk:bubble},
we might expect that
\begin{equation}
d_c = \inf \{ d \in \R : B_{m^2} < \infty \},
\end{equation}
where $B_{m^2}$ is the bubble diagram, defined as the $\ell^2(\Zd)$ norm of the
massive Green function $C = (C' + m^2)^{-1}_{0x}$. Thus, in order to achieve
$d_c = d + \epsilon$, we choose $C'$ to decay like
\begin{equation}
C'_{0x} \asymp |x|^{-(d-\alpha)}
\end{equation}
with $\alpha = \tfrac12 (d + \epsilon)$. Such a choice is given by
\begin{equation}
C' = (-\Delta)^{-\alpha/2}
\end{equation}
for $\alpha\in(0, 2)$ (so $d \le 3$).

This approach has been used to implement the renormalisation group below the
upper-critical dimensions in \cite{BDH98,MS00,BMS03,Abde07}. Recently, Slade
\cite{Slad17} has extended the approach discussed in this thesis to compute
\emph{anomalous} (non-Gaussian) critical exponents for long-range versions of
the weakly self-avoiding walk and the $|\varphi|^4$ model. In particular, he
showed that, as $\nu\downarrow\nu_c$ for these models, the susceptibility $\chi$
satisfies
\begin{equation}
\label{e:chi-anom}
\chi
	\asymp
(\nu - \nu_c)^{-\left(1 + \tfrac{n+2}{n+8} \tfrac{\epsilon}{\alpha} + O(\epsilon^2)\right)}.
\end{equation}
By extensions of \cite{Slad17} to use observable fields, we think it should
be possible to identify the scaling behaviour of the two-point function and possibly
other correlation functions for these long-range models. In particular, this would
make it possible to confirm (if true) the intriguing prediction of \cite{FMN72}, which
states that
\begin{equation}
\eta = 2 - \alpha
\end{equation}
if $d = d_c - \epsilon$ for small $\epsilon$. In other words, unlike the susceptibility,
deviations from mean-field behaviour of the two-point function cannot be detected
to any order in $\epsilon$.

\begin{rk}
Models at and above the upper-critical dimension exhibit \emph{asymptotic freedom}.
In our context, this means that $\|K_j\|_{\Wcal_j} \to 0$,
$\nu_j, z_j \to 0$, and $g_j \to 0$ in the massless regime $m^2 = 0$. Below $d_c$, we do
not have asymptotic freedom, as reflected by the lack of exact asymptotics in
\eqref{e:chi-anom}. In some ways, this is advantageous (see \cite{Slad17}), but
in others it creates additional difficulties that must be overcome.
\end{rk}

%%%%%%%%%%%%%%%%%%%%%%%%%%%%%%%%%%%%%%%%%%%%%%%%%%%%%%%%%%%%%%%%%%%%%%%%%%%%%%%

\subsection{The \texorpdfstring{$O(n)$}{O(n)} model and self-avoiding walk}
\label{sec:hard-core}

Recall that the Hamiltonian of the $O(n)$ model was defined in \eqref{e:on-model}.
On $\Lambda$, it takes the form
\begin{equation}
H_J(\sigma)
	=
-\frac12 \sigma \cdot J \sigma.
\end{equation}
This was derived from the $|\varphi|^4$ model by taking a suitable
$g\to\infty$ limit. The restriction to small coupling $g$ is deeply embedded into
the method we use, but the Kac-Siegert transformation (see \cite{Bryd09}) offers
an alternative approach to the study of this model.

Namely, let $\Omega = (S^{n-1})^\Lambda$ and let $d\sigma$ denote the product measure on $\Omega$,
where $S^{n-1}$ is equipped with the uniform
% \footnote{In our derivation of the $O(n)$ model, we get an unnormalized surface measure on
% the sphere, but this is an unimportant difference.}
measure. The partition function of the $O(n)$ model is given by
\begin{equation}
Z_J = \int_\Omega e^{-H_J(\sigma)} \; d\sigma.
\end{equation}
When $J$ is a positive-definite symmetric matrix, the Gaussian measure $d\mu_J(\varphi)$
with covariance $J$ is well-defined and satisfies the elementary identity
\begin{equation}
e^{-H_J(\varphi)}
	=
e^{\frac12 \sigma \cdot J \sigma}
	=
\int_{(\R^n)^\Lambda} e^{\sigma \cdot \varphi} \; d\mu_J(\varphi).
\end{equation}
Interchanging the order of integration, we can write
\begin{equation}
Z_J
	=
\int_{(\R^n)^\Lambda}
e^{-\sum_{x\in\Lambda} L(\varphi_x)}
% \left(\prod_{x\in\Lambda} L(\varphi_x)\right)
\; d\mu_J(\varphi),
\end{equation}
where
\begin{equation}
L(t)
	=
-\log
\int_{S^{n-1}} e^{\sigma_0 \cdot t} \; d\sigma_0,
	\quad
t\in\R^n
\end{equation}
is the negative logarithm of the Laplace transform of the sphere. Since $L$ is
a rotation- and reflection-invariant analytic function and $L(0) = 0$, it has
the form
\begin{equation}
\label{e:logLaplace}
L(t) = \nu |t|^2 + g |t|^4 + \sum_{k=3}^\infty c_{2k} |t|^{2k}.
\end{equation}
Letting $J = (-\Delta + \gamma^2)^{-1}$, we have
\begin{equation}
d\mu_J(\varphi)
	\propto
e^{-\frac12 (\gamma^2 |\varphi|^2 + \varphi \cdot (-\Delta \varphi))}.
\end{equation}
Thus, we can express the partition function as a perturbed $|\varphi|^4$ model.

By a procedure as in Section~\ref{sec:reformulation}, the analysis of this model
can be reformulated in terms of the evolution of an effective interaction $Z_j$
with initial condition $Z_0 = (I_0 \circ K_0)(\Lambda)$. Once again, the initial
error coordinate $K_0$ will be coupled to $I_0$, but we expect that the critical
parameters $\nu_0^c, z_0^c$ can be identified by an implicit function theorem
argument as in Section~\ref{sec:nu0z0c}.

However, estimates on $K_0$ (which are straightforward
to obtain by a more careful computation of \eqref{e:logLaplace}) indicate that
$K_0$ is not of order $g_0^3$, which is required to invoke Theorem~\ref{thm:flow-flow}.
Thus, an extension of the ideas in \cite{BBS-rg-flow} would be needed to study this
case.

\begin{rk}
Similarly, it is possible to re-cast the strictly self-avoiding walk as a
perturbation of weakly self-avoiding walk using a supersymmetric integral
representation obtained in \cite{BIS09}. The covariance of the form
$(-\Delta + \gamma^2)^{-1}$ in this case corresponds to a model of \emph{spread-out}
self-avoiding walk with exponentially decaying jump probabilities. Once again,
the initial error coordinate is not of order $O(g_0^3)$.
\end{rk}

%%%%%%%%%%%%%%%%%%%%%%%%%%%%%%%%%%%%%%%%%%%%%%%%%%%%%%%%%%%%%%%%%%%%%%%%%%%%%%%
%%%%%%%%%%%%%%%%%%%%%%%%%%%%%%%%%%%%%%%%%%%%%%%%%%%%%%%%%%%%%%%%%%%%%%%%%%%%%%%

\section{Other observable quantities}

Here we discuss some problems concerning the models studied in this thesis.

%%%%%%%%%%%%%%%%%%%%%%%%%%%%%%%%%%%%%%%%%%%%%%%%%%%%%%%%%%%%%%%%%%%%%%%%%%%%%%%

\subsection{The correlation length}

Our results concerning the finite-order correlation lengths $\xi_p$ are insufficient
for recovering the predicted behaviour of the \emph{true} correlation length $\xi$.
The estimate \eqref{e:Rab-bound} gives super-polynomial decay of the errors in the
approximation \eqref{e:Gab-to-sum-Rqj} of the two-point function, but this is not
sufficient for studying $\xi$, which would need exponentially decaying errors.
The current estimates follow from the covariance bounds \eqref{e:scaling-estimate}
on the decomposition of \cite{Baue13a}. Although it may not be possible to improve
the bounds for this particular decomposition, this should be possible for the
decomposition of \cite{BGM04} (see \cite[p.~445]{BGM04}).

However, even if this were possible, exponentially decaying errors would require
exponential decay of the weights $\ell_j$ above the mass scale, which would contradict
in a major way the central hypotheses \eqref{e:h-assumptions-IE} on these weights.
Thus, it seems new ideas would be needed to study the correlation length (note,
however, that the correlation length for the $1$-component $|\varphi|^4$ model
was successfully studied by a block-spin renormalisation group method in \cite{HT87}).

%%%%%%%%%%%%%%%%%%%%%%%%%%%%%%%%%%%%%%%%%%%%%%%%%%%%%%%%%%%%%%%%%%%%%%%%%%%%%%%

\subsection{Inversion of the Laplace transform}

One of the main motivations for studying the
susceptibility and finite-order correlation length for walks is the possibility
of recovering information about the growth of the partition function $c_T$ and
the mean-squared distance $\langle |X(T)|^2 \rangle$ as $T\to\infty$. In
particular, recalling the discussion in Section~\ref{sec:asymp}, one may
try to derive logarithmic corrections to the predicted scaling relations
\eqref{e:cT-asymp}--\eqref{e:XT-asymp} as a consequence of Theorem~\ref{thm:mr}(ii)--(iii).

This approach was successfully used in \cite{BI03c}, where the mean-squared displacement
of a hierarchical model of weakly self-avoiding walk is recovered by inversion of
the Laplace transform. This requires control over the two-point function in a
sector of the complex plane larger than what has been achieved here on the
Euclidean lattice.

% Other: tricritical model, WSAW-SA phase diagram, magnetization, other correlation
% functions, scaling limits



%%%% Bibliography %%%%
% Note: the bibliography must come before the appendices
\bibliographystyle{abbrv}
\bibliography{thesis}


%%%% Appendices %%%%

%% Use this to reset the appendix counter.  Note that the FoGS
%% requires that the word ``Appendices'' appear in the table of
%% contents either before each appendix lable or as a division
%% denoting the start of the appendices.  We take the latter option
%% here.  This is ensured by making the \texttt{appendicestoc} option
%% a default option to the UBC thesis class.


%%%% If you only have one appendix, please uncomment the following line.
% \renewcommand{\appendicesname}{Appendix}

% \appendix 					% Start appendices
% % First appendix (from saw-sa)
\chapter{Finite-volume approximation}
\label{sec:finvol}

In this appendix, we prove Proposition~\ref{prop:finvol}.

%%%%%%%%%%%%%%%%%%%%%%%%%%%%%%%%%%%%%%%%%%%%%%%%%%%%%%%%%%%%%%%%%%%%%%%%%%%%%%%
%%%%%%%%%%%%%%%%%%%%%%%%%%%%%%%%%%%%%%%%%%%%%%%%%%%%%%%%%%%%%%%%%%%%%%%%%%%%%%%

% \section{The critical point}

% \subsection{Fekete's subadditivity lemma}

% \subsection{The critical point of WSAW-SA}

%%%%%%%%%%%%%%%%%%%%%%%%%%%%%%%%%%%%%%%%%%%%%%%%%%%%%%%%%%%%%%%%%%%%%%%%%%%%%%%
%%%%%%%%%%%%%%%%%%%%%%%%%%%%%%%%%%%%%%%%%%%%%%%%%%%%%%%%%%%%%%%%%%%%%%%%%%%%%%%

\section{A monotonicity lemma}

Before proving the proposition, we require some preliminaries.
Let $P^n$ be the projection
of $\Zd$ onto the discrete torus of side $n$,
which we denote $\Z_n^d$.
Then $P^n$ has a natural action
on the path space $(\Zd)^{[0,\infty)}$. We let
$X^n = P^n(X)$ be the projection of $X$
and note that $X^n$ is a simple random walk on $\Z^d_n$.

We call $h = (h_x)_{x\in\Zd}$ a \emph{field of path functionals} if
$h_x : (\Zd)^{[0,\infty)} \to \R$ is a function on continuous-time paths
for each $x \in \Zd$;
a simple example is given by the local time functional.
We assume that the \emph{random} field $h(X) = (h_x(X))_{x\in\Zd}$
has finite support almost surely, i.e.,
with probability $1$, $h_x(X) = 0$ for all but finitely many $x$.
Denote by $h(X^n)$ the corresponding random field for $X^n$, i.e., for $x \in \Z_n^d$,
\begin{equation}
h_x(X^n) = \sum_{y\in\Zd} h_{x+ny}(X).
\end{equation}

Given a positive integer $k$, we define
$Q_k \subset \Z^d$ by $Q_k = \{y \in \Z^d : 0 \le y_i < k, \;   i=1,\ldots,d\}$.
Then, for integers $n,k \ge 1$,
\begin{equation}
\label{e:ffold1}
    \sum_{y \in Q_k} h_{x+ny}(X^{kn})
  = \sum_{y \in Q_k} \sum_{z\in\Zd} h_{x+ny+knz}(X)
  = \sum_{y\in\Zd} h_{x+ny}(X)
  = h_x(X^n),
\end{equation}
and it follows by summation over $x \in \Z^d_n$ that
\begin{equation}
\label{e:ffold2}
\sum_{x\in\Z^d_{kn}} h_x(X^{kn})
  =
\sum_{x\in\Z^d_n} h_x(X^n).
\end{equation}

\begin{lemma}
\label{lem:mono}
Let $n,k \ge 1$ and let $f$ and $g$ be nonnegative fields of path functionals
with finite support almost surely.
Then
\begin{equation}
\sum_{x\in\Z^d_{kn}} f_x(X^{kn}) g_x(X^{kn})
  \leq
\sum_{x\in\Z^d_n} f_x(X^n) g_x(X^n).
\end{equation}
\end{lemma}

\begin{proof}
By \eqref{e:ffold2} and \eqref{e:ffold1},
\begin{equation}
\sum_{x\in\Z_{kn}^d} f_x(X^{kn}) g_x(X^{kn})
  =
\sum_{x\in\Z_n^d}
\sum_{y \in Q_k}
  f_{x+ny}(X^{kn}) g_{x+ny}(X^{kn}).
\lbeq{mono}
\end{equation}
By nonnegativity and two more applications of \eqref{e:ffold1},
\begin{align}
\sum_{x\in\Z_n^d}
\sum_{y \in Q_k}
f_{x+ny}(X^{kn}) g_{x+ny}(X^{kn})
  &\le \sum_{x\in\Z_n^d}
      \left(\sum_{y \in Q_k} f_{x+ny}(X^{kn})\right)
      \left(\sum_{y \in Q_k} g_{x+ny}(X^{kn})\right) \nonumber \\
  &= \sum_{x\in\Z_n^d} f_x(X^n) g_x(X^n).
\end{align}
This completes the proof.
\end{proof}

%%%%%%%%%%%%%%%%%%%%%%%%%%%%%%%%%%%%%%%%%%%%%%%%%%%%%%%%%%%%%%%%%%%%%%%%%%%%%%%
%%%%%%%%%%%%%%%%%%%%%%%%%%%%%%%%%%%%%%%%%%%%%%%%%%%%%%%%%%%%%%%%%%%%%%%%%%%%%%%

\section{Convergence of the finite-volume approximation}

For $L \geq 2$ and $N \geq 1$
note that $\Lambda_N$ is the torus $\Z^d_n$ with $n=L^N$.
Thus, $X^{L^N}$ is the simple random walk on $\Lambda_N$.
For $F_T = F_T(X)$ any one of the functions $L_T^x,I_T,C_T$
of $X$ defined in \eqref{e:LTx-def}--\eqref{e:CTdef},
we write $F_{N,T} = F_T(X^{L^N})$. For instance, with $n=L^N$,
\begin{equation}
    L^x_{N,T} = \int_0^T \1_{X^{n}_t=\;x} \; dt,
    \quad I_{N,T} = \sum_{x \in \Lambda_N}(L_{N,T}^x)^2 .
\end{equation}
We apply Lemma~\ref{lem:mono} with $k = L$ and $n = L^N$ for three
choices of $f$, $g$:
\begin{alignat}{2}
\label{e:ILT-mon}
I_{N+1,T} &\leq I_{N,T}
	\quad
&&(f_x=g_x=L_T^x),
	\\
\label{e:CSA-mon}
C_{N+1,T} &\leq C_{N,T}
	\quad
&&(f_x=\textstyle{\sum_{e\in \Ucal}L_T^{x+e}},\; g_x=L_T^x),
	\\
\lbeq{nabL}
\sum_{x\in\Lambda_{N+1}} |\nabla^e L^x_{N+1,T}|^2
	&\leq
\sum_{x\in\Lambda_N} |\nabla^e L^x_{N,T}|^2
	\quad
&& (f_x = g_x = \left|\nabla^e L_T^x\right|).
\end{alignat}
Summation of \refeq{nabL} over unit vectors $e\in\Zd$ also gives
\begin{align}
\label{e:gradLT-mon}
\sum_{x\in\Lambda_{N+1}} |\nabla L^x_{N+1,T}|^2
  \leq
\sum_{x\in\Lambda_N} |\nabla L^x_{N,T}|^2.
\end{align}

% centred at the origin (approximately if $L$ is even),
% with $\Lambda_{N+1}$ paved by $L^d$ translates of $\Lambda_N$.
% We denote the expectation of $X^{L^N}$ started from $0$ by $E^{\Lambda_N}_0$
% and define
% \begin{align}
% \label{e:cN}
% c_{N,T}(x)
%     &=
% E^{\Lambda_N}_0 \left( e^{-U_{\gcc,\gamma,T}} \1_{X(T)=b} \right),
% 	\quad
% x \in \Lambda_N \\
% c_{N,T}
%     &=
% E^{\Lambda_N}_0 \left( e^{-U_{\gcc,\gamma,T}} \right).
% \end{align}
% The finite-volume two-point function and susceptibility
% are defined by
% \begin{align}
% G_x(\gcc,\gamma,\nu)
%     &= \int_0^\infty c_{N,T}(x) e^{-\nu T} \; dT, \\
% \chi_N(\gcc, \gamma, \nu)
%     &= \int_0^\infty c_{N,T} e^{-\nu T} \; dT
%     .
%     \label{e:chiNdef}
% \end{align}
Here and throughout this section we drop the parameter $n$
from the notation since $n = 0$.

\begin{prop}
\label{prop:finvol-re}
Let $d >0$, $\gcc >0$ and $\gamma < \gcc$. For all $\nu \in \R$,
\begin{equation}
\label{e:Givlc-re}
\lim_{N \to \infty}
G_x(\gcc,\gamma,\nu)
=
G_x(\gcc,\gamma,\nu)
\end{equation}
and
\begin{equation}
\label{e:chilim-re}
\lim_{N\to\infty}\chi_N(\gcc,\gamma,\nu)=   \chi(\gcc,\gamma,\nu).
\end{equation}
\end{prop}

\begin{proof}
We only prove the case $\gamma \ge 0$. The proof for $\gamma < 0$ can be found in
\cite{BSW-saw-sa}.

Fix $x \in \Zd$, and consider $N$ sufficiently large that $x$ can be identified
with points in $\Lambda_N$.
By \eqref{e:V2}, \eqref{e:ILT-mon} and \eqref{e:gradLT-mon}
% (if $0 \le \gamma < \gcc$),
% or by \eqref{e:V}, \eqref{e:ILT-mon} and \eqref{e:CSA-mon} (if $\gamma < 0$),
\begin{equation}
\label{e:ctmon}
c_{N,T}(x) \leq c_{N+1,T}(x).
\end{equation}
Thus, \eqref{e:Givlc-re} follows by monotone convergence, once we show that
\begin{equation}
\lim_{N\to\infty} c_{N,T}(x) = c_T(x).
\end{equation}

% This follows as in \cite[(2.8)]{BBS-saw4}.
To show this, recall that we are identifying the vertices of $\Lambda_N$
with nested subsets of $\Zd$. We can thus define $\partial \Lambda_N$ to be
the inner vertex boundary of $\Lambda_N$. We set
\begin{align}
c_{N,T}^*(x)
  &=
E^{\Lambda_N}_0
\left(
  e^{-U_{\gcc,\gamma,T}} \1_{X(T)=b} \1_{\{X([0, T]) \cap \partial \Lambda_N \neq \varnothing\}}
\right) \\
c_T^*(x)
  &=
E_0
\left(
  e^{-U_{\gcc,\gamma,T}} \1_{X(T)=b} \1_{\{X([0, T]) \cap \partial \Lambda_N \neq \varnothing\}}
\right).
\end{align}
Since walks which do not reach $\partial \Lambda_N$ make equal contributions to both
$c_T(x)$ and $c_{N,T}(x)$,
we have
\begin{equation}
c_T(x) - c_T^*(x) = c_{N,T}(x) - c_{N,T}^*(x).
\end{equation}
Thus,
\begin{align}
|c_T(x) - c_{N,T}(x)|
= |c_T^*(x) - c_{N,T}^*(x)|
\leq c_T^*(x) + c_{N,T}^*(x).
\end{align}
Let $P^{\Lambda_N}_0$ and $P_0$ be the measures
associated with $E^{\Lambda_N}_0$ and $E_0$, respectively.
With $Y_t$ a rate-$2d$ Poisson process with measure ${\sf P}$,
\begin{align}
 c_T^*(x) + c_{N,T}^*(x)
  &\leq P_0 (X([0, T]) \cap \partial\Lambda_N \neq \varnothing)
    + P^{\Lambda_N}_0 (X([0, T]) \cap \partial\Lambda_N \neq \varnothing) \nonumber \\
  &\leq 2 {\sf P} (Y_T \geq \diam{\Lambda_N}) \to 0
\end{align}
as $N\to\infty$.
This completes the proof of \eqref{e:Givlc-re}.

Finally, by monotone convergence of $G_N$ to $G$,
for $\nu \in \R$,
\begin{equation}
\lim_{N\to\infty} \chi_N(g, \gamma, \nu)
    = \sum_{x\in\Zd} \lim_{N\to\infty} G_{x,N}(\gcc,\gamma,\nu) \1_{x\in\Lambda_N}
    = \chi(g, \gamma, \nu),
\end{equation}
which proves \eqref{e:chilim-re}.
\end{proof}
% \chapter{Moments of the free Green function}
\label{app:free-moments}
% from clp

In this appendix we prove Proposition~\ref{prop:Gab-free-moment-estimate}.

%%%%%%%%%%%%%%%%%%%%%%%%%%%%%%%%%%%%%%%%%%%%%%%%%%%%%%%%%%%%%%%%%%%%%%%%%%%%%%%
%%%%%%%%%%%%%%%%%%%%%%%%%%%%%%%%%%%%%%%%%%%%%%%%%%%%%%%%%%%%%%%%%%%%%%%%%%%%%%%

\section{Main result}

The following is a re-statement of Proposition~\ref{prop:Gab-free-moment-estimate}.
Since we are only dealing with the free Green function, we set
\begin{equation}
G_x(m^2) = G_x(0, 0, m^2).
\end{equation}

\begin{prop}\label{prop:Gab-free-moment-estimate-bis}
Let ${\sf c}_p$ be the constant defined by \eqref{e:cpdef}.
For all dimensions $d>2$ and all $p>0$,
as $m^2 \downarrow 0$,
\begin{equation}
\label{e:Gab-free-moment-estimate-bis}
\sum_{x\in\Zd} |x|^p G_{x}(0, m^2)
=
{\sf c}_p^p m^{-(p + 2)} (1 + O(m)).
\end{equation}
In particular, $\xi_p(0,\varepsilon) = {\sf c}_p \varepsilon^{-1/2}
(1+O(\varepsilon^{1/2}))$ as $\varepsilon \downarrow 0$.
\end{prop}

The last sentence in the the proposition follows immediately from
\refeq{Gab-free-moment-estimate-bis} and the fact that $\chi(0,m^2)=m^{-2}$
(recall \eqref{e:chi-free}), so it suffices to prove \refeq{Gab-free-moment-estimate-bis}.

The case $p = 2$ of \refeq{Gab-free-moment-estimate-bis}
can be obtained easily from the identity
\begin{equation}
\sum_{x\in\Zd} |x|^2 G_x(m^2) = -\Delta_\Rd \hat{G}(0),
\end{equation}
where $\hat G$ is the Fourier transform of $G$.
Higher even moments could in principle
be computed by further differentiating $\hat G$.
We adopt a different approach
for general $p>0$,
based on the finite range decomposition of $(-\Delta_{\Zd}+m^2)^{-1}$
given in \cite{BGM04,Baue13a}.
This finite range decomposition also provides the basis for the renormalisation group method.

%%%%%%%%%%%%%%%%%%%%%%%%%%%%%%%%%%%%%%%%%%%%%%%%%%%%%%%%%%%%%%%%%%%%%%%%%%%%%%%
%%%%%%%%%%%%%%%%%%%%%%%%%%%%%%%%%%%%%%%%%%%%%%%%%%%%%%%%%%%%%%%%%%%%%%%%%%%%%%%

\section{Riemann sum approximation}

We will make use of the following elementary result.

\begin{lemma}
Let $f : \Rd \to \R$ be a Lipschitz function with support in a box of side $t$
centred at the origin.
Then there is a constant $C$ such that for any $\epsilon > 0$,
\begin{equation}
\left|\int_\Rd f(x) \; dx - \epsilon^d \sum_{x\in\Zd} f(\epsilon x)\right|
	\le
C (t \epsilon)^d.
\end{equation}
\end{lemma}

\begin{proof}
For any $x\in\Zd$, let $S_x(\epsilon)$ denote the square of side $\epsilon$
centred at $\epsilon x\in\Rd$. Then
\begin{equation}
\int_{\Rd} f(x) \; dx
	=
\sum_{x\in\Zd} \int_{S_x(\epsilon)} f(y) \; dy.
\end{equation}
By the mean value theorem, there exists $y_x = y_x(\epsilon) \in S_x(\epsilon)$ such that
\begin{equation}
\int_{S_x(\epsilon)} f(y) \; dy
	=
\epsilon^d f(y_x).
\end{equation}
Thus,
\begin{equation}
\left|\int_\Rd f(x) \; dx - \epsilon^d \sum_{x\in\Zd} f(\epsilon x)\right|
	\le
\epsilon^d \sum_{x\in\Zd} |f(y_x) - f(\epsilon x)|.
\end{equation}
By the Lipschitz condition on $f$, each summand on the right-hand side is
$O(\epsilon)$. By the support assumption on $f$, there are at most
$O(t^d/\epsilon)$ such summands and the result follows.
\end{proof}

%%%%%%%%%%%%%%%%%%%%%%%%%%%%%%%%%%%%%%%%%%%%%%%%%%%%%%%%%%%%%%%%%%%%%%%%%%%%%%%
%%%%%%%%%%%%%%%%%%%%%%%%%%%%%%%%%%%%%%%%%%%%%%%%%%%%%%%%%%%%%%%%%%%%%%%%%%%%%%%

\section{Covariance decomposition}

The finite-range decomposition of the finite-volume covariance discussed in
Section~\ref{sec:prog} is derived from a decomposition of the infinite-volume
covariance (whose construction is the main result of \cite{Baue13a}) of the form
\begin{equation}
    G_x(m^2) = \sum_{j=1}^\infty C_{j;x}(m^2).
\end{equation}
Recall that the finite-range property refers to the fact that $C_{j;x}(m^2) = 0$ if
$|x| \ge \frac 12 L^j$,
where $L>1$ is fixed arbitrarily.
 We review some properties of this decomposition, from
\cite{BBS-rg-pt,Baue13a}, before
proving Proposition~\ref{prop:Gab-free-moment-estimate-bis}.
The positive-definiteness of the finite range
decomposition is not needed here, and $L$ need not be large.

The terms $C_{j;x}(m^2)$ are defined in \cite[Section~\ref{pt-sec:Cdecomp}]{BBS-rg-pt} by
\begin{equation} \label{e:Cdef}
  C_{j;x}(m^2) = \left\{\begin{aligned}
      &\int_{0}^{\frac{1}{2} L}  \phi_{t}^*(x; m^2) \; \frac{dt}{t}
      &\quad& (j=1)
      \\
      &\int_{\frac{1}{2} L^{j-1}}^{\frac{1}{2} L^{j}}  \phi_{t}^*(x; m^2) \; \frac{dt}{t}
      && (j\ge 2)
    \end{aligned}\right.
\end{equation}
(in \cite{BBS-rg-pt}, the notation $C_{j;0,x}$ and $\phi^*_t(0, x; m^2)$ was used instead).
Here, $\phi_t^*$ is a function of $x \in \Rd$ and $m^2 > 0$ given in
\cite[Example 1.1]{Baue13a}. It satisfies the finite range property that
$\phi_t^*(x; m^2) = 0$ for $|x| > t$.
It was also shown in \cite{Baue13a} that there exists a function $\phi_t$
satisfying the same finite range property but giving a decomposition of the
\emph{continuum} Green function:
\begin{align}
\label{e:frd-cont-int}
    (-\Delta_{\Rd} + m^2)^{-1}_{0x}
    &=
    \int_0^\infty \phi_t(x; m^2) \frac{dt}{t} .
\end{align}
Moreover, by \cite[(1.37)]{Baue13a}, for $|x| \leq t$,
\begin{equation}
\label{e:frd-Zd-Rd}
\phi^*_t(x; m^2) = \phi_t(x; m^2) + O(t^{-(d-1)} (1 + m^2 t^2)^{-k}).
\end{equation}
This allows us to approximate the discrete Green function by the continuum one, for which the moments are easily computed.
We have set the constant $c$ present in \cite{Baue13a} equal to $1$, which we can do by rescaling $\phi_t^*$.

As $t$ approaches $0$, the error bound in \eqref{e:frd-Zd-Rd} degenerates.
However, to estimate \eqref{prop:Gab-free-moment-estimate-bis}, it suffices to
restrict to $x \neq 0$.
Then, since $x \in \Z^d$, the finite range property permits replacement of the lower bound
in the range of integration for $j=1$ in \eqref{e:Cdef} by $\frac12$, and the contribution
due to $j=1$ can be estimated in the same way as the terms $j\geq 2$.

Also, by \cite[(1.34)]{Baue13a}, for any $k$ there is a constant $C_k$ such that
\begin{equation}
\label{e:frd-deriv-est}
|D_x \phi_t(x; m^2)| \leq C_k t^{-(d - 1)} (1 + m^2 t^2)^{-k}.
\end{equation}
We fix a choice of $k$ which obeys $k > \frac 12 (p+1)$ and use only this choice.
By \cite[(1.38)]{Baue13a}, there exists a function $\bar\phi$ such that
\begin{equation}
\label{e:frd-scaling}
\phi_t(x; m^2) = t^{-(d - 2)} \bar\phi\left(\frac{x}{t}; m^2 t^2\right).
\end{equation}

%%%%%%%%%%%%%%%%%%%%%%%%%%%%%%%%%%%%%%%%%%%%%%%%%%%%%%%%%%%%%%%%%%%%%%%%%%%%%%%
%%%%%%%%%%%%%%%%%%%%%%%%%%%%%%%%%%%%%%%%%%%%%%%%%%%%%%%%%%%%%%%%%%%%%%%%%%%%%%%

\section{Proof of main result}

\begin{proof}[Proof of Proposition~\ref{prop:Gab-free-moment-estimate}]

We begin by writing
\begin{align}
\sum_{x\in\Zd} |x|^p G_x(m^2)
    &=
    \sum_{x\in\Zd} |x|^p
    \sum_{j=1}^\infty C_{j;x}(m^2)
    =
M(m^2)
+
E(m^2)
,
\end{align}
where the main and error terms are respectively
\begin{align}
M(m^2) &=
\sum_{x\in\Zd} |x|^p \sum_{j=1}^\infty
\int_{\frac 12 L^{j-1}}^{\frac 12 L^j} \phi_t(x; m^2) \;\frac{dt}{t},
\\
E(m^2) &= \sum_{x\in\Zd} |x|^p \sum_{j=1}^\infty \left(C_{j;x} - \int_{\frac{1}{2} L^{j-1}}^{\frac{1}{2} L^j} \phi_t(x; m^2) \frac{dt}{t}\right).
\end{align}

We first compute the main term $M$. By \eqref{e:frd-scaling},
\begin{equation}
\phi_t(x; m^2) = m^{d-2} \phi_{mt}(mx; 1).
\end{equation}
Therefore, by Riemann sum approximation,
\begin{align}
\sum_{x\in\Zd} &|x|^p \int_{\frac{1}{2} L^{j-1}}^{\frac{1}{2} L^j} \phi_t(x; m^2) \; \frac{dt}{t} \\
  &= m^{-(p+2)} m^d \sum_{x\in\Zd} |mx|^p \int_{\frac{1}{2} L^{j-1}}^{\frac{1}{2} L^j} \phi_{mt}(mx; 1) \; \frac{dt}{t}
  \\ \nonumber
  &= m^{-(p + 2)} \int_\Rd |x|^p \int_{\frac{1}{2} L^{j-1}}^{\frac{1}{2} L^j} \phi_{mt}(x; 1) \; \frac{dt}{t}
    + O(L^{(p + 1) j}
    L^{-2k(j-j_m)_+})
    ,
\end{align}
where the error estimate follows from \eqref{e:frd-deriv-est} and \eqref{e:mass-decay}.
Summation over $j$ gives
\begin{align}
M(m^2)
&= {\sf c}_p^p m^{-(p + 2)} + O(m^{-(p + 1)}),
\end{align}
where we used
\eqref{e:frd-cont-int} for the first term, and we used $2k>p+1$ and
Lemma~\ref{lem:mass-scale-sum} for the second term.

For the error term,
it follows from
\eqref{e:Cdef}, \eqref{e:frd-Zd-Rd},
and the observation that the lower bound in the range of integration for the $j=1$ term in \eqref{e:Cdef}
can be changed to $\frac12$ that
\begin{equation}
\label{e:frd-integral-approx}
C_{j;x}
=
\int_{\frac{1}{2} L^{j-1}}^{\frac{1}{2} L^j} \phi_t(x; m^2) \frac{dt}{t}
+
O(L^{-j (d - 1)} (1 + m^2 L^{2j})^{-k})
\1_{|x|\leq L^j}.
\end{equation}
Therefore, again using \eqref{e:mass-decay}, we have
\begin{align}
E(m^2)
&= \sum_{j=1}^\infty \sum_{|x|\leq L^j} |x|^p O(L^{-j (d - 1)}
L^{-2k(j-j_m)_+})
\\
&= \sum_{j=1}^\infty O(L^{(p+1)j}L^{-2k(j-j_m)_+})
\label{e:free-error-estimate}
.
\end{align}
With $2k>p+1$ and Lemma~\ref{lem:mass-scale-sum}, this gives
$E(m^2) = O(m^{-(p+1)})$,
and the proof is complete.
\end{proof}
% \chapter{An implicit function theorem}
\label{sec:IFT}

In this appendix, we prove Proposition~\ref{prop:IFT}.

\section{Implicit function theorem with a parameter}

We make use of \cite[Chapter 4, Theorem~9.3]{LS14},
which is a version of the implicit function theorem that allows
for a continuous, rather than differentiable, parameter.
While the precise statement of \cite[Chapter 4, Theorem~9.3]{LS14}
takes this parameter from an open subset of a Banach space,
by \cite[Chapter 4, Theorem~9.2]{LS14}, the
parameter can in fact be taken from an arbitrary metric space.
With this minor change, we restate \cite[Chapter 4, Theorem~9.3]{LS14}
as the following proposition.

\begin{prop}
\label{prop:LS-IFT}
Let $A$ be a metric space, let $W,X$ be Banach spaces,
and let $B \subset W$ be an open subset.
Let $F : A \times B \to X$ be continuous,
and suppose that $F$ is $C^1$ in its second argument.
Let $(\alpha, \beta) \in A \times B$ be a point such that
$F(\alpha, \beta) = 0$ and $D_2 F(\alpha, \beta)^{-1}$ exists.
Then there are open balls $M \ni \alpha$ and $N \ni \beta$
and a unique continuous mapping $f : M \to N$ such that $F(\xi, f(\xi)) = 0$
for all $\xi \in M$.
\end{prop}

We also use the following lemma, which is a small modification of
\cite[Chapter 3, Theorem~11.1]{LS14}. In particular, it considers
functions that may only be left- or right-differentiable.

\begin{lemma}
\label{lem:IFT-C1}
Let $F$ be a mapping as in the previous proposition
with $A \subset \R^{m_1} \times \R^{m_2}$.
In addition, suppose that $F$ is left-differentiable (respectively, right-differentiable)
in $\alpha_2$ at $(\alpha, \beta)$, with $\alpha = (\alpha_1, \alpha_2)$.
If $f$ is a continuous mapping defined in a neighbourhood of
$\alpha$, such that $F(\xi, f(\xi)) = 0$,
then $f$ is left-differentiable (respectively, right-differentiable) in $\alpha_2$ at $\alpha$.
\end{lemma}

\section{Main result}

The above results lead to the following proposition, which we apply
in the proofs of Propositions~\ref{prop:nuzhat} and \ref{prop:changevariables1}.
Recall that $D(\delta, r)$ is defined in \refeq{Ddef}.

\begin{prop}
\label{prop:IFT-re}
Let $\delta > 0$, and let $r_1, r_2$ be continuous positive-definite functions on $[0, \delta]$.
Set
\begin{equation}
    D(\delta, r_1, r_2)
    =
    \{ (w, x, y, z) \in D(\delta, r_1) \times \R^n : |z| \leq r_2(x) \},
\end{equation}
and let $F$ be a continuous function on $D(\delta, r_1, r_2)$ that is $C^1$ in $(x, z)$.
Suppose that for all $(\bar w, \bar x) \in [0, \delta]^2$ there exists $\bar z$
such that both $F(\bar w, \bar x, 0, \bar z) = 0$
and $D_Y F(\bar w, \bar x, 0, \bar z)$ is invertible.
Then there is a continuous positive-definite function $r$ on $[0, \delta]$ and
a continuous map $f : D(\delta, r) \to \R^n$
that is $C^1$ in $x$
and such that $F(w, x, y, f(w, x, y)) = 0$
for all $(w, x, y) \in D(\delta, r)$.
Moreover, if $F$ is left-differentiable
(respectively, right-differentiable) in $y$ at some point $(w, x, y, z)$,
then $f$ is left-differentiable (respectively, right-differentiable) at $(w, x, y)$.
\end{prop}

\begin{proof}
Take any $(\bar w, \bar x) \in [0, \delta] \times (0, \delta]$
and let $R(\bar w, \bar x)$ be the maximal radius $s$ such that
for all $(w, x, y) \in B(\bar w, \bar x, 0; s)$ there exists $z$
such that both $F(w, x, y, z) = 0$ and $D_Z F(w, x, y, z)$ is
invertible. By continuity of $(D_Z F(w, x, y, z))^{-1}$ near
$(\bar w, \bar x, 0, \bar z)$, and by Proposition~\ref{prop:LS-IFT}
(applied to the restriction of $F$ to $A \times B$, for some
$A \ni (\bar w, \bar x, 0)$ and an open set $B \ni \bar z$),
we have $R(\bar w, \bar x) > 0$ and there is a continuous function
\begin{equation}
f_{\bar w,\bar x} : B(\bar w, \bar x, 0; R(\bar w, \bar x)) \to \R^n
\end{equation}
such that $F(w, x, y, f_{\bar w,\bar x}(w, x, y)) = 0$
for all $(w, x, y) \in B(\bar w, \bar x, 0; R(\bar w, \bar x))$.
Moreover, the unique solution to $F(w, x, y, z) = 0$
is given by $z = f_{\bar w,\bar x}(w, x, y)$ for all such $(w, x, y)$.
By an application of Lemma~\ref{lem:IFT-C1}
(with $\alpha_1 = (w, x), \alpha_2 = y$),
we see that $f_{\bar w, \bar x}$ is
left- or right-differentiable in $y$ wherever $F$ is.
By another application of Lemma~\ref{lem:IFT-C1} (with $\alpha_1 = (w, y), \alpha_2 = x$),
we see that $f_{\bar w, \bar x}$ is $C^1$ in $x$.

Set $R(\bar w, 0) = 0$ for all $\bar w \in [0, \delta]$, and
let
\begin{equation}
D_f = \bigcup_{(\bar w,\bar x)\in [0, \delta]^2} B(\bar w, \bar x, 0; R(\bar w, \bar x)).
\end{equation}
We define $f(w, 0, 0) = 0$ and, for $x > 0$,
\begin{equation}
f(w, x, y) = f_{\bar w,\bar x}(w, x, y)
  \quad\text{for}\quad
(w, x, y) \in B(\bar w, \bar x, 0; R(\bar w, \bar x)).
\end{equation}
By uniqueness, this function is well-defined.
Continuity of $f$ at $(w, 0, 0)$
follows from the fact that $|f(w, x, y)| \le r_2(x)$.
The remaining desired regularity properties of $f$
follow from those of the $f_{\bar w,\bar x}$.
It remains to show that $D(\delta,r) \subset D_f$
for some continuous positive-definite function $r$ on $[0, \delta]$.

First, let us show that $R$ is continuous on $[0, \delta]^2$.
Let $\bar x > 0$ and fix $0 < \epsilon < R(\bar w, \bar x)$.
Then for any $(\bar w', \bar x') \in [0,\delta] \times (0, \delta]$ such that
$|(\bar w, \bar x) - (\bar w', \bar x')| < \epsilon$,
we have $B(\bar w', \bar x', 0; R(\bar w, \bar x) - \epsilon) \subset B(\bar w, \bar x, 0; R(\bar w, \bar x))$
by maximality of $R$.
It follows that $R(\bar w', \bar x') \geq R(\bar w, \bar x) - \epsilon$.
By a similar argument, $R(\bar w', \bar x') \leq R(\bar w, \bar x) + \epsilon$,
so $|R(\bar w, \bar x) - R(\bar w', \bar x')| \leq \epsilon$.
Thus, $R$ is continuous on $[0, \delta] \times (0, \delta]$.
Continuity at $\bar x = 0$ follows from the fact that $R(\bar w, \bar x) \le r_1(\bar x)$
uniformly in $\bar w$.

For $\bar x \in [0,\delta]$, let
\begin{equation}
r(\bar x) = \inf (R(\bar w, \bar x) : \bar w \in [0, \delta]).
\end{equation}
Since $R(\cdot, \bar x)$ is continuous, $r(\bar x) > 0$
for $\bar x > 0$. Moreover, $0 \le r(0) \le r_1(0) = 0$, so $r$ is positive-definite.
Continuity of $r$ follows from joint continuity of $R$.
For any $(w, x, y) \in D(\delta, r)$ (with this choice of $r$),
\begin{equation}
|(w, x, y) - (w, x, 0)| = |y| < r(x) \leq R(w, x),
\end{equation}
so $(w, x, y) \in B(w, x, 0; R(w, x))$.
We conclude that $D(\delta, r) \subset D_f$.
\end{proof}

\end{document}
