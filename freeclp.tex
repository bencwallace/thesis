\chapter{Moments of the free Green function}
\label{app:free-moments}
% from clp

In this appendix we prove Proposition~\ref{prop:Gab-free-moment-estimate}.

%%%%%%%%%%%%%%%%%%%%%%%%%%%%%%%%%%%%%%%%%%%%%%%%%%%%%%%%%%%%%%%%%%%%%%%%%%%%%%%
%%%%%%%%%%%%%%%%%%%%%%%%%%%%%%%%%%%%%%%%%%%%%%%%%%%%%%%%%%%%%%%%%%%%%%%%%%%%%%%

\section{Main result}

The following is a re-statement of Proposition~\ref{prop:Gab-free-moment-estimate}.
Since we are only dealing with the free Green function, we set
\begin{equation}
G_x(m^2) = G_x(0, 0, m^2).
\end{equation}

\begin{prop}\label{prop:Gab-free-moment-estimate-bis}
Let ${\sf c}_p$ be the constant defined by \eqref{e:cpdef}.
For all dimensions $d>2$ and all $p>0$,
as $m^2 \downarrow 0$,
\begin{equation}
\label{e:Gab-free-moment-estimate-bis}
\sum_{x\in\Zd} |x|^p G_{x}(0, m^2)
=
{\sf c}_p^p m^{-(p + 2)} (1 + O(m)).
\end{equation}
In particular, $\xi_p(0,\varepsilon) = {\sf c}_p \varepsilon^{-1/2}
(1+O(\varepsilon^{1/2}))$ as $\varepsilon \downarrow 0$.
\end{prop}

The last sentence in the the proposition follows immediately from
\refeq{Gab-free-moment-estimate-bis} and the fact that $\chi(0,m^2)=m^{-2}$
(recall \eqref{e:chi-free}), so it suffices to prove \refeq{Gab-free-moment-estimate-bis}.

The case $p = 2$ of \refeq{Gab-free-moment-estimate-bis}
can be obtained easily from the identity
\begin{equation}
\sum_{x\in\Zd} |x|^2 G_x(m^2) = -\Delta_\Rd \hat{G}(0),
\end{equation}
where $\hat G$ is the Fourier transform of $G$.
Higher even moments could in principle
be computed by further differentiating $\hat G$.
We adopt a different approach
for general $p>0$,
based on the finite range decomposition of $(-\Delta_{\Zd}+m^2)^{-1}$
given in \cite{BGM04,Baue13a}.
This finite range decomposition also provides the basis for the renormalisation group method.

%%%%%%%%%%%%%%%%%%%%%%%%%%%%%%%%%%%%%%%%%%%%%%%%%%%%%%%%%%%%%%%%%%%%%%%%%%%%%%%
%%%%%%%%%%%%%%%%%%%%%%%%%%%%%%%%%%%%%%%%%%%%%%%%%%%%%%%%%%%%%%%%%%%%%%%%%%%%%%%

\section{Riemann sum approximation}

We will make use of the following elementary result.

\begin{lemma}
Let $f : \Rd \to \R$ be a Lipschitz function with support in a box of side $t$
centered at the origin.
Then there is a constant $C$ such that for any $\epsilon > 0$,
\begin{equation}
\left|\int_\Rd f(x) \; dx - \epsilon^d \sum_{x\in\Zd} f(\epsilon x)\right|
	\le
C (t \epsilon)^d.
\end{equation}
\end{lemma}

\begin{proof}
For any $x\in\Zd$, let $S_x(\epsilon)$ denote the square of side $\epsilon$
centered at $\epsilon x\in\Rd$. Then
\begin{equation}
\int_{\Rd} f(x) \; dx
	=
\sum_{x\in\Zd} \int_{S_x(\epsilon)} f(y) \; dy.
\end{equation}
By the mean value theorem, there exists $y_x = y_x(\epsilon) \in S_x(\epsilon)$ such that
\begin{equation}
\int_{S_x(\epsilon)} f(y) \; dy
	=
\epsilon^d f(y_x).
\end{equation}
Thus,
\begin{equation}
\left|\int_\Rd f(x) \; dx - \epsilon^d \sum_{x\in\Zd} f(\epsilon x)\right|
	\le
\epsilon^d \sum_{x\in\Zd} |f(y_x) - f(\epsilon x)|.
\end{equation}
By the Lipschitz condition on $f$, each summand on the right-hand side is
$O(\epsilon)$. By the support assumption on $f$, there are at most
$O(t^d/\epsilon)$ such summands and the result follows.
\end{proof}

%%%%%%%%%%%%%%%%%%%%%%%%%%%%%%%%%%%%%%%%%%%%%%%%%%%%%%%%%%%%%%%%%%%%%%%%%%%%%%%
%%%%%%%%%%%%%%%%%%%%%%%%%%%%%%%%%%%%%%%%%%%%%%%%%%%%%%%%%%%%%%%%%%%%%%%%%%%%%%%

\section{Covariance decomposition}

The finite-range decomposition of the finite-volume covariance discussed in
Section~\ref{sec:prog} is derived from a decomposition of the infinite-volume
covariance (whose construction is the main result of \cite{Baue13a}) of the form
\begin{equation}
    G_x(m^2) = \sum_{j=1}^\infty C_{j;x}(m^2).
\end{equation}
Recall that the finite-range property refers to the fact that $C_{j;x}(m^2) = 0$ if
$|x| \ge \frac 12 L^j$,
where $L>1$ is fixed arbitrarily.
 We review some properties of this decomposition, from
\cite{BBS-rg-pt,Baue13a}, before
proving Proposition~\ref{prop:Gab-free-moment-estimate-bis}.
The positive-definiteness of the finite range
decomposition is not needed here, and $L$ need not be large.

The terms $C_{j;x}(m^2)$ are defined in \cite[Section~\ref{pt-sec:Cdecomp}]{BBS-rg-pt} by
\begin{equation} \label{e:Cdef}
  C_{j;x}(m^2) = \left\{\begin{aligned}
      &\int_{0}^{\frac{1}{2} L}  \phi_{t}^*(x; m^2) \; \frac{dt}{t}
      &\quad& (j=1)
      \\
      &\int_{\frac{1}{2} L^{j-1}}^{\frac{1}{2} L^{j}}  \phi_{t}^*(x; m^2) \; \frac{dt}{t}
      && (j\ge 2)
    \end{aligned}\right.
\end{equation}
(in \cite{BBS-rg-pt}, the notation $C_{j;0,x}$ and $\phi^*_t(0, x; m^2)$ was used instead).
Here, $\phi_t^*$ is a function of $x \in \Rd$ and $m^2 > 0$ given in
\cite[Example 1.1]{Baue13a}. It satisfies the finite range property that
$\phi_t^*(x; m^2) = 0$ for $|x| > t$.
It was also shown in \cite{Baue13a} that there exists a function $\phi_t$
satisfying the same finite range property but giving a decomposition of the
\emph{continuum} Green function:
\begin{align}
\label{e:frd-cont-int}
    (-\Delta_{\Rd} + m^2)^{-1}_{0x}
    &=
    \int_0^\infty \phi_t(x; m^2) \frac{dt}{t} .
\end{align}
Moreover, by \cite[(1.37)]{Baue13a}, for $|x| \leq t$,
\begin{equation}
\label{e:frd-Zd-Rd}
\phi^*_t(x; m^2) = \phi_t(x; m^2) + O(t^{-(d-1)} (1 + m^2 t^2)^{-k}).
\end{equation}
This allows us to approximate the discrete Green function by the continuum one, for which the moments are easily computed.
We have set the constant $c$ present in \cite{Baue13a} equal to $1$, which we can do by rescaling $\phi_t^*$.

As $t$ approaches $0$, the error bound in \eqref{e:frd-Zd-Rd} degenerates.
However, to estimate \eqref{prop:Gab-free-moment-estimate-bis}, it suffices to
restrict to $x \neq 0$.
Then, since $x \in \Z^d$, the finite range property permits replacement of the lower bound
in the range of integration for $j=1$ in \eqref{e:Cdef} by $\frac12$, and the contribution
due to $j=1$ can be estimated in the same way as the terms $j\geq 2$.

Also, by \cite[(1.34)]{Baue13a}, for any $k$ there is a constant $C_k$ such that
\begin{equation}
\label{e:frd-deriv-est}
|D_x \phi_t(x; m^2)| \leq C_k t^{-(d - 1)} (1 + m^2 t^2)^{-k}.
\end{equation}
We fix a choice of $k$ which obeys $k > \frac 12 (p+1)$ and use only this choice.
By \cite[(1.38)]{Baue13a}, there exists a function $\bar\phi$ such that
\begin{equation}
\label{e:frd-scaling}
\phi_t(x; m^2) = t^{-(d - 2)} \bar\phi\left(\frac{x}{t}; m^2 t^2\right).
\end{equation}

%%%%%%%%%%%%%%%%%%%%%%%%%%%%%%%%%%%%%%%%%%%%%%%%%%%%%%%%%%%%%%%%%%%%%%%%%%%%%%%
%%%%%%%%%%%%%%%%%%%%%%%%%%%%%%%%%%%%%%%%%%%%%%%%%%%%%%%%%%%%%%%%%%%%%%%%%%%%%%%

\section{Proof of main result}

\begin{proof}[Proof of Proposition~\ref{prop:Gab-free-moment-estimate}]

We begin by writing
\begin{align}
\sum_{x\in\Zd} |x|^p G_x(m^2)
    &=
    \sum_{x\in\Zd} |x|^p
    \sum_{j=1}^\infty C_{j;x}(m^2)
    =
M(m^2)
+
E(m^2)
,
\end{align}
where the main and error terms are respectively
\begin{align}
M(m^2) &=
\sum_{x\in\Zd} |x|^p \sum_{j=1}^\infty
\int_{\frac 12 L^{j-1}}^{\frac 12 L^j} \phi_t(x; m^2) \;\frac{dt}{t},
\\
E(m^2) &= \sum_{x\in\Zd} |x|^p \sum_{j=1}^\infty \left(C_{j;x} - \int_{\frac{1}{2} L^{j-1}}^{\frac{1}{2} L^j} \phi_t(x; m^2) \frac{dt}{t}\right).
\end{align}

We first compute the main term $M$. By \eqref{e:frd-scaling},
\begin{equation}
\phi_t(x; m^2) = m^{d-2} \phi_{mt}(mx; 1).
\end{equation}
Therefore, by Riemann sum approximation,
\begin{align}
\sum_{x\in\Zd} &|x|^p \int_{\frac{1}{2} L^{j-1}}^{\frac{1}{2} L^j} \phi_t(x; m^2) \; \frac{dt}{t} \\
  &= m^{-(p+2)} m^d \sum_{x\in\Zd} |mx|^p \int_{\frac{1}{2} L^{j-1}}^{\frac{1}{2} L^j} \phi_{mt}(mx; 1) \; \frac{dt}{t}
  \\ \nonumber
  &= m^{-(p + 2)} \int_\Rd |x|^p \int_{\frac{1}{2} L^{j-1}}^{\frac{1}{2} L^j} \phi_{mt}(x; 1) \; \frac{dt}{t}
    + O(L^{(p + 1) j}
    L^{-2k(j-j_m)_+})
    ,
\end{align}
where the error estimate follows from \eqref{e:frd-deriv-est} and \eqref{e:mass-decay}.
Summation over $j$ gives
\begin{align}
M(m^2)
&= {\sf c}_p^p m^{-(p + 2)} + O(m^{-(p + 1)}),
\end{align}
where we used
\eqref{e:frd-cont-int} for the first term, and we used $2k>p+1$ and
Lemma~\ref{lem:mass-scale-sum} for the second term.

For the error term,
it follows from
\eqref{e:Cdef}, \eqref{e:frd-Zd-Rd},
and the observation that the lower bound in the range of integration for the $j=1$ term in \eqref{e:Cdef}
can be changed to $\frac12$ that
\begin{equation}
\label{e:frd-integral-approx}
C_{j;x}
=
\int_{\frac{1}{2} L^{j-1}}^{\frac{1}{2} L^j} \phi_t(x; m^2) \frac{dt}{t}
+
O(L^{-j (d - 1)} (1 + m^2 L^{2j})^{-k})
\1_{|x|\leq L^j}.
\end{equation}
Therefore, again using \eqref{e:mass-decay}, we have
\begin{align}
E(m^2)
&= \sum_{j=1}^\infty \sum_{|x|\leq L^j} |x|^p O(L^{-j (d - 1)}
L^{-2k(j-j_m)_+})
\\
&= \sum_{j=1}^\infty O(L^{(p+1)j}L^{-2k(j-j_m)_+})
\label{e:free-error-estimate}
.
\end{align}
With $2k>p+1$ and Lemma~\ref{lem:mass-scale-sum}, this gives
$E(m^2) = O(m^{-(p+1)})$,
and the proof is complete.
\end{proof}