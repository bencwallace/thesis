\section{Relations between models}
\label{sec:spin-walk}

One way to understand universality is via representation theorems that relate
different models. For instance, the \emph{Kac-Siegert transformation}
can be used to write the partition function of the $O(n)$ model as a partition function
for a perturbation of the $|\varphi|^4$ model (we will discuss this further in
Section~\ref{sec:hard-core}). In the other direction, the
\new{Simon-Griffiths} construction \cite{SG73} can be used to approximate the $1$-component
$|\varphi|^4$ model \new{as a suitable limit of Ising models.}
Such theorems do not necessarily imply universality
(in the sense that models related in this way have the same critical exponents or
scaling limit), but tend to be suggestive of it and may in some cases
be used as the basis for the proof of a universality-type result.

We have already noted in Example~\ref{ex:gff-asymp} the close relationship between
the simple random walk
and the Gaussian free field, which ultimately stems from the representation of
matrix powers in terms of walks and which is familiar to anyone
who has studied Markov chains. Namely, if $M$ is a $\vertices\times\vertices$
matrix, then
\begin{equation}
M^n_{ab} = \sum_{x_1,\ldots,x_n\in\vertices} M_{ax_1} M_{x_1x_2} \ldots M_{x_nb}.
\end{equation}
When $M$ is indexed by $\edges$, the
sum above can be replaced by a sum over $n$-step walks from $a$ to $b$ on $\graph$.
When the entries of $M$ are nonnegative, such a sum acquires a probabilistic
interpretation as an expectation with respect to the random walk whose steps
are weighted by the entries of $M$.

It was discovered by Symanzik \cite{Syma69} that certain spin systems could be
represented as models of interacting walks in a background of interacting loops.
Symanzik used this insight to study quantum field theories in terms of walks.
Such representations were also studied, e.g.\ in \cite{BFS82,Dynk83}. A comprehensive
reference is \cite{FFS92}.

In the opposite direction, one can consider studying walks by looking for corresponding
spin systems. In \cite{Genn72}, \new{de Gennes} argued that the self-avoiding walk
corresponds to an $n \to 0$ ``limit'' of the $O(n)$ spin model and used this
to predict the values of its critical exponents. Since $n$ is
the number of components of the spins, the $O(n)$ model is only well-defined
for $n$ a positive integer and it is not clear how to make sense of such a limit.

Parisi and Sourlas \cite{PS80} and McKane \cite{McKa80} discovered
an alternative approach to the predictions of \new{de Gennes}. They argued that the weakly
self-avoiding walk two-point function could be represented as the two-point function
for a version of the $|\varphi|^4$ model, involving boson and
fermion fields (we discuss these below). The formal appearance of $n = 0$ quantities
was then explained as a consequence of a symmetry between the bosons and fermions known
as \emph{supersymmetry}.

In Section~\ref{sec:ntozero}, we provide a brief description of the heuristic
relation between spin systems and self-avoiding walk. Then in Section~\ref{sec:intrep},
we describe the rigorous representation of WSAW-SA as a supersymmetric field
theory.

%%%%%%%%%%%%%%%%%%%%%%%%%%%%%%%%%%%%%%%%%%%%%%%%%%%%%%%%%%%%%%%%%%%%%%%%%%%%%%%

\subsection{The \texorpdfstring{$n\to0$}{n approaches 0} limit}
\label{sec:ntozero}

The heuristic relation between self-avoiding walk and spin systems is most easily
treated on \new{finite} graphs $\graph$ of degree $3$
% (e.g.\ the \new{hexagonal} lattice),
so we restrict our attention to this case. In
addition, we consider a version of the $O(n)$ model with spins normalized to
lie on the sphere of radius $\sqrt n$, which we equip with the uniform measure.
We denote the product measure on the resulting configuration space by
\begin{equation}
d\sigma = \prod_{x\in\vertices} d\sigma_x.
\end{equation}

\begin{rk}
This normalization of the spins was in fact used when the $O(n)$ model was originally
introduced in \cite{Stanley68}. Moreover, in \cite{KT71}, it was shown that this
normalization is necessary for the study of the $n\to\infty$ limit.
\end{rk}

The \emph{high-temperature expansion} of a spin system is based on the expansion of the
Boltzmann weight \new{$e^{-\beta H}$} around $\beta = 0$.
For the $O(n)$ spin model with interaction $J_{xy} = \1_{x\sim y}$,
\new{neglecting higher-order terms in the high-temperature expansion yields}
% an uncontrolled approximation of this expansion yields
\begin{align}
Z 
	&=
\int d\sigma \prod_{xy\in\edges} e^{\beta \sigma_x \cdot \sigma_y} \nonumber \\
	&\approx
\int d\sigma \prod_{xy\in\edges} (1 + \beta \sigma_x \cdot \sigma_y) \nonumber \\
	&=
\sum_{E \subset \edges} \beta^{|E|} \int d\sigma \prod_{xy \in E} \sigma_x \cdot \sigma_y
\end{align}
By reflection-invariance, the last integral above is non-zero
if and only if every vertex in the product over $E$ appears an even number of times.
On a graph of degree $3$, this is only possible if $E$ is a (possibly empty) collection
of mutually avoiding (i.e.\ disjoint) self-avoiding loops (walks from a vertex to itself
that are self-avoiding everywhere except this vertex).
% Thus, the sum over subsets of $\edges$ can be replaced by a sum over collections of such loops.

Moreover, for any loop $L$, invariance under orthogonal transformations and the fact
that spins have radius $\sqrt n$ implies that
\begin{equation}
\int d\sigma \; \prod_{xy\in L} \sigma_x \cdot \sigma_y
	=
\sum_{i=1}^n \prod_{x\in\vertices(L)} \int d\sigma_x \; (\sigma^i_x)^2
	=
n,
\end{equation}
where $\vertices(L)$ is the set of vertices in $L$.
Thus,
\begin{equation}
\label{e:loop-partition}
Z
	\approx
1
	+
\sum_{N \ge 1}
\frac{n^N}{N!}
\sum_{L_1,\ldots,L_N}
\beta^{|L_1|+\cdots+|L_N|},
\end{equation}
where the inner sum is over all collections of \new{disjoint} loops $L_1, \ldots, L_N$
and permutations of these loops are accounted for by the $1/N!$ factor.
Notice that the final expression on the right-hand side of \eqref{e:loop-partition}
makes sense for any $N$ and equals $1$ when $n = 0$.

The two-point function for the $O(n)$ model can be defined analogously to
\eqref{e:two-point-function-phi4} and \eqref{e:Ising-2pt}. By a similar expansion
as was used to study the partition function above, the numerator in the two-point
function becomes
\begin{equation}
\label{e:2pt-num}
n^{-1} \int d\sigma (\sigma_a \cdot \sigma_b) \prod_{xy\in\edges} e^{\beta\sigma_x\cdot\sigma_y}
	\approx
n^{-1}
\sum_{E\subset\edges} \beta^{|E|}
\int d\sigma (\sigma_a \cdot \sigma_b)
\prod_{xy\in E} \sigma_x \cdot \sigma_y.
\end{equation}
Once again, every vertex must appear twice on the right-hand side in order to make a
non-zero contribution to the sum. Due to the presence of the factor $\sigma_a \cdot \sigma_b$,
this means (unless $a = b$) that the sum can be replaced by a sum over subsets $E$ containing
a self-avoiding walk from $a$ to $b$ together with with a (possibly empty) family of mutually
avoiding self-avoiding loops that also avoid this walk. (As a very simple example, if $a \sim b$,
then there is a non-zero contribution from $E = \{ a, b \}$.) For any such configuration $E$
containing $N$ loops,
\begin{equation}
\int d\sigma \; (\sigma_a \cdot \sigma_b) \prod_{xy\in E} \sigma_x \cdot \sigma_y
	=
n^{1 + N}.
\end{equation}
The extra factor of $n$ arises from the walk in $E$ but is
cancelled by the normalization in \eqref{e:2pt-num}.
Thus, after formally setting $n = 0$ (so that $Z = 1$), the two-point function
is approximately given by
\begin{equation}
1 + \sum_{\omega\in\Scal_n(a, b)} \beta^{|\omega|},
\end{equation}
which is the two-point function, i.e.\ the generating function for all self-avoiding
walks from $a$ to $b$.
% Thus, the two-point function is approximately
% \begin{equation}
% \frac{1}{1 + O(n)}
% \left(
% 	1
% 		+
% 	\sum_{\omega : a \to b} \beta^{|\omega|}
% 		+
% 	O(n)
% \right).
% \end{equation}

%%%%%%%%%%%%%%%%%%%%%%%%%%%%%%%%%%%%%%%%%%%%%%%%%%%%%%%%%%%%%%%%%%%%%%%%%%%%%%%

\subsection{Self-avoiding walk representation}
% This section based on saw-sa
\label{sec:intrep}

In this section we describe an integral representation of the of WSAW-SA on the
discrete torus $\Lambda$. We begin with the necessary background on Grassmann
integration, which was introduced in \cite{Bere66}. However, we follow the
treatment of \cite{BIS09} in terms of differential forms.

\subsubsection{Boson and fermion fields}
\label{sec:forms}

Let $\phi_x$, $\bar\phi_x$ denote complex variables indexed by $x\in\Lambda$.
We refer to $(\phi, \phib)$ as a \emph{boson} field. Let $u_x, v_x$ denote
the real and imaginary parts of $\phi_x$ and define the differentials
$d\phi_x = du_x + i dv_x$ and likewise for $d\phib_x$.
We multiply differential forms in the usual way via the anticommutative wedge product
$\wedge$ but drop this in our notation; in particular,
\begin{equation}
d\phib_x d\phi_x = 2 i du_x dv_x.
\end{equation}

\begin{example}
Let $C$ be a positive-definite symmetric $\Lambda\times\Lambda$ matrix. The
\emph{complex Gaussian measure} with covariance $C$ is the probability measure
on $\R^{2\Lambda}$ given by
\begin{equation}
d\mu_C(\phi, \phib)
	=
\frac{d\phib d\phi}{\det(2\pi i C)} e^{-\phi \cdot A \phib}
\end{equation}
where $A = C^{-1}$ and
\begin{equation}
d\phib d\phi \coloneqq \prod_{x\in\Lambda} d\phib_x d\phi_x
\end{equation}
The order in which the product over $x\in\Lambda$ is taken does not matter
since the $d\phib_x d\phi_x$ commute. The complex Gaussian satisfies a
version of Wick's theorem. In particular,
\begin{equation}
\label{e:wick-complex2}
\int \phib_x \phi_y \; d\mu_C(\phi, \phib) = C_{xy}.
\end{equation}
\end{example}

Let
\begin{equation}
\psi_x = \frac{1}{\sqrt{2\pi i}} d\phi_x,
\quad
\bar\psi_x = \frac{1}{\sqrt{2\pi i}} d\bar\phi_x,
\end{equation}
where we fix a choice of complex square root. We refer to $(\psi_x, \psib_x)_{x\in\Lambda}$
as a \emph{fermion} field.
A differential form that is the
product of a function of $(\phi, \bar\phi)$
with $p$ differentials is said to have \emph{degree} $p$.
A sum of forms of even degree is said to be \emph{even}.

We introduce a copy $\bar\Lambda$ of $\Lambda$
and we denote the copy of $X \subset \Lambda$ by $\bar X \subset \bar\Lambda$.
We also denote the copy of $x \in \Lambda$
by $\bar x \in \bar\Lambda$ and define $\phi_{\bar x} = \bar\phi_x$ and $\psi_{\bar x} = \bar\psi_x$.
Then any differential form $F$ can be written
\begin{equation}
\lbeq{FinNcal}
F
=
\sum_{\vec y}
F_{\vec y} (\phi, \bar\phi)
\psi^{\vec y}
\end{equation}
where the sum is over finite sequences $\vec y$ over $\Lambda\sqcup\bar\Lambda$,
and $\psi^{\vec y} = \psi_{y_1} \ldots \psi_{y_p}$
% (in some canonical order)
when $\vec y = (y_1, \ldots, y_p)$. Here, we take the sequences to be ordered in
some \new{fixed but arbitrary} fashion.
\new{We let $F^\zero$ denote the $0$-degree (bosonic) part of $F$, given by the
coefficient $F_{\vec y}$ with $\vec y = \varnothing$ the empty sequence.}
% When $\vec y = \varnothing$ is the empty sequence,
% $F_\varnothing$ denotes the $0$-degree (bosonic) part of $F$.

In order to apply the results of \cite{BBS-saw4-log,BBS-saw4,BSTW-clp}, we require
smoothness of the coefficients $F_{\vec y}$ of $F$.  For Theorem~\ref{thm:mr}(i,ii),
we need these coefficients to be $C^{10}$, and for Theorem~\ref{thm:mr}(iii) we require
a $p$-dependent number of derivatives for the analysis of $\xi_p$.
In either case, we let $p_\Ncal$ denote the desired degree of smoothness.
We will discuss this further in Section~\ref{sec:newnorm}.

We let $\Ncal^\bulk$ be the algebra of even forms with sufficiently smooth coefficients
and we let $\Ncal^\bulk(X) \subset \Ncal^\bulk$ be the sub-algebra of even forms only depending on fields
in $X$. Thus, for $F \in \Ncal^\bulk(X)$, the sum in \eqref{e:FinNcal} runs over sequences
$\vec y$ over $X \sqcup \bar X$.

Now let $F = (F_j)_{j \in J}$ be a finite collection of even forms
indexed by a set $J$
and write $F^\zero = (F^\zero_j)_{j \in J}$.
Given a $C^\infty$ function $f : \R^J \to \C$, we define
$f(F)$ by its Taylor expansion about $F^\zero$:
\begin{equation}
f(F) = \sum_\alpha \frac{1}{\alpha!} f^{(\alpha)}(F^\zero) (F - F^\zero)^\alpha.
\end{equation}
The summation terminates as a finite sum, since $\psi_x^2 = \bar\psi_x^2 = 0$
by anticommutativity.
% due to the anti-commut\-ative product.

We define the integral $\int F$ of a differential form $F$ in the usual way
as the Riemann integral of its top-degree part (which may be regarded as a function
of the boson field).
In particular, given a positive-definite symmetric
$\Lambda \times \Lambda$ matrix $C$ with inverse $A = C^{-1}$,
we define the \emph{Gaussian expectation} (or \emph{super-expectation}) of $F$ by
\begin{equation}
\lbeq{ExCF}
\Ex_C F = \int e^{-S_A} F,
\end{equation}
where
\begin{equation}
\label{e:action}
S_A = \sum_{x\in\Lambda} \Big(\phi_x (A\bar\phi)_x + \psi_x (A \bar\psi)_x\Big).
\end{equation}
% Note that the computation of \eqref{e:ExCF} always reduces to an integral of the form
% \begin{equation}
% \int f(\phi, \phib) \psi^{\Lambda\sqcup\bar\Lambda},
% \end{equation}
% where $\psi^{\Lambda\sqcup\bar\Lambda}$ denotes the product of all the fermionic fields
% in their canonical ordering. Thus, in order to make the change of variables
% $\psi_x \mapsto a \psi_x$ for some $x$ we must simply scale the integral by the factor
% $a^{-1}$. This is the opposite effect of a bosonic change of variables $\phi_x \mapsto a \phi_x$

\new{The super-expectation has the following self-normalizing property:
\begin{equation}
\label{e:self-normal}
\Ex_C 1 = \int e^{-S_A} = 1.
\end{equation}}
Moreover, if $F$ is a degree-$0$ form, then
\begin{equation}
\Ex_C F = \int F \; d\mu_C.
\end{equation}
There is also a version of Wick's theorem for fermions. In particular,
\begin{equation}
\label{e:wick-fermion2}
\int e^{-S_A} \psi_x \psib_x = C_{xx}.
\end{equation}
\new{Proofs of the statements \eqref{e:self-normal}--\eqref{e:wick-fermion2}
can be found in \cite{BIS09}.}

For $F = f(\phi, \bar\phi) \psi^{\vec y}$, we let
\begin{equation}
\theta F = f(\phi + \xi, \bar\phi + \bar\xi) (\psi + \eta)^{\vec y},
\end{equation}
where $\xi$ is a new boson field, $\eta = (2\pi i)^{-1/2} d\xi$ a new fermion field,
and $\bar\xi, \bar\eta$ are the corresponding conjugate fields.
We extend $\theta$ to all $F \in \Ncal^\bulk$ by linearity
and define the convolution operator $\Ex_C\theta$ by letting
$\Ex_C\theta F \in \Ncal^\bulk$ denote the Gaussian expectation of $\theta F$ with respect
to $(\xi, \bar\xi, \eta, \bar\eta)$, with $\phi,\phib,\psi,\psib$ held fixed.

\subsubsection{Integral representation of the two-point function}
\label{sec:Gintrep}

An integral representation formula applying to general local time functionals
is given in \cite{BEI92,BIS09}. We state the result we need in the proposition below.
A direct proof can be obtained by a small modification to the proof in
\cite[Appendix~A]{ST-phi4}.

We define the differential forms:
\begin{align}
\label{e:taudef}
\tau_x
	&=
\phi_x \bar\phi_x + \psi_x \bar\psi_x
	\\
\label{e:addDelta}
\tau_{\Delta,x}
	&=
\frac 12
\Big(
	\phi_{x} (- \Delta \bar{\phi})_{x} + (- \Delta \phi)_{x} \bar{\phi}_{x}
		+
	\psi_{x}  (- \Delta \bar{\psi})_{x} + (- \Delta \psi)_{x}  \bar{\psi}_{x}
\Big)
	\\
\label{e:nablatau}
|\nabla \tau_x|^2
	&=
\sum_{|e|=1} (\nabla^e \tau)_x^2.
\end{align}
\new{The forms $\tau_x$ are special due to
the following remarkable property
of the super-expectation (see \cite{BIS09}):
\begin{equation}
\label{e:tau-iso}
\int e^{-S_A} F(\tau) = F(0).
\end{equation}}
Recall \eqref{e:Udef-pos} and define
\begin{equation}
\label{e:Vdef2}
V_{\gcc,\gamma,\nu,N}
	=
U_{\gcc,\gamma}(\tau)
	+
\sum_{x\in\Lambda_N}
\Big(
	\nu \tau_x + \tau_{\Delta,x}
\Big)
% \sum_{x\in\Lambda_N}
% \Big(
% 	\gcc \tau_x^2 + \nu \tau_x + \tau_{\Delta,x} - \tfrac{1}{2 d} \gamma |\nabla \tau_x|^2
% \Big)
\end{equation}

\begin{prop}
Let $d > 0$ and $\gcc > 0$. For $\gamma < \gcc$ and $\nu \in \R$,
\begin{equation}
\label{e:Grep-pos-bis}
G_{x,N}(\gcc, \gamma, \nu; 0)
	=
\int e^{-V_{\gcc,\gamma,\nu,N}} \phib_0 \phi_x.
% \int e^{-U_{\gcc,\gamma,\nu,N}} \bar\phi_a \phi_b.
\end{equation}
\end{prop}

\subsubsection{Finite-volume approximation}

In order to make use of the integral representation above, we must approximate the
WSAW-SA on $\Zd$ by a model on $\Lambda_N$.

Let $X^{L^N}$ denote the simple random walk on $\Lambda_N$.
For $F_T = F_T(X)$ any one of the functions $L_T^x,I_T,C_T$
of $X$ defined in \eqref{e:LTx-def}--\eqref{e:CTdef},
we write $F_{N,T} = F_T(X^{L^N})$. For instance, with $n=L^N$,
\begin{equation}
    L^x_{N,T} = \int_0^T \1_{X^{n}_t=\;x} \; dt,
    \quad I_{N,T} = \sum_{x \in \Lambda_N}(L_{N,T}^x)^2 .
\end{equation}

As before, we identify the vertices of $\Lambda_N$ with nested subsets of $\Zd$,
centred at the origin (approximately if $L$ is even),
with $\Lambda_{N+1}$ paved by $L^d$ translates of $\Lambda_N$.
% We can thus define $\partial \Lambda_N$ to be the inner vertex boundary of $\Lambda_N$.
We denote the expectation of $X^{L^N}$ started from $0 \in \Lambda_N$ by $E^{\Lambda_N}_0$
and define
\begin{align}
\label{e:cN}
c_{N,T}(x)
    &= E^{\Lambda_N}_0 \left( e^{-U_{\gcc,\gamma,T}} \1_{X(T)=b} \right),
    \quad x \in \Lambda_N \\
c_{N,T}
    &= E^{\Lambda_N}_0 \left( e^{-U_{\gcc,\gamma,T}} \right).
\end{align}
The finite-volume two-point function and susceptibility
are defined by
\begin{align}
G_{x,N}(\gcc,\gamma,\nu; 0)
    &=
\int_0^\infty c_{N,T}(x) e^{-\nu T} \; dT, \\
\label{e:chiNdef-pre}
\chi_N(\gcc, \gamma, \nu; 0)
    &=
\int_0^\infty c_{N,T} e^{-\nu T} \; dT.
\end{align}
The proof of the following proposition is given in Appendix~\ref{sec:finvol}.

\begin{prop}
\label{prop:finvol}
Let $d >0$, $\gcc >0$ and $\gamma < \gcc$. For all $\nu \in \R$,
\begin{equation}
\label{e:Givlc}
\lim_{N \to \infty}
G_{x,N}(\gcc,\gamma,\nu; 0)
=
G_x(\gcc,\gamma,\nu; 0)
\end{equation}
and
\begin{equation}
\label{e:chilim-pre}
\lim_{N\to\infty}\chi_N(\gcc,\gamma,\nu; 0) =   \chi(\gcc,\gamma,\nu; 0).
\end{equation}
In fact, $\chi_N$ and $\chi$ are analytic in ${\rm Re} \nu > \nu_c$ and
$\chi_N \to \chi$ uniformly on compact subsets of this domain.
\end{prop}

%%%%%%%%%%%%%%%%%%%%%%%%%%%%%%%%%%%%%%%%%%%%%%%%%%%%%%%%%%%%%%%%%%%%%%%%%%%%%%%
%%%%%%%%%%%%%%%%%%%%%%%%%%%%%%%%%%%%%%%%%%%%%%%%%%%%%%%%%%%%%%%%%%%%%%%%%%%%%%%

\section{Outline}

\new{
Chapter~\ref{sec:rg} introduces the elements and formalism of the renormalisation
group method developed in \cite{BS-rg-norm,BS-rg-loc,BBS-rg-pt,BS-rg-IE,BS-rg-step}.
However, we proceed differently from these papers in two regards.

Firstly, in Section~\ref{sec:weights} we employ a different choice of norm weights
from that used in \cite{BS-rg-step}, where the renormalisation group map was constructed.
However, these new weights cannot be used with the same norms as in \cite{BS-rg-step}.
In Chapter~\ref{sec:RGstep}, we explain how to overcome this obstacle by a new choice of norm
and we provide a detailed verification that the estimates on the renormalisation group map
are improved by this choice. The result is summarized as
Theorem~\ref{thm:step-mr-fv}, which is the first main technical achievement of this
thesis. The improved estimates that we obtain are required for the proof of
Theorem~\ref{thm:mr}(iii) even with $\gamma = 0$. This result first appeared in \cite{BSTW-clp}.

Secondly, the initial coordinates for the renormalisation group that we define in
Section~\ref{sec:initIK} involve a non-trivial error coordinate that captures
the self-attraction term in the WSAW-SA and the $\gamma (\nabla |\phi_x|^2)^2$
term in the generalized $|\varphi|^4$ model. This error coordinate is coupled to
the parameters $g_0, \nu_0, z_0$ and a tailored version of the implicit function
theorem is consequently required for the identification of critical parameters
such that the renormalisation group is initialized on its stable manifold when
$\gamma \ne 0$. The construction of these critical parameters is carried out in
Chapter~\ref{sec:RGflow} and the result is summarized as Theorem~\ref{thm:rhatflow}.
This is the second main technical achievement of this thesis and is required for the
proof of Theorem~\ref{thm:mr} with $\gamma \ne 0$. This result first appeared in
\cite{BSW-saw-sa} for $n = 0$; here, we have extended it to all $n \ge 0$.

In Chapter~\ref{sec:chi-G-xi}, we discuss the proof of Theorem~\ref{thm:mr} assuming
Theorem~\ref{thm:step-mr-fv} and Theorem~\ref{thm:rhatflow} hold. The proof of
Theorem~\ref{thm:mr}(iii) is a novel contribution even when $\gamma = 0$. The case
$\gamma = 0$ first appeared in \cite{BSTW-clp} and the extension to $\gamma \ne 0$.

The proof of Theorem~\ref{thm:mr}(i)--(ii) was previously obtained for $\gamma = 0$
by Bauerschmidt, Brydges, and Slade in \cite{BBS-saw4-log,BBS-phi4-log,BBS-saw4}
and Slade and Tomberg in \cite{ST-phi4}. The extension to $\gamma \ne 0$ involves,
in addition to the proof of Theorem~\ref{thm:rhatflow}, a change of variables result
stated and proved in Section~\ref{sec:nuztilde}.

We conclude in Chapter~\ref{sec:conclusion} with a discussion of some open problems.
}

% \commentbw{General comments:

% - the structure of the thesis and proof of main result needs
%   to be presented more clearly early on

% - you should give definitions more often rather than referring
%   to other papers
  
% - I will be more than happy to sign off on your revised version.
%   I won't have more time before Apr 21 to carefully read the revision.}
