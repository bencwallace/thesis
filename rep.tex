\section{Spin systems and walks}

We have already discussed the close relationship between the simple random walk
and the Gaussian free field, which ultimately stems from the representation of
matrix powers in terms of walks and which is familiar to anyone
who has studied Markov chains. Namely, if $M$ is a $\vertices\times\vertices$
matrix, then
\begin{equation}
M^n_{ab} = \sum_{x_1,\ldots,x_n\in\vertices} M_{ax_1} M_{x_1x_2} \ldots M_{x_nb}.
\end{equation}
When $M$ is indexed by edges (i.e.\ when $M_{xy} = 0$ for $x \not\sim y$), the
above sum can be replaced by a sum over $n$-step walks from $a$ to $b$ on $\graph$.
When the entries of $M$ are non-negative, such a sum acquires a probabilistic
interpretation as an expectation with respect to the random walk whose steps
are weighted by the entries of $M$.

This idea can be extended to obtain relationships between certain models of
interacting walks and spin systems.

%%%%%%%%%%%%%%%%%%%%%%%%%%%%%%%%%%%%%%%%%%%%%%%%%%%%%%%%%%%%%%%%%%%%%%%%%%%%%%%

\subsection{High-temperature expansion of the \texorpdfstring{$O(n)$}{O(n)} model}

The high-temperature expansion of a spin system is based on the expansion of the
Boltzmann weight $e^{\beta H}$ around $\beta = 0$.
For the $O(n)$ spin model, an uncontrolled high-temperature expansion yields
\begin{align}
Z 	&= \int d\lambda(\sigma) \prod_{xy\in\edges} e^{\beta \sigma_x \cdot \sigma_y} \\
	&\approx \int d\lambda(\sigma) \prod_{xy\in\edges} (1 + \beta \sigma_x \cdot \sigma_y) \\
	&= \sum_{E \subset \edges} \beta^{|E|} \int d\lambda(\sigma) \prod_{xy \in E} \sigma_x \cdot \sigma_y
\end{align}
By reflection-invariance of the sphere measure, the last integral above is non-zero
if and only if every vertex in the product over $E$ appears an even number of times;
thus, the sum over subsets of $\edges$ can be replaced by a sum over collections of
loops (walks from a vertex to itself) in $\graph$.

A similar expansion can be performed for the numerator in the definition of the
two-point function, yielding:
\begin{equation}
\int d\lambda(\sigma) \sigma_a \cdot \sigma_b
	\approx
\sum_{E\subset\edges} \beta^{|E|}
\int d\lambda(\sigma) \sigma_a \cdot \sigma_b
\prod_{xy\in E} \sigma_x \cdot \sigma_y.
\end{equation}
Once again, every vertex must appear twice on the right-hand side. This time,
there are additional non-zero contributions from subsets $E$ of edges containg
a path from $a$ to $b$. For instance, if $a \sim b$, there is a non-zero
contribution from $E = \{ \{ a, b \} \}$.

%%%%%%%%%%%%%%%%%%%%%%%%%%%%%%%%%%%%%%%%%%%%%%%%%%%%%%%%%%%%%%%%%%%%%%%%%%%%%%%

\subsection{The \texorpdfstring{$n\to0$}{n approaches 0} limit}

A more careful analysis of the high-temperature expansion of the $O(n)$ model
above will reveal the $n$-dependence of the non-zero contributions to the
partition function and two-point function. When the spins are restricted to
the sphere of radius of $\sqrt n$, it can be shown that formally setting $n = 0$
results in $Z = 1$ and the two-point function receives contributions only from
self-avoiding walks from $a$ to $b$. See \cite[Section 2.3]{MS93} for details.

Based on this idea, De Gennes predicted the critical exponents of the self-avoiding
walk by setting $n = 0$ in the predicted exponents for the $O(n)$ spin model.
This is known as the $n \to 0$ ``limit''.

%%%%%%%%%%%%%%%%%%%%%%%%%%%%%%%%%%%%%%%%%%%%%%%%%%%%%%%%%%%%%%%%%%%%%%%%%%%%%%%

\subsection{Self-avoiding walk representation}
% This section based on saw-sa
\label{sec:intrep}

\todo{Discuss complex Gaussian integral}

It is not clear how to make rigorous the $n\to0$ limit of De Gennes. Parisi and
Sourlas and, independently, McKane, discovered an alternative approach to the
predictions of De Gennes. They argued that the weakly
self-avoiding walk \todo{two-point function} could be represented as the two-point
function for a version of the $|\varphi|^4$ model involving both boson and fermion
fields. The formal appearance of $n = 0$ quantities was then explained as a
consequence of a symmetry between the bosons and fermions known as \emph{supersymmetry}.

In this section we describe an integral representation of the of WSAW-SA, which
is a special case of a result of Brydges, Imbrie, and Slade \cite{BIS09}.
We restrict out attention to the graph $\Lambda = \Lambda_N = \Zd/L^N\Zd$.
We begin by introducing the notions of bosons and fermion fields on $\Lambda$.

\subsubsection{Boson and fermion fields}
\label{sec:forms}

We fix $N$ and write $\Lambda = \Lambda_N$.
Given complex variables $\phi_x, \bar\phi_x$
(the boson field) for $x \in \Lambda$,
we define the differentials (the fermion field)
\begin{equation}
\psi_x = \frac{1}{\sqrt{2\pi i}} d\phi_x,
\quad
\bar\psi_x = \frac{1}{\sqrt{2\pi i}} d\bar\phi_x,
\end{equation}
where we fix a choice of complex square root.
The fermion fields are multiplied with each other
via the anti-commutative wedge product,
though we suppress this in our notation.

A differential form that is the
product of a function of $(\phi, \bar\phi)$
with $p$ differentials is said to have degree $p$.
A sum of forms of even degree is said to be \emph{even}.
We introduce a copy $\bar\Lambda$ of $\Lambda$
and we denote the copy of $X \subset \Lambda$ by $\bar X \subset \bar\Lambda$.
We also denote the copy of $x \in \Lambda$
by $\bar x \in \bar\Lambda$ and define $\phi_{\bar x} = \bar\phi_x$ and $\psi_{\bar x} = \bar\psi_x$.
Then any differential form $F$ can be written
\begin{equation}
\lbeq{FinNcal}
F
=
\sum_{\vec y}
F_{\vec y} (\phi, \bar\phi)
\psi^{\vec y}
\end{equation}
where the sum is over finite sequences $\vec y$ over $\Lambda\sqcup\bar\Lambda$,
and $\psi^{\vec y} = \psi_{y_1} \ldots \psi_{y_p}$ when $\vec y = (y_1, \ldots, y_p)$.
When $\vec y = \varnothing$ is the empty sequence,
$F_\varnothing$ denotes the $0$-degree (bosonic) part of $F$.

In order to apply the results of \cite{BBS-saw4-log,BBS-saw4,BSTW-clp}, we require
smoothness of the coefficients $F_{\vec y}$ of $F$.  For Theorem~\ref{thm:mr}(i,ii),
we need these coefficients to be $C^{10}$, and for Theorem~\ref{thm:mr}(iii) we require
a $p$-dependent number of derivatives for the analysis of $\xi_p$, as discussed in \cite{BSTW-clp}.
In either case, we let $p_\Ncal$ denote the desired degree of smoothness.

We let $\Ncal^\varnothing$ be the algebra of even forms with sufficiently smooth coefficients
and we let $\Ncal^\varnothing(X) \subset \Ncal^\varnothing$ be the sub-algebra of even forms only depending on fields
in $X$. Thus, for $F \in \Ncal^\varnothing(X)$, the sum in \eqref{e:FinNcal} runs over sequences $\vec y$
over $X \sqcup \bar X$.
Note that $\Ncal^\varnothing = \Ncal^\varnothing(\Lambda)$.


Now let $F = (F_j)_{j \in J}$ be a finite collection of even forms
indexed by a set $J$
and write $F_\varnothing = (F_{\varnothing,j})_{j \in J}$.
Given a $C^\infty$ function $f : \R^J \to \C$, we define
$f(F)$ by its Taylor expansion about $F_\varnothing$:
\begin{equation}
f(F) = \sum_\alpha \frac{1}{\alpha!} f^{(\alpha)}(F_\varnothing) (F - F_\varnothing)^\alpha.
\end{equation}
The summation terminates as a finite sum,
since $\psi_x^2 = \bar\psi_x^2 = 0$ due to the anti-commut\-ative product.

We define the integral
$\int F$
of a differential form $F$ in the usual way
as the Riemann integral of its top-degree part
(which may be regarded as a function
of the boson field).
In particular, given a positive-definite
$\Lambda \times \Lambda$ symmetric matrix $C$
with inverse $A = C^{-1}$,
we define the \emph{Gaussian expectation}
(or \emph{super-expectation}) of $F$ by
\begin{equation}
\lbeq{ExCF}
\Ex_C F = \int e^{-S_A} F,
\end{equation}
where
\begin{equation}
\label{e:action}
S_A = \sum_{x\in\Lambda} \Big(\phi_x (A\bar\phi)_x + \psi_x (A \bar\psi)_x\Big).
\end{equation}
{The super-expectation has the remarkable self-normalization property that}
\begin{equation}
\label{e:self-normal}
\int e^{-S_A} = 1.
\end{equation}

Finally, for $F = f(\phi, \bar\phi) \psi^{\vec y}$,
we let
\begin{equation}
\theta F = f(\phi + \xi, \bar\phi + \bar\xi) (\psi + \eta)^{\vec y},
\end{equation}
where $\xi$ is a new boson field, $\eta = (2\pi i)^{-1/2} d\xi$ a new fermion field,
and $\bar\xi, \bar\eta$ are the corresponding conjugate fields.
We extend $\theta$ to all $F \in \Ncal^\varnothing$ by linearity
and define the convolution operator $\Ex_C\theta$ by letting
$\Ex_C\theta F \in \Ncal^\varnothing$ denote the Gaussian expectation of $\theta F$ with respect
to $(\xi, \bar\xi, \eta, \bar\eta)$, with $\phi,\phib,\psi,\psib$ held fixed.

\subsubsection{Integral representation of the two-point function}
\label{sec:Gintrep}

An integral representation formula applying to general local time functionals
is given in \cite{BEI92,BIS09}; see also \cite[Appendix~A]{ST-phi4}.
We state the result we need in the proposition below.

Let $\Delta$ denote the Laplacian on $\Lambda$,
i.e.\ $\Delta_{xy}$ is given by the right-hand side of
\eqref{e:Deltaxy} for $x, y \in \Lambda$.
We define the differential forms:
\begin{align}
\label{e:taudef}
\tau_x
	&=
\phi_x \bar\phi_x + \psi_x \bar\psi_x
	\\
\label{e:addDelta}
\tau_{\Delta,x}
	&=
\frac 12
\Big(
	\phi_{x} (- \Delta \bar{\phi})_{x} + (- \Delta \phi)_{x} \bar{\phi}_{x}
		+
	\psi_{x}  (- \Delta \bar{\psi})_{x} + (- \Delta \psi)_{x}  \bar{\psi}_{x}
\Big)
	\\
\label{e:nablatau}
|\nabla \tau_x|^2
	&=
\sum_{|e|=1} (\nabla^e \tau)_x^2.
\end{align}
Recall \eqref{e:Udef-pos} and define
\begin{equation}
\label{e:Vdef2}
V_{\gcc,\gamma,\nu,N}
	=
U_{\gcc,\gamma}(\tau)
	+
\sum_{x\in\Lambda_N}
\Big(
	\nu \tau_x + \tau_{\Delta,x}
\Big)
% \sum_{x\in\Lambda_N}
% \Big(
% 	\gcc \tau_x^2 + \nu \tau_x + \tau_{\Delta,x} - \tfrac{1}{2 d} \gamma |\nabla \tau_x|^2
% \Big)
\end{equation}

\begin{prop}
Let $d > 0$ and $\gcc > 0$. For $\gamma < \gcc$ and $\nu \in \R$,
\begin{align}
\label{e:Grep-pos-bis}
G_{x,N}(\gcc, \gamma, \nu)
	&=
\int e^{-V_{\gcc,\gamma,\nu,N}} \phib_0 \phi_x.
% \int e^{-U_{\gcc,\gamma,\nu,N}} \bar\phi_a \phi_b.
\end{align}
\end{prop}

\subsubsection{Finite-volume approximation}

In order to make use of the integral representation above, we must approximate the
WSAW-SA on $\Zd$ by a model on $\Lambda_N$.

Let $X^{L^N}$ denote the simple random walk on $\Lambda_N$.
For $F_T = F_T(X)$ any one of the functions $L_T^x,I_T,C_T$
of $X$ defined in \eqref{e:LTx-def}--\eqref{e:CTdef},
we write $F_{N,T} = F_T(X^{L^N})$. For instance, with $n=L^N$,
\begin{equation}
    L^x_{N,T} = \int_0^T \1_{X^{n}_t=\;x} \; dt,
    \quad I_{N,T} = \sum_{x \in \Lambda_N}(L_{N,T}^x)^2 .
\end{equation}

As before, we identify the vertices of $\Lambda_N$ with nested subsets of $\Zd$,
centred at the origin (approximately if $L$ is even),
with $\Lambda_{N+1}$ paved by $L^d$ translates of $\Lambda_N$.
% We can thus define $\partial \Lambda_N$ to be the inner vertex boundary of $\Lambda_N$.
We denote the expectation of $X^{L^N}$ started from $a \in \Lambda_N$ by $E^{\Lambda_N}_a$
and define
\begin{align}
\label{e:cN}
c_{N,T}(x)
    &= E^{\Lambda_N}_a \left( e^{-U_{\gcc,\gamma,T}} \1_{X(T)=b} \right)
    \quad (x \in \Lambda_N), \\
c_{N,T}
    &= E^{\Lambda_N}_0 \left( e^{-U_{\gcc,\gamma,T}} \right).
\end{align}
The finite-volume two-point function and susceptibility
are defined by
\begin{align}
G_{x,N}(\gcc,\gamma,\nu)
    &=
\int_0^\infty c_{N,T}(x) e^{-\nu T} \; dT, \\
\label{e:chiNdef-pre}
\chi_N(\gcc, \gamma, \nu)
    &=
\int_0^\infty c_{N,T} e^{-\nu T} \; dT.
\end{align}

\begin{prop}
\label{prop:finvol}
Let $d >0$, $\gcc >0$ and $\gamma < \gcc$. For all $\nu \in \R$,
\begin{equation}
\label{e:Givlc}
\lim_{N \to \infty}
G_{x,N}(\gcc,\gamma,\nu)
=
G_x(\gcc,\gamma,\nu)
\end{equation}
and
\begin{equation}
\label{e:chilim-pre}
\lim_{N\to\infty}\chi_N(\gcc,\gamma,\nu) =   \chi(\gcc,\gamma,\nu).
\end{equation}
\end{prop}

The proof is in the appendix.

%%%%%%%%%%%%%%%%%%%%%%%%%%%%%%%%%%%%%%%%%%%%%%%%%%%%%%%%%%%%%%%%%%%%%%%%%%%%%%%
%%%%%%%%%%%%%%%%%%%%%%%%%%%%%%%%%%%%%%%%%%%%%%%%%%%%%%%%%%%%%%%%%%%%%%%%%%%%%%%

\section{Outline of remaining chapters}

\commentbw{Outline the remainder}
