\chapter{Conclusion}

Several problems immediately present themselves as possible extensions of
our results.

%%%%%%%%%%%%%%%%%%%%%%%%%%%%%%%%%%%%%%%%%%%%%%%%%%%%%%%%%%%%%%%%%%%%%%%%%%%%%%%
%%%%%%%%%%%%%%%%%%%%%%%%%%%%%%%%%%%%%%%%%%%%%%%%%%%%%%%%%%%%%%%%%%%%%%%%%%%%%%%

\section{Other models}

We begin by describing a possible approach to the analysis of the $O(n)$ model
and strictly self-avoiding walk. We then discuss long-range models to which
renormalisation group methods have been applied with some success.

%%%%%%%%%%%%%%%%%%%%%%%%%%%%%%%%%%%%%%%%%%%%%%%%%%%%%%%%%%%%%%%%%%%%%%%%%%%%%%%

\subsection{Models with hard-core constraints}

Recall that the Hamiltonian of the $O(n)$ model was defined in \eqref{e:on-model}.
On $\Lambda$, it takes the form
\begin{equation}
H_J(\sigma)
	=
-\frac12 \sigma \cdot J \sigma.
\end{equation}
This was obtained from a suitable limiting case of the $|\varphi|^4$ model which involved
taking $g\to\infty$. Although the restriction of our results to small $g$ is fundamental
to the renormalisation group approach, the Kac-Siegert transformation opens up an
alternative way to study the $O(n)$ model.

Namely, let $\Omega = (S^{n-1})^\Lambda$ and let $d\sigma$ denote the product measure on $\Omega$,
where $S^{n-1}$ is equipped with the uniform measure on the sphere. The partition function
for the $O(n)$ model is then given by
\begin{equation}
Z_J = \int_\Omega e^{-H_J(\sigma)} \; d\sigma.
\end{equation}
When $J$ is a positive-definite symmetric
matrix and $d\mu_J(\varphi)$ is the Gaussian measure with covariance $\varphi$, the elementary
identity
\begin{equation}
e^{-H_J(\varphi)}
	=
e^{\frac12 \sigma \cdot J \sigma}
	=
\int_{(\R^n)^\Lambda} e^{\sigma \cdot \varphi} \; d\mu_J(\varphi)
\end{equation}
holds. Interchanging the order of integration, we get
\begin{equation}
Z_J
	=
\int_{(\R^n)^\Lambda}
e^{-\sum_{x\in\Lambda} L(\varphi_x)}
% \left(\prod_{x\in\Lambda} L(\varphi_x)\right)
\; d\mu_J(\varphi),
\end{equation}
where
\begin{equation}
L(t)
	=
-\log
\int_{S^{n-1}} e^{\sigma_0 \cdot t} \; d\sigma_0,
	\quad
t\in\R^n
\end{equation}
is the negative logarithm of the Laplace transform of the sphere. Since $L$ is
a rotation- and reflection-invariant analytic function and $L(0) = 0$, it has
the form
\begin{equation}
\label{e:logLaplace}
L(t) = \nu |t|^2 + g |t|^4 + \sum_{k=3}^\infty c_{2k} |t|^{2k}.
\end{equation}
Letting $J = (-\Delta + \gamma^2)^{-1}$, we get
\begin{equation}
d\mu_J(\varphi)
	\propto
e^{-\frac12 (\gamma^2 |\varphi|^2 + \varphi \cdot (-\Delta \varphi))}.
\end{equation}
Thus, we can express the partition function as a perturbed $|\varphi|^4$ model
and a similar expression can be obtained for the two-point function.

By a procedure as in Section~\ref{sec:reformulation}, the analysis of this model
can be reformulated in terms of the evolution of an effective interaction $Z_j$
with initial condition $Z_0 = (I_0 \circ K_0)(\Lambda)$. Once again, the initial
error coordinate $K_0$ will be coupled to $I_0$, but we expect that the critical
parameters $\nu_0^c, z_0^c$ can be identified by an implicit function argument
as in Section~\ref{sec:nu0z0c}.

However, estimates on $K_0$ (which are straightforward
to obtain by a more careful computation of \eqref{e:logLaplace}) indicate that
$K_0$ is not of order $g_0^3$, which is required to invoke Theorem~\ref{thm:flow-flow}.
Thus, an extension of the ideas in \cite{BBS-rg-flow} would be needed to study this
case.

\begin{rk}
Similarly, it is possible to re-cast the strictly self-avoiding walk as a
perturbation of weakly self-avoiding walk using a supersymmetric integral
representation obtained in \cite{BIS09}. The covariance of the form
$(-\Delta + \gamma^2)^{-1}$ in this case corresponds to a model of \emph{spread-out}
self-avoiding walk with exponentially decaying jump probabilities.
\end{rk}

%%%%%%%%%%%%%%%%%%%%%%%%%%%%%%%%%%%%%%%%%%%%%%%%%%%%%%%%%%%%%%%%%%%%%%%%%%%%%%%

\subsection{Long-range models}

In \cite{WF72}, Wilson and Fisher suggested studying models in $d_c - \epsilon$
dimensions with $\epsilon > 0$ small and $d_c = 4$ the upper-critical dimension.
By setting $\epsilon = 1$, they obtained approximate values for critical exponents
in $3$ dimensions. One approach to the rigorous implementation of this idea
involves studying models in dimension $d$ (an integer) whose upper-critical
dimension is given by $d_c + \epsilon$. This is not as problematic as considering
models in fractional dimensions, as the upper-critical dimension $d_c$ need not
be considered the actual of dimension of some ambient space.

In the context of spin systems, given a massless covariance $C'$, we think of $d_c$
as simple a number such that some class of models undergoes Gaussian scaling with
covariance $C'$ whenever $d > d_c$. Recalling Remark~\ref{rk:bubble}, we might
expect that
\begin{equation}
d_c = \inf \{ d \in \R : B_{m^2} < \infty \},
\end{equation}
where $B_{m^2}$ is the bubble diagram, defined as the $\ell^2(\Zd)$ norm of the
massive Green function $C = (C' + m^2)^{-1}_{0x}$. Thus, in order to achieve
$d_c = d + \epsilon$, we choose $M$ to decay like $C'_{0x} \asymp |x|^{-(d-\alpha)}$
with $\alpha = \tfrac12 (d + \epsilon)$. Such a choice is given by
\begin{equation}
C' = (-\Delta)^{-\alpha/2}.
\end{equation}

This approach has been used to implement the renormalisation group below the
upper-critical dimensions in \cite{BDH98,MS00,BMS03,Abde07}. Recently, Slade
\cite{Slad17} has extended the approach discussed in this thesis to compute
some critical exponents asymptotics for long-range versions of the weakly
self-avoiding walk and the $|\varphi|^4$ model. In particular, he showed that,
as $\nu\downarrow\nu_c$ for these models, the susceptibility $\chi$ satisfies
\begin{equation}
\chi
	\asymp
(\nu - \nu_c)^{-\left(1 + \tfrac{n+2}{n+8} \tfrac{\epsilon}{\alpha} + O(\epsilon^2)\right)}.
\end{equation}
By extensions of \cite{Slad17} to use observables as in \REF, it should be
possible to identify the scaling behaviour of the two-point function and possibly
other correlation functions for these long-range models.

%%%%%%%%%%%%%%%%%%%%%%%%%%%%%%%%%%%%%%%%%%%%%%%%%%%%%%%%%%%%%%%%%%%%%%%%%%%%%%%
%%%%%%%%%%%%%%%%%%%%%%%%%%%%%%%%%%%%%%%%%%%%%%%%%%%%%%%%%%%%%%%%%%%%%%%%%%%%%%%

\section{Other observable quantities}

%%%%%%%%%%%%%%%%%%%%%%%%%%%%%%%%%%%%%%%%%%%%%%%%%%%%%%%%%%%%%%%%%%%%%%%%%%%%%%%

\subsection{Inversion of Laplace transform}

One of the main motivations for studying the
susceptibility and finite-order correlation length for walks is the possibility
of recovering information about the growth of partition function $c_T$ and
the mean-squared distance $\langle |X(T)|^2 \rangle$ as $T\to\infty$. In
particular, recalling the discussion in Section~\ref{sec:asymp}, one may
try to derive logarithmic corrections to the predicted scaling relations
\eqref{e:cT-asymp}--\eqref{e:XT-asymp} as a consequence of Theorem~\ref{thm:mr}(ii)--(iii).
This approach was successfully used in \REF to identify the asymptotics of the
mean-squared displacement of a hierarchical model of weakly self-avoiding walk.
The authors of \REF proceed by inverting the Laplace transform, which requires
control over this quantity in a sector of the complex plane larger than what
has been achieved here.

%%%%%%%%%%%%%%%%%%%%%%%%%%%%%%%%%%%%%%%%%%%%%%%%%%%%%%%%%%%%%%%%%%%%%%%%%%%%%%%

\subsection{The correlation length}

Another natural problem is to study the true correlation length $\xi$, defined in
\REF. The estimate \REF gives super-polynomial decay of the errors in the approximation
\REF of the two-point function, but this is not sufficient for studying $\xi$, which
would need exponentially decaying errors. The current estimates follow from the
covariance bounds \REF. It does not seem likely that these bounds can be improved
for the covariance decomposition of \cite{Baue13a}. It may be possible to obtain
sufficiently good bounds for the decomposition of \cite{BGM04}, but this is not the
only limitation: Exponentially decaying errors would require that the weights $\ell_j$
decay exponentially above the mass scale, but this would contradict the requirement \REF.
It is possible that this requirement reflects a limitation of the renormalisation group
method used here. Another possibility is that the renormalisation group is not well-suited
to the analysis of models away from the critical point, as has been suggested in \REF.
Upon reaching the mass scale, the effective interaction $Z_j$ essentially describes
a high-temperature model and it may be that methods more suited to such models would be
preferable. Also, If \REF held uniformly in $g$ for all $p$, one could consider proving
an analogue of \REF for $\xi$ as a consequence of \REF.
