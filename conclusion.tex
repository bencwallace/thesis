\chapter{Conclusion}

\setcounter{footnote}{0}

We end with a discussion of some open problems that may be accessible by
variants of the renormalisation group method discussed in this thesis.
Many of these require significant extensions to this method. We will try to
point out some of the main obstacles that must be overcome.

%%%%%%%%%%%%%%%%%%%%%%%%%%%%%%%%%%%%%%%%%%%%%%%%%%%%%%%%%%%%%%%%%%%%%%%%%%%%%%%
%%%%%%%%%%%%%%%%%%%%%%%%%%%%%%%%%%%%%%%%%%%%%%%%%%%%%%%%%%%%%%%%%%%%%%%%%%%%%%%

\section{Other models}

In order to apply the renormalisation group to the models we have considered,
we had to express them as perturbations of a Gaussian measure whose covariance
admits an appropriate finite-range decomposition. Here we discuss other models
that can be written in this way.

%%%%%%%%%%%%%%%%%%%%%%%%%%%%%%%%%%%%%%%%%%%%%%%%%%%%%%%%%%%%%%%%%%%%%%%%%%%%%%%

\subsection{Long-range models}

In \cite{WF72}, Wilson and Fisher suggested studying models in $d_c - \epsilon$
dimensions with $\epsilon > 0$ small and $d_c = 4$ the upper-critical dimension.
By setting $\epsilon = 1$, they obtained approximate values for critical exponents
in $3$ dimensions. One approach to the rigorous implementation of this idea
involves studying models in dimension $d$ (an integer) whose upper-critical
dimension is $d_c + \epsilon$. This is not as problematic as considering
models in fractional dimensions, as the upper-critical dimension $d_c$ need not
be considered the actual of dimension of an ambient space.

Given a massless covariance $C'$, the upper-critical
dimension is simply a number $d_c$ such that some class of models undergoes Gaussian
scaling with covariance $C'$ if and only if $d > d_c$. Recalling Remark~\ref{rk:bubble},
we might expect that
\begin{equation}
d_c = \inf \{ d \in \R : B_{m^2} < \infty \},
\end{equation}
where $B_{m^2}$ is the bubble diagram, defined as the $\ell^2(\Zd)$ norm of the
massive Green function $C = (C' + m^2)^{-1}_{0x}$. Thus, in order to achieve
$d_c = d + \epsilon$, we choose $C'$ to decay like
\begin{equation}
C'_{0x} \asymp |x|^{-(d-\alpha)}
\end{equation}
with $\alpha = \tfrac12 (d + \epsilon)$. Such a choice is given by
\begin{equation}
C' = (-\Delta)^{-\alpha/2}
\end{equation}
for $\alpha\in(0, 2)$ (so $d \le 3$).

This approach has been used to implement the renormalisation group below the
upper-critical dimensions in \cite{BDH98,MS00,BMS03,Abde07}. Recently, Slade
\cite{Slad17} has extended the approach discussed in this thesis to compute
\emph{anomalous} (non-Gaussian) critical exponents for long-range versions of
the weakly self-avoiding walk and the $|\varphi|^4$ model. In particular, he
showed that, as $\nu\downarrow\nu_c$ for these models, the susceptibility $\chi$
satisfies
\begin{equation}
\label{e:chi-anom}
\chi
	\asymp
(\nu - \nu_c)^{-\left(1 + \tfrac{n+2}{n+8} \tfrac{\epsilon}{\alpha} + O(\epsilon^2)\right)}.
\end{equation}
By extensions of \cite{Slad17} to use observable fields, we think it should
be possible to identify the scaling behaviour of the two-point function and possibly
other correlation functions for these long-range models. In particular, this would
make it possible to confirm (if true) the intriguing prediction of \cite{FMN72}, which
states that
\begin{equation}
\eta = 2 - \alpha
\end{equation}
if $d = d_c - \epsilon$ for small $\epsilon$. In other words, unlike the susceptibility,
deviations from mean-field behaviour of the two-point function cannot be detected
to any order in $\epsilon$.

\begin{rk}
Models at and above the upper-critical dimension exhibit \emph{asymptotic freedom}.
In our context, this means that $\|K_j\|_{\Wcal_j} \to 0$ (as in \eqref{e:WKbd}),
$\nu_j, z_j \to 0$, and (as $m^2\downarrow0$) $g_{j_m} \to 0$. Below $d_c$, we do
not have asymptotic freedom, as reflected by the lack of exact asymptotics in
\eqref{e:chi-anom}. In some ways, this is advantageous (see \cite{Slad17}), but
in others it creates additional difficulties that must be overcome.
\end{rk}

%%%%%%%%%%%%%%%%%%%%%%%%%%%%%%%%%%%%%%%%%%%%%%%%%%%%%%%%%%%%%%%%%%%%%%%%%%%%%%%

\subsection{The \texorpdfstring{$O(n)$}{O(n)} model and self-avoiding walk}
\label{sec:hard-core}

Recall that the Hamiltonian of the $O(n)$ model was defined in \eqref{e:on-model}.
On $\Lambda$, it takes the form
\begin{equation}
H_J(\sigma)
	=
-\frac12 \sigma \cdot J \sigma.
\end{equation}
This was derived from the $|\varphi|^4$ model by taking a suitable
$g\to\infty$ limit. The restriction to small coupling $g$ is deeply embedded into
the method we use, but the Kac-Siegert transformation (see \cite{Bryd09}) offers
an alternative approach to the study of this model.

Namely, let $\Omega = (S^{n-1})^\Lambda$ and let $d\sigma$ denote the product measure on $\Omega$,
where $S^{n-1}$ is equipped with the uniform
% \footnote{In our derivation of the $O(n)$ model, we get an unnormalized surface measure on
% the sphere, but this is an unimportant difference.}
sphere measure. The partition function of the $O(n)$ model is given by
\begin{equation}
Z_J = \int_\Omega e^{-H_J(\sigma)} \; d\sigma.
\end{equation}
When $J$ is a positive-definite symmetric matrix, the Gaussian measure $d\mu_J(\varphi)$
with covariance $J$ is well-defined and satisfies the elementary identity
\begin{equation}
e^{-H_J(\varphi)}
	=
e^{\frac12 \sigma \cdot J \sigma}
	=
\int_{(\R^n)^\Lambda} e^{\sigma \cdot \varphi} \; d\mu_J(\varphi).
\end{equation}
Interchanging the order of integration, we can write
\begin{equation}
Z_J
	=
\int_{(\R^n)^\Lambda}
e^{-\sum_{x\in\Lambda} L(\varphi_x)}
% \left(\prod_{x\in\Lambda} L(\varphi_x)\right)
\; d\mu_J(\varphi),
\end{equation}
where
\begin{equation}
L(t)
	=
-\log
\int_{S^{n-1}} e^{\sigma_0 \cdot t} \; d\sigma_0,
	\quad
t\in\R^n
\end{equation}
is the negative logarithm of the Laplace transform of the sphere. Since $L$ is
a rotation- and reflection-invariant analytic function and $L(0) = 0$, we can
write
\begin{equation}
\label{e:logLaplace}
L(t) = \nu |t|^2 + g |t|^4 + \sum_{k=3}^\infty c_{2k} |t|^{2k}.
\end{equation}
Letting $J = (-\Delta + \gamma^2)^{-1}$, we have
\begin{equation}
d\mu_J(\varphi)
	\propto
e^{-\frac12 (\gamma^2 |\varphi|^2 + \varphi \cdot (-\Delta \varphi))}.
\end{equation}
Thus, we can express the partition function as a perturbed $|\varphi|^4$ model.

By a procedure as in Section~\ref{sec:reformulation}, the analysis of this model
can be reformulated in terms of the evolution of an effective interaction $Z_j$
with initial condition $Z_0 = (I_0 \circ K_0)(\Lambda)$. Once again, the initial
error coordinate $K_0$ will be coupled to $I_0$, but we expect that the critical
parameters $\nu_0^c, z_0^c$ can be identified by an implicit function argument
as in Section~\ref{sec:nu0z0c}.

However, estimates on $K_0$ (which are straightforward
to obtain by a more careful computation of \eqref{e:logLaplace}) indicate that
$K_0$ is not of order $g_0^3$, which is required to invoke Theorem~\ref{thm:flow-flow}.
Thus, an extension of the ideas in \cite{BBS-rg-flow} would be needed to study this
case.

\begin{rk}
Similarly, it is possible to re-cast the strictly self-avoiding walk as a
perturbation of weakly self-avoiding walk using a supersymmetric integral
representation obtained in \cite{BIS09}. The covariance of the form
$(-\Delta + \gamma^2)^{-1}$ in this case corresponds to a model of \emph{spread-out}
self-avoiding walk with exponentially decaying jump probabilities.
\end{rk}

%%%%%%%%%%%%%%%%%%%%%%%%%%%%%%%%%%%%%%%%%%%%%%%%%%%%%%%%%%%%%%%%%%%%%%%%%%%%%%%
%%%%%%%%%%%%%%%%%%%%%%%%%%%%%%%%%%%%%%%%%%%%%%%%%%%%%%%%%%%%%%%%%%%%%%%%%%%%%%%

\section{Other observable quantities}

Here we discuss a some problems concerning the models studied in this thesis.

%%%%%%%%%%%%%%%%%%%%%%%%%%%%%%%%%%%%%%%%%%%%%%%%%%%%%%%%%%%%%%%%%%%%%%%%%%%%%%%

\subsection{The correlation length}

Our results concerning the finite-order correlation lengths $\xi_p$ are insufficient
for recovering the predicted behaviour of the \emph{true} correlation length $\xi$.
The estimate \eqref{e:Rab-bound} gives super-polynomial decay of the errors in the
approximation \eqref{e:Gab-to-sum-Rqj} of the two-point function, but this is not
sufficient for studying $\xi$, which would need exponentially decaying errors.
The current estimates follow from the covariance bounds \eqref{e:scaling-estimate}
on the decomposition of \cite{Baue13a}. It may not be possible to improve the bounds
for this particular decomposition, this should be possible for the decomposition of
\cite{BGM04} (see \cite[p.~445]{BGM04}).

However, even if this were possible, exponentially decaying errors would require exponential
decay of the weights $\ell_j$ above the mass scale, which would contradict in a major way the
central hypotheses \ref{e:h-assumptions-IE} on these weights. Thus, it seems new ideas would
be needed to study the correlation length (note, however, that the correlation length for the
$1$-component $|\varphi|^4$ model was successfully studied by a renormalisation group method
in \cite{HT87}).

%%%%%%%%%%%%%%%%%%%%%%%%%%%%%%%%%%%%%%%%%%%%%%%%%%%%%%%%%%%%%%%%%%%%%%%%%%%%%%%

\subsection{Inversion of the Laplace transform}

One of the main motivations for studying the
susceptibility and finite-order correlation length for walks is the possibility
of recovering information about the growth of the partition function $c_T$ and
the mean-squared distance $\langle |X(T)|^2 \rangle$ as $T\to\infty$. In
particular, recalling the discussion in Section~\ref{sec:asymp}, one may
try to derive logarithmic corrections to the predicted scaling relations
\eqref{e:cT-asymp}--\eqref{e:XT-asymp} as a consequence of Theorem~\ref{thm:mr}(ii)--(iii).

This approach was successfully used in \cite{BI03c}, where the mean-squared displacement
of a hierarchical model of weakly self-avoiding walk is recovered by inversion of
the Laplace transform. This requires control over the two-point function in a
sector of the complex plane larger than what has been achieved on here on the
Euclidean lattice.

% Other: tricritical model, WSAW-SA phase diagram, magnetization, other correlation
% functions, scaling limits
