\chapter{The renormalisation group step}
\label{sec:RGstep}

\renewcommand{\Vpt}{\Vp_\pt}

The main purpose of this chapter is to prove that Theorem~\ref{thm:step-mr-fv}
holds with arbitrary $s$ in the definitions \eqref{e:elldef-zz} of the weights
$\ell, \ell_\sigma$. The proof of this fact requires involves numerous changes
to results in \cite{BS-rg-loc,BS-rg-IE,BS-rg-step}.
Consequently, the arguments presented here will not be completely self-contained;
for instance, we will not detail the construction of the renormalisation group
map, which makes up the bulk of \cite{BS-rg-step}.

For this reason, we begin in Section~\ref{sec:rgmech} with an informal overview
of some of the key ideas used in these papers. We will sometimes state supporting
results used to prove Theorem~\ref{thm:step-mr-fv} although we will not make direct
use of them in this thesis.

In Section~\ref{sec:Rpf1}, we define new norms above the mass scale and  prove
that they satisfy two key hypotheses required by results in \cite{BS-rg-IE}.
The remaining modifications needed to prove Theorem~\ref{thm:step-mr-fv} are
given in Section~\ref{sec:Rpf2}.

%%%%%%%%%%%%%%%%%%%%%%%%%%%%%%%%%%%%%%%%%%%%%%%%%%%%%%%%%%%%%%%%%%%%%%%%%%%%%%%
%%%%%%%%%%%%%%%%%%%%%%%%%%%%%%%%%%%%%%%%%%%%%%%%%%%%%%%%%%%%%%%%%%%%%%%%%%%%%%%

\section{Simplified renormalisation group step}
\label{sec:rgmech}

For this discussion, let us drop $\lambda, q, u$ from the notation and write
$\Vc = \Vp$. In this setting, our goal, given $(\Vp, K)$, is to construct
$(\Vp_+, K_+)$ such that, with $I = I(\Vp)$ and $I_+ = I(\Vp_+)$,
\begin{equation}
\label{e:IcircKsimp}
\Ex_+ \theta (I \circ K)(\Lambda) = (I_+ \circ K_+)(\Lambda)
\end{equation}
with $K \mapsto K_+$ contractive in some sense; this is needed not only to
control the error produced by $K$ in the computation of critical exponents
(e.g.\ recall the use of Lemma~\ref{lem:deriv-norm-bds} in Section~\ref{sec:chi-G-xi})
but also so that the map
$(\Vp, K) \mapsto (\Vp_+, K_+)$ can be iterated an arbitrary number of times as in
Theorem~\ref{thm:rhatflow}. The algebraic problem \eqref{e:IcircKsimp} admits
a multitude of solutions and the main difficulty is the construction of a
solution that admits appropriate bounds.

A possible definition of the map $(\Vp, K) \mapsto \Vp_+$ is suggested by perturbation
theory, as discussed in Section~\ref{sec:pt}. For now, let us suppose this is
the correct definition and set $\Ipt = I(\Vpt)$. Then we can define $K_+$ in terms
of $\Ipt$ such that \eqref{e:IcircKsimp} holds as follows.

For $B\in\Bcal$, let $\delta I(B) = \theta I(B) - \Ipt(B)$ and extend this to $X\in\Pcal$
by imposing block-factorization. Then by associativity of the circle product,
\begin{align}
(I \circ K)(\Lambda)
	&= \sum_{X\in\Pcal} (\Ipt + \delta I)^{\Lambda \setminus X} K(X) \\
	&= [(\Ipt \circ \delta I) \circ K](\Lambda) \\
	&= [\Ipt \circ (\delta I \circ K)](\Lambda).
\end{align}
Note that the fluctuation fields at scale $j$ have been integrated out in the
definition of $\Vpt$. Thus,
\begin{equation}
\Ex_+ \theta(I \circ K)(\Lambda) = [\Ipt \circ \tilde K](\Lambda)
\end{equation}
with
\begin{equation}
\tilde K = \Ex_+ (\delta I \circ \theta K).
\end{equation}
This has the form \eqref{e:IcircKsimp} but with the circle product on the right-hand
side at the wrong scale. This is remedied by a simple resummation:
\begin{align}
\Ex_+ \theta (I \circ K)(\Lambda)
	= \sum_{Y\in\Pcal} \Ipt(\Lambda \setminus Y) \tilde K(Y)
	= \sum_{X\in\Pcal_+} \Ipt^{\Lambda \setminus X} K_+(X)
	= (\Ipt \circ K_+)(\Lambda)
\end{align}
where
\begin{equation}
\label{e:Kplus3}
K_+(X) = \sum_{Y \in \bar\Pcal(X)} \Ipt^{X \setminus Y} \tilde K(Y).
\end{equation}
Here, $\bar\Pcal(X)$ is the collection of polymers $Y\in\Pcal(X)$ such that
$X$ is the smallest polymer in $\Pcal_+$ containing $Y$.

%%%%%%%%%%%%%%%%%%%%%%%%%%%%%%%%%%%%%%%%%%%%%%%%%%%%%%%%%%%%%%%%%%%%%%%%%%%%%%%

\subsection{Main contributions to \texorpdfstring{$K_+$}{K+}}

By expanding the circle product in the definition of $\tilde K$, we can write
\begin{equation}
K_+(X) = \Ipt^X [h(X) + k(X) + R(X)]
\end{equation}
where (letting $\bar\Ccal(X) = \Ccal\cap\bar\Pcal(X)$)
\begin{align}
\label{e:hpt}
h(X)	&= \sum_{Y\in\bar\Pcal(X)} \Ipt^{-Y} \Ex_+ \delta I(Y) \\
\label{e:k}
k(X)	&= \sum_{Y\in\bar\Ccal(X)} \Ipt^{-Y} \Ex_+ \theta K(Y)
\end{align}
and $R$ is the remainder. If $\delta I$ and $K$ are sufficiently small (in an
appropriate sense), then it is reasonable to view the terms $h$ and $k$ as
first-order contributions to $K_+$. Note that the sum defining $k$ is restricted
to connected polymers as component-factorization of $K$ should imply that terms
involving $K$ on disconnected polymers will be of higher order in this sense.
Our main task then is to bound $h$ and $k$.

\subsubsection{Perturbative contribution and covariance bound}

By the definition \eqref{e:Idef} of $I$, we expect that $\delta I = O(\delta\Vp^3)$
where $\delta\Vp = \theta\Vp - \Vpt$. Indeed, a version of this statement is true
by \cite[Proposition~\ref{IE-prop:ip}]{BS-rg-IE}.
Thus, in order bound the contrinution $h$ defined in \eqref{e:h}, we must consider
the size of $\delta\Vp$. By definition,
\begin{equation}
\|\delta\Vp(B)\|_{T_0(\ell)}
	\le
\|\theta\Vp(B) - \Vp(B)\|_{T_0(\ell)} + \|\Vp(B) - \Vpt(B)\|_{T_0(\ell)}.
\end{equation}
By \eqref{e:Vpt-def}, $\Vpt - \Vp = (\Ex \theta\Vp - \Vp) - P(\Vp)$ and
the first term on the right-hand side can be bounded term-by-term. For
instance, the difference between a single quartic term and its expectation
is given by
\begin{equation}
\varphi_x^4 - \Ex \theta\varphi_x^4
	=
6 C_{00} \varphi_x^2 + 3 C_{00}^2.
\end{equation}
Covariance terms such as those above occur at higher-order and can be
estimated using Proposition~\ref{prop:Cdecomp}.

The method of \cite{BS-rg-IE} is more flexible, however, and does not require
the precise bounds in \eqref{e:scaling-estimate}. Rather, necessary bounds on
the covariance and its derivatives are encoded in the hypothesis
\cite[\eqref{IE-e:CLbd}]{BS-rg-IE} on the $\Phi(\ell)$
norm estimate of the covariance. This constraint naturally ensures that the $T_0(\ell)$
norm estimates properly reflects the size of the expectation of a field as discussed
in Section~\ref{sec:weights}.
Our main bound on the covariance, which extends \cite[\eqref{IE-e:CLbd}]{BS-rg-IE},
will be stated in Section~\ref{sec:Cbds}.

\begin{rk}
The generality provided by predicating the results of \cite{BS-rg-IE} on a
norm estimate on the covariance is very useful, e.g.\ as in \cite{Slad17}.
\end{rk}

\subsubsection{Extraction and contraction}

The term $k$ in \eqref{e:k} is the contribution to $K_+$ that is linear in $K$. Thus, its
control is essential to obtaining the contractivity estimate \eqref{e:DKkappa}.

In the simple case that $K = 0$, we will have $K_+ = \Ipt h$, which is a Taylor
remainder that contains terms at all orders in the fields. Thus, it includes
relevant and marginal terms as well as non-local irrelevant terms.
% For instance, integration of of the irrelevant term $\sum_{x\in B} \varphi_x^6$ yields
% \begin{equation}
% \binom{6}{2} \sum_{x\in B} \varphi_x^4 C_{00},
% \end{equation}
% which is marginal.
The size and number of such terms will prevent $K$ from shrinking under the action
of the renormalisation group unless they are somehow dealt with.

This situation is not unlike what was discussed in Section~\ref{sec:pt}, where
integration of $e^{-\Vp}$ yielded a non-local term that was split into a local
part $P$ and a remainder $W$ using the operator $\Loc_x$ constructed in
\cite{BBS-rg-pt}. This same operator is used in \cite{BS-rg-step} to \emph{extract}
a marginal/relevant part from $K$ prior to integration.

Thus, the true definition of the renormalisation group map constructed in
\cite{BS-rg-step} involves several more steps than \eqref{e:Kplus3}. In fact,
the definition of the map $(\Vp, K) \mapsto (\Vp_+, K_+)$ is a composition of
$6$ maps, called Maps 1--6. In Maps 1--2, the operator $\Loc$ is used to perform
extraction. Map 3 then implements the expectation and change of scale in
\eqref{e:Kplus3}.

Since a sufficiently large portion of the marginal and relevant terms have been
extracted, the expectation now successfully causes $K$ to shrink as it should
based on the heuristics of Section~\ref{sec:dimensional}. The fact that irrelevant
terms shrink under expectation and change of scale is formally captured by
\cite[Proposition~\ref{IE-prop:cl}]{BS-rg-IE}, which we refer to as the
\emph{crucial contraction}. In Section~\ref{sec:cc}, we prove that the
crucial contraction continues to hold when $s > 1$.

\begin{rk}
In order to maintain the form \eqref{e:IcircKsimp}, the extraction step in Maps 1--2
must make a corresponding adjustment to $\Vp$. This adjustment ultimately results in a
map $\Vp \mapsto \Vp_+$, which is a small perturbation of $\Vp$, as expressed
by the first bound of \eqref{e:RKplus} (recall \eqref{e:Rplusdef}).
\end{rk}

%%%%%%%%%%%%%%%%%%%%%%%%%%%%%%%%%%%%%%%%%%%%%%%%%%%%%%%%%%%%%%%%%%%%%%%%%%%%%%%
%%%%%%%%%%%%%%%%%%%%%%%%%%%%%%%%%%%%%%%%%%%%%%%%%%%%%%%%%%%%%%%%%%%%%%%%%%%%%%%

\section{Improved norm}
\label{sec:Rpf1}

% The proof of Theorem~\ref{thm:step-mr-fv} is based on the observation that
% it is possible to use the parameters \refeq{elldef-zz} in the norm used in
% \cite{BS-rg-IE}, instead of the $s=0$ version used previously.
In this section, we first
state improved covariance estimates, thereby indicating why it is possible
to improve the norm.
This leads to a discussion of simplified norm pairs beyond the mass
scale.  A lemma concerning the fluctuation-field regulator indicates why the
simplification is possible.
In the following, we use the notation appropriate for the spin field
$\varphi \in (\R^n)^\Lambda$ for $n \ge 1$; only notational modifications are needed for
$n=0$.

%%%%%%%%%%%%%%%%%%%%%%%%%%%%%%%%%%%%%%%%%%%%%%%%%%%%%%%%%%%%%%%%%%%%%%%%%%%%%%%

\subsection{Covariance bounds}
\label{sec:Cbds}

% The estimate in \cite{ST-phi4} which yields the $s = 0$ case of \refeq{Rab-bound}
% uses the norms defined in \cite{BS-rg-IE}.
Recall from Section~\ref{sec:Ncal} that the $\Phi_j(\ell_j)$ norm defined by
\begin{equation}
\lbeq{phinorm}
\|\varphi\|_{\Phi_j(\ell_j)}
=
\ell_j^{-1}
\sup_{x\in \Lambda}
\sup_{|\alpha|_1  \le p_\Phi}
L^{j|\alpha|_1}
|\nabla^{\alpha} \varphi_x|,
\end{equation}
Since we are working with functions $F$ of a boson field $\varphi\in(\R^n)^\Lambda$, we have
$F_{\vec x, \vec y} = 0$ whenever $\vec y \ne \varnothing$. Thus, \eqref{e:pairing}
becomes
\begin{equation}
\label{e:pairing-re}
\langle F, g \rangle_\varphi
	=
\sum_{|\vec x| \le p_\Ncal} \frac{1}{|\vec x|!} F_{\vec x}(\varphi) g_{\vec x}.
\end{equation}
% which depends on the parameter $\ell_j$,
% and on the maximal number of discrete derivatives $p_\Phi$
% (fixed to be at least $4$ in \cite{BS-rg-IE}).
As in \refeq{elldef-zz}, we now define
\begin{align}
\label{e:elldef}
\ell_j &= \ell_0 L^{-j - s (j - j_m)_+}, \quad
\ell_{\sigma,j}
=
\ell_{j \wedge j_{x}}^{-1} 2^{(j - j_{x})_+} \ggen_j.
\end{align}
The analysis of \cite{BS-rg-IE,BS-rg-step} uses the norm parameters $\ell_j$ and $\ell_{\sigma,j}$ with $s = 0$.
To distinguish these from our
new choice \refeq{elldef} of $\ell_j$ and $\ell_{\sigma,j}$, we write
\begin{equation}
\label{e:ell-old}
    \ell_j^\oldrm = \ell_0 L^{-j},
    \quad
    \ell_{\sigma,j}^\oldrm  =
    (\ell_{j \wedge j_{x}}^{\rm old})^{-1}2^{(j - j_{x})_+}\ggen_j.
\end{equation}

% In the more general terminology and notation of \cite{BS-rg-norm,BS-rg-IE},
We may regard a covariance $C_j$ in the decomposition \eqref{e:NCj}
as a test function depending on
two arguments $x,y$, and with this identification its $\Phi_j(\ell_j)$
norm is
\begin{equation}
    \label{e:Phinorm}
    \|C_j\|_{\Phi_{j}(\ell_j)}  =
    \ell_j^{-2}
    \sup_{x,y\in \Lambda}
    \;
    \sup_{|\alpha|_1 + |\beta|_1 \le p_\Phi}
    L^{(|\alpha|_1+  |\beta|_1)j}
    |\nabla_x^{\alpha} \nabla_y^{\beta} C_{j;x,y}|.
\end{equation}
The purpose of the $\Phi_j(\ell_j)$ norm is to measure the size of typical
fluctuation fields $\varphi$ with covariance $C_j$.
The parameter $\ell_j$ is chosen so that the norm of a typical field should
be $O(1)$, independent of $j$.

The following lemma justifies our choice of $\ell_j$
in \refeq{elldef}, by showing that the
bound \cite[\eqref{IE-e:CLbd}]{BS-rg-IE}, proved there only for the $s=0$ version
$\ell_j^\oldrm$ of \refeq{ell-old},
remains true with the stronger
choice of norm parameter $\ell_j$ that permits arbitrary $s \ge 0$.
%The sequence $\chicCov_j$ in the lemma is called $\chi_j$ in \cite{BS-rg-IE}, but here we use
%a different symbol to avoid confusion with the susceptibility.
% In its statement, the bounded sequence $\chicCov_j$ decays exponentially after the
% mass scale and may be taken to be equal to
% $2^{-(j-j_m)_+}$; its details are given
% in \cite[Section~\ref{IE-sec:frp}]{BS-rg-IE} (where it is called $\chi_j$ rather
% than $\chicCov_j$).

\begin{lemma}
[{Extension of \cite[\eqref{IE-e:CLbd}]{BS-rg-IE}}]
\label{lem:Cbd}
Given $\ellconst \in (0, 1]$, $\ell_0$ can be chosen large (depending on $L,\ellconst,s$)
so that
\begin{equation}
\lbeq{Cbd}
\|C_j\|_{\Phi_{j}(\ell_j)} \leq \min(\ellconst, \chicCov_j).
\end{equation}
\end{lemma}

\begin{proof}[Proof of Lemma~\ref{lem:Cbd}]
For $d=4$, insertion of \refeq{scaling-estimate} into \refeq{Phinorm} gives
\begin{equation}
    \label{e:Phinorm2}
    \|C_j\|_{\Phi_{j}(\ell_j)}
    \le
    c
    L^{p_\Phi}
    \ell_j^{-2}(1+m^2L^{2(j-1)})^{-k}
    L^{-2(j-1)}.
\end{equation}
With $s=0$ in \eqref{e:elldef}, \refeq{Phinorm2} gives
$\|C_j\|_{\Phi_{j}(\ell_j)} \le c_L \ell_0^{-2} (1+m^2L^{2(j-1)})^{-k}$
for an $L$-dependent constant $c_L$ (whose value may now change from line to line).
The estimate \cite[\eqref{IE-e:CLbd}]{BS-rg-IE}
is wasteful in that it does not make any use of the factor
$(1+m^2L^{2(j-1)})^{-k}$ in \refeq{Phinorm2} beyond extraction of the factor $\chicCov_j$.
% To improve this, we now allow arbitrary $s$, and fix the arbitrary parameter $k$ to be $k=s+1$
% in \refeq{Phinorm2} so that
% \begin{equation} \label{e:mass-decay}
% (1 + m^2 L^{2j})^{-k} \le c_L L^{-2(s+1)(j - j_m)_+}.
% \end{equation}
We insert \refeq{mass-decay} and the definition $\ell_j=\ell_0 L^{-j-s(j-j_m)_+}$ from
\refeq{elldef} into
\eqref{e:Phinorm2}, to conclude that there exists $c_0 = c_0(s, L)$ such that
\begin{equation}
    \|C_j\|_{\Phi_{j}(\ell_j)} \leq c_0 \ell_0^{-2} L^{-2(j - j_m)_+}
    .
\end{equation}
By definition of $\chicCov_j$ (see \cite[Section~\ref{IE-sec:frp}]{BS-rg-IE}),
$L^{-2(j - j_m)_+}$ is bounded by a multiple of $\chicCov_j$.
It thus suffices to choose $\ell_0$ large enough that
$\ell_0^2 \ge c_0 \ellconst^{-1}$.
\end{proof}

%%%%%%%%%%%%%%%%%%%%%%%%%%%%%%%%%%%%%%%%%%%%%%%%%%%%%%%%%%%%%%%%%%%%%%%%%%%%%%%

\subsection{New choice of norm beyond the mass scale}
\label{sec:newnorm}

A field $\varphi$ can be viewed as a test function supported on sequences with
$|\vec x| = 1$ and $|\vec y| = 0$.
In particular, $\|\varphi\|_\Phi$ is defined as the norm of a test function.
As in \cite[(\ref{IE-e:PhiXdef})]{BS-rg-IE}, we use the localised version
of \eqref{e:phinorm}, defined for subsets $X \subset \Lambda$  by
\begin{align}
\label{e:PhiXdef}
    \|\varphi\|_{\Phi_j(X)}
    &=
    \inf \{ \|\varphi -f\|_{\Phi_j} :
    \text{$f \in \C^\Lambda$ such that $f_{x} = 0$
    $\forall x\in X$}\}.
\end{align}
A similar definition is given for general test functions.
% A \emph{small set} is defined to be a connected polymer $X \in \Pcal_j$
% consisting of at most $2^d$ blocks (the specific number $2^d$ plays no
% direct role here),
% and $\Scal_j \subset \Pcal_j$ denotes the set of small sets.
% The \emph{small set neighbourhood} of $X \subset \Lambda $ is
% the enlargement of $X$ defined by
% $
%     X^{\Box}
% =
%     \bigcup_{Y\in \Scal_{j}:X\cap Y \not =\varnothing } Y$.
Given $X \subset \Lambda$ and $\varphi \in (\R^n)^{\Lambda}$,
we recall from \cite[\eqref{IE-e:GPhidef}]{BS-rg-IE}
that the
\emph{fluctuation-field regulator} $G_j$
is defined by
\begin{align}
\label{e:GPhidef}
    G_j(X,\varphi)
    =
    \prod_{x \in X} \exp
    \left(|B_{x}|^{-1}\|\varphi\|_{\Phi_j (B_{x}^\Box,\ell_j )}^2 \right)
    ,
\end{align}
where $B_{x}\in \Bcal_j$ is the unique block that contains $x$,
and hence $|B_x| = L^{dj}$.
The \emph{large-field regulator} is defined in \cite[\eqref{IE-e:9Gdef}]{BS-rg-IE} by
\begin{align}
\label{e:9Gdef}
    \tilde G_j  (X,\varphi)
    =
    \prod_{x \in X}
    \exp \left(
    \frac 12 |B_{x}|^{-1}\|\varphi\|_{\tilde\Phi_j (B_{x}^\Box,\ell_j)}^2
    \right)
    .
\end{align}
The $\tilde\Phi_j$ norm appearing on the right-hand side of \refeq{9Gdef} will be defined
in Section~\ref{sec:cc}.
% similar to the $\Phi_j$ norm, with the important difference that it is insensitive to
% shifts by linear test functions; see \cite[\eqref{IE-e:Phitilnorm}]{BS-rg-IE} for the precise definition.
The two regulators serve as weights in the \emph{regulator norms} of
\cite[Definition~\ref{IE-def:Gnorms}]{BS-rg-IE}.
The regulator norms are defined,  with ${\sf t} \in (0,1]$ and
for $F \in \Ncal(X^\Box)$ by
% in the space $\Ncal(X^\Box)$ of functionals of the field (see \cite[\eqref{norm-e:NXdef}]{BS-rg-norm}),
\begin{align}
\label{e:Gnormdef1}
    \| F\|_{G_j(\ell_j)}
    &=
    \sup_{\varphi \in (\R^n)^\Lambda}
    \frac{\|F\|_{T_{\varphi,j}(\ell_j)}}{G_{j}(X,\varphi)}
    ,
\\
\label{e:Gnormdef2}
    \|F\|_{\tilde G_j^{\Gtilp}(h_j)}
    &=
    \sup_{\varphi \in (\R^n)^\Lambda}
    \frac{\|F \|_{T_{\varphi,j}(h_j)}}{\tilde{G}^{\Gtilp}_{j}(X,\varphi)}
    .
\end{align}
The parameter $\ell_j$ that appears in the regulators \refeq{GPhidef}--\refeq{9Gdef} and
in the numerator of \refeq{Gnormdef1} was taken to be $\ell_j^\oldrm$ in \cite{BS-rg-IE},
but now we use $\ell_j$ instead. As in \cite{BS-rg-IE},
the parameter $h_j$ and its observable counterpart $h_{\sigma,j}$ are given by
\begin{align}
\label{e:h}
    h_{j} &= k_0 \ggen_j^{-1/4}L^{-j},
    \quad
    h_{\sigma,j}  = (\ell_{j \wedge j_{x}}^{\rm old})^{-1}
    2^{(j - j_{x})_+}\ggen_j^{1/4}.
\end{align}

In \cite{BS-rg-IE}, estimates on $\|\cdot\|_{j+1}$ are given in terms of
$\|\cdot\|_j$, where the pair $(\|\cdot\|_j, \|\cdot\|_{j+1})$ refers to
either of the norm pairs
\begin{equation}
\label{e:np1}
    \|F\|_j = \|F\|_{G_j(\ell_j^\oldrm)}
    \quad \text{and} \quad
    \|F\|_{j+1} = \|F\|_{T_{0,j+1}(\ell_{j+1}^\oldrm)},
\end{equation}
or
\begin{equation}
\label{e:np2}
    \|F\|_j = \|F\|_{\tilde{G}_j(h_j)}
    \quad \text{and} \quad
    \|F\|_{j+1} = \|F\|_{\tilde{G}_{j+1}^{\Gtilp}(h_{j+1})}.
\end{equation}
We will show that, above the mass scale $j_m$, the results of \cite{BS-rg-IE} hold with
both norm pairs in \eqref{e:np1} and \eqref{e:np2} replaced by the single new norm pair
\begin{equation}
\label{e:npmass}
    \|F\|_j = \|F\|_{G_j(\ell_j)}
    \quad \text{and} \quad
    \|F\|_{j+1} = \|F\|_{G_{j+1}(\ell_{j+1})},
\end{equation}
with the improved $\ell_j$ of \eqref{e:elldef} with $s > 1$ fixed as large as desired.


The use of two norm pairs adds intricacy to \cite{BS-rg-IE,BS-rg-step}.
The pair \refeq{np1} is insufficient, on its own, because the scale-$(j+1)$ norm
is the $T_0$ semi-norm which controls only small fields, and an estimate in this norm
does not imply an estimate for the $G_{j+1}$ norm.  The norm pair \refeq{np2} is
used to supplement the norm pair \refeq{np1}, and estimates in both of the scale-$(j+1)$
norms can be combined to provide an estimate for the $G_{j+1}$ norm.  This then
sets the stage for the next renormalisation group step.  Above the mass scale,
the use of \refeq{npmass} now bypasses many issues.  For example, for $j>j_m$
 the $\Wcal_j$ norm of \cite[\eqref{step-e:9Kcalnorm}]{BS-rg-step} is replaced
 simply by the $\Fcal_j(G)$ norm, and there is no need for the $\Ycal_j$ norm of
\cite[\eqref{step-e:Ycaldef}]{BS-rg-step} nor for \cite[Lemma~\ref{step-lem:KKK}]{BS-rg-step}.

The need for both norm pairs \eqref{e:np1}--\eqref{e:np2} is discussed in
\cite[Section~\ref{IE-sec:lfp}]{BS-rg-IE} and is related to the
so-called \emph{large-field problem}. Roughly speaking, the
norm pair \refeq{np2} is used to take advantage of the quartic term in the interaction to
suppress the effects of large values of the fields. This approach
relies on the fact that the interaction polynomial is dominated by the
quartic term in the $h$-norm, as expressed by
\cite[\eqref{IE-e:tau2dom}]{BS-rg-IE}, together with the lower bound
\cite[\eqref{IE-e:epVbark0}]{BS-rg-IE} on the quartic term.
However, above the mass scale, large fields are naturally suppressed
by the rapid decay of the covariance.
This idea is captured in Lemma~\ref{lem:mart} below, which replaces
\cite[Lemma~\ref{IE-lem:mart}]{BS-rg-IE} above the mass scale.
The regulators in its statement are defined by \refeq{GPhidef} with the $s$-dependent
$\ell_j$ of \refeq{elldef}.


\begin{lemma}[{Replacement for \cite[Lemma~\ref{IE-lem:mart}]{BS-rg-IE}}]
\label{lem:mart}
Let $X \subset \Lambda$ and assume that $s > 1$.
For any $q >0$, if $L$ is sufficiently large depending on $q$, then for $j_m \leq j < N$,
\begin{equation}
\label{e:mart}
    G_{j}(X, \varphi)^{q}
    \le
    G_{j+1}(X, \varphi).
\end{equation}
\end{lemma}
\begin{proof}
By \eqref{e:GPhidef}, it suffices to show that, for any scale-$j$ block $B_j$ and any scale-$(j+1)$ block $B_{j+1}$ containing $B_j$,
\begin{equation}
q \|\varphi\|^2_{\Phi_j (B_j^\Box,\ell_j )}
\leq
L^{-4} \|\varphi\|^2_{\Phi_{j+1} (B_{j+1}^\Box,\ell_{j+1})}.
\end{equation}
In fact, since $\|\varphi\|_{\Phi_j (B_j^\Box,\ell_j )}
\leq \|\varphi\|_{\Phi_j (B_{j+1}^\Box,\ell_j )}$ by definition,
it suffices to prove the above bound with $B_j$ replaced by $B_{j+1}$ on the left-hand side.
According to the definition of the norm in \eqref{e:PhiXdef},
to show this it suffices to prove that
\begin{equation}
\lbeq{martwant}
q \|\varphi\|_{\Phi_j(\ell_j)}^2 \leq L^{-4} \|\varphi\|_{\Phi_{j+1}(\ell_{j+1})}^2
\end{equation}
(then we replace $\varphi$ by $\varphi -f$ in the above and take the infimum).

By definition,
\begin{equation}
\|\varphi\|_{\Phi_j(\ell_j)}
\le
\ell_j^{-1} \ell_{j+1}
\sup_{x\in \Lambda} \sup_{|\alpha| \leq p_\Phi}
\ell_{j+1}^{-1}
L^{(j+1) |\alpha|}
|\nabla^\alpha \varphi_x|,
\end{equation}
with the inequality due to replacement of $L^{j |\alpha|}$ on the left-hand
side by $L^{(j+1) |\alpha|}$ on the right-hand side.
Since $\ell_j^{-1} \ell_{j+1} = L^{-1 - s \1_{j \geq j_m}}$,
\begin{equation}
\|\varphi\|_{\Phi_j(\ell_j)} \leq L^{-1 - s \1_{j \geq j_m}} \|\varphi\|_{\Phi_{j+1}(\ell_{j+1})}.
\end{equation}
Thus,
\begin{equation}
q \|\varphi\|_{\Phi_j(\ell_j)}^2
\leq q L^{-4} L^{2 - 2s \1_{j \geq j_m}} \|\varphi\|^2_{\Phi_{j+1}(\ell_{j+1})},
\end{equation}
and then \refeq{martwant}
follows once $L$ is large enough that $q L^{2 - 2s} \leq 1$.
\end{proof}

\begin{rk}
The elimination of the $h$-norm after the mass scale is more than a convenience.
It becomes a necessity when we improve the $\ell$-norm.
Briefly, the reason is as follows. In the proof of
\cite[Lemma~\ref{step-lem:KKK}]{BS-rg-step}, the ratio
$\ell_{\sigma}/h_{\sigma}$
must be bounded. For this, we would need
to increase $h_{\sigma}$
beyond the mass scale  (since $\ell_{\sigma}$ has been increased).
This forces a compensating decrease in $h$
beyond $j_m$, to keep the product $hh_{\sigma}$ bounded for stability
(as in Section~\ref{sec:stability1}
below). But if we do this, we lose the lower bound required on $\epsilon_{g\tau^2}$
required for stability in the $h$-norm (see \cite[\eqref{IE-e:epVbardefz-app}]{BS-rg-IE}).
\end{rk}

%%%%%%%%%%%%%%%%%%%%%%%%%%%%%%%%%%%%%%%%%%%%%%%%%%%%%%%%%%%%%%%%%%%%%%%%%%%%%%%
%%%%%%%%%%%%%%%%%%%%%%%%%%%%%%%%%%%%%%%%%%%%%%%%%%%%%%%%%%%%%%%%%%%%%%%%%%%%%%%

\section{Proof of Theorem~\ref{thm:step-mr-fv}}
\label{sec:Rpf2}

In this section, we show that Theorem~\ref{thm:step-mr-fv}
holds, thereby completing the proof of Proposition~\ref{prop:R}.
The key steps in
the proof of the $s = 0$ case of Theorem~\ref{thm:step-mr-fv}
are contained in \cite{BS-rg-IE,BS-rg-step}.
% To do so, we make use of the renormalisation group flow $(V_j,K_j)$
% constructed in \cite{BS-rg-step,BBS-rg-flow,BBS-saw4-log} and used
% in \cite{BBS-saw4-log,BBS-phi4-log,BBS-saw4,ST-phi4}.
% This includes the flow of the observable coupling constants $q_0, q_x$, which include a non-perturbative contribution from the remainder terms \eqref{e:Rabdef}, defined here in the same way as in \cite{ST-phi4}.
% The estimates on the renormalisation group flow and remainder terms provided in those papers
% are consequences of the estimates proved in \cite{BS-rg-IE,BS-rg-step}
% with norm parameter $\ell_{\sigma,j}^\oldrm$.
Our main objective in this section is to show that the results in \cite{BS-rg-IE,BS-rg-step}
continue to hold with the new norm parameters $\ell_j,\ell_{\sigma,j}$.
To this end, we may and do use the fact that the estimates of \cite{BS-rg-IE} have already been
established with the old norm parameters.

In the following,
we indicate the changes in the analysis of
\cite{BS-rg-IE,BS-rg-step} that arise due to the new choice of norm parameters \refeq{elldef}
beyond the mass scale, and due to the reduction from two norm pairs to one.
This requires repeated reference to previous papers.

%%%%%%%%%%%%%%%%%%%%%%%%%%%%%%%%%%%%%%%%%%%%%%%%%%%%%%%%%%%%%%%%%%%%%%%%%%%%%%%

\subsection{Norm parameter ratios}
\label{sec:norm-parameter-ratios}

The analysis of \cite{BS-rg-IE} assumes that the norm parameters $\h_j,\h_{\sigma,j}$,
for $\h = \ell$ or $\h = h$,
satisfy the estimates \cite[\eqref{IE-e:h-assumptions}]{BS-rg-IE}; these assert that
\begin{align}
\label{e:h-assumptions-IE}
    \h_j \ge \ell_{j},
    \quad\quad
    \frac{\h_{j+1}}{\h_j}
    &\le 2 L^{-1},
    \quad\quad
    \frac{\h_{\sigma,j+1}}{\h_{\sigma,j}}
    \le
    {\rm const}\,
    \begin{cases}
     L  & (j < j_{x})
     \\
     1 & (j \ge j_{x}).
     \end{cases}
\end{align}
We do not change $\h_j$ or $\h_{\sigma,j}$ for $j$ below the mass scale, so there
can be no difficulty until above the mass scale.  Above the mass scale, the parameters
$h_j,h_{\sigma,j}$ are eliminated, and requirements involving them become vacuous.
Thus, for \refeq{h-assumptions-IE}, we need only verify the second and third
inequalities for the case $\h=\ell$.
By definition,
\begin{equation}\label{e:h-assumptions}
\frac{\ell_{j+1}}{\ell_j} = L^{-(1 + s \1_{j \geq j_m})},
\qquad
\frac{\ell_{\sigma, j+1}}{\ell_{\sigma,j}} = \frac{\ggen_{j+1}}{\ggen_j}
\times
\begin{cases} L^{1 + s \1_{j \geq j_m}} & (j < j_x) \\ 2 & (j \geq j_x). \end{cases}
\end{equation}
According to \cite[\eqref{IE-e:gbarmono}]{BS-rg-IE},
$\frac 12 \ggen_{j+1} \le \ggen_j \le 2 \ggen_{j+1}$.
Thus, the second estimate of \eqref{e:h-assumptions-IE}
is satisfied (the ratio being improved when $j\ge j_m$),
while the third is \emph{not} when $s > 1$ and $j_m < j_x$.
This potentially dangerous third
estimate in \refeq{h-assumptions-IE} is used to prove the scale monotonicity lemma
\cite[Lemma \ref{IE-lem:Imono}]{BS-rg-IE}, as well
as the crucial contraction.
We discuss
\cite[Lemma \ref{IE-lem:Imono}]{BS-rg-IE} next, and return to the crucial contraction
in Section~\ref{sec:cc} below.

\paragraph{\cite[Lemma \ref{IE-lem:Imono}]{BS-rg-IE}}
There is actually no problem with the scale monotonicity lemma.
Indeed, for the case $\alpha =ab$ of the proof of
\cite[Lemma \ref{IE-lem:Imono}]{BS-rg-IE}, the hypothesis that
$\pi_{0x}F=0$ for $j<j_{x}$ ensures that this case only relies on the dangerous
estimate for $j \ge j_x$ where the danger is absent in \refeq{h-assumptions}.
For the cases $\alpha=a$ and $\alpha =b$ of the proof of
\cite[Lemma \ref{IE-lem:Imono}]{BS-rg-IE}, what is
important is the inequality
$\ell_{\sigma,j+1}\ell_{j+1} \le {\rm const}\, \ell_{\sigma,j}\ell_{j}$, which
continues to hold with
\refeq{elldef} for all scales $j$, both above and below the mass scale, since
the products in this inequality are the same for the new and the old choices of $\ell$.
So \cite[Lemma \ref{IE-lem:Imono}]{BS-rg-IE} continues to hold with the choice
\refeq{elldef}.
In addition,
\begin{equation}
\label{e:norm-change}
\|F\|_{T_\varphi(\ell_j)} \leq  \|F\|_{T_\varphi(\ell^\oldrm_j)}.
\end{equation}
This strengthened special case of the first inequality of
\cite[\eqref{IE-e:scale-change}]{BS-rg-IE} (strengthened due to the constant
$1$ on the right-hand side of \refeq{norm-change} compared to
the generic constant in \cite[\eqref{IE-e:scale-change}]{BS-rg-IE})
can be seen from an examination of the proof of the $\alpha =a,b$ case of
\cite[Lemma \ref{IE-lem:Imono}]{BS-rg-IE}, together with the observation that
$\ell_{\sigma,j}\ell_{j} = \ell_{\sigma,j}^{\rm old} \ell_j^{\rm old}$ by definition.

%%%%%%%%%%%%%%%%%%%%%%%%%%%%%%%%%%%%%%%%%%%%%%%%%%%%%%%%%%%%%%%%%%%%%%%%%%%%%%%

\subsection{Stability domains}
\label{sec:stability1}

In \cite[\eqref{IE-e:DV1-bis}]{BS-rg-IE}, an extension of the domain \eqref{e:DVdef}
is defined. By some abuse of notation, we will also denote this extended domain by
$\DV_j$. We modify $\DV_j$ only for the coupling constant $q$, by replacing $r_q$
in \cite[\eqref{IE-e:h-coupling-def-1-bis}]{BS-rg-IE} by
\begin{equation}
\lbeq{newrq}
L^{2j_x + 2 s (j_x - j_m)_+} 2^{2(j-j_x)} r_{q,j} =
\begin{cases}
  0 & j < j_x \\
  C_{\DV}  & j \ge j_x.
\end{cases}
\end{equation}

\paragraph{\cite[Proposition~\ref{IE-prop:monobd}]{BS-rg-IE}}
With \refeq{newrq}, \cite[Proposition~\ref{IE-prop:monobd}]{BS-rg-IE}
as it pertains to $\h=\ell$ (omitting all reference to $\h=h$) continues to hold beyond
the mass scale by the same proof.
In particular, with the smaller choice for the domain of $q$,
\cite[\eqref{IE-e:qhsig}]{BS-rg-IE} holds with the larger $s$-dependent $\ell_{\sigma,j}$.

\medskip
Note that we do not need to change the domain of $\lambda$.
This is because the bound \cite[\eqref{IE-e:hsigh}]{BS-rg-IE}
continues to hold with the new norm parameters. Indeed, while $\ell_j$
and $\ell_{\sigma,j}$ have been modified, their product $\ell_j \ell_{\sigma,j}$
has not.
This guarantees that the $T_0$ semi-norm
$\|\sigma\bar\varphi_a\|_{T_0} = \ell_\sigma \ell$ remains identical to what it was
with the old norm parameters, and therefore there is no new stability requirement
arising from this.

The choice \refeq{newrq} places a more stringent requirement on the domain
than does the $s=0$ version.  To see that this requirement is actually met
by the renormalisation group flow,
we note a minor improvement to the proof of \cite[Lemma~\ref{step-lem:K7a}(ii)]{BS-rg-step},
where the bound $|\delta q| \leq c L^{-2j}$ is used to show that $v(X)$
(defined there) satisfies
\begin{equation}
\lbeq{vXbd}
\|v(X)\| \leq c L^{-2j} (\ell_{\sigma,j}^\oldrm)^2 \leq c'.
\end{equation}
Here the factor $L^{-2j}$ arises
as a bound on the covariance $C_{j+1;00}$ in the perturbative flow
\cite[\eqref{pt-e:qpt2}]{BS-rg-IE} of $q$ and it can therefore be improved to $L^{-2j-2s(j - j_m)_+}$
by Lemma~\ref{lem:Cbd}.
Thus also with $\ell^\oldrm, \ell_{\sigma}^{\rm old}$ replaced by $\ell,\ell_{\sigma}$,
the required bound $\|v(X)\| \leq c'$ remains valid.

%%%%%%%%%%%%%%%%%%%%%%%%%%%%%%%%%%%%%%%%%%%%%%%%%%%%%%%%%%%%%%%%%%%%%%%%%%%%%%%

\subsection{Extension of stability analysis}
\label{sec:stability2}

In this and the next section,
we verify that the results of \cite[Section~\ref{IE-sec:IE}]{BS-rg-IE}
remain valid with $\ell^\oldrm$ replaced by $\ell$.
In this section, we deal with the results whose proofs need only minor
modification.

First, we note that the supporting results of \cite[Section~\ref{IE-sec:W}]{BS-rg-IE} hold with the new norms.
Indeed,
it is immediate from \eqref{e:norm-change} that analogues of
\cite[Proposition~\ref{IE-prop:Wnorms}]{BS-rg-IE} and
\cite[Lemmas~\ref{IE-lem:epdV}, \ref{IE-lem:W-logwish}--\ref{IE-lem:Wbil}]{BS-rg-IE} hold with the new $\ell_j$.
Moreover, \cite[Lemma~\ref{IE-lem:Fpibd-bis}]{BS-rg-IE} and \cite[Proposition~\ref{IE-prop:Wbounds}]{BS-rg-IE} hold
for general values of the parameters $\h_j$ (which are implicit in the $T_{0,j}$-norm).
We discuss \cite[Proposition~\ref{IE-prop:1-LTdefXY}]{BS-rg-IE}
in Section~\ref{sec:cc} below,
and the remaining results of \cite[Section~\ref{IE-sec:W}]{BS-rg-IE} do not make use of norms.

\paragraph{\cite[Proposition~\ref{IE-prop:Iupper}]{BS-rg-IE}}
With $\h = \ell$, \cite[\eqref{IE-e:Iupper-a}]{BS-rg-IE} continues to hold with the same proof;
in fact the proof does not depend on the explicit choice of $\h$.
We do not need \cite[\eqref{IE-e:Iupper-b}]{BS-rg-IE} as it is only applied with $\h = h$.

\paragraph{\cite[Proposition~\ref{IE-prop:Istab}]{BS-rg-IE}}
The only change to the proof is for the case $j_* = j + 1$.
To get \cite[\eqref{IE-e:IF}]{BS-rg-IE},
we proceed as previously in the case $\h=h$ but applying Lemma~\ref{lem:mart}
rather than \cite[Lemma~\ref{IE-lem:mart}]{BS-rg-IE} following \cite[(5.22)]{BS-rg-IE}.
In the same way, we get \cite[\eqref{IE-e:Iass}]{BS-rg-IE} and the remaining parts of
the proposition follow without changes to the proof.

\paragraph{\cite[Proposition~\ref{IE-prop:Ianalytic1:5}]{BS-rg-IE}}
Again the only required change in the proof is the use of
Lemma~\ref{lem:mart} in the case $j_* = j + 1$,
for which as previously we use Lemma~\ref{lem:mart} instead of \cite[Lemma~\ref{IE-lem:mart}]{BS-rg-IE}.

\paragraph{\cite[Proposition~\ref{IE-prop:JCK-app-1}]{BS-rg-IE}}
No changes need to be made to the proof.
In fact, it is necessary \emph{not} to use the $\h = \ell$ case
of the estimate \cite[(5.32)]{BS-rg-IE}. Instead, the
$\h = \ell^\oldrm$ case of this estimate should be used for $g_Q$.
This is possible since the renormalisation
group map, and in particular the coupling constants, are independent of the choice of norm.

\paragraph{\cite[Proposition~\ref{IE-prop:hldg}]{BS-rg-IE}}
Using \refeq{norm-change}, we see that the proof
continues to hold
above the mass scale.
The only change to the proof is that in the application of
\cite[Proposition~\ref{IE-prop:Istab}]{BS-rg-IE}, $j$ should be replaced by
$j + 1$ in \cite[\eqref{IE-e:IF}]{BS-rg-IE} with $j_* = j + 1$ (corresponding
to the $G_{j+1}$ norm). This yields \cite[\eqref{IE-e:2Lprimeh1}]{BS-rg-IE}
with a $G_{j+1}$ norm on the left-hand side.

\paragraph{\cite[Proposition~\ref{IE-prop:h}]{BS-rg-IE}}
A version of \cite[Lemma~\ref{IE-lem:dIipV}]{BS-rg-IE} with the new $\ell$
continues to hold. This lemma makes use of $\hat\ell$,
which superficially depends on the choice of $\ell$ in its definition
\cite[\eqref{IE-e:ellhatdef}]{BS-rg-IE}. However, brief scrutiny of
\cite[\eqref{IE-e:ellhatdef}]{BS-rg-IE} reveals that
the apparent dependence on $\ell$ actually cancels and there is in fact no dependence.
Similarly,
\cite[Lemma~\ref{IE-lem:epdV}]{BS-rg-IE} continues to hold without
any changes to its proof.
The proof of \cite[Proposition~\ref{IE-prop:h}]{BS-rg-IE} then applies without change.

\paragraph{\cite[Proposition~\ref{IE-prop:ip}]{BS-rg-IE}}
With the new choice of $\ell$ (and $\Gcal = G$),
\cite[Lemma~\ref{IE-lem:dIip}]{BS-rg-IE} continues to hold with no changes to its proof.
Thus, by \cite[\eqref{IE-e:scale-change}]{BS-rg-IE}
and \cite[Lemma~\ref{IE-lem:dIip}]{BS-rg-IE},
\begin{align}
  \label{e:integration-property-pf}
     &\|\Ex_{j+1} \delta I^X \theta F(Y) \|_{T_{\varphi,j+1}(\ell_{j+1})}
     \nnb
     &\quad \leq
     \|\Ex_{j+1} \delta I^X \theta F(Y) \|_{T_{\varphi,j}(\ell_{j})} \nnb
     &\quad \leq
     \Econst^{|X|_j+|Y|_j}
     (C_{\delta V} \epdV)^{|X|_j}
     \| F(Y) \|_{G_j(\ell_{j})}
    G_j(X\cup Y,\varphi)^5
    .
\end{align}
By Lemma~\ref{lem:mart}, $G_j(X\cup Y,\varphi)^5 \le G_{j+1}(X\cup Y,\varphi)$.
Now we divide both sides by
$G_{j+1}(X \cup Y, \varphi)$ and take the supremum over $\varphi$ to complete the proof.

%%%%%%%%%%%%%%%%%%%%%%%%%%%%%%%%%%%%%%%%%%%%%%%%%%%%%%%%%%%%%%%%%%%%%%%%%%%%%%%

\subsection{Extension of the crucial contraction}
\label{sec:cc}

The proof of the ``crucial contraction''
\cite[Proposition \ref{IE-prop:cl}]{BS-rg-IE}
makes use of the  third estimate in
\eqref{e:h-assumptions-IE}, which is now violated above the mass scale
due to our new choice of $\ell_j$.
On the other hand, the second estimate of \eqref{e:h-assumptions-IE} is
improved by the new choice and compensates for the degraded third estimate,
as we explain in this section.

\subsubsection{The operator $\Loc$}

The space $\Phi$ of test functions and the space $\Ncal$ of field functionals
are dual to one another via the pairing \eqref{e:pairing}. By exploting this,
in \cite{BS-rg-IE}, the operator $\Loc$ is defined as a kind of adjoint to
an operator $\Tay_a$, which replaces test functions by a lattice Taylor expansion
at $a\in\Lambda$.
Non-constant polynomials are not well-defined on the whole torus $\Lambda$, but
such a Taylor expansion can nevertheless be defined for test functions supported
on sequences whose components lie in a sufficiently ``small'' subset of $\Lambda$.
These are referred to in \cite{BS-rg-IE} as \emph{coordinate patches}. By definition,
they are nonempty and any element of the set $\Scal$ of small sets is a coordinate
patch (in this thesis, we are ultimately only concerned with the case of small sets).

Suppose we fix a coordinate patch $\Lambda' \subset \Lambda$. By definition, it
can be identified with a rectangle in $\Zd$.
% Via this identification,
Then given a local monomial $M$ of the form \eqref{e:field-mon},
we define $p_M\in\Phi$ by
\begin{equation}
\label{e:lattice-mon}
% p(\vec x) = \prod_{k=1}^m x_k^{\alpha_k}
p_M(x_1, \ldots, x_p)
	=
x_1^{\alpha_1} \ldots x_p^{\alpha_p},
	\qquad
x_1, \ldots, x_p \in \Lambda'
\end{equation}
% when $x_1, \ldots, x_m \in \Lambda'$ and we set $p(\vec x) = 0$ otherwise;
and set $p_M(\vec x) = 0$ if $|\vec x| \ne p$ or if the lattice points in $\vec x$
do not all lie in $\Lambda'$. Following \eqref{e:mon-dim},
we define the \emph{dimension} of such a monomial to be the dimension of $M$,
i.e.\ $m [\varphi] + |\alpha|$.
We let $d_+ \ge 0$ and let $\Pi = \Pi[\Lambda']$ denote the span of the monomials
of this form with dimension at most $d_+$. For $X \subset \Lambda'$, we can also define
$\Pi(X) = \Pi[\Lambda'](X)$ as the subset of $\Pi$ consisting of test functions
supported on sequences over $X$.

For $a\in\Lambda'$, in \cite{BS-rg-loc} an operator $\Tay_a : \Phi \to \Pi$ is defined
by a lattice analogue of Taylor expansion; although the monomials $p_M$ form an obvious
basis with respect to which this expansion can be defined, a different basis is used in
\cite{BS-rg-loc}. The operator $\Tay_a g$ satisfies lattice analogues of the usual properties
of Taylor polynomials. We will not discuss this in further detail as we will only use the fact
that $\Tay_a g \in \Pi$ here.

\begin{rk}
\label{rk:TayX}
By expansion in a basis of $\Pi(X)$, one can also defne a lattice Taylor expansion map
$\Phi \to \Pi(X)$. We will also denote this map by $\Tay_a$.
\end{rk}

The following, which is a restatement of \cite[Proposition~\ref{loc-prop:LTsymexists}]{BS-rg-IE},
defines $\Loc_X F$ as the unique element of $\Vcalp(X)$ that agrees with $F$ to
order $d_+$ in an appropriate sense.

\begin{prop}
\label{prop:LTsymexists}
Let $X\subset\Lambda$ be a coordinate patch and let $F\in\Ncal(X)$.
Then there is a unique polynomial $\Vp\in\Vcalp$ such that
\begin{equation}
\langle F, g \rangle_0 = \langle \Vp(X), g \rangle_0
\end{equation}
for all $g\in\Pi$. We write $\Loc_X F = \Vp(X)$.
\end{prop}

Define the seminorm
\begin{equation}
\label{e:Phitilde}
\|g\|_{\tilde\Phi(X)} = \inf\{ \|g - f\|_\Phi : f\in\Pi(X) \}
\end{equation}
on $\Phi$ (this is used in the definition of the large-field
regulator \eqref{e:9Gdef}).
We will need the following lemma, which is a restatement of
\cite[Lemma~\ref{loc-lem:testfndecomp}]{BS-rg-loc}.

\begin{lemma}
\label{lem:testfndecomp}
Let $X$ be a coordinate patch and let $g\in\Phi$.
There exists $f\in\Pi(X)$ such that, with $h = g - f$, we have
$\|g\|_{\tilde\Phi(X)} \le \|h\|_\Phi \le (1 + \epsilon) \|g\|_{\tilde\Phi(X)}$
and $\|f\|_\Phi \le (2 + \epsilon) \|g\|_\Phi$.
\end{lemma}

\subsubsection{Proof of the crucial contraction}

The space $\Ncal$ containing the functionals $F$ appearing above requires control on
up to $p_\Ncal$ derivatives of $F$ with respect to the field $\varphi$,
where $p_\Ncal$ is a parameter of the $T_\varphi$-seminorm.
In the proof of Proposition~\ref{prop:cl}, we must choose $p_\Ncal$ to be large
depending on $p$, in order to analyse the correlation length
of order $p$.  The renormalisation group analysis is predicated on fixed (but arbitrary)
$p_\Ncal$, so it can proceed with this modification.  However,
we do not prove that constants are uniform in $p_\Ncal$,
and in particular we do not prove that the required smallness of $g$ in
Theorem~\ref{thm:mr}(iii) is uniform in the choice of $p_\Ncal$.
Thus we do not have a result for \emph{all} $p>0$ for any fixed $g$.

Below the mass scale, we continue to use the crucial contraction as stated in
\cite[Proposition \ref{IE-prop:cl}]{BS-rg-IE} in terms of two norm pairs.
Next, we state a version of the crucial contraction for use above the mass
scale using the new norm pair \eqref{e:npmass}.
% The statement uses the notation of \cite{BS-rg-IE} (which we do not redefine here),
% with the exception that now we have replaced $a$ by $0$, $b$ by $x$, and $j_{ab}$ by $j_x$
% for consistency with our present notation.
Throughout this section, we
sometimes write the dimension as $d$ for emphasis, although we only consider $d=4$.

\commentbw{Some missing notation.}

\begin{prop}[{Improvement of \cite[Proposition \ref{IE-prop:cl}]{BS-rg-IE}}]
\label{prop:cl} Let $j_m \leq j<N$ and $V\in \DV_j$.  Let $X \in \Scal_j$ and
$U = \overline X$.  Let $F(X) \in \Ncal(X^\Box)$ be such that
$\pi_\alpha F(X) =0$ when $X(\alpha)=\varnothing$, and such that
$\pi_{0x}F(X)=0$ unless $j \ge j_x$.
{There is a constant $C$ (independent of $L$) such that}
\begin{align}
    \label{e:contraction3z-new}
    \|\Ipttil^{U\setminus X} \Ex_{C_{j+1}} \theta F (X) \|_{{G_{j+1}(\ell_{j+1})}}
    &
    \leq C \Big(
    ( L^{-d -1} +  L^{-1}\1_{X \cap \{0,x\} \not = \varnothing} )
    \kappa_F
    + \kappa_{\LT F}
    \Big)
    ,
\end{align}
with $\kappa_F=\|F (X)\|_{{G_{j}(\ell_{j})}}$ and
$\kappa_{\LT F} =\|\Ipttil^X \LT_X \Ipttil^{-X} F(X) \|_{{G_{j}(\ell_{j})}}$.
\end{prop}

An ingredient in the proof of Proposition~\ref{prop:cl} is
\cite[Lemma~\ref{loc-lem:phij}]{BS-rg-loc}, which is the $s=0$ version of
the following lemma.  For simplicity, we state only the conclusion of the lemma,
\todo{and the notation and hypotheses are those in}
\cite[Lemma~\ref{loc-lem:phij}]{BS-rg-loc}, except now we use the
$s$-dependent norm parameters
$\h_j=\ell_j$ of \refeq{elldef} ($h_j$ is not needed above the mass scale, and
the $s=0$ case applies below the mass scale).

The proof of Lemma~\ref{loc-lem:phij} with $s = 0$
is based on the assumption  $\ell_{j+1}/\ell_j \leq  cL^{-1}$.
(we take $[\varphi_i]=1$; the parameters $\ell_{\sigma,j}$ are not used).
For our new values of $\ell$, the stronger assumption
$\ell_{j+1}/\ell_j \leq L^{-1 - s \1_{j \geq j_m}}$ holds.
The unique change to the proof occurs in the transition from
\cite[\eqref{loc-e:gTay1}]{BS-rg-loc} to
\cite[\eqref{loc-e:rhognew}]{BS-rg-loc}, where the ratio
$\ell_{j+1}/\ell_j$ is used.

In the following, we let $\Phi = \Phi_j(\h_j)$ norm and let $\Phi' = \Phi_{j+1}(\h_{j+1})$.
We employ similar conventions for $\Phi(X)$ and $\tilde\Phi(X)$. The constant $d_+'$
is defined in \cite[\eqref{loc-e:dplusprimedef}]{BS-rg-loc} and in this context becomes
$d_+' = d_+ + 1$. The enlargement $X_+$ of a polymer $X\in\Pcal_j$ is defined by replacing
each block $B\in\Bcal_j(X)$ by a cube of twice the side length of $B$ (minus $1$ if $L^j$
is odd) that is centered at $B$.

\begin{lemma}[{Improvement of \cite[Lemma~\ref{loc-lem:phij}]{BS-rg-loc}}]
\label{lem:phiij-improved}
With the same hypotheses and notation as in \cite[Lemma~\ref{loc-lem:phij}]{BS-rg-loc},
\begin{equation}
\lbeq{gbd-improved}
\|g\|_{\tilde{\Phi} (X)}
\leq
\bar C_3
L^{- (1 + s \1_{j \geq j_m}) d_+'}  \|g\|_{\tilde{\Phi}' (X_+)}.
\end{equation}
\end{lemma}


\begin{proof}
Assume without loss of generality that $X$ is connected.
Let $f\in\Pi(X)$ and $h\in\Phi$ be as in Lemma~\ref{lem:testfndecomp}.
Thus, $g = f + h$ and so (with $\Tay_a : \Phi \to \Pi(X)$ as in Remark~\ref{rk:TayX})
we have $g - (h - \Tay_a h) = f + \Tay_a h \in \Pi(X)$, where $a$ is the largest
point which is lexicographically no larger than any point in $X$.
By definition of the $\tilde\Phi(X)$ seminorm,
\begin{equation}
\|g\|_{\tilde\Phi(X)}
	=
\|h - \Tay_a h\|_{\tilde\Phi(X)}
	\le
\|h - \Tay_a h\|_{\Phi(X)}.
\end{equation}
By the bound on $h$ (from Lemma~\ref{lem:testfndecomp}), it suffices to
show that
\begin{equation}
\|h - \Tay_a h\|_{\Phi(X)}
	\le
\frac12 \bar C_3 L^{-(1 + s \1_{j \ge j_m}) d'_+} \|g\|_{\tilde\Phi'(X_+)}.
\end{equation}

To this end, let $r = h - \Tay_a h$. By \cite[Lemma~\ref{loc-lem:gX}]{BS-rg-loc}
with $t = 1/2$, there exists $K > 1$ such that
\begin{equation}
\|r\|_{\Phi(X)}
	\le
\sup_{\vec x \in \bf{X}_+}
	(K \ell_j^{-1})^{\vec x}
\sup_{|\beta|_\infty \le p_\Phi}
	L^{j |\beta|_1} |\nabla^\beta r_{\vec x}|
\end{equation}
where $A^{\vec x} = A^{|\vec x|}$ and $\bf{X}_+$ is the set of sequences whose
components lie in $X_+$. In other words, we can estimate the $\Phi(X)$
norm of $r$ in terms of the values of $r$ and its derivatives in the enlargement
$X_+$ of $X$.

With the new ratio \eqref{e:h-assumptions},
% \cite[\eqref{loc-e:rhognew}]{BS-rg-loc}
we can rewrite this as
\begin{align}
\label{e:rhognew-improved}
\|r\|_{\Phi (X)}
	& \le
\sup_{{\vec x} \in {\mathbf X}_+}
	(K\ell_{j+1}^{-1})^{\vec x}
\sup_{|\beta|_\infty \le p_\Phi}
	L^{-(|{\vec x}| + |{\vec x}| s \1_{{j\geq j_m}} +|\beta|_1)}
	L^{(j+1) |\beta|_1}
	| \nabla^\beta  r_{{\vec x}}  |,
\end{align}
replacing \cite[\eqref{loc-e:rhognew}]{BS-rg-loc}.
% Here $r = h - \Tay_a h$, where $h$ is an arbitrary test function and $a$ is the largest
% point which is lexicographically no larger than any point in $X$.
% The test function $h$ depends on sequences of points $(x_1, \dots, x_p)$,
% and $\Tay_a h$ is a discrete version of Taylor's approximation which approximates $h$ by a
% discrete Taylor polynomial localised at point $a$ in each argument (see \cite{BS-rg-loc} for details).

By definition, for the empty sequence $\varnothing$, $(\Tay_a h)_\varnothing = h_\varnothing$,
and thus $r_\varnothing = 0$.
It follows that we can take $|\vec x| \geq 1$ in the supremum over ${\vec x} \in \mathbf{X}_+$ in
\eqref{e:rhognew-improved}. Thus,
\begin{align}
    \|r\|_{\Phi (X)}
    & \le
    L^{-s \1_{{j\geq j_m}}}
    \sup_{{\vec x} \in {\mathbf X}_+}
    (K\ell_{j+1}^{-1})^{\vec x}
    \sup_{|\beta|_\infty \le p_\Phi}
    L^{-(|\vec x| +|\beta|_1)}
    L^{(j+1) |\beta|_1}
    | \nabla^\beta  r_{{\vec x}}  |.
\end{align}
The quantity
\begin{align}
\label{e:rhognew}
    \sup_{{\vec x} \in {\mathbf X}_+}
    (K\ell_{j+1}^{-1})^{\vec x}
    \sup_{|\beta|_\infty \le p_\Phi}
    L^{-(|\vec x| +|\beta|_1)}
    L^{(j+1) |\beta|_1}
    | \nabla^\beta  r_{{\vec x}}  |
\end{align}
is identical to the right-hand side of \cite[\eqref{loc-e:rhognew}]{BS-rg-loc}
when $[\varphi_i] = 1$ and is bounded in the same way. Namely, it is shown in
\cite{BS-rg-loc} that this quantity can be bounded by a constant times
\begin{align}
    L^{-d_{+}'}
    \|h\|_{\Phi'( X_+)}.
\end{align}
% Thus,
% \begin{equation}
%     \|r\|_{\Phi (X)}
%     \leq \bar C_3
%     L^{-s \1_{{j\geq j_m}}}
%     L^{-d_{+}'}
%     \|h\|_{\Phi'( X_+)}.
% \end{equation}
% The result follows from the fact that $\|h\|_{\Phi'(X_+)} \le 2 \|g\|_{\tilde\Phi'(X_+)}$.
% With this improvement to \cite[\eqref{loc-e:rhognew}]{BS-rg-loc} in the proof of
% \cite[Lemma~\ref{loc-lem:phij}]{BS-rg-loc}, the conclusion of
% \cite[Lemma~\ref{loc-lem:phij}]{BS-rg-loc} is improved to \refeq{gbd-improved}.
\end{proof}

Roughly speaking, the $L$-dependent factor in \eqref{e:gbd-improved} implements the dimensional gain
for irrelevant directions in a renormalisation group step, when passing from one scale to the next.
In other words, we may regard the dimension of the field as improving from $1$ below the
mass scale to $1+s$ above the mass scale.
The $s=0$ version of Lemma~\ref{lem:phiij-improved} is adapted to the scaling at the critical point, where $m^2=0$.
In the noncritical case $m^2>0$, the dimensional gain improves greatly for $j>j_m$,
as apparent from \eqref{e:scaling-estimate}, and is
captured more accurately by the general-$s$ version of \eqref{e:gbd-improved}.

As a consequence of the former improvement we have the following two further improvements.
From now on, we always assume $\h=\ell$ and $j>j_m$, as this is the only case relevant for
the improvement of \cite[Proposition~\ref{IE-prop:cl}]{BS-rg-IE}.

\paragraph{\cite[Proposition~\ref{loc-prop:1-LTdefXY}]{BS-rg-loc}}
The improvement in Lemma~\ref{lem:phiij-improved} propagates to
\cite[Proposition~\ref{loc-prop:1-LTdefXY}]{BS-rg-loc}, which now holds
as stated except with
$\gamma_{\alpha,\beta}$
improved to
\begin{equation}
\label{e:cgamobs}
    \gamma_{\alpha,\beta}
        =
    \left(
    L^{-(d_\alpha' + s \1_{j \geq j_m})} +  L^{-(A+1)}
    \right)
    \left( \frac{\ell_{\sigma,j+1}}{\ell_{\sigma,j}} \right)^{|\alpha \cup \beta|}
    .
\end{equation}
The right-hand side can be estimated as follows.
By \eqref{e:h-assumptions},
\begin{equation}
\frac{\ell_{\sigma,j+1}}{\ell_{\sigma,j}} \leq
4
  \begin{cases}
  L^{1 + s \1_{{j \geq j_m}}} & j < j_x \\
  1 & j \geq j_x,
  \end{cases}
\end{equation}
and hence
\begin{equation}
\label{e:cgamobs-L}
    \gamma_{\alpha,\beta}
    \le C''
    \left(
    L^{-(d_\alpha' + s \1_{j \geq j_m})} +  L^{-(A+1)}
    \right)
    \times
    \begin{cases}
    L^{(1 + s \1_{{j \geq j_m}})(|\alpha \cup \beta|)} & j < j_x \\
    1 & j \geq j_x.
    \end{cases}
\end{equation}

\paragraph{\cite[Proposition~\ref{IE-prop:1-LTdefXY}]{BS-rg-IE}}
As we explain next, using \refeq{cgamobs} and
identical notation to that defined in and around
\cite[Proposition~\ref{IE-prop:1-LTdefXY}]{BS-rg-IE},
the proposition holds as stated also for
the improved norms, provided we take
$A \ge 5+s$.
For this, what is required is to show that under the hypotheses of
\cite[Proposition~\ref{IE-prop:1-LTdefXY}]{BS-rg-IE},
the $\gamma_{\alpha,\beta}$ that arise in its proof obey
\begin{equation}
\lbeq{gamalphbet}
\gamma_{\alpha,\beta} \leq C
  \begin{cases}
  L^{-5} & |\alpha\cup\beta| = 0 \\
  L^{-1}  & |\alpha\cup\beta| = 1, 2.
  \end{cases}
\end{equation}
For $|\alpha\cup\beta| = 0$,
the first term of \refeq{cgamobs-L} obeys the bound of \refeq{gamalphbet},
since $d_\varnothing'=d+1$.
For the remaining cases, $d_\alpha'=2$ for $j < j_x$ and $d_\alpha'=1$ for $j \ge j_x$.
For $|\alpha\cup\beta| = 2$, the assumption that $F_1,F_2,F_1F_2$ have no component
in $\Ncal_{0x}$ unless $j \geq j_x$ means that we are in the case with no
growth due the ratio $\ell_{\sigma,j+1}/\ell_{\sigma,j}$ in \refeq{cgamobs-L},
and its first term again obeys the bound
\refeq{gamalphbet} with room to spare.
Finally, when $|\alpha\cup\beta| = 1$,
the first term of \refeq{cgamobs-L} also obeys the estimate
\refeq{gamalphbet}, and again with room to spare.
Concerning the second term of \refeq{cgamobs-L}, given
our choice of $A$ and the fact that we need only consider the growing factor in
\refeq{cgamobs-L} for $|\alpha\cup\beta|=1$, it suffices to observe that
\begin{equation}
L^{-(A + 1)}
L^{1 + s \1_{j \geq j_m}} \leq L^{- 5}.
\end{equation}
This completes the proof of the improved version of
\cite[Proposition~\ref{IE-prop:1-LTdefXY}]{BS-rg-IE}.

\begin{proof}[Proof of Proposition~\ref{prop:cl}]
We complete the proof of Proposition~\ref{prop:cl} by modifying the proof of
\cite[Proposition~\ref{IE-prop:cl}]{BS-rg-IE} above the mass scale.
The estimate \cite[\eqref{IE-e:Lkpfii-1zbis}]{BS-rg-IE}
follows from \cite[Proposition~\ref{IE-prop:ip}]{BS-rg-IE}
as an estimate in terms of the modified norm pair \eqref{e:npmass},
for which \cite[Proposition~\ref{IE-prop:ip}]{BS-rg-IE} was verified in Section~\ref{sec:stability2}.
The bound \cite[\eqref{IE-e:cl-1}]{BS-rg-IE} with improved $\gamma$ is obtained by applying the improved
version of \cite[Proposition~\ref{IE-prop:1-LTdefXY}]{BS-rg-IE}. In the remainder
of the proof of \cite[Proposition~\ref{IE-prop:cl}]{BS-rg-IE}, we specialise each
occurrence of $\Gcal$ to the case $\Gcal = G$ and we conclude by obtaining an
analogue of \cite[\eqref{IE-e:FXbdKzzz}]{BS-rg-IE} with $\tilde G$ replaced by
$G$ by applying Lemma~\ref{lem:mart} rather than \cite[Lemma~\ref{IE-lem:mart}]{BS-rg-IE}.

An additional detail is that it is required that we choose the parameter defining the space $\Ncal$ to
obey $p_\Ncal >A$.
Since we have changed $A$ (depending on $s$), we must make a corresponding change to $p_\Ncal$. This does not pose problems
(beyond the previously discussed requirement that $g$ needs to be chosen small depending on $p$),
as this parameter may be
fixed to be an arbitrary and sufficiently large integer
(see \cite[Section~\ref{phi4-sec:pNcal}]{ST-phi4} where this point is addressed
in a different context).  Similarly, the value of $A$ is immaterial and can be
any fixed number in the proof of
\cite[Proposition \ref{IE-prop:cl}]{BS-rg-IE}.
\end{proof}