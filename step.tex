\chapter{Renormalisation group step}

%%%%%%%%%%%%%%%%%%%%%%%%%%%%%%%%%%%%%%%%%%%%%%%%%%%%%%%%%%%%%%%%%%%%%%%%%%%%%%%
%%%%%%%%%%%%%%%%%%%%%%%%%%%%%%%%%%%%%%%%%%%%%%%%%%%%%%%%%%%%%%%%%%%%%%%%%%%%%%%

\section{Improved norm}
\label{sec:Rpf1}

The proof of Theorem~\ref{thm:step-mr-fv} is based on the observation that
it is possible to use the parameters \refeq{elldef-zz}
in the norm used in \cite{BS-rg-IE}, instead of the $s=0$ version used
previously.  In this section, we first
state improved covariance estimates, thereby indicating why it is possible
to improve the norm.
This leads to a discussion of simplified norm pairs beyond the mass
scale.  A lemma concerning the fluctuation-field regulator indicates why the
simplification is possible.
In the following, we use the notation appropriate for the spin field
$\varphi \in (\R^n)^\Lambda$ for $n \ge 1$; only notational modifications are needed for
$n=0$.

%%%%%%%%%%%%%%%%%%%%%%%%%%%%%%%%%%%%%%%%%%%%%%%%%%%%%%%%%%%%%%%%%%%%%%%%%%%%%%%

\subsection{Covariance bounds}
\label{sec:Cbds}

% The starting point for the renormalisation group method leading to \eqref{e:vq-new}
% is a finite range decomposition of the covariance
% $(-\Delta_\Lambda + m^2)^{-1} =C_1+C_2+\ldots + C_{N-1}+C_{N,N}$, provided by \cite{BGM04,Baue13a}.  Properties
% of this covariance decomposition can be found in
% \cite[Section~\ref{pt-sec:Cdecomp}]{BBS-rg-pt}, and we do not repeat them all here.

The estimate in \cite{ST-phi4} which yields the $s = 0$ case of \refeq{Rab-bound}
uses the norms defined in \cite{BS-rg-IE}.
One of these norms is the $\Phi_j(\ell_j)$ norm defined by
\begin{equation}
\lbeq{phinorm}
\|\varphi\|_{\Phi_j(\ell_j)}
=
\ell_j^{-1}
\sup_{x\in \Lambda}
\sup_{|\alpha|_1  \le p_\Phi}
L^{j|\alpha|_1}
|\nabla^{\alpha} \varphi_x|,
\end{equation}
which depends on the parameter $\ell_j$,
and on the maximal number of discrete derivatives $p_\Phi$
(fixed to be at least $4$ in \cite{BS-rg-IE}).
% In this paper, we set
% \begin{equation}
% % \lbeq{elldefs}
% \ell_j = \ell_0 L^{-j - s (j - j_m)_+}.
% \end{equation}
As in \refeq{elldef-zz}, we now define
\begin{align}
\label{e:elldef}
\ell_j &= \ell_0 L^{-j - s (j - j_m)_+}, \quad
\ell_{\sigma,j}
=
\ell_{j \wedge j_{x}}^{-1} 2^{(j - j_{x})_+} \ggen_j.
\end{align}
The analysis of \cite{BS-rg-IE,BS-rg-step} uses the norm parameters $\ell_j$ and $\ell_{\sigma,j}$ with $s = 0$.
To distinguish these from our
new choice \refeq{elldef} of $\ell_j$ and $\ell_{\sigma,j}$, we write
\begin{equation}
\label{e:ell-old}
    \ell_j^\oldrm = \ell_0 L^{-j},
    \quad
    \ell_{\sigma,j}^\oldrm  =
    (\ell_{j \wedge j_{x}}^{\rm old})^{-1}2^{(j - j_{x})_+}\ggen_j.
\end{equation}

In the more general terminology and notation of \cite{BS-rg-norm,BS-rg-IE},
we may regard a covariance $C_j$
in the decomposition \eqref{e:NCj}
as a test function depending on
two arguments $x,y$, and with this identification its $\Phi_j(\ell_j)$
norm is
\begin{equation}
    \label{e:Phinorm}
    \|C_j\|_{\Phi_{j}(\ell_j)}  =
    \ell_j^{-2}
    \sup_{x,y\in \Lambda}
    \;
    \sup_{|\alpha|_1 + |\beta|_1 \le p_\Phi}
    L^{(|\alpha|_1+  |\beta|_1)j}
    |\nabla_x^{\alpha} \nabla_y^{\beta} C_{j;x,y}|.
\end{equation}
The purpose of the $\Phi_j(\ell_j)$ norm is to measure the size of typical
fluctuation fields $\varphi$ with covariance $C_j$.
The parameter $\ell_j$ is chosen so that the norm of a typical field should
be $O(1)$, independent of $j$.

The following lemma justifies our choice of $\ell_j$
in \refeq{elldef}, by showing that the
bound \cite[\eqref{IE-e:CLbd}]{BS-rg-IE}, proved there only for the $s=0$ version
$\ell_j^\oldrm$ of \refeq{ell-old},
remains true with the stronger
choice of norm parameter $\ell_j$ that permits arbitrary $s \ge 0$.
%The sequence $\chicCov_j$ in the lemma is called $\chi_j$ in \cite{BS-rg-IE}, but here we use
%a different symbol to avoid confusion with the susceptibility.
In its statement, the bounded sequence $\chicCov_j$ decays exponentially after the
mass scale and may be taken to be equal to
$2^{-(j-j_m)_+}$; its details are given
in \cite[Section~\ref{IE-sec:frp}]{BS-rg-IE} (where it is called $\chi_j$ rather
than $\chicCov_j$).

\begin{lemma}
[{Extension of \cite[\eqref{IE-e:CLbd}]{BS-rg-IE}}]
\label{lem:Cbd}
Given $\ellconst \in (0, 1]$, $\ell_0$ can be chosen large (depending on $L,\ellconst,s$)
so that
\begin{equation}
\lbeq{Cbd}
\|C_j\|_{\Phi_{j}(\ell_j)} \leq \min(\ellconst, \chicCov_j).
\end{equation}
\end{lemma}



The proof of Lemma~\ref{lem:Cbd} uses an estimate from
\cite[Proposition~\ref{pt-prop:Cdecomp}]{BBS-rg-pt}, which we repeat here as
the following proposition.

\begin{prop}[{Restatement of \cite[Proposition~\ref{pt-prop:Cdecomp}(a)]{BBS-rg-pt}}]
\label{prop:Cdecomp}
  Let $d >2$, $L\geq 2$, $j \ge 1$, $\bar m^2 >0$.
  For multi-indices $\alpha,\beta$ with
  $\ell^1$ norms $|\alpha|_1,|\beta|_1$ at most
  some fixed value $p$,
  and for any $k$, and for $m^2 \in [0,\bar m^2]$,
  \begin{equation}
    \label{e:scaling-estimate}
    |\nabla_x^\alpha \nabla_y^\beta C_{j;x,y}|
    \leq c(1+m^2L^{2(j-1)})^{-k}
    L^{-(j-1)(d-2 +|\alpha|_1+|\beta|_1)},
  \end{equation}
  where $c=c(p,k,\bar m^2)$ is independent of $m^2,j,L$.
  The same bound holds for $C_{N,N}$ if
  $m^2L^{2(N-1)} \ge \varepsilon$ for some $\varepsilon >0$,
  with $c$ depending on $\varepsilon$ but independent of $N$.
\end{prop}

\begin{proof}[Proof of Lemma~\ref{lem:Cbd}]
For $d=4$, insertion of \refeq{scaling-estimate} into \refeq{Phinorm} gives
\begin{equation}
    \label{e:Phinorm2}
    \|C_j\|_{\Phi_{j}(\ell_j)}
    \le
    c
    L^{p_\Phi}
    \ell_j^{-2}(1+m^2L^{2(j-1)})^{-k}
    L^{-2(j-1)}.
\end{equation}
With $s=0$ in \eqref{e:elldef}, \refeq{Phinorm2} gives
$\|C_j\|_{\Phi_{j}(\ell_j)} \le c_L \ell_0^{-2} (1+m^2L^{2(j-1)})^{-k}$
for an $L$-dependent constant $c_L$ (whose value may now change from line to line).
The estimate \cite[\eqref{IE-e:CLbd}]{BS-rg-IE}
is wasteful in that it does not make any use of the factor
$(1+m^2L^{2(j-1)})^{-k}$ in \refeq{Phinorm2} beyond extraction of the factor $\chicCov_j$.
To improve this, we now allow arbitrary $s$, and fix the arbitrary parameter $k$ to be $k=s+1$
in \refeq{Phinorm2} so that
\begin{equation} \label{e:mass-decay}
(1 + m^2 L^{2j})^{-k} \le c_L L^{-2(s+1)(j - j_m)_+}.
\end{equation}
We insert \refeq{mass-decay} and the definition $\ell_j=\ell_0 L^{-j-s(j-j_m)_+}$ from
\refeq{elldef} into
\eqref{e:Phinorm2}, to conclude that there exists $c_0 = c_0(s, L)$ such that
\begin{equation}
    \|C_j\|_{\Phi_{j}(\ell_j)} \leq c_0 \ell_0^{-2} L^{-2(j - j_m)_+}
    .
\end{equation}
By definition of $\chicCov_j$ (see \cite[Section~\ref{IE-sec:frp}]{BS-rg-IE}),
$L^{-2(j - j_m)_+}$ is bounded by a multiple of $\chicCov_j$.
It thus suffices to choose $\ell_0$ large enough that
$\ell_0^2 \ge c_0 \ellconst^{-1}$.
\end{proof}

%%%%%%%%%%%%%%%%%%%%%%%%%%%%%%%%%%%%%%%%%%%%%%%%%%%%%%%%%%%%%%%%%%%%%%%%%%%%%%%

\subsection{New choice of norm beyond the mass scale}
\label{sec:newnorm}


As in \cite[(\ref{IE-e:PhiXdef})]{BS-rg-IE}, we use the localised version
of \eqref{e:phinorm}, defined for subsets
$X \subset \Lambda$  by
\begin{align}
\label{e:PhiXdef}
    \|\varphi\|_{\Phi_j(X)}
    &=
    \inf \{ \|\varphi -f\|_{\Phi_j} :
    \text{$f \in \C^\Lambda$ such that $f_{x} = 0$
    $\forall x\in X$}\}.
\end{align}
A \emph{small set} is defined to be a
connected polymer $X \in \Pcal_j$ consisting of at most $2^d$ blocks
(the specific number
$2^d$ plays no direct role here),
and $\Scal_j \subset \Pcal_j$ denotes the set of small sets.
The \emph{small set neighbourhood} of $X \subset \Lambda $ is
the enlargement of $X$ defined by
$
    X^{\Box}
=
    \bigcup_{Y\in \Scal_{j}:X\cap Y \not =\varnothing } Y$.

Given $X \subset \Lambda$ and $\varphi \in (\R^n)^{\Lambda}$,
we recall from \cite[\eqref{IE-e:GPhidef}]{BS-rg-IE}
that the
\emph{fluctuation-field regulator} $G_j$
is defined by
\begin{align}
\label{e:GPhidef}
    G_j(X,\varphi)
    =
    \prod_{x \in X} \exp
    \left(|B_{x}|^{-1}\|\varphi\|_{\Phi_j (B_{x}^\Box,\ell_j )}^2 \right)
    ,
\end{align}
where $B_{x}\in \Bcal_j$ is the unique block that contains $x$,
and hence $|B_x| = L^{dj}$.
The \emph{large-field regulator} is defined in \cite[\eqref{IE-e:9Gdef}]{BS-rg-IE} by
\begin{align}
\label{e:9Gdef}
    \tilde G_j  (X,\varphi)
    =
    \prod_{x \in X}
    \exp \left(
    \frac 12 |B_{x}|^{-1}\|\varphi\|_{\tilde\Phi_j (B_{x}^\Box,\ell_j)}^2
    \right)
    .
\end{align}
The $\tilde\Phi_j$ norm appearing on the right-hand side of \refeq{9Gdef} is
similar to the $\Phi_j$ norm, with the important difference that it is insensitive to
shifts by linear test functions; see \cite[\eqref{IE-e:Phitilnorm}]{BS-rg-IE} for the
precise definition.
The two regulators serve as weights in the \emph{regulator norms} of
\cite[Definition~\ref{IE-def:Gnorms}]{BS-rg-IE}.
The regulator norms are defined,  with $\gamma \in (0,1]$ and
for $F$ in the space $\Ncal(X^\Box)$ of functionals
of the field (see \cite[\eqref{norm-e:NXdef}]{BS-rg-norm}), by
\begin{align}
\label{e:Gnormdef1}
    \| F\|_{G_j(\ell_j)}
    &=
    \sup_{\varphi \in (\R^n)^\Lambda}
    \frac{\|F\|_{T_{\varphi,j}(\ell_j)}}{G_{j}(X,\varphi)}
    ,
\\
\label{e:Gnormdef2}
    \|F\|_{\tilde G_j^{\Gtilp}(h_j)}
    &=
    \sup_{\varphi \in (\R^n)^\Lambda}
    \frac{\|F \|_{T_{\varphi,j}(h_j)}}{\tilde{G}^{\Gtilp}_{j}(X,\varphi)}
    .
\end{align}
The parameter $\ell_j$ that appears in the regulators \refeq{GPhidef}--\refeq{9Gdef} and
in the numerator of \refeq{Gnormdef1} was taken to be $\ell_j^\oldrm$ in \cite{BS-rg-IE},
but now we use $\ell_j$ instead. As in \cite{BS-rg-IE},
the parameter $h_j$ and its observable counterpart $h_{\sigma,j}$ are given by
\begin{align}
\label{e:h}
    h_{j} &= k_0 \ggen_j^{-1/4}L^{-j},
    \quad
    h_{\sigma,j}  = (\ell_{j \wedge j_{x}}^{\rm old})^{-1}
    2^{(j - j_{x})_+}\ggen_j^{1/4}.
\end{align}

In \cite{BS-rg-IE}, estimates on $\|\cdot\|_{j+1}$ are given in terms of
$\|\cdot\|_j$, where the pair $(\|\cdot\|_j, \|\cdot\|_{j+1})$ refers to
either of the norm pairs
\begin{equation}
\label{e:np1}
    \|F\|_j = \|F\|_{G_j(\ell_j^\oldrm)}
    \quad \text{and} \quad
    \|F\|_{j+1} = \|F\|_{T_{0,j+1}(\ell_{j+1}^\oldrm)},
\end{equation}
or
\begin{equation}
\label{e:np2}
    \|F\|_j = \|F\|_{\tilde{G}_j(h_j)}
    \quad \text{and} \quad
    \|F\|_{j+1} = \|F\|_{\tilde{G}_{j+1}^{\Gtilp}(h_{j+1})}.
\end{equation}
We will show that, \emph{above the mass scale} $j_m$ (see \eqref{e:jmdef}), the results of \cite{BS-rg-IE} hold  with
both norm pairs in \eqref{e:np1} and \eqref{e:np2} replaced by the single new norm pair
\begin{equation}
\label{e:npmass}
    \|F\|_j = \|F\|_{G_j(\ell_j)}
    \quad \text{and} \quad
    \|F\|_{j+1} = \|F\|_{G_{j+1}(\ell_{j+1})},
\end{equation}
with the improved $\ell_j$ of \eqref{e:elldef} with $s>0$ fixed as large as desired.

The space $\Ncal$ containing the functionals $F$ appearing above requires control on
up to $p_\Ncal$ derivatives of $F$ with respect to the field $\varphi$,
where $p_\Ncal$ is a parameter of the $T_\varphi$-norm.
In the proof of Proposition~\REF % \ref{prop:cl}
below, we must choose $p_\Ncal$ to be large depending on $p$,
in order to analyse the correlation length
of order $p$.  The renormalisation group analysis is predicated on fixed (but arbitrary)
$p_\Ncal$, so it can proceed with this modification.  However,
we do not prove that constants are uniform in $p_\Ncal$,
and in particular we do not prove that the required smallness of $g$ in
Theorem~\ref{thm:mr} is uniform in the choice of $p_\Ncal$.
Thus we do not have a result for \emph{all} $p>0$ for any fixed $g$.


The use of two norm pairs adds intricacy to \cite{BS-rg-IE,BS-rg-step}.
The pair \refeq{np1} is insufficient, on its own, because the scale-$(j+1)$ norm
is the $T_0$ semi-norm which controls only small fields, and an estimate in this norm
does not imply an estimate for the $G_{j+1}$ norm.  The norm pair \refeq{np2} is
used to supplement the norm pair \refeq{np1}, and estimates in both of the scale-$(j+1)$
norms can be combined to provide an estimate for the $G_{j+1}$ norm.  This then
sets the stage for the next renormalisation group step.  Above the mass scale,
the use of \refeq{npmass} now bypasses many issues.  For example, for $j>j_m$
 the $\Wcal_j$ norm of \cite[\eqref{step-e:9Kcalnorm}]{BS-rg-step} is replaced
 simply by the $\Fcal_j(G)$ norm, and there is no need for the $\Ycal_j$ norm of
\cite[\eqref{step-e:Ycaldef}]{BS-rg-step} nor for \cite[Lemma~\ref{step-lem:KKK}]{BS-rg-step}.

The need for both norm pairs \eqref{e:np1}--\eqref{e:np2} is discussed in
\cite[Section~\ref{IE-sec:lfp}]{BS-rg-IE} and is related to the
so-called \emph{large-field problem}. Roughly speaking, the
norm pair \refeq{np2} is used to take advantage of the quartic term in the interaction to
suppress the effects of large values of the fields. This approach
relies on the fact that the interaction polynomial is dominated by the
quartic term in the $h$-norm, as expressed by
\cite[\eqref{IE-e:tau2dom}]{BS-rg-IE}, together with the lower bound
\cite[\eqref{IE-e:epVbark0}]{BS-rg-IE} on the quartic term.
However, above the mass scale, large fields are naturally suppressed
by the rapid decay of the covariance.
This idea is captured in Lemma~\ref{lem:mart} below, which replaces
\cite[Lemma~\ref{IE-lem:mart}]{BS-rg-IE} above the mass scale.
The regulators in its statement are defined by \refeq{GPhidef} with the $s$-dependent
$\ell_j$ of \refeq{elldef}.


\begin{lemma}[{Replacement for \cite[Lemma~\ref{IE-lem:mart}]{BS-rg-IE}}]
\label{lem:mart}
Let $X \subset \Lambda$ and assume that $s > 1$.
For any $q >0$, if $L$ is sufficiently large depending on $q$, then for $j_m \leq j < N$,
\begin{equation}
\label{e:mart}
    G_{j}(X, \varphi)^{q}
    \le
    G_{j+1}(X, \varphi).
\end{equation}
\end{lemma}
\begin{proof}
By \eqref{e:GPhidef}, it suffices to show that, for any scale-$j$ block $B_j$ and any scale-$(j+1)$ block $B_{j+1}$ containing $B_j$,
\begin{equation}
q \|\varphi\|^2_{\Phi_j (B_j^\Box,\ell_j )}
\leq
L^{-4} \|\varphi\|^2_{\Phi_{j+1} (B_{j+1}^\Box,\ell_{j+1})}.
\end{equation}
In fact, since $\|\varphi\|_{\Phi_j (B_j^\Box,\ell_j )}
\leq \|\varphi\|_{\Phi_j (B_{j+1}^\Box,\ell_j )}$ by definition,
it suffices to prove the above bound with $B_j$ replaced by $B_{j+1}$ on the left-hand side.
According to the definition of the norm in \eqref{e:PhiXdef},
to show this it suffices to prove that
\begin{equation}
\lbeq{martwant}
q \|\varphi\|_{\Phi_j(\ell_j)}^2 \leq L^{-4} \|\varphi\|_{\Phi_{j+1}(\ell_{j+1})}^2
\end{equation}
(then we replace $\varphi$ by $\varphi -f$ in the above and take the infimum).

By definition,
\begin{equation}
\|\varphi\|_{\Phi_j(\ell_j)}
\le
\ell_j^{-1} \ell_{j+1}
\sup_{x\in \Lambda} \sup_{|\alpha| \leq p_\Phi}
\ell_{j+1}^{-1}
L^{(j+1) |\alpha|}
|\nabla^\alpha \varphi_x|,
\end{equation}
with the inequality due to replacement of $L^{j |\alpha|}$ on the left-hand
side by $L^{(j+1) |\alpha|}$ on the right-hand side.
Since $\ell_j^{-1} \ell_{j+1} = L^{-1 - s \1_{j \geq j_m}}$,
\begin{equation}
\|\varphi\|_{\Phi_j(\ell_j)} \leq L^{-1 - s \1_{j \geq j_m}} \|\varphi\|_{\Phi_{j+1}(\ell_{j+1})}.
\end{equation}
Thus,
\begin{equation}
q \|\varphi\|_{\Phi_j(\ell_j)}^2
\leq q L^{-4} L^{2 - 2s \1_{j \geq j_m}} \|\varphi\|^2_{\Phi_{j+1}(\ell_{j+1})},
\end{equation}
and then \refeq{martwant}
follows once $L$ is large enough that $q L^{2 - 2s} \leq 1$.
\end{proof}

\begin{rk}
The elimination of the $h$-norm after the mass scale is more than a convenience.
It becomes a necessity when we improve the $\ell$-norm.
Briefly, the reason is as follows. In the proof of
\cite[Lemma~\ref{step-lem:KKK}]{BS-rg-step}, the ratio
$\ell_{\sigma}/h_{\sigma}$
must be bounded. For this, we would need
to increase $h_{\sigma}$
beyond the mass scale  (since $\ell_{\sigma}$ has been increased).
This forces a compensating decrease in $h$
beyond $j_m$, to keep the product $hh_{\sigma}$ bounded for stability
(as in Section~\REF % \ref{sec:stability1}
below). But if we do this, we lose the lower bound required on $\epsilon_{g\tau^2}$
required for stability in the $h$-norm (see \cite[\eqref{IE-e:epVbardefz-app}]{BS-rg-IE}).
\end{rk}